\documentclass[letterpaper,twocolumn,amsmath,amssymb,pre,aps,10pt]{revtex4-1}
\usepackage{graphicx} % Include figure files
\usepackage{color}
\usepackage{nicefrac} % Include for in-line fractions
\usepackage{harpoon}% <---

\newcommand{\red}[1]{{\bf \color{red} #1}}
\newcommand{\green}[1]{{\bf \color{green} #1}}
\newcommand{\blue}[1]{{\bf \color{blue} #1}}
\newcommand{\cyan}[1]{{\bf \color{cyan} #1}}

\newcommand{\davidsays}[1]{{\color{red} [\green{David:} \emph{#1}]}}
\newcommand{\jpsays}[1]{{\color{red} [\blue{Jordan:} \emph{#1}]}}
\newcommand{\tssays}[1]{{\color{red} [\cyan{Tanner:} \emph{#1}]}}

\newcommand{\Int}{\int}
\newcommand\Emin{E_{\min}}
\newcommand\Nmax{N_{\max}}
\newcommand\Tmin{T_{\min}}
\newcommand\mumax{\mu_{\max}}

\begin{document}
\title{Two-dimensional Broad Histogram Monte Carlo}

\author{Jordan K. Pommerenck}
\author{Cade MIDDLE INITIAL? Trotter}
\author{David Roundy}
\affiliation{Department of Physics, Oregon State University,
  Corvallis, OR 97331}

\begin{abstract}
We present several histogram methods and compare the performance and
efficiency at treating the square-well fluid.
\end{abstract}

\maketitle

In component notation, the free energy can be broken down into its constituent components.
\begin{align}
\overrightharp{g} = \bigg(\frac{\partial S}{\partial U}\bigg)_N \hat{U} + \bigg(\frac{\partial S}{\partial S}\bigg)_U \hat{N}
\end{align}
\begin{align}
\overrightharp{g} = \frac{1}{\Tmin} \hat{U} - \frac{\mumax}{\Tmin}  \hat{N}
\end{align}
The gradient of the free energy can be written as follows (thinking of a $\Delta S$ near the minimum entropy $S_{\min}$:
\begin{align}
\Delta S = \left|\overrightharp{g} \cdot \overrightharp{r} \right| = \left|\frac{\Emin}{\Tmin} \right| + \left|\frac{\mumax\Nmax}{\Tmin} \right|
\end{align}
\begin{align}
\langle\Delta S \rangle = \frac{1}{3}\bigg( \left|\frac{\Emin}{\Tmin} \right| + \left|\frac{\mumax\Nmax}{\Tmin} \right| \bigg)
\end{align}
We can write the entropy as a parabolic function with free parameters $A$ and $B$ in terms of the energy and particle number of the
system.
\begin{align}
S = -A E^2 - B N^2
\end{align}
By taking to partial derivatives of the entropy, we can solve for the free parameters.
\begin{align}
\bigg(\frac{\partial S}{\partial E}\bigg)_N = \frac{1}{\Tmin}
    = - 2A\Emin
\end{align}
\begin{align}
\bigg(\frac{\partial S}{\partial N}\bigg)_E = - \frac{\mumax}{\Tmin}
    = - 2B\Nmax
\end{align}
Therefore the parabolic equation can now be written in terms of known parameters.
\begin{align}
S &= \left( \frac{1}{2 \Tmin \Emin}\right) E^2
    - \left( \frac{\mumax}{2 \Tmin \Nmax}\right) N^2
\end{align}
Now we integrate this functional form of entropy in terms of the internal energy $E$ and particle number $N$.
\begin{align}
\left<S-S_{\min}\right>
  &=\frac{1}{|\Emin|\Nmax}\Int_{0}^{\Nmax} \Int_{\Emin}^{0} \bigg( S - S_{\min}\bigg) \,dE\,dN
\end{align}
Now we insert our derived expressions into the integral and obtain a functional form for the update factor $\gamma$.
\begin{multline}
\left<S-S_{\min}\right>
  = \frac1{2|\Emin|\Nmax}\times\\
  \Int_{0}^{\Nmax} \!\!\!\!\!\int_{\Emin}^{0}\!\!
   \Bigg(\left( \frac{E^2}{\Tmin\Emin}
         - \frac{\mumax N^2}{\Tmin\Nmax}\right)
    \\
    - \bigg( \frac{\Emin}{\Tmin}  - \frac{\mumax\Nmax}{\Tmin} \bigg)\Bigg) \,dE\,dN
\end{multline}
We integrate this!
\begin{multline}
\left<S-S_{\min}\right>
  =
   \frac1{2}\times\Bigg(\frac13\left(\frac{\Emin}{\Tmin}
    -\frac{\mumax \Nmax}{\Tmin}
  \right)
    \\
    - \frac{\Emin}{\Tmin}  + \frac{\mumax\Nmax}{\Tmin} \Bigg)
\end{multline}
With a little algebra...
\begin{align}
\left<S-S_{\min}\right>
  = \frac13\frac{|\Emin|}{\Tmin}  + \frac13\frac{\mumax\Nmax}{\Tmin}
\end{align}
We then find that the result of the integral is equal to the following:
\begin{align}
\gamma =\frac{1}{t} \frac{\Emin\Nmax}{\Tmin}\bigg(\Emin + 7 \mumax\Nmax \bigg)
\end{align}

\bibliography{paper}% Produces the bibliography via BibTeX.

\end{document}
