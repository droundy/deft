\thispagestyle{plain}
\begin{center}
    \Large
    \textbf{Application of Generalized Renormalization Group Methods to the Square-Well Liquid Near Criticality}
    \\or\\
    \textbf{Decomposition of the Critical Square-Well Liquid Into Consitutent Cells}    \\or\\
    \textbf{Free Energy Decomposition of the Critical Square-Well Liquid Using a Renormalization Group Method}
   
    \vspace{0.8cm}
    \large
    A Computational Project
    
    \vspace{0.8cm}
    \textbf{Brenden Vischer}
    
    \vspace{1.2cm}
    \textbf{Abstract}
\end{center}

The free energy of a bulk liquid is a known quantity [cite], near and far from the critical region [cite. true?]. Computing the free energy of a liquid requires lengthy, difficult computations that [something]. These computations generally involve renormalization methods which [do things]. Analytic models using renormalization methods for fluids have been [time scale] developed [cite], but computational models [something]. [something about including all lengths, not knowning what each contributes]. We present a method to determine the free energy contributions of each length scale by exploiting both the behavior of the critical fluid and the periodicity of Monte-Carlo simultations. We further detail a method by which to compute the ``absolute'' free energy, with both the ideal gas free energy and the excess free energy, for the square-well liquid. [more] We find that the square-well liquid near the critical region is well-characterized by a base cell length of $\sqrt2\sigma$, the side length of an FCC lattice.  
\clearpage