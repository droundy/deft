\documentclass[double,12pt]{revtex4-2}
\usepackage{graphicx}
\usepackage{color}
\usepackage{amsmath}
\usepackage{subcaption}

\begin{document}
%WORK ALL SECTIONS in Parallel!

\title{RESEARCH PROPOSAL}
\maketitle
\large
\begin{center}Toward completion of the Doctor of Philosophy degree in Physics 
~~~~~~~~~~~~~~~~~~~~~~~~~~~~~~~~~~~~~~at Oregon State University for Kirstie Finster
\end{center}

\normalsize

\section{Part 1: Overview}
My research involves creating models for use in the study of liquids. 
Although there are many approaches that can be taken, the simplest and 
most useful approach is through the use of the perturbation theory first
applied to the study of liquids by Zwanzig in 1954. 
Zwanzig demonstrated that a real liquid can be modeled by using the well 
known and highly developed hard sphere model as a reference fluid for the 
repulsive portion of the particle interaction to which 
a small attractive portion could be added as a perturbation. 
In 1964, Rowlinson introduced the idea of making the diameter of the hard sphere 
fluid temperature-dependent to better fit certain classes of soft interatomic potentials
to the behavior that would be expected as the temperature changes and particles 
come ever closer together during collisions at increasing temperatures.
%SHOW FIGURE
In 1967, Barker and Henderson made a monumental break-through 
by generalizing 
Rowlinson's use of the temperature-dependent hard sphere diameter and 
developing a method by which \textit{any} soft repulsive can be modeled 
by mapping it to a hard-sphere fluid with a temperature-dependent diameter. %\textcolor{red}{(cite paper)} 
This highly successful model, known as the 
Barker-Henderson fluid, has been in use now for around fifty years and
is the standard used to model fluids today. %\textcolor{red}{(cite papers)}.
But there is a problem.
The hard-sphere fluid model around which the Barker-Henderson fluid is 
constructed uses delta functions which makes the model difficult and 
inconvenient to apply in computer simulations. 
My research aims at constructing a model for fluids around a reference 
fluid that has a smooth function instead of the sharply discontinuous 
repulsive hard-sphere interatomic potential that is zero when the 
particles are not in contact and infinity upon contact. 
 
In selecting a soft interatomic potential to use as a reference fluid, 
there is one choice to consider that stands out above others, and that is 
the repulsive portion of the interatomic potential of the 
Weeks-Chandler-Andersen (WCA) fluid.
The WCA potential reproduces the repulsive force of the Lennard-Jones 
interatomic potential that is used to model neutral atoms. 
The thermodynamic properties of the fluid can be obtained from the 
Helmholtz free energy which
can be expressed as a functional of the average number 
density profile $n(\vec r)$ 
which gives the probability of finding a particle at each position $\vec r$. 
The Helmholtz free energy, along with the corresponding equilibrium number density profile,
can be found using density functional theory by 
varying the number density profile until the free energy is minimized.
For classical fluids, Rosenfeld formulated a highly successful 
density functional theory called Fundamental Measure Theory (FMT)
by noting that the Mayer function of the hard-sphere fluid is a step function
that can be decomposed into a set of weight functions that can be used 
to express the Helmholtz free energy functional in terms of a set of weighted densities. 
Schmidt created Soft Fundamental Measure Theory (SFMT) for soft sphere 
fluids which is based on FMT but reformulated for soft potentials by 
meeting the limiting conditions for a 0-dimensional fluid (where a sphere 
is confined to a cavity large enough for only one sphere) 
and for low-density. 
Emerging from his work is an "error-function" potential he formed to 
demonstrate the application of SFMT to a soft-sphere fluid and which is
similar to the repulsive potential of the WCA fluid. However, 
Schmidt's error function potential can only be made similar to the WCA fluid 
at one temperature -
it will not work for a variety of temperatures. 
Therefore, we use the error function potential but make
its parameters temperature-dependent (similar to the way Barker-Henderson made the
diameter of the hard sphere fluid temperature-dependent). 
%One parameter makes the softness of the error function potential comparable 
%with the softness (steepness) of the WCA potential, while the other parameter 
%sets the length-scale of the particle interaction and is
%derived by setting 
%We design the parameters make
Our parameters are derived in part by making 
the second virial coefficient of our error function 
potential equal to the second virial coefficient of a WCA fluid. 
Our approach is easily extensible to other soft repulsive potentials.

\section{Part 2: Research Done}
We have a working model of a soft-sphere reference fluid that depends 
on a smooth potential function instead of on the discontinuous hard-sphere potential 
which has been the standard in use for the last fifty years. 
In our paper prepared for publication we have: \\
$\bullet$ Demonstrated good agreement between our WCA fluid and Barker-Henderson fluid \\
$\bullet$ Demonstrated good agreement between our WCA fluid and Monte-Carlo simulations \\
$\bullet$ Demonstrated good agreement between our WCA fluid and experiments with Argon under high pressure (so attractive forces can be ignored) \\

\section{Part 3: Research To Do}
\noindent %Toward a better model and a better understanding of our model, 
I propose some possible paths for wrapping up my research: \\
$\bullet$ Finish up the paper we have and graduate. \\
$\bullet$ Write a follow-on paper and graduate. In a second paper, I would 
like to look at finding the internal energy for our WCA fluid using 
classical Density Functional Theory and our SFMT functional 
for the Helmholtz free energy that we developed in our first paper 
\begin{align}
  U[n(\vec r), s(\vec r)] = F[n(\vec r)] + T\int s(\vec r) d\vec r
\end{align}
 Alternatively (or if more is needed, or time allows), I could: \\
$\bullet$ Look at how to improve results at low temperature, high density 
by adjusting our functional and/or potential and our pressure calculations.\\ %pressure calculations  \\
%$\bullet$ Possibly derive temperature-dependent parameter with B3 and B2  \\
%4- another functional for better pressure in freezing with improved pressure calculation\\
%4- another function (instead of Verf) for better pressure in freezing \\
$\bullet$ Examine and demonstrate the generality and usefulness of our 
method by applying it to other soft repulsive fluid potentials (Dr. Roundy famous!) \\
%$\bullet$ Apply our model  (ie. show by example) to demonstrate usefulness \\ %and market \\
%better understand and meet the needs of those who would benefit from our approach and market \\

\noindent Papers would be submitted to Physical Review E.

\section{Part 4: Means/timeline/program}
\noindent Requirements of the PhD program \\
$\bullet$ One class left: PH575 Spring 2023 \\
$\bullet$ Allowed time left: 4 years \\
$\bullet$ Resources needed: Bingley computer in Dr. Roundy's lab \\
$\bullet$ Funding: Currently I am a TA, but I'm going to apply for a grant from NSF \\

\end{document}
