\documentclass{beamer}
\usepackage{amsmath}
\usepackage{graphicx}
\usepackage{color}
\usepackage{cancel}
\graphicspath{ {/home/krebse/deft-soft/papers/fuzzy-fmt/figs/} }
\usetheme{Warsaw}
\newcommand{\rr}{\textbf{r}}
\title{Softening the Hard-Sphere Fluid}
\author{Eric Krebs, Samuel Loomis, Patrick Kreitzberg, David Roundy}
\institute{Oregon State University}
\date{May 3, 2014}
\begin{document}

\begin{frame}
 \titlepage
\end{frame}

\begin{frame}{Motivation}
  \begin{itemize}
    \item Want a theory that can deal with an inhomogeneous liquid
    \item The hard-sphere fluid is well studied and understood
    \item Hard spheres are used as a reference for real liquids
      \begin{itemize}
        \item Hard spheres are not very physical
        \item Computationally difficult (deta and step functions)
      \end{itemize}
  \end{itemize}
  \begin{block}{Classical density functional theory (cDFT)}
    \begin{align}
    \Omega(T) = \underset{n(\rr)}{min} \left\{F[n(\rr),T] + \int
                \left(V_{ext}(\rr) - \mu \right)n(\rr)d\rr \right\}
    \end{align}
    \begin{itemize}
      \item $n(\rr)$ can be an inhomogeneous particle density
    \end{itemize}
  \end{block}
\end{frame}

\begin{frame}{Fundamental measure theory (FMT)}
  \begin{itemize}
    \item Standard cDFT for hard-sphere
      fluid
    \item Hard spheres represented by ``fundamental measures''
      \begin{itemize}
        \item Volume, surface area, mean curvature
      \end{itemize}
    \item Weighted densities calculated using weight functions
      \begin{itemize}
        \item $n_{\alpha}(\rr) = \int n(\rr')w_{\alpha}(\rr -
        \rr')d\rr'$
      \end{itemize}
      \begin{block}{Some weight functions for sphere of radius R}
        \begin{align}
          w_3(\rr) = \theta (R - |\rr|) \\
          w_2(\rr) = \delta (R - |\rr|) \\
          \notag
        \end{align}
      \end{block}
    \item Other weight functions are simply related:
      \begin{itemize}
        \item $w_1(\rr) = \frac{w_2(\rr)}{4\pi R}$, $w_0(\rr) =
          \frac{w_2(\rr)}{4\pi R^2}$,  etc.
      \end{itemize}
  \end{itemize}
\end{frame}

\begin{frame}{Soft fundamental measure theory}
  \begin{itemize}
    \item A modification of FMT to treat penetrating spheres
    \item Weight functions are modified to be determined by the Mayer
      Function
    \item Reduces to hard spheres as $T \to 0$
      \begin{align}
        \int_{-\infty}^{\infty}w_2(r')w_2(r - r') dr' = \frac{\partial f(r)}{\partial r}
      \end{align}
      \begin{block}{Mayer Function}
        \begin{center}
          $f(r) = e^{-\beta V(r)} - 1$ \\
        \end{center}
      \end{block}
    \item If $w_2$ is known, all weight functions, $w_{\alpha}$'s, can
      be derived
  \end{itemize}
\end{frame}

\begin{frame}{Choosing a $w_2(\rr)$}
 \begin{itemize}
   \item Want spheres to overlap slightly at ambient temperatures
   \item We use a harmonic potential for $r < 2R$
     \begin{center}
       \includegraphics[trim = 0 0 0 20, clip, scale = 0.40]{figs/harmonic}
     \end{center}
   \item However, we could not find a simple analytic solution for $w_2(\rr)$
 \end{itemize}
\end{frame}

\begin{frame}{Error function method}
  \begin{itemize}
    \item Necessary to discuss this?
  \end{itemize}
\end{frame}

\begin{frame}{The homogeneous fluid}
  \begin{itemize}
    \item We compare homogeneous equation of state against Monte-Carlo
      (MC) simulation
    \item Good agreement for moderate temperatures and lower
      packing fraction
  \end{itemize}
  \begin{center}
    \includegraphics[trim = 0 13 0 37, clip, scale = 0.40]{figs/p-vs-packing}
  \end{center}
\end{frame}

\begin{frame}{Radial distribution}
  \begin{itemize}
    \item Compare radial distribution of homogeneous fluid with MC
      simulation
      \item Good agreement, especially in our range of interest
  \end{itemize}
  \begin{center}
    \includegraphics[trim = 35 17 35 25, clip, scale = .32]{figs/radial-distribution-10}
    \includegraphics[trim = 35 17 -1000 25 , clip, scale = .32]{figs/radial-distribution-30}
  \end{center}
\end{frame}

\begin{frame}{Soft-sphere fluid near a hard wall}
  \begin{itemize}
    \item We compare filling fraction near a hard wall against MC simulation
    \item Again we find good agreement.
  \end{itemize}
  \begin{center}
    \includegraphics[trim = 0 13 0 25, clip, scale = .40]{figs/walls-30}
  \end{center}
\end{frame}

\end{document}
