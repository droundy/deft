%This document address an alternative way to caluclate pressure

\documentclass[double,12pt]{revtex4-2}
\usepackage{graphicx}
\usepackage{color}
\usepackage{amsmath}
\usepackage{subcaption}


\begin{document}
We want to find the pressure of our system from the chemical potential 
$\mu$ given the number density $n$ and the Helmholtz free energy per 
volume $f_v$. 

\begin{align}
	F &= U - TS  \\
	  &= TS -PV + \mu N - TS  \\
	  &= -PV + \mu N  \\ \nonumber\\
%
	P &= \frac{\mu N}{V} - \frac{F}{V} \\
	  &= \mu n - f_v   \\ \nonumber\\
%
	dF &= dU - TdS -SdT \\
	   &= TdS - PdV + \mu N  - TdS -SdT\\
	   &=  -SdT - PdV + \mu N \\
	   &= \frac{\partial F}{\partial T}\bigg|_{V,N}dT 
	       + \frac{\partial F}{\partial V}\bigg|_{T,N}dV 
	       + \frac{\partial F}{\partial N}\bigg|_{T,V}dN \\ \nonumber\\
%
    \mu &= \frac{\partial F}{\partial N}\bigg|_{T,V} \\
        &= \frac{\partial \frac{F}{V}}{\partial \frac{N}{V}}\bigg|_{T,V}\\
       % &= \frac{\partial f_v}{\partial n}\bigg|_{T,V} \\
        &= \frac{1}{V}\frac{\partial F}{\partial n}\bigg|_{T,V}        
\end{align}

In terms of virtual variations of the functional $F[n(\vec r)]$, 
%$\mu(\vec r)$ is given by
%\begin{align}
    %\mu(\vec r) &= \int \frac{\delta f_v[n(\vec r)]}{\delta n(\vec r~'')}
    %\bigg|_{T,V} d\vec r~''
%\end{align}
%\begin{align}
    %\mu(\vec r) &= \frac{\delta f_v[n(\vec r)]}{\delta n(\vec r)}
    %\bigg|_{T,V} \\
                %&= \int \frac{\delta f_v[n(\vec r)]}{\delta n_i(\vec r~'')}
    %\bigg|_{T,V} \frac{\delta n_i(\vec r~'')}{\delta n(\vec r)}~~d\vec r~'' \color{red}\mbox{~~~UNITS?}
%\end{align}

\begin{align}
  \mu(\vec r) &= \frac{\delta F[n(\vec r)]}{\delta n(\vec r)}\bigg|_{T,V}
\end{align}
where there is no 1/V in front of the functional derivative because 
functional derivatives are terms
over which an integration must take place and thus has a 1/volume 
incorrporated into it similar to a delta function which
has units of 1/volume in 3 dimensions so that it integrates to 1.

%save below, but is not correct ******
%\begin{align}
    %\mu(\vec r) &=  \frac{\delta F[n(\vec r)]}{\delta n(\vec r)}\bigg|_{T,V}  
%\end{align}
%where there is no 1/V in front because functional derivatives are terms
%over which an integration must take place and thus has a 1/volume 
%incorrporated into it similar to a delta function which
%has units of 1/volume in 3 dimensions so that it integrates to 1.
The average chemical potential over all space is
\begin{align}
    \mu &= \left<\mu(\vec r)\right> = \frac{1}{V}\int \mu(\vec r)~d\vec r~\\
        &= \frac{1}{V} \int \frac{\delta F[n(\vec r)]}{\delta n(\vec r)}
           \bigg|_{T,V}~d\vec r~  
        %&= \frac{1}{V} \int d\vec r \int d\vec r~'' ~\frac{\delta 
        %F[n(\vec r)]}{\delta n_i(\vec r~'')}\bigg|_{T,V} 
        %\frac{\delta n_i(\vec r~'')}{\delta n(\vec r)}   
\end{align}
%end save ******

The Helmholtz free energy can be broken into an ideal and an excess part.
\begin{align}
    F = F_{id}+ F_{ex}
\end{align} 
\begin{align}
   F_{ideal}  &= k_BTN[\ln\left(\frac{N\Lambda^{3}}{V}\right)-1]  \\
%
  \mu_{ideal} &= \frac{\partial F}{\partial N}\bigg|_{T,V}   \\
              &= k_BT\ln\left(\frac{N\Lambda^{3}}{V}\right)  \\ 
              &= k_BT\ln\left(\frac{n}{n_Q}\right)
\end{align} 
For a spatially varying number density, as in an inhomogenious ideal gas,
this becomes
\begin{align}
    F_{ideal}[n(\vec{r})] &= k_BT\int_{Volume}n(\vec r~')
      \left[\ln\left(\frac{n(\vec r~')}{n_Q}\right)-1\right] ~d\vec r~' 
\end{align}
% p.64 EQ 3.1.22 Simple Liquids
\begin{align}
 \mu_{ideal}(\vec r) &= \frac{\delta F[n(\vec r)]}{\delta 
              n(\vec r)}\bigg|_{T,V} \\ \nonumber\\
    &= \frac{\delta}{\delta n(\vec r)}
    \left(k_BT\int d\vec r~'~ n(\vec r~')\left[\ln\left(\frac{n(\vec r~')}
    {n_Q}\right)-1\right]\right) \\ \nonumber\\
    &= k_BT\int d\vec r~'~ \frac{\delta}{\delta n(\vec r)}\left(n(\vec r~')\left[\ln\left(\frac{n(\vec r~')}
    {n_Q}\right)-1\right]\right) \\ \nonumber\\
    &= k_BT\left(\int d\vec r~'~ \frac{\delta n(\vec r~')}
    {\delta n(\vec r)}\left[\ln\left(\frac{n(\vec r~')}
    {n_Q}\right)-1\right] + n(\vec r~')\frac{n_Q}{n(\vec r~')}
    \frac{1}{n_Q} \frac{\delta n(\vec r~')}{\delta n(\vec r)}
    \right)\\ \nonumber\\
    &= k_BT\left(\int d\vec r~'~ \frac{\delta n(\vec r~')}
    {\delta n(\vec r)}\left[\ln\left(\frac{n(\vec r~')}
    {n_Q}\right)-1\right] + \frac{\delta n(\vec r~')}{\delta n(\vec r)}
    \right)\\ \nonumber\\
    &= k_BT\int d\vec r~'~ \frac{\delta n(\vec r~')}
    {\delta n(\vec r)}\left(\ln\left(\frac{n(\vec r~')}
    {n_Q}\right)-1 + 1\right)\\ \nonumber\\
    &= k_BT\int d\vec r~'~ \frac{\delta n(\vec r~')}
    {\delta n(\vec r)}\ln\left(\frac{n(\vec r~')}
    {n_Q}\right)\\ \nonumber\\
    &= k_BT\int d\vec r~'~ \delta(\vec r~'-
    \vec r)\ln\left(\frac{n(\vec r~')}
    {n_Q}\right)\\ \nonumber\\
    &= k_BT\ln\left(\frac{n(\vec r)}
    {n_Q}\right)  ~~~~~~\text{EQ 3.1.20 Theory of Simple Liquids}
\end{align}
and the %\textcolor{red}{ensemble?} 
average is
\begin{align}
    \mu_{ideal} &= \left<\mu(\vec r)\right>  \\
        &=  \frac{1}{V}\int \mu(\vec r)~d\vec r~ \\
        &= \frac{1}{V}\int k_BT\ln\left(\frac{n(\vec r)}{n_Q}\right)~d\vec r~ 
        %&= \frac{1}{V} \int d\vec r \int d\vec r~'' ~\frac{\delta 
        %F[n(\vec r)]}{\delta n_i(\vec r~'')}\bigg|_{T,V} 
        %\frac{\delta n_i(\vec r~'')}{\delta n(\vec r)}   
\end{align}
which for the homogenious case, $n(r)=n$, gives becomes
\begin{align}
    \mu_{ideal} &= \frac{1}{V}\int k_BT\ln\left(\frac{n}{n_Q}\right)~d\vec r~ \\
        &= \frac{1}{V}k_BT\ln\left(\frac{n}{n_Q}\right)\int ~d\vec r~ \\
        &= \frac{1}{V}k_BT\ln\left(\frac{n}{n_Q}\right)(V) \\
        &= k_BT\ln\left(\frac{n}{n_Q}\right) 
\end{align}

%below is incorrect!
%\begin{align}
 %\mu_{ideal} &= \frac{1}{V^2}\int \frac{\delta F[n(\vec r)]}{\delta 
              %n(\vec r)} d\vec r \\ \nonumber\\
    %&= \frac{k_BT}{V^2} \int d\vec r ~\frac{\delta}{\delta n(\vec r)}
    %\left(\int d\vec r~'~ n(\vec r~')\left[\ln\left(\frac{n(\vec r~')}
    %{n_Q}\right)-1\right]\right) \\ \nonumber\\
    %&= \frac{k_BT}{V^2} \int d\vec r ~
    %\left(\int d\vec r~'~ \frac{\delta n(\vec r~')}{\delta n(\vec r)}
    %\left[\ln\left(\frac{n(\vec r~')}
    %{n_Q}\right)-1\right] + n(\vec r~')\frac{1}{\frac{n(\vec r~')}
    %{n_Q}}\frac{1}{n_Q} \frac{\delta n(\vec r~')}{\delta n(\vec r)}
    %\right)\\ \nonumber\\
    %&= \frac{k_BT}{V^2} \int d\vec r ~
    %\left(\int d\vec r~'~ \frac{\delta n(\vec r~')}{\delta n(\vec r)}
    %\left[\ln\left(\frac{n(\vec r~')}
    %{n_Q}\right)\right]\right) - \frac{\delta n(\vec r~')}{\delta 
    %n(\vec r)}+\frac{\delta n(\vec r~')}{\delta n(\vec r)}\\ \nonumber\\
    %&= \frac{k_BT}{V^2} \int d\vec r ~ \left(\int d\vec r~'~\delta(\vec r~'-
    %\vec r)\left[\ln\left(\frac{n(\vec r~')}{n_Q}\right)\right]\right) 
    %\\ \nonumber\\  
    %&= \frac{k_BT}{V^2} \int d\vec r ~\ln\left(\frac{n(\vec r)}{n_Q}\right)  
    %%&= \frac{k_BT}{V} (V) ~\ln\left(\frac{n(\vec r)}{n_Q}\right) \\
    %%&= k_BT~\ln\left(\frac{n(\vec r)}{n_Q}\right) %p. 64 EQ 3.1.20 Simple Liquids
%\end{align} 
%end incorrect!

The excess Helmholtz free energy is given by
\begin{align}
    F_{{ex}}[n(\vec r)] &= k_BT\int\Phi_1
    +\Phi_2+\Phi_3~~d\vec r
\end{align}     
where
\begin{align}
    \Phi_1 &= -n_{0}\ln(1-n_{3}) \\
    \Phi_2 &= \frac{n_1n_2-\vec n_1\cdot\vec n_2}{1-n_3} \\
    \Phi_3 &= \frac{{n_2}^3-3n_2\vec n_{v2}\cdot\vec n_{v2}+\frac{9}{2}
       [\vec n_{v2}\cdot{\overleftrightarrow{n}_{m2}}\cdot{\vec n_{v2}}
       -\operatorname{Tr}({\overleftrightarrow n^3_{m2}})]}{24\pi(1-n_3)^2}  
\end{align}
and the weighted densities are given by 
\begin{align}
    n_\alpha(\vec r) &= \int n(\vec {r}~')w_\alpha(\vec r-\vec {r}~')
                    ~d\vec {r}~'  \label{weighted_densities}  \\
                     &= n\otimes w_\alpha(\vec r)
\end{align}  
where $n(\vec r)$ = n for the homogeneous case.
The weight functions are
\begin{align}\label{eq:weights}
  w_{0}(r) &=\frac{w_{2}}{4\pi{r}^2} \\
  w_{1}(r) &=\frac{w_{2}}{4\pi{r}} 
\end{align}
\begin{align}
  w_2(r) &=\frac{\sqrt{2}}{\Xi\sqrt\pi}\exp^{-\left(\frac{r-\frac{\alpha}
           {2}}{\Xi/\sqrt{2}}\right)^2}  \\
  w_3(r) &=\frac{1}{2}\left[1-\operatorname{erf}\left(\frac{r
          -\frac{\alpha}{2}}{\frac{\Xi}{\sqrt{2}}}\right)\right]  \\
%    
      \vec {w}_2 &= w_2~\hat r \\
      \vec {w}_1 &= w_1~\hat r \\
      \overleftrightarrow{w}_{m2}(\vec{r}) &= w_2(r)\left(\frac{\vec{r}
                                        \vec{r}}{r^2}-\frac{I}{3}\right) 
\end{align}

\begin{equation}
    {\widetilde{w}_{{m2}_{x'x'}}(k)=\left(\frac{-\pi{\Xi}^2}{3}
   -\frac{4\pi}{k^2}\right)e^{-\frac{k^2\Xi^2}{8}}\cos(\frac{k\alpha}{2})
   -\frac{2\pi\alpha}{3k}e^{-\frac{k^2\Xi^2}{8}}\sin(\frac{k\alpha}{2})}
   \nonumber
\end{equation} 
\begin{equation} %equation continued...
   {+\frac{{(2\pi)}^{\frac{3}{2}}}{k^3\Xi}e^{-\left(\frac{\alpha^2}
   {2\Xi^2}\right)}\left[\operatorname{erf}\left(\frac{k\Xi}{2^\frac{3}{2}}
   +i\frac{\alpha}{\sqrt{2}\Xi}\right)+\operatorname{erf}\left(\frac{k\Xi}
   {2^\frac{3}{2}}-i\frac{\alpha}{\sqrt{2}\Xi}\right)\right]}
\end{equation} 

\begin{align}
    \widetilde{w}_{m2_{ij}}(\vec{k}) &= (\delta{ij}-3\frac{k_ik_j}{k^2})
                                    \widetilde{w}_{{m2}_{x'x'}}(\vec{k}) \\ \nonumber \\
%\end{align}
%\begin{align}
    n_{m2ij}(\vec r) &=  \frac{1}{\left(2\pi\right)^3}\int d\vec k\, 
                       n(\vec k) w_{m2ij}(\vec k)e^{i\vec k\cdot \vec r}
\end{align} 

%\begin{align} old primes
  %n(\vec r) &= \int \delta (\vec r~''-\vec r)n(\vec r~'') d\vec r~''\\
  %\delta n(\vec r) &= \int \delta (\vec r~''-\vec r)\delta n(\vec r~'')
  %d\vec r~''
%\end{align} 
%But for a functional,
%\begin{align} 
%\delta n(\vec r) &= \int \frac{\delta n(\vec r)}{\delta n(\vec r~'')}
%\delta n(\vec r~'')d\vec r~''
%\end{align} 
%and so
%\begin{align} 
  %\frac{\delta n(\vec r)}{\delta n(\vec r~'')} &= \delta (\vec r~''-\vec r) \\
                                               %&= \delta (\vec r-\vec r~'') 
%\end{align} 
%\begin{align} 
   %\frac{\delta n_i(\vec r)}{\delta n(\vec r~'')}  &= \frac{\delta}
   %{\delta n(\vec r~'')}\int n(\vec r~')w_i(\vec r~-\vec r~')d\vec r~'\\
   %&= \int \frac{\delta n(\vec r~')}{\delta n(\vec r~'')}w_i(\vec r
   %-\vec r~') d\vec r~' \\
   %&= \int \delta (\vec r~' -\vec r~'')w_i(\vec r-\vec r~') d\vec r~'\\
   %&= w_i(\vec r-\vec r~'') 
%\end{align}
%Or, more simply,
\begin{align} 
  n(\vec r) &= \int \delta (\vec r~'-\vec r)n(\vec r~') d\vec r~'\\
  \delta n(\vec r) &= \int \delta (\vec r~'-\vec r)\delta n(\vec r~')
  d\vec r~'
\end{align} 
But for a functional,
\begin{align} 
\delta n(\vec r) &= \int \frac{\delta n(\vec r)}{\delta n(\vec r~')}
\delta n(\vec r~')d\vec r~'
\end{align} 
and so
\begin{align} 
  \frac{\delta n(\vec r)}{\delta n(\vec r~')} &= \delta (\vec r~'-\vec r) \\
                                               &= \delta (\vec r-\vec r~') 
\end{align} 
\begin{align} 
   \frac{\delta n_i(\vec r)}{\delta n(\vec r~'')}  &= \frac{\delta}
   {\delta n(\vec r~'')}\int n(\vec r~')w_i(\vec r~-\vec r~')d\vec r~'\\
   &= \int \frac{\delta n(\vec r~')}{\delta n(\vec r~'')}w_i(\vec r
   -\vec r~') d\vec r~' \\
   &= \int \delta (\vec r~' -\vec r~'')w_i(\vec r-\vec r~') d\vec r~'\\
   &= w_i(\vec r-\vec r~'') 
\end{align}
Or, more simply,
\begin{align} 
   n_i(\vec r)  &= \int n(\vec r~')w_i(\vec r~-\vec r~')d\vec r~'\\
   \delta n_i(\vec r)  &= \int \delta n(\vec r~')w_i(\vec r~-\vec r~')d\vec r~'\\
   \delta n_i(\vec r)  &= \int \delta n(\vec r~')\frac{\delta n_i(\vec r)}{\delta n(\vec r~')}d\vec r~'\\
   \frac{\delta n_i(\vec r)}{\delta n(\vec r~')}&= w_i(\vec r-\vec r~') 
\end{align}

%\begin{align}   %old primes
    %F_{1ex} &= k_BT\int \Phi_1~d\vec r \\ \nonumber\\
    %\mu_1 &= \frac{1}{V}\int \frac{\delta F[n(\vec r)]}{\delta n(\vec r)} d\vec r \\ \nonumber\\
%%
        %&= -\frac{k_BT}{V} \int d\vec r~'' \int d\vec r~ \frac{\delta}
        %{\delta n(\vec r~'')}n_o(\vec r)\ln(1-n_3(\vec r)) \\ \nonumber\\
%%
        %&= -\frac{k_BT}{V}\int d\vec r~''\int d\vec r~\frac{\delta n_o(\vec r)}
        %{\delta n(\vec r~'')}\ln(1-n_3(\vec r))+\frac{n_o(\vec r)}
          %{1-n_3(\vec r)}\left(-\frac{\delta n_3(\vec r)}{\delta n(\vec r~'')}
          %\right) \\ \nonumber\\
%%
        %&= -\frac{k_BT}{V} \int d\vec r~'' \int d\vec r~w_0(\vec r - \vec r~'')
          %\ln(1-n_3(\vec r)) - w_3(\vec r - \vec r~'')\frac{n_o(\vec r)}
          %{1-n_3(\vec r)} \\ \nonumber\\
%%
        %&= -\frac{k_BT}{V} \int d\vec r~\ln(1-n_3(\vec r)) \int d \vec r~''  
          %~w_0(\vec r - \vec r~'') \nonumber\\
        %&- \frac{k_BT}{V} \int d\vec r ~\frac{n_o(\vec r)}{1-n_3(\vec r)} 
          %\int d \vec r~''~w_3(\vec r - \vec r~'')  \\ \\
%%
        %&= -\frac{k_BT}{V} \int d\vec r ~\ln(1-n_3(\vec r))(1)
          %- \frac{k_BT}{V} \int d\vec r ~\frac{n_o(\vec r)}{1-n_3(\vec r)}
          %\left(\frac{\pi\Xi^2\alpha}{2}\right) \\ \\
%%
        %&= -\frac{k_BT}{V}\int\ln(1-n_3(\vec r))+\left(
        %\frac{\pi\Xi^2\alpha}{2}\right)\frac{n_o(\vec r)}{1-n_3(\vec r)}~d\vec r
%\end{align}
 
\begin{align}
    F_{1ex} &= k_BT\int \Phi_1~d\vec r~' \\ \nonumber\\
    \mu_{1ex} &= \frac{1}{V}\int \frac{\delta F[n(\vec r)]}{\delta n(\vec r)} d\vec r \\ \nonumber\\
%
        &= -\frac{k_BT}{V} \int d\vec r ~\frac{\delta}
        {\delta n(\vec r)}\left(\int d\vec r~'~n_o(\vec r~')\ln(1-n_3(\vec r~'))\right) \\ \nonumber\\
%
        &= -\frac{k_BT}{V}\int d\vec r\int d\vec r~'~\frac{\delta n_o(\vec r~')}
        {\delta n(\vec r)}\ln(1-n_3(\vec r~'))+\frac{n_o(\vec r~')}
          {1-n_3(\vec r)}\left(-\frac{\delta n_3(\vec r~')}{\delta n(\vec r)}
          \right) \\ \nonumber\\
%
        &= -\frac{k_BT}{V} \int d\vec r \int d\vec r~' ~w_0(\vec r - \vec r~')
          \ln(1-n_3(\vec r~')) - \frac{n_o(\vec r~')}
          {1-n_3(\vec r~')}w_3(\vec r - \vec r~') \\ \nonumber\\
%
        &= -\frac{k_BT}{V} \int d\vec r~'\ln(1-n_3(\vec r~')) \int d \vec r  
          ~w_0(\vec r - \vec r~') \nonumber\\
        &+ \frac{k_BT}{V} \int d\vec r~' ~\frac{n_o(\vec r~')}{1-n_3(\vec r~')} 
          \int d \vec r~w_3(\vec r - \vec r~')  \\ \\
%
        &= -\frac{k_BT}{V} \int d\vec r~' ~\ln(1-n_3(\vec r~'))(1)
          + \frac{k_BT}{V} \int d\vec r~' ~\frac{n_o(\vec r~')}{1-n_3(\vec r~')}
          \left(\frac{\pi\Xi^2\alpha}{2}\right) \\ \\
%
        &= \frac{k_BT}{V}\int d\vec r~'~ \left[-\ln(1-n_3(\vec r~'))+\left(
        \frac{\pi\Xi^2\alpha}{2}\right)\frac{n_o(\vec r~')}{1-n_3(\vec r~')}\right]
\end{align}

\begin{align}
    F_{2ex} &= k_BT\int \Phi_2 ~d\vec r~' \\ \nonumber\\
    \mu_{2ex} &= \frac{1}{V}\int \frac{\delta F_{2ex}[n(\vec r)]}{\delta 
          n(\vec r)}\bigg|_{T,V} d\vec{r} \\
%
          &= \frac{k_BT}{V} \int d\vec r ~\frac{\delta }{\delta n(\vec r)}
          \left(\int d \vec r~' ~\frac{n_1(\vec r~')n_2(\vec r~')-\vec n_{v1}
          (\vec r~')\cdot \vec n_{v2}(\vec r~')}{1-n_3(\vec r~')}\right) 
          \nonumber\\ \\        
%
          &= \frac{k_BT}{V} \int d\vec r~ \int d \vec r~' ~
          \frac{\delta n_1(\vec r~')}{\delta n(\vec r)}
          \frac{n_2(\vec r~')}{1-n_3(\vec r~')}   \nonumber\\
          &+\frac{k_BT}{V} \int d\vec r~ \int d \vec r~' ~
          \frac{\delta n_2(\vec r~')}{\delta n(\vec r)}
          \frac{n_1(\vec r~')}{1-n_3(\vec r~')}   \nonumber\\
          &+\frac{k_BT}{V} \int d\vec r~ \int d \vec r~'~
          \frac{\delta n_3(\vec r~')}{\delta n(\vec r)}
          \frac{n_1(\vec r)n_2(\vec r~')}{(1-n_3(\vec r~'))^2}  \nonumber\\
          &-\frac{k_BT}{V} \int d\vec r~ \int d \vec r~'~ \frac{\frac{\delta 
          \vec n_{v1}(\vec r~')}{\delta n(\vec r)}\cdot \vec n_{v2}(\vec r~')}
          {1-n_3(\vec r~')}  \nonumber\\
          &-\frac{k_BT}{V} \int d\vec r~ \int d \vec r~'~ \frac{\frac{\delta 
          \vec n_{v2}(\vec r~')}{\delta n(\vec r)}\cdot \vec n_{v1}(\vec r~')}
          {1-n_3(\vec r~')}  \nonumber\\
          &-\frac{k_BT}{V} \int d\vec r~\int d \vec r~'~
          \frac{\delta n_3(\vec r~')}{\delta n(\vec r)}
           \frac{\vec n_{v1}(\vec r~')\cdot \vec n_{v2}(\vec r~')}
           {(1-n_3(\vec r~'))^2} \\ \nonumber\\
%
          &= \frac{k_BT}{V} \int d\vec r~ \int d \vec r~' ~w_1(\vec r-\vec r~')
          \frac{n_2(\vec r~')}{1-n_3(\vec r~')}   \nonumber\\
          &+\frac{k_BT}{V} \int d\vec r~ \int d \vec r~' ~w_2(\vec r-\vec r~')
          \frac{n_1(\vec r~')}{1-n_3(\vec r~')}   \nonumber\\
          &+\frac{k_BT}{V} \int d\vec r~ \int d \vec r~'~w_3(\vec r - \vec r~') 
          \frac{n_1(\vec r~')n_2(\vec r~')}{(1-n_3(\vec r~'))^2}  \nonumber\\
          &-\frac{k_BT}{V} \int d\vec r~ \int d \vec r~' ~\frac{\vec w_1(\vec r 
          - \vec r~')\cdot \vec n_{v2}(\vec r~')}{1-n_3(\vec r~')}  \nonumber\\
          &-\frac{k_BT}{V} \int d\vec r~ \int d \vec r~'~ \frac{\vec w_2(\vec r 
          - \vec r~')\cdot \vec n_{v1}(\vec r~')}{1-n_3(\vec r~')}  \nonumber\\
          &-\frac{k_BT}{V} \int d\vec r~\int d \vec r~'~ w_3(\vec r - \vec r~')
           \frac{\vec n_{v1}(\vec r~')\cdot \vec n_{v2}(\vec r~')}
           {(1-n_3(\vec r~'))^2} 
\end{align}
\begin{align} % mu_2 equation continued...
          &= \frac{k_BT}{V} \int d\vec r~'~ \frac{n_1(\vec r~')\int d \vec r
           ~w_2(\vec r - \vec r~')
          +n_2(\vec r~')\int d \vec r ~w_1(\vec r - \vec r~')}
          {1-n_3(\vec r~')} \nonumber\\
          &+\frac{k_BT}{V} \int d\vec r~'~\frac{n_1(\vec r~')n_2(\vec r~')
          -\vec n_{v1}(\vec r~')
          \cdot \vec n_{v2}(\vec r~')}{(1-n_3(\vec r~'))^2} \int d \vec r 
          ~w_3(\vec r - \vec r~') \\ \nonumber \\
%
           &= \frac{k_BT}{V}\int d\vec r~'~ \frac{n_1(\vec r~')
           \pi(\Xi^2 + \alpha^2)
          +n_2(\vec r~')\left(\frac{\alpha}{2}\right)}{1-n_3(\vec r~')} 
          \nonumber \\
          &+ \frac{n_1(\vec r~')n_2(\vec r~')
          -\vec n_{v1}(\vec r~')
          \cdot \vec n_{v2}(\vec r~')}{(1-n_3(\vec r~'))^2}
          \left(\frac{\pi\Xi^2\alpha}{2}\right)   
\end{align}

\begin{align}
  F_{3ex} &= k_BT\int \Phi_3 ~d\vec r~'  \\ 
  \nonumber\\ 
  \mu_{3ex} &= \frac{1}{V}\int \frac{\delta F_{3ex}[n(\vec r)]}{\delta n(\vec r)}
  \bigg|_{T,V} d\vec r \\ \nonumber \\
%
   &= \frac{k_BT}{V}\int d\vec r~\frac{\delta}{\delta n(\vec r)}~\int 
   d\vec r~'~\frac{{n_2}^3-3n_2\vec{n}_{v2}\cdot\vec{n}_{v2}+\frac{9}{2}
    [\vec{n}_{v2}\cdot{\overleftrightarrow{n}_{m2}}\cdot{\vec{n}_{v2}}
    -\operatorname{Tr}({\overleftrightarrow{n}^3_{m2}})]}{24\pi(1-n_3)^2}
     \nonumber \\ \\
%
     &=\frac{k_BT}{V}\int d\vec r~\int d\vec r~'~ \color{blue}\left( {n_2}^3-3n_2
     \vec{n}_{v2}\cdot\vec{n}_{v2} 
     +\frac{9}{2}[\vec{n}_{v2}\cdot{\overleftrightarrow{n}_{m2}}\cdot{
     \vec{n}_{v2}}-\operatorname{Tr}({\overleftrightarrow{n}^3_{m2}})]
     \right) \nonumber\\
      &~\color{blue} \frac{\delta}{\delta n(\vec r)}\left(\frac{1}{24\pi(1-n_3)^2}
      \right) \nonumber\\
     &+ \color{red}\left(\frac{1}{24\pi(1-n_3)^2}\right)\frac{\delta}
     {\delta n(\vec r)}\left({n_2}^3-3n_2\vec{n}_{v2}\cdot\vec{n}_{v2}
     +\frac{9}{2}[\vec{n}_{v2}\cdot{\overleftrightarrow{n}_{m2}}\cdot{
     \vec{n}_{v2}}-\operatorname{Tr}({\overleftrightarrow{n}^3_{m2}})]
     \right) \\ \nonumber \\
%
    &=\frac{k_BT}{V}\int d\vec r~\int d\vec r~'~\color{blue}\left( {n_2}^3-3n_2
    \vec{n}_{v2}\cdot\vec{n}_{v2} 
     +\frac{9}{2}[\vec{n}_{v2}\cdot{\overleftrightarrow{n}_{m2}}\cdot{
     \vec{n}_{v2}}-\operatorname{Tr}({\overleftrightarrow{n}^3_{m2}})]
     \right) \nonumber\\
     &~\color{blue}\left(\frac{2}{24\pi(1-n_3)^3}\frac{\delta n_3(\vec r~')}{\delta n(\vec r)}
     \right) 
     \color{red}
     + \left(\frac{1}{24\pi(1-n_3)^2}\right)\left[3n_2(\vec r~')^2
     \frac{\delta n_2(\vec r~')}{\delta n(\vec r)} \right.
      \nonumber\\
     &\color{red} - 3\left(\frac{\delta n_{2}(\vec r~')}{\delta n(\vec r)}\vec n_{v2}
     (\vec r~') \cdot \vec n_{v2}(\vec r~')\right) - 2*3n_2(\vec r~')\frac{\delta
     \vec n_{v2}(\vec r~')}{\delta n(\vec r)}\cdot \vec n_{v2}(\vec r~') 
     \nonumber\\
     &\color{red}+\frac{9}{2}\left(\frac{\delta \vec n_{v2}(\vec r~')}{\delta n(\vec r)}
     \cdot \overleftrightarrow n_{m2}(\vec r~')\cdot \vec n_{v2}( \vec r~')
     +\vec n_{v2}(\vec r~')\cdot \frac{\delta \overleftrightarrow n_{m2}
     (\vec r~')}{\delta n(\vec r)}\cdot \vec n_{v2}(\vec r~')\color{white}
     \right)\nonumber\\
     &\color{red}+ \left.\vec n_{v2}(\vec r~')\cdot 
     \overleftrightarrow n_{m2}(\vec r~')
     \cdot \frac{\vec n_{v2}(\vec r~')}{\delta n(\vec r)}-
     \frac{\delta}{\delta n(\vec r)}\sum_x\sum_y\sum_z n_{m2ij}n_{m2jk}
     n_{m2ik}\right)
     \Bigg]
\end{align}
\begin{align}   % mu_3 equation continued...
&=\frac{k_BT}{V}\int d\vec r~\int d\vec r~'~ \color{blue}\Bigg( {n_2}^3-3n_2
    \vec{n}_{v2}\cdot\vec{n}_{v2} 
     +\frac{9}{2}[\vec{n}_{v2}\cdot{\overleftrightarrow{n}_{m2}}\cdot{
     \vec{n}_{v2}}-\operatorname{Tr}({\overleftrightarrow{n}^3_{m2}})]
     \Bigg) \nonumber\\
     &~\color{blue}\Bigg(\frac{2}{24\pi(1-n_3)^3} w_3(\vec r-\vec r~')
     \Bigg) 
     + \color{red}\Bigg(\frac{1}{24\pi(1-n_3)^2}\Bigg)
     \Bigg[
     3n_2(\vec r~')^2
     w_2(\vec r-\vec r~') \nonumber\\
     &\color{red}- 3w_2(\vec r-\vec r~')\vec n_{v2}
     (\vec r~') \cdot \vec n_{v2}(\vec r~') - 6n_2(\vec r~')
     \vec w_{v2}(\vec r-\vec r~')\cdot \vec n_{v2}(\vec r~') \nonumber \\
     &\color{red}+\frac{9}{2}\Bigg(\vec w_{v2}
     (\vec r-\vec r~')
     \cdot \overleftrightarrow n_{m2}(\vec r~')\cdot \vec n_{v2}( \vec r~')
     +\vec n_{v2}(\vec r~')\cdot \overleftrightarrow w_{m2}(\vec r-\vec r~')
     \cdot \vec n_{v2}( \vec r~') \nonumber\\
     &\color{red}+ \vec n_{v2}(\vec r~')\cdot \overleftrightarrow n_{m2}(\vec r~') 
     \cdot \vec w_{v2}(\vec r-\vec r~')
     -\frac{\delta n_{m2ij}(\vec r~')}{\delta n(\vec r)}n_{m2jk}(\vec r~')
     n_{m2ik}(\vec r~') \nonumber \\     
     &\color{red}-n_{m2ij}(\vec r~')\frac{\delta n_{m2jk}(\vec r~')}
     {\delta n(\vec r)}n_{m2ik}(\vec r~') - n_{m2ij}(\vec r~')n_{m2jk}
     (\vec r~')\frac{\delta n_{m2ik}(\vec r~')}
     {\delta n(\vec r)}\Bigg)
     \Bigg]   \\ \nonumber \\
%
&=\frac{k_BT}{V}\int d\vec r~' \color{blue}\left( {n_2}^3-3n_2
    \vec{n}_{v2}\cdot\vec{n}_{v2} 
     +\frac{9}{2}[\vec{n}_{v2}\cdot{\overleftrightarrow{n}_{m2}}\cdot{
     \vec{n}_{v2}}-\operatorname{Tr}({\overleftrightarrow{n}^3_{m2}})]
     \right) \nonumber\\
     &\color{blue}~\left(\frac{2}{24\pi(1-n_3)^3}\right) \int d\vec r~ w_3(\vec r
     -\vec r~') \nonumber\\
     &\color{red}+\left(\frac{1}{24\pi(1-n_3)^2}\right)
     \Bigg[ 
     3n_2(\vec r~')^2
     \int d\vec r~ w_2(\vec r-\vec r~') 
     \nonumber \\
     &\color{red}-3\left(\int d\vec r ~w_2(\vec r-\vec r~')\right)\vec n_{v2}(\vec r~')
     \cdot \vec n_{v2}(\vec r~') \nonumber\\
     &\color{red}-6n_2(\vec r~')\left(\int d\vec r ~\vec 
     w_{v2}(\vec r-\vec r~')\right)\cdot \vec n_{v2}(\vec r~')\nonumber \\
     &\color{red}+\frac{9}{2}\Bigg(\Bigg(\int d\vec r ~\vec w_{v2}(\vec r-\vec r~')
     \Bigg)
     \cdot \overleftrightarrow n_{m2}(\vec r~')\cdot \vec n_{v2}(\vec r~') 
     \nonumber\\
     &\color{red}+\vec n_{v2}(\vec r~')\cdot \left(\int d\vec r ~\overleftrightarrow 
     w_{m2}(\vec r-\vec r~')\right) \cdot \vec n_{v2}( \vec r~') \nonumber\\    
     &\color{red}+\vec n_{v2}( \vec r~')\cdot \overleftrightarrow n_{m2}(\vec r~')
     \cdot \left(\int d\vec r ~\vec w_{v2}(\vec r-\vec r~')\right)
     \nonumber\\
     &\color{red}-n_{m2jk}(\vec r~')n_{m2ik}(\vec r~')\left(\int d\vec r~w_{m2ij}
     (\vec r-\vec r~')\right) \nonumber \\
     &\color{red}-n_{m2ij}(\vec r~')n_{m2ik}(\vec r~')\left(\int d\vec r~w_{m2jk}
     (\vec r-\vec r~')\right)\nonumber \\
     &\color{red}-n_{m2ij}(\vec r~')n_{m2ji}(\vec r)\left(\int d\vec r~w_{m2ik}
     (\vec r-\vec r~')\right)     
     \Bigg)\Bigg] 
\end{align} 

\begin{align} % mu_3 equation continued...
%
&=\frac{k_BT}{V}\int d\vec r~' \color{blue}\left( {n_2}^3-3n_2
    \vec{n}_{v2}\cdot\vec{n}_{v2} 
     +\frac{9}{2}[\vec{n}_{v2}\cdot{\overleftrightarrow{n}_{m2}}\cdot{
     \vec{n}_{v2}}-\operatorname{Tr}({\overleftrightarrow{n}^3_{m2}})]
     \right) \nonumber\\
     & \color{blue}~\left(\frac{2}{24\pi(1-n_3)^3}\left(\frac{\pi\Xi^2\alpha}{2}\right)
     \right) +\color{red} \left(\frac{1}{24\pi(1-n_3)^2}\right)\Bigg[3n_2(\vec r~')^2
     \pi(\Xi^2 + \alpha^2) \color{white}\Bigg] \nonumber\\
     &\color{red}- 3\pi(\Xi^2 + \alpha^2)\vec n_{v2}(\vec r~') \cdot \vec n_{v2}
     (\vec r~') -6n_2(\vec r~')(0)\cdot \vec n_{v2}(\vec r~')\nonumber \\
     &\color{red}+\frac{9}{2}\left(0\cdot \overleftrightarrow n_{m2}(\vec r~')\cdot 
     \vec n_{v2}( \vec r~')+\vec n_{v2}(\vec r~')\cdot 0 \cdot \vec n_{v2}
     (\vec r~')\color{white}\right)+ \color{white}\left(\color{black}\color{red}
     \vec n_{v2}(\vec r~')\cdot\overleftrightarrow n_{m2}(\vec r~')
     \cdot 0-0\right) 
     \color{white}\Bigg[\color{black}\color{red}\Bigg] \\
%
&=\frac{k_BT}{V}\int d\vec r~' \color{blue}\left({n_2}^3-3n_2
   \vec{n}_{v2}\cdot\vec{n}_{v2} 
    +\frac{9}{2}[\vec{n}_{v2}\cdot{\overleftrightarrow{n}_{m2}}\cdot{
    \vec{n}_{v2}}-\operatorname{Tr}({\overleftrightarrow{n}^3_{m2}})]
    \right) \nonumber\\
    &\color{blue}~\left(\frac{\Xi^2\alpha}{24(1-n_3)^3}
    \right) + \color{black}\color{red}\left(\frac{1}{24\pi(1-n_3)^2}\right)\Bigg[\color{white}
    \color{red}3n_2(\vec r~')^2
    \pi(\Xi^2 + \alpha^2) \color{white}\Bigg] \\
    &\color{red}- 3\pi(\Xi^2 + \alpha^2)\vec n_{v2}
    (\vec r~') \cdot \vec n_{v2}(\vec r~')
    \color{white}\Bigg[\color{black}\color{red}\Bigg]
\end{align}

Final equation for $\mu$:  
\begin{align}
  %\mu &= \frac{1}{V}\int \frac{\delta F_{ex}[n(\vec r)]}{\delta n(\vec r)}
  %\bigg|_{T,V} d\vec r \\ \\
   \mu &= \mu_1 + \mu_2 + \mu_3  \\ \nonumber \\
%
   \mu &=\frac{k_BT}{V}\int d\vec r~' (-1)\ln(1-n_3)+\left(
        \frac{\pi\Xi^2\alpha}{2}\right)\frac{n_o}
        {1-n_3} \\
    &+ \frac{n_1\pi(\Xi^2 + \alpha^2)
       +n_2\left(\frac{\alpha}{2}\right)}{1-n_3} + \frac{n_1n_2-\vec n_{v1}
         \cdot \vec n_{v2}}{(1-n_3)^2}
          \left(\frac{\pi\Xi^2\alpha}{2}\right) \\
   &\color{blue}+\left({n_2}^3-3n_2
   \vec{n}_{v2}\cdot\vec{n}_{v2} 
    +\frac{9}{2}[\vec{n}_{v2}\cdot{\overleftrightarrow{n}_{m2}}\cdot{
    \vec{n}_{v2}}-\operatorname{Tr}({\overleftrightarrow{n}^3_{m2}})]
    \right) \\
    &\color{blue}~\left(\frac{\Xi^2\alpha}{24(1-n_3)^3}
    \right) + \color{red}\left(\frac{1}{24\pi(1-n_3)^2}\right)\Bigg[
    \color{white}\color{black}
    \color{red}3n_2^2\pi(\Xi^2 + \alpha^2) \color{white}\Bigg] \\
    &\color{red}- 3\pi(\Xi^2 + \alpha^2)\vec n_{v2} \cdot \vec n_{v2}
    \color{white}\Bigg[\color{black}\color{red}\Bigg]
\end{align}

\[\]
\[\]
\[\]
\[\]
\[\]
\[\]


\begin{align}
   \int_{all space} w_i(\vec r) ~d\vec r &= \int w_i(\vec r)e^{i\vec k \cdot 
   \vec r}~d\vec r 
    = \widetilde{w}_i(\vec k) \mbox{~~with $\vec k = 0$}
\end{align}

\begin{align}
    \widetilde{w}_0(k) &= \frac{\sqrt{\pi}}{\sqrt{2}k\Xi}e^{-\left(
    \frac{\alpha^2}{2\Xi^2}\right)}\left[\operatorname{erf}
    \left(\frac{k\Xi}{2\sqrt{2}}+i\frac{\alpha}{\sqrt{2}\Xi}\right)
    +\operatorname{erf}\left(\frac{k\Xi}{2\sqrt{2}}-i\frac{\alpha}
    {\sqrt{2}\Xi}\right)\right] \\
    & \approx   \frac{\sqrt{\pi}}{\sqrt{2}k\Xi}e^{-\left(\frac{\alpha^2}
    {2\Xi^2}\right)}\left[2\operatorname{Re}\left(\operatorname{erf}
    \left(\frac{k\Xi}{2\sqrt{2}}+i\frac{\alpha}{\sqrt{2}\Xi}\right)
    \right)\right] \\
    & \approx   \frac{\sqrt{\pi}}{\sqrt{2}k\Xi}e^{-\left(\frac{\alpha^2}
    {2\Xi^2}\right)}\left[2e^{\left(\frac{\alpha^2}{2\Xi^2}\right)}
    \frac{2}{\sqrt{\pi}}\left(\frac{k\Xi}{2\sqrt{2}} -\frac{2\left(
    \frac{\alpha^2}{2\Xi^2}\right)+1}{3}\left(\frac{k\Xi}{2\sqrt{2}}
    \right)^3\right)\right] \\
    & \approx   \frac{\sqrt{\pi}}{\sqrt{2}k\Xi}\left(\frac{4}{\sqrt{\pi}}
    \right)\left(\frac{k\Xi}{2\sqrt{2}} -\frac{\frac{\alpha^2}{\Xi^2}+1}
    {3}\left(\frac{k^3\Xi^3}{16\sqrt{2}}\right)\right) \\
     & \approx   \frac{\sqrt{\pi}}{\sqrt{2}\Xi}\left(\frac{4}{\sqrt{\pi}}
    \right)\left(\frac{\Xi}{2\sqrt{2}} -\frac{\frac{\alpha^2}{\Xi^2}+1}
    {3}\left(\frac{k^2\Xi^3}{16\sqrt{2}}\right)\right) \\
     \widetilde{w}_0(k=0) & = 1
\end{align}

\begin{align}
    \widetilde{w}_1(k) &= \frac{1}{k}e^{-\frac{k^2\Xi^2}{8}}
    \sin\left(\frac{k\alpha}{2}\right)  \\
  & \approx\left(\frac{1}{k}\right)\left(1-\frac{k^2\Xi^2}{8}\right)
  \left(\frac{k\alpha}{2}\right)  \\
  \widetilde{w}_1(k=0) & \approx\frac{\alpha}{2}
\end{align}

\begin{align}
     \widetilde{w}_2(k) &= \frac{2\pi}{k}e^{-\frac{k^2\Xi^2}{8}}
       \left[\frac{k\Xi^2}{2}\cos\left(\frac{k\alpha}{2}\right)
       +\alpha\sin\left(\frac{k\alpha}{2}\right)\right] \\
       &  \approx\left(\frac{2\pi}{k}\right)\left(1-\frac{k^2\Xi^2}{8}
       \right)\left[\frac{k\Xi^2}{2}\left(1\right)+\alpha\left(
       \frac{k\alpha}{2}\right)\right] \\
       \widetilde{w}_2(k=0) & \approx \pi(\Xi^2 + \alpha^2)
\end{align}

\begin{align}
  \widetilde{w}_3(k) &= \frac{4\pi}{k^3}e^{-\frac{k^2\Xi^2}{8}}
    \left[\left(1+\frac{k^2\Xi^2}{4}\right)\sin\left(\frac{k\alpha}{2}
    \right)-\frac{k\alpha}{2}\cos\left(\frac{k\alpha}{2}\right)\right]\\
    &\approx \left(\frac{4\pi}{k^3}\right)\left(1-\frac{k^2\Xi^2}{8}
    \right)\left[\left(1+\frac{k^2\Xi^2}{4}\right)\left(\frac{k\alpha}
    {2}\right)-\frac{k\alpha}{2}\left(1\right)\right] \\ 
    &\approx \left(\frac{4\pi}{k^3} -\frac{4\pi\Xi^2}{k8}\right)
    \left[\frac{k\alpha} {2}+\frac{k^3\Xi^2\alpha}{8}
    -\frac{k\alpha}{2}\right]  \\
    &\approx \left(\frac{4\pi}{k^3} -\frac{4\pi\Xi^2}{k8}\right)
    \left[\frac{k^3\Xi^2\alpha}{8}\right]  \\
     \widetilde{w}_3(k=0) & \approx \frac{\pi\Xi^2\alpha}{2}
\end{align}

\begin{align}
   \widetilde{\vec{w}}_2(\vec{k})&= \frac{4\pi{i}}{k^2}
   e^{-\frac{k^2\Xi^2}{8}}
   \left[\left(\frac{\Xi^2k}{4}+\frac{1}{k}\right)\sin\left(
   \frac{k\alpha}{2}\right)
   -\frac{\alpha}{2}\cos\left(\frac{k\alpha}{2}\right)\right]
   {~}\vec{k} \\
    & \approx  \frac{4\pi{i}}{k^2}\left(1-\frac{k^2\Xi^2}{8}\right)
    \left[\left(\frac{\Xi^2k}{4}+\frac{1}{k}\right)\left(\frac{k\alpha}
    {2}\right)-\frac{\alpha}{2}\left(1\right)\right]{~}\vec{k}\\
    & \approx  \left(\frac{4\pi{i}}{k^2}-\frac{4\pi i\Xi^2}{8}\right)
    \left[\frac{\Xi^2k^2\alpha}{8}+\frac{\alpha}{2}-\frac{\alpha}{2}
    \right]{~}\vec{k} \\
    \widetilde{\vec w}_2(\vec k=0) & \approx \frac{\pi\Xi^2\alpha i}{4} 
    {~}\vec{k} = 0
\end{align}

\begin{align}
   \widetilde{\vec{w}}_1(\vec{k}) &= \frac{i}{k^2}\left[\frac{\sqrt{\pi}}
   {\sqrt{2}k\Xi}
   e^{-\left(\frac{\alpha}{\sqrt{2}\Xi}\right)^2}\left[\operatorname{erf}
   \left(\frac{k\Xi}{(\sqrt{2})^3}+\frac{i\alpha}{\sqrt{2}\Xi}\right)
   +\operatorname{erf}\left(\frac{k\Xi}{(\sqrt{2})^3}-\frac{i\alpha}
   {\sqrt{2}\Xi}
   \right)\right]\color{white}\right]\color{black}  \nonumber \\
 & \color{white}\left[\color{black}-e^{-\frac{k^2\Xi^2}{8}}\cos\left(
    \frac{k\alpha}{2}\right)\right]{~}\color{black}\vec{k} \\
    & \approx  \frac{i\sqrt{\pi}}{\sqrt{2}k^3\Xi}e^{-\left(\frac{\alpha}
    {\sqrt{2}\Xi}\right)^2}\left[2\operatorname{Re}\left(\operatorname{erf}
    \left(\frac{k\Xi}{(\sqrt{2})^3}+\frac{i\alpha}{\sqrt{2}\Xi}\right)
    \right)\right] \nonumber \\
    &-\frac{i}{k^2}\left(1-\frac{k^2\Xi^2}{8}\right)\left(1\right) 
    {~}\vec{k} \\
    & \approx  \frac{i\sqrt{\pi}}{\sqrt{2}k^3\Xi}e^{-\left(\frac{\alpha}
    {\sqrt{2}\Xi}\right)^2}\left[2e^{\left(\frac{\alpha^2}{2\Xi^2}\right)}
    \frac{2}{\sqrt{\pi}}   \left(\frac{k\Xi}{2\sqrt{2}}   -\frac{2\left(
    \frac{\alpha^2}{2\Xi^2}\right)+1}{3}\left(\frac{k\Xi}{2\sqrt{2}}
    \right)^3\right)\right] \nonumber \\
    &-\left(\frac{i}{k^2}-\frac{i\Xi^2}{8}\right) {~}\vec{k}\\
    & \approx  \frac{i\sqrt{\pi}}{\sqrt{2}k^3\Xi}\left[\frac{4}{\sqrt{\pi}}
    \left(\frac{k\Xi}{(\sqrt{2})^3} -\frac{(\frac{\alpha^2}{\Xi^2}+1)}{3}
    \frac{k^3\Xi^3}{(\sqrt{2})^3}\right)\right] -\left(\frac{i}{k^2}
    -\frac{i\Xi^2}{8}\right) {~}\vec{k}\\
    & \approx  \frac{4i}{\sqrt{2}k^3\Xi}\left(\frac{k\Xi}{(\sqrt{2})^3} 
    -\frac{(\frac{\alpha^2}{\Xi^2}+1)}{3}\frac{k^3\Xi^3}{(\sqrt{2})^3}
    \right)-\left(\frac{i}{k^2}-\frac{i\Xi^2}{8}\right) {~~~}\vec{k}\\
    & \approx  \frac{i}{k^2} -\frac{(\frac{\alpha^2}{\Xi^2}+1)i\Xi^2}{3}
    -\frac{i}{k^2}+\frac{i\Xi^2}{8} {~~~}\vec{k}\\
    & \approx  -\frac{(\frac{\alpha^2}{\Xi^2}+1)i\Xi^2}{3}
    +\frac{i\Xi^2}{8} {~~~}\vec{k}\\
     \widetilde{\vec w}_1(\vec k=0) &= 0 
\end{align}

As shown in the thesis,
\begin{align}
  \widetilde{\overleftrightarrow w}_{m2}(\vec k=0) &= 0  \\
  \widetilde w_{m2ik}(\vec k = 0) &= 0
\end{align}

Compare to the homogeneous case,
\begin{align}
    F_{1ex} &= K_BT\Phi_1 V\\
           &= k_BT\left(-n_0\ln\left(1-n_3\right)\right)V  \\
           &= -k_BT\left(\int nw_0 \right)\left(\ln\left(1-\int nw_3\right)\right)V  \\
           &= -k_BT\left(n\int w_0 \right)\left(\ln\left(1-n\int w_3\right)\right)V  \\
 \mu_{1ex} &= \frac{1}{V}\frac{\partial F_{1ex}}{\partial n} \bigg|_{T,V}\\
           &= \frac{-k_BTV}{V} \frac{\partial}{\partial n}\left[ \left(n\int w_0 \right)\left(\ln\left(1-n\int w_3\right)\right)\right]\\
           &= - k_BT\left[\left(\int w_0\right)\ln\left(1-n\int w_3\right) +\left(n\int w_0 \right)\frac{\left(-\int w_3\right)}{1-\left(n\int w_3\right)}\right]\\
           &= - k_BT\left[(1)\ln\left(1-n\frac{\pi\Xi^2\alpha}{2}\right) -n\left(1\right)\frac{\frac{\pi\Xi^2\alpha}{2}}{1-n\frac{\pi\Xi^2\alpha}{2}}\right]\\
           &= k_BT\left[-\ln\left(1-n\frac{\pi\Xi^2\alpha}{2}\right) +\frac{n\frac{\pi\Xi^2\alpha}{2}}{1-n\frac{\pi\Xi^2\alpha}{2}}\right]
\end{align}

\begin{align}
    F_{1ex} &= k_BT\int \Phi_1~d\vec r~' \\ \nonumber\\
    \mu_{1ex} &= \frac{1}{V}\int \frac{\delta F[n(\vec r)]}{\delta n(\vec r)} d\vec r \\ \nonumber\\
%
        &= -\frac{k_BT}{V} \int d\vec r ~\frac{\delta}
        {\delta n(\vec r)}\left(\int d\vec r~'~n_o(\vec r~')\ln(1-n_3(\vec r~'))\right) \\ \nonumber\\
%
        &= -\frac{k_BT}{V}\int d\vec r\int d\vec r~'~\frac{\delta n_o(\vec r~')}
        {\delta n(\vec r)}\ln(1-n_3(\vec r~'))+\frac{n_o(\vec r~')}
          {1-n_3(\vec r)}\left(-\frac{\delta n_3(\vec r~')}{\delta n(\vec r)}
          \right) \\ \nonumber\\
%
        &= -\frac{k_BT}{V} \int d\vec r \int d\vec r~' ~w_0(\vec r - \vec r~')
          \ln(1-n_3(\vec r~')) - \frac{n_o(\vec r~')}
          {1-n_3(\vec r~')}w_3(\vec r - \vec r~') \\ \nonumber\\
%
        &= -\frac{k_BT}{V} \int d\vec r~'\ln(1-n_3(\vec r~')) \int d \vec r  
          ~w_0(\vec r - \vec r~') \nonumber\\
        &+ \frac{k_BT}{V} \int d\vec r~' ~\frac{n_o(\vec r~')}{1-n_3(\vec r~')} 
          \int d \vec r~w_3(\vec r - \vec r~')  \\ \\
%
        &= -\frac{k_BT}{V} \int d\vec r~' ~\ln(1-n_3(\vec r~'))(1)
          + \frac{k_BT}{V} \int d\vec r~' ~\frac{n_o(\vec r~')}{1-n_3(\vec r~')}
          \left(\frac{\pi\Xi^2\alpha}{2}\right) \\ \\
%
        &= \frac{k_BT}{V}\int d\vec r~'~ \left[-\ln(1-n_3(\vec r~'))+\left(
        \frac{\pi\Xi^2\alpha}{2}\right)\frac{n_o(\vec r~')}{1-n_3(\vec r~')}\right]   \text{Good!}
\end{align}

\begin{align}
    F_{2ex} &= K_BT\Phi_2 V\\
            &= k_BT\left(\frac{n_1n_2-\vec n_1\cdot\vec n_2}{1-n_3}\right)V  \\
            &= k_BT\left(\frac{\int nw_1\int nw_2-\int n\vec w_1\cdot \int n \vec w_2}{1-\int nw_3}\right)V  \\
            &= k_BT\left(\frac{n\left(\int w_1\right) n\left(\int w_2\right)- n\left(\int \vec w_1\right)\cdot n\left(\int \vec w_2\right)}{1-n\left(\int w_3\right)}\right)V  \\
            %&= k_BT\left(\frac{n\left(\int w_1\right) n\left(\int w_2\right) -n\left(0\right)\cdot n\left(0\right)}{1-n\left(\int w_3\right)}\right)V  \\
            %&= k_BT\left(\frac{n\left(\int w_1\right) n\left(\int w_2\right)}{1-n\left(\int w_3\right)}\right)V  \\
  \mu_{2ex} &= \frac{1}{V}\frac{\partial F_{2ex}}{\partial n} \bigg|_{T,V}\\
            &= \frac{k_BTV}{V} \frac{\partial}{\partial n}\left[\frac{n\left(\int w_1\right) n\left(\int w_2\right)- n\left(\int \vec w_1\right)\cdot n\left(\int \vec w_2\right)}{1-n\left(\int w_3\right)}\right] \\ 
            &= k_BT \frac{\partial}{\partial n}\left[\frac{n\left(\int w_1\right) n\left(\int w_2\right)- n\left(\int \vec w_1\right)\cdot n\left(\int \vec w_2\right)}{1-n\left(\int w_3\right)}\right] \\ \\          
%            
            &= k_BT \left(\frac{\partial}{\partial n}\left[n\left(\int w_1\right)\right]\frac{n\left(\int w_2\right)}{1-n\left(\int w_3\right)} \color{white}\right]\color{black} \\
            &+ \frac{\partial}{\partial n}\left[n\left(\int w_2\right)\right]\frac{n\left(\int w_1\right)}{1-n\left(\int w_3\right)} \\
%            
            &+ \frac{\partial}{\partial n}\left[n\left(\int \vec w_1\right)\right]\frac{n\left(\int \vec w_2\right)}{1-n\left(\int w_3\right)} \color{white}\right]\color{black} \\
            &+ \frac{\partial}{\partial n}\left[n\left(\int \vec w_2\right)\right]\frac{n\left(\int \vec w_1\right)}{1-n\left(\int w_3\right)} \\
 %           
            &+ \color{white}\left(\color{black}\frac{\partial}{\partial n}\left[\frac{1}{1-n\left(\int w_3\right)}\right]n\left(\int w_1\right)n\left(\int w_2\right) \right)\\ \\
%            
            &= k_BT \left[\left(\int w_1\right)\frac{n\left(\int w_2\right)}{1-n\left(\int w_3\right)} \color{white}\right]\color{black} \\
            &+ \left(\int w_2\right)\frac{n\left(\int w_1\right)}{1-n\left(\int w_3\right)} \\
            &+ \color{white}\left[\color{black}\left(-\frac{-\int w_3}{\left(1-n\left(\int w_3\right)\right)^2}\right)n\left(\int w_1\right)n\left(\int w_2\right)\right]
\end{align} 
\begin{align}          
            &= k_BT \left[\left(\frac{\alpha}{2}\right)\frac{n\left(\pi\left(\Xi^2+\alpha^2\right)\right)}{1-n\left(\int w_3\right)} \color{white}\right]\color{black} \\
            &+ \left(\pi\left(\Xi^2+\alpha^2\right)\right)\frac{n\left(\frac{\alpha}{2}\right)}{1-n\left(\int w_3\right)} \\
            &+ \color{white}\left[\color{black}\left(\frac{\int w_3}{\left(1-n\left(\int w_3\right)\right)^2}\right)n\left(\frac{\alpha}{2}\right)n\left(\int w_2\right)\right] \\ \\
%            
            &= k_BT \left[\frac{\left(\frac{\alpha}{2}\right)n\left(\pi\left(\Xi^2+\alpha^2\right)\right) + \left(\pi\left(\Xi^2+\alpha^2\right)\right)n\left(\frac{\alpha}{2}\right)}{1-n\left(\int w_3\right)} \color{white}\right]\color{black} \\
            &+ \color{white}\left[\color{black}\left(\frac{\int w_3}{\left(1-n\left(\int w_3\right)\right)^2}\right)n\left(\frac{\alpha}{2}\right)n\left(\int w_2\right)\right] \\ \\
\end{align}

\begin{align}
\mu_{2ex} &= \frac{k_BT}{V}\int d\vec r~'~ \frac{n_1(\vec r~')
           \pi(\Xi^2 + \alpha^2)
          +n_2(\vec r~')\left(\frac{\alpha}{2}\right)}{1-n_3(\vec r~')} 
          \nonumber \\
          &+ \frac{n_1(\vec r~')n_2(\vec r~')
          -\vec n_{v1}(\vec r~')
          \cdot \vec n_{v2}(\vec r~')}{(1-n_3(\vec r~'))^2}
          \left(\frac{\pi\Xi^2\alpha}{2}\right)  ~~~ \text{Good!}
\end{align}

\begin{align}
    %F_{3ex} &= K_BT\Phi_3 V  \\
            %&= k_BT\left(\frac{{n_2}^3-3n_2\vec n_{v2}\cdot\vec n_{v2}+\frac{9}{2}
       %\left[\vec n_{v2}\cdot{\overleftrightarrow{n}_{m2}}\cdot{\vec n_{v2}}
       %-\operatorname{Tr}({\overleftrightarrow n^3_{m2}})\right]}{24\pi(1-n_3)^2} \right)V  \\
%%       
       %\mu_{3ex} &= \frac{1}{V}\frac{\partial F_{3ex}}{\partial n} \bigg|_{T,V} \\
         %&= \frac{k_BTV}{V} \frac{\partial}{\partial n}\left[
       %\frac{{n_2}^3-3n_2\vec n_{v2}\cdot\vec n_{v2}+\frac{9}{2}
       %[\vec n_{v2}\cdot{\overleftrightarrow{n}_{m2}}\cdot{\vec n_{v2}}
       %-\operatorname{Tr}({\overleftrightarrow n^3_{m2}})]}{24\pi(1-n_3)^2}\right]  \\       
       %&= \frac{k_BTV}{V} \frac{\partial}{\partial n}
       %\left[\frac{\left(\int n w_2\right)^3-3\int n w_2\int \vec n w_{v2}\cdot\int 
       %\vec n w_{v2}+\frac{9}{2}\left[\int n\vec w_{v2}\cdot{\int n\overleftrightarrow{w}_{m2}}
       %\cdot{\int n\vec w_{v2}}-\operatorname{Tr}({\left(\int n\overleftrightarrow w_{m2}}
       %\right)^3)\right]
       &= \frac{k_BT}{24\pi(1-\int n w_3)^2}\left[3\left(\int n w_2\right)^2\left(\int w_2\right)
       -3\int w_2\int \vec n w_{v2}\cdot\int \vec n w_{v2} \color{white}\right]\color{black} \\ 
       &- 2*3\int nw_2\int \vec w_{v2}\cdot\int \vec n w_{v2}
       +\frac{9}{2}\left[\color{white}\right]\color{black}
       \int \vec n w_{v2}\cdot{\int n\overleftrightarrow{w}_{m2}}
       \cdot{\int \vec n w_{v2}}-\operatorname{Tr}({\left(\int n\overleftrightarrow w_{m2}}
       \right)^3)\color{white}\left[\color{black}\right]\color{white}\left[\color{black}\right]
\end{align} 

NEW METHOD - Jan -------------------------------------------------------
\begin{align}
  \Phi_1 &= -n_0\ln\left(1-n_3\right)                            \\
  \frac{\partial \Phi_1}{\partial n_0} &= -\ln\left(1-n_3\right) \\
  \frac{\partial \Phi_1}{\partial n_1} &= 0                      \\
  \frac{\partial \Phi_1}{\partial n_2} &= 0                      \\
  \frac{\partial \Phi_1}{\partial n_3} &= \frac{n_0}{1-n_3}      \\
  \frac{\partial \Phi_1}{\partial \vec n_1} &= 0                 \\
  \frac{\partial \Phi_1}{\partial \vec n_2} &= 0                 \\
  \frac{\partial \Phi_1}{\partial \overleftrightarrow  n_{m2}} &= 0   \\
\end{align} 

\begin{align}
  \Phi_2 &= \frac{n_1n_2-\vec n_1\cdot\vec n_2}{1-n_3}          \\
  \frac{\partial \Phi_2}{\partial n_0} &=  0                    \\
  \frac{\partial \Phi_2}{\partial n_1} &=  \frac{n_2}{1-n_3}    \\
  \frac{\partial \Phi_2}{\partial n_2} &=  \frac{n_1}{1-n_3}    \\
  \frac{\partial \Phi_2}{\partial n_3} &=   \frac{n_1n_2-\vec n_1\cdot\vec n_2}{\left(1-n_3\right)^2} \\
  \frac{\partial \Phi_2}{\partial \vec n_1} &=   \frac{\vec n_2}{1-n_3}  \\
  \frac{\partial \Phi_2}{\partial \vec n_2} &=   \frac{\vec n_1}{1-n_3}  \\
  \frac{\partial \Phi_2}{\partial \overleftrightarrow  n_{m2}} &= 0      \\
\end{align} 

\begin{align}
  \Phi_3 &= \frac{{n_2}^3-3n_2\vec n_{v2}\cdot\vec n_{v2}+\frac{9}{2}
       [\vec n_{v2}\cdot{\overleftrightarrow{n}_{m2}}\cdot{\vec n_{v2}}
       -\operatorname{Tr}({\overleftrightarrow n^3_{m2}})]}{24\pi(1-n_3)^2}   \\
  \frac{\partial \Phi_3}{\partial n_0} &= 0   \\
  \frac{\partial \Phi_3}{\partial n_1} &= 0   \\
  \frac{\partial \Phi_3}{\partial n_2} &=  \frac{3n_2^2-3\vec n_{v2}\cdot\vec n_{v2}}{24\pi(1-n_3)^2}  \\
  \frac{\partial \Phi_3}{\partial n_3} &=  \frac{{n_2}^3-3n_2\vec n_{v2}\cdot\vec n_{v2}+\frac{9}{2}
       [\vec n_{v2}\cdot{\overleftrightarrow{n}_{m2}}\cdot{\vec n_{v2}}
       -\operatorname{Tr}({\overleftrightarrow n^3_{m2}})]}{12\pi(1-n_3)^3}  \\
  \frac{\partial \Phi_3}{\partial \vec n_1} &=  0  \\
  \frac{\partial \Phi_3}{\partial \vec n_2} &= \frac{-6n_2\vec n_2 +9\vec n_{v2}\cdot{\overleftrightarrow{n}_{m2}}}{24\pi(1-n_3)^2} ~~~~\text{CHECK} \\
  \frac{\partial \Phi_3}{\partial \overleftrightarrow  n_{m2}} &= \frac{\frac{9}{2}
       [\vec n_{v2}\cdot{\vec n_{v2}}
       -3\operatorname{Tr}({\overleftrightarrow n^2_{m2}})]}{24\pi(1-n_3)^2}   ~~~~\text{CHECK} \\
\end{align}
Collecting terms
\begin{align}
 &= k_BT\int \sum_{i=0}^3 \frac{\partial \Phi_1}{\partial n_i} + \frac{\partial \Phi_2}{\partial n_i} 
  + \frac{\partial \Phi_3}{\partial n_i} \\
 &= k_BT\int -\ln\left(1-n_3\right) + \frac{n_0}{1-n_3}      \\
%
&+ \frac{n_2}{1-n_3} + \frac{n_1}{1-n_3} + \frac{n_1n_2-\vec n_1\cdot\vec n_2}{\left(1-n_3\right)^2} 
+ \frac{\vec n_2}{1-n_3} + \frac{\vec n_1}{1-n_3}  \\
%  
 &+ \frac{3n_2^2-3\vec n_{v2}\cdot\vec n_{v2}}{24\pi(1-n_3)^2}  \\
 &+ \frac{{n_2}^3-3n_2\vec n_{v2}\cdot\vec n_{v2}+\frac{9}{2}
       [\vec n_{v2}\cdot{\overleftrightarrow{n}_{m2}}\cdot{\vec n_{v2}}
       -\operatorname{Tr}({\overleftrightarrow n^3_{m2}})]}{12\pi(1-n_3)^3}  \\
&+ \frac{-6n_2\vec n_2 +9\vec n_{v2}\cdot{\overleftrightarrow{n}_{m2}}}{24\pi(1-n_3)^2} ~~~~\text{CHECK} \\
&+ \frac{\frac{9}{2}[\vec n_{v2}\cdot{\vec n_{v2}}
       -3\operatorname{Tr}({\overleftrightarrow n^2_{m2}})]}{24\pi(1-n_3)^2}   ~~~~\text{CHECK} \\
\end{align}

\begin{align}
  \mu &= \frac{\partial F}{\partial N} = \frac{dF[Cn(\vec r)]}{d(CN)} = \frac{1}{N}\frac{dF[Cn(\vec r)]}{dC}\\
   F[Cn(\vec r)] &= k_BT\int \Phi_1 +\Phi_2+\Phi_3 ~d\vec r \\
   &= k_BT\int -Cn_0\ln\left(1-Cn_3\right) + \frac{Cn_1Cn_2-C\vec n_1\cdotC\vec n_2}{1-Cn_3} \\
   &+ \frac{{C^3n_2}^3-3Cn_2C\vec n_{v2}\cdot C\vec n_{v2}+\frac{9}{2}
       [C\vec n_{v2}\cdot{C\overleftrightarrow{n}_{m2}}\cdot{C\vec n_{v2}}
       -\operatorname{Tr}({C^3\overleftrightarrow n^3_{m2}})]}{24\pi(1-Cn_3)^2}    \\ \\
%   
  \frac{dF[n(\vec r)]}{dC}    &= k_BT\int -n_0\ln\left(1-Cn_3\right) + C^2\frac{n_0}{1-Cn_3} \\
   &+ 2C\frac{n_1n_2-\vec n_1\cdot\vec n_2}{1-Cn_3} +C^3 \frac{n_1n_2-\vec n_1\cdot\vec n_2}{(1-Cn_3)^2} \\   
   &+ \frac{{3C^2n_2}^3-9C^2n_2\vec n_{v2}\cdot \vec n_{v2} 
   + \frac{27}{2}[C^2\vec n_{v2}\cdot{\overleftrightarrow{n}_{m2}}\cdot{\vec n_{v2}} 
       -\operatorname{Tr}({C^2\overleftrightarrow n^3_{m2}})]}{24\pi(1-Cn_3)^2}   \\ 
    &+ \frac{{C^4n_2}^3-3C^4n_2\vec n_{v2}\cdot \vec n_{v2} 
    + \frac{9}{2}[C^4\vec n_{v2}\cdot{\overleftrightarrow{n}_{m2}}\cdot{\vec n_{v2}}   
       -\operatorname{Tr}({C^4\overleftrightarrow n^3_{m2}})]}{12\pi(1-Cn_3)^3}     
\end{align} 
Setting C=1 gives
\begin{align}
   \frac{dF[n(\vec r)]}{dC} &= k_BT\int -n_0\ln\left(1-n_3\right) + \frac{n_0}{1-n_3} \\
   &+ \frac{n_1n_2-\vec n_1\cdot\vec n_2}{1-n_3} + \frac{n_1n_2-\vec n_1\cdot\vec n_2}{(1-n_3)^2} \\   
   &+ \frac{{3n_2}^3-9n_2\vec n_{v2}\cdot \vec n_{v2} 
   + \frac{27}{2}[\vec n_{v2}\cdot{\overleftrightarrow{n}_{m2}}\cdot{\vec n_{v2}} 
       -\operatorname{Tr}({\overleftrightarrow n^3_{m2}})]}{24\pi(1-n_3)^2}   \\ 
    &+ \frac{{n_2}^3-3n_2\vec n_{v2}\cdot \vec n_{v2} 
    + \frac{9}{2}[\vec n_{v2}\cdot{\overleftrightarrow{n}_{m2}}\cdot{\vec n_{v2}}   
       -\operatorname{Tr}({\overleftrightarrow n^3_{m2}})]}{12\pi(1-n_3)^3}     
\end{align} 

\end{document}
