\documentclass{beamer}
\usepackage{beamerthemelined}
\usepackage{pstricks}

\setbeamertemplate{navigation symbols}{}
\addtobeamertemplate{navigation symbols}{}{%
    \usebeamerfont{footline}%
    \usebeamercolor[fg]{footline}%
    \hspace{1em}%
    \insertframenumber/\inserttotalframenumber
}
\setbeamertemplate{caption}[numbered]

\usepackage{caption}
\captionsetup{textfont={small,bf,color=blue}}
\captionsetup{labelformat=empty}

%%% physics and math
\usepackage[boldvectors,braket]{physymb}
\newcommand{\bk}{\Braket} % shorthand for braket notation
\usepackage{amsmath}
\newcommand{\p}[1]{\left(#1\right)} % parenthesis
\renewcommand{\sp}[1]{\left[#1\right]} % square parenthesis
\renewcommand{\set}[1]{\left\{#1\right\}} % curly parenthesis
\newcommand{\f}[2]{\dfrac{#1}{#2}}
\renewcommand{\d}{\partial}
\renewcommand{\t}[1]{\text{#1}}

\let\olditem\item
\renewcommand{\item}{\setlength{\itemsep}{6pt}\olditem}
\usepackage{graphicx}

\usepackage{xcolor}

\title[Optimizing MC Simulation of the SW Fluid]
{Optimizing Monte Carlo Simulation of the Square-Well Fluid}
\author[M. A. Perlin, D. Roundy]
{Michael A. Perlin\\Advisor: David Roundy}
\institute{\bf Department of Physics, Oregon State University}
\date{02 June 2015}

\begin{document}

\begin{frame}
  \maketitle
\end{frame}

\begin{frame}
  \frametitle{Preview}
  \begin{itemize}
  \item Square-well fluid
  \item Monte Carlo fluid simulation
  \item Histogram methods
  \item Simulation results
    \begin{itemize}
    \item Method comparison
    \end{itemize}
  \end{itemize}
\end{frame}

\begin{frame}
  \frametitle{Phase transitions}
  \begin{figure}
    \centering
    \includegraphics[height=0.75\textheight]<1>{figs/phase-diagram.pdf}
    \includegraphics[height=0.75\textheight]
    <2>{figs/phase-diagram-critical-point.pdf}
    \caption{A generic phase diagram}
  \end{figure}
\end{frame}

\begin{frame}
  \frametitle{Square-well fluid}
  \begin{itemize}
  \item Simplest system with a liquid-vapor phase transition
  \item Spheres of diameter $\sigma$ with piece-wise pair potential
  \begin{align*}
    v_{sw}\p{r}=\left\{
      \begin{array}{ll}
        \infty & r<\sigma \\
        -\epsilon & \sigma<r<\lambda\sigma \\
        0 & r>\sigma
      \end{array}
    \right.
  \end{align*}
  \end{itemize}
\end{frame}

\begin{frame}
  \frametitle{Square-well fluid}
  \begin{figure}
    \centering
    \includegraphics[height=0.75\textheight]<1>{figs/square-well.pdf}
    \includegraphics[height=0.75\textheight]<2>{figs/square-well-v1.pdf}
    \includegraphics[height=0.75\textheight]<3>{figs/square-well-v2.pdf}
    \includegraphics[height=0.75\textheight]<4>{figs/square-well-v3.pdf}
    \caption{Pair potential of the square-well fluid}
  \end{figure}
\end{frame}

\begin{frame}
  \frametitle{Monte Carlo fluid simulation}
  \begin{itemize}
  \item Model systems vs. the real world
  \item Unbiased Monte Carlo: random walk in configuration space
  \item Histogram $H\p{E}$ of observed energies is proportional to the
    density of states $D\p{E}$
  \end{itemize}
\end{frame}

\begin{frame}
  \frametitle{Thermodynamic properties}
  \begin{itemize}
  \item Partition function
    \begin{align*}
      Z\p{T}=\sum_se^{-E_s/kT}=\sum_E{\color{red}D\p{E}}e^{-E/kT}
    \end{align*}
  \item Temperature dependence of observable properties
    \begin{align*}
      \bk{X}_T=\sum_E{\color{red} X\p{E}}P\p{E,T}
      =\f1{Z\p{T}}\sum_E{\color{red}X\p{E}D\p{E}}e^{-E/kT}
    \end{align*}
  \end{itemize}
\end{frame}

\begin{frame}
  \frametitle{Thermodynamic properties}
  \begin{itemize}
  \item Internal energy $U\p{T}=\bk{E}_T$
  \item Heat capacity
    \begin{align*}
      C_V\p{T}=\p{\f{\d U}{\d T}}_V
      =\bk{\p{\f{E}{kT}}^2}_T-\bk{\f{E}{kT}}_T^2
    \end{align*}
  \item Configurational entropy
  \begin{align*}
    S_{\text{config}}\p{T} =k\bk{\f{E}{kT}+\ln Z}_T
    -k\bk{\f{E}{kT}+\ln Z}_{T\to\infty}
  \end{align*}

  \end{itemize}
\end{frame}

\begin{frame}
  \frametitle{Density of states}
  \begin{figure}
    \centering
    \includegraphics[height=0.75\textheight]
    {figs/dos-poster-example.pdf}
    \caption{Sample density of states ($N=25$, $\eta=0.3$)}
  \end{figure}
\end{frame}

\begin{frame}
  \frametitle{Density of states}
  \begin{figure}
    \centering
    \includegraphics[height=0.75\textheight]
    {figs/dos-thesis-example.pdf}
    \caption{Sample density of states ($N=100$, $\eta=0.1$)}
  \end{figure}
\end{frame}

\begin{frame}
  \frametitle{Histogram methods}
  \begin{itemize}
  \item {\it Biased} walk in configuration space
    \begin{itemize}
    \item To each energy $E$, assign a weight $w\p{E}$
    \item Accept Monte Carlo moves $s_i\to s_f$ with probability
      \begin{align*}
        P\p{s_i\to s_f}=\min\set{\f{w\p{E_f}}{w\p{E_i}},1}
      \end{align*}
    \end{itemize}
  \item $w\p{E}$ is proportional to the bias on states with energy $E$
  \item Histogram method: an algorithm for finding $w\p{E}$
  \end{itemize}
\end{frame}

\begin{frame}
  \frametitle{Canonical weights}
  \begin{itemize}
  \item Boltzmann distribution: $w\p{E}=e^{-E/kT_0}$
  \item ``Fixed-temperature'' simulation with $T=T_0$
  \item Temperature limitations, autocorrelation problems
  \end{itemize}
\end{frame}

\begin{frame}
  \frametitle{Flat histogram}
  \begin{itemize}
  \item Broad energy sampling: flat energy histogram $H\p{E}$
  \item Weights $w\p{E}=1/D\p{E}$
  \item ``Optimal'' weights for a flat energy histogram
  \item Assumes knowledge of $D\p{E}$
  \item Na\"ive implementation
  \end{itemize}
\end{frame}

\begin{frame}
  \frametitle{Simple flat method}
  \begin{itemize}
  \item ``Simplest viable'' histogram method
  \item Iterate flat histogram initialization procedure
  \item Standard of comparison
  \end{itemize}
\end{frame}

\begin{frame}
  \frametitle{Wang-Landau}
  \begin{itemize}
  \item Standard histogram method
  \item Weight adjustment on-the-fly
  \item Four free parameters
  \item Requires specifying an energy range
    \begin{itemize}
    \item ``Vanilla'' Wang-Landau
    \end{itemize}
  \end{itemize}
\end{frame}

\begin{frame}
  \frametitle{Transition matrix Monte Carlo (TMMC)}
  \begin{itemize}
  \item Transition matrix $T\p{E_i\to E_j}$: probability of a
    transition $E_i\to E_j$ in one Monte Carlo move
  \item Ensemble of simulations at equilibrium satisfy
    \begin{align*}
      D\p{E_j}=\sum_iT\p{E_i\to E_j}D\p{E_i}
    \end{align*}
  \item Data on $T\p{E_i\to E_j}$ is easy to collect
  \end{itemize}
\end{frame}

\begin{frame}
  \frametitle{Transition matrix Monte Carlo (TMMC)}
  \begin{itemize}
  \item Initialize with probability of accepting a move $E_i\to E_f$
    \begin{align*}
      P\p{E_i\to E_f}\approx\f{T\p{E_i\to E_f}}{T\p{E_f\to E_i}}
    \end{align*}
  \item One free parameter
  \item Flat energy histogram
    \begin{itemize}
    \item Initialization: for sampling $T\p{E_i\to E_f}$
    \item Simulation: via $w\p{E}=1/D\p{E}$
    \end{itemize}
  \end{itemize}
\end{frame}

\begin{frame}
  \frametitle{Optimized ensemble}
  \begin{itemize}
  \item Motivation: reduce autocorrelation of low energy statistics
  \item Optimize time for ``round trip'' in energy range
  \item Weights $w\p{E}=\f1{D\p{E}\alpha\p{E}}$
  \item Tricky multi-step initialization procedure
  \item OE-TMMC hybrid: compute $\alpha\p{E}$ from transition matrix
  \end{itemize}
\end{frame}

\begin{frame}
  \frametitle{Histograms and arrays}
  \begin{figure}
    \centering
    \includegraphics[height=0.75\textheight]{figs/hist-example.pdf}
    \caption{Example energy histograms ($N=25$, $\eta=0.3$)}
  \end{figure}
\end{frame}

\begin{frame}
  \frametitle{Histograms and arrays}
  \begin{figure}
    \centering
    \includegraphics[height=0.75\textheight]{figs/dos-example.pdf}
    \caption{Example densities of states ($N=25$, $\eta=0.3$)}
  \end{figure}
\end{frame}

\begin{frame}
  \frametitle{Comparison and controls}
  \begin{itemize}
  \item Thermodynamics properties of the $\eta=0.3$ square-well fluid
  \item ``Gold standard'' simulations
  \item ``Converged flat'' weight simulations
  \end{itemize}
\end{frame}

\begin{frame}
  \frametitle{Thermodynamic properties}
  \begin{figure}
    \centering
    \includegraphics[height=0.75\textheight]{figs/internal-energy.pdf}
    \caption{Specific internal energy ($N=25$)}
  \end{figure}
\end{frame}

\begin{frame}
  \frametitle{Errors}
  \begin{figure}
    \centering
    \includegraphics[height=0.75\textheight]{figs/u-err.pdf}
    \caption{Error in specific internal energy ($N=25$)}
  \end{figure}
\end{frame}

\begin{frame}
  \frametitle{Error comparisons}
  \begin{figure}
    \centering
    \includegraphics[height=0.75\textheight]{figs/u-comps.pdf}
    \caption{Error comparisons in specific internal energy}
  \end{figure}
\end{frame}

\begin{frame}
  \frametitle{Summary}
  \begin{itemize}
  \item Wang-Landau best, but requires specified energy range
  \item Simple flat surprisingly decent, but unreliable
  \item OETMMC worse than TMMC
    \begin{itemize}
    \item Worth further investigation
    \end{itemize}
  \item TMMC best bet for square-well fluid
  \end{itemize}
\end{frame}

\begin{frame}
  \frametitle{Acknowledgments}
  \begin{itemize}
  \item David Roundy
  \item Paho Lurie-Gregg
  \item David McIntyre
  \item Henri Jansen
  \end{itemize}
\end{frame}

\end{document}
