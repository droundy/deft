\documentclass[letterpaper,twocolumn,amsmath,amssymb,pre]{revtex4-1}
\usepackage{graphicx}% Include figure files
%\usepackage{breqn}
\usepackage{color}

%% \newcommand{\red}[1]{{\bf \color{red} #1}}
%% \newcommand{\blue}[1]{{\bf \color{blue} #1}}
%% \newcommand{\green}[1]{{\bf \color{green} #1}}
\newcommand{\rr}{\textbf{r}}
\newcommand{\refnote}{\red{[ref]}}

\newcommand{\fixme}[1]{\red{[#1]}}

%\newcommand{\derivation}[1]{#1} % Use this to show all derivations in detail
\newcommand{\derivation}[1]{} % Use this for nice pegagogical paper...

% needsworklater is used to annotate bits that need work, but that we
% can postpone for a while.
\newcommand{\needsworklater}[1]{\emph{[#1]}}
% needsworknow is intended to prioritize stuff that needs fixing.
\newcommand{\needsworknow}[1]{\textcolor{red}{[\emph{#1}]}}

\begin{document}
\title{A Simple Theory for the Melting of Ice}

\author{David Roundy}
\affiliation{Department of Physics, Oregon State University}
%\pacs{61.20.Ne, 61.20.Gy, 61.20.Ja}
%%%%%%%%%%%%%%%%%%%%%%%%%%%%%%%%%%%%%%%%%%%%%%%%%%%%%%%%%%%%
\begin{abstract}
  We develop a theory for ice.
\end{abstract}

\maketitle


%%%%%%%%%%%%%%%%%%%%%%%%%%%%%%%%%%%%%%%%%%%%%%%%%%%%%%%%%%%%
\section{Introduction}

Wertheim's Thermodynamic Pertubation Theory (TPT) provides a formalism
for treating systems with short-range interactions combined with a
hard-core repulsion~\cite{wertheim1984fluidsI, *wertheim1984fluidsII,
  *wertheim1986fluidsIII}.  Various groups have
applied Wertheim's TPT to four-site systems such as water (see
\cite{peery2003association} for references).  Peery and Evans (and
others) applied Wertheim's methods to a four-site model for a
water-like fluid~\cite{peery2003association}.  Alas, all such models
studied to date fail to enforce tetrahedral coordination, instead
allowing the four association sites to float freely on the surface of
the hard sphere.

\subsection{Equations of state for water}

An interesting paper in which they use a single hydrogen bond to
``explain'' water, and claim that tetrahedral coordination is not so
important~\cite{truskett1999single}.

A molecular dynamics paper, in which they  model water using
Stillinger-Weber, and also talk about Si and
diamond\cite{molinero2008water, *hujo2011rise}.

A lattice model for water that makes impressive claims, but is a bit
hard to read at first glance\cite{hakem2007temperature}.  Here is a
second lattice model that also predicts thermodynamic anomalies, and
gives what looks like a good discussion of the low-temperature
properties of water\cite{pretti2004thermodynamic}.

A recent paper that introduces a new model for water to explain
low-temperature behavior of supercooled water\cite{tu2009anomalies}.
Has some nice references and explanation of supercooled behavior.

A key early paper in the ``chemical theory'' of associating liquids,
which is closely aligned to the theory I am
proposing\cite{heidemann1976van}.

An excellent review article, which outlines and explains the primary
approaches to modelling associating fluids, in particular giving
background on non-SAFT approaches such as the ``chemical
theory''\cite{muller2001molecular}.

\section{Theory}

\subsection{A look at dimers}

We will begin by outlining some aspects Wertheim's Thermodynamic
Pertubation Theory (TPT)~\cite{wertheim1984fluidsI,
  *wertheim1984fluidsII, *wertheim1986fluidsIII},
with a view towards extending it towards tetrahedrally coordinated
molecules such as water.

\newcommand\nI{\ensuremath{{n^{(1)}}}}
\newcommand\nII{\ensuremath{{n^{(2)}}}}

\newcommand\PI{\ensuremath{{P^{(1)}}}}
\newcommand\PII{\ensuremath{{P^{(2)}}}}

The idea of Wertheim's is to construct multiple densities for
different bonding states of molecules.  One then constructs
self-consistent equations which are solved for those densities, which
then appear in the free energy itself.  In Wertheim's approach, the
density is a ``bonding-state'' density.  We will use a related
approach in which we consider ``cluster'' densities.  Here, we will
describe Wertheim's approach with our ``cluster'' language, which will
be simplest to describe in the case where there is a single
bonding site on each molecule, so that at most dimers may be formed.
In this case, we can envision the fluid as composed of two densities,
a density of unbonded monomers $\nI$ and a density of dimers $\nII$.
Together they make up the total density:
\begin{align}
  n = \nI + 2\nII
\end{align}
We find the self-consistent equation for $\nII$ by considering the
relative probability of finding a dimer, versus finding two monomers
that are touching in a volume equal to the bonding volume, but in a
particular wrong orientation.  This ratio is given by a Boltzmann
factor
\begin{align}
  \frac{\PII}{\PI_{\textit{touching}}} &= e^{\beta\epsilon}
\end{align}
where $\epsilon$ is the energy difference between the two
configurations, which by construction have the same entropy.  The
probability of two spheres touching within a volume $\kappa$ is
determined by the monomer-monomer distribution function
\begin{align}
  \PI_{\textit{touching}} &= \PI \nI\kappa g_{11}(\sigma)
\end{align}
Thus, with a little algebra, we find
\begin{align}
  \PII &= \PI \nI\kappa g_{11}(\sigma) e^{\beta\epsilon} \\
  \nII &= \nI^2\kappa g_{11}(\sigma) e^{\beta\epsilon} \\
  n - \nI &= \nI^2\kappa g_{11}(\sigma) e^{\beta\epsilon} \\
  \nI\left(1 + \nI\kappa g_{11}(\sigma) e^{\beta\epsilon}\right) &= n \\
  \nI &= \frac{n}{\left(1 + \nI\kappa g_{11}(\sigma) e^{\beta\epsilon}\right)} \\
  \PI &= \frac{1}{\left(1 + \nI\kappa g_{11}(\sigma) e^{\beta\epsilon}\right)}
\end{align}
This quantity is normally called $X$ in the SAFT literature.  Once we
have these densities (and related probabilities), we are most of the
way towards finding the free energy and equation of state.

\subsection{Energies of clumps}

We will treat tetrahedral coordination by examining larger clusters
than just two, and taking into account the geometry of coordination.
There are different conformations, but we will begin by focusing on
just the lowest energy conformations, which are roughly spherical.
Later, we will consider the highest energy conformations, which are
linear chains.  The energy is equal to $\epsilon B_s$, where $B_s$ is
the number of bonds in a cluster of $s$ molecules.  We will
approximate $B_s$ by assuming that molecules within a specified
distance of the surface of the sphere have no bonds apart from those
to molecules on the interior, while the molecules on the interior have
four bonds each, keeping in mind that each bond is shared by two
molecules.
\begin{align}
  B_s &= 2(s^{\frac13} - 0.45)^3
\end{align}
The constant $0.45$ in this formula is reached by fitting the
thickness of the ``exterior'' in order to produce approximately
correct results for small values of $s$, as shown in
Table~\ref{tab:Bs}.
\begin{table}
  \begin{tabular}{c|cc}
    $s$ & $B_s$ & true $B_s$ \\
    \hline
    1 & 0.33 & 0 \\
    2 & 1.06 & 1 \\
    3 & 1.95 & 2 \\
    4 & 2.94 & 3 \\
    5 & 4.00 & 4 \\
    6 & 5.11 & 5 \\
    7 & 6.26 & 6 \\
    8 & 7.45 & 7 \\
  \end{tabular}
  \caption{Number of bonds $B_s$ as a function of cluster
    size.}\label{tab:Bs}
\end{table}

\section{Theory, take II}

I have another approach that sounds interesting, with a different way
of describing clusters.  The idea is to describe a cluster (as defined
aboce) in terms of connected \emph{rigid clusters}.  A rigid cluster
is a set of molecules any two of which are connected by more than one
bond.  A rigid cluster should be unable to be rotated at all, beyond
the amount of flex that comes about because each hydrogen bond has a
certain volume of give.  The smallest possible rigid cluster will be
composed of a single molecule, and I believe the next largest will
have six molecules in it.

The rigid clusters that make up a given cluster will be connected by a
single hydrogen bond.  Since this bond rotates freely, it will have a
certain amount of rotational entropy.  Moreover, there can be many
rigid clusters in a particular cluster, and there may be many ways of
putting together these rigid clusters (which could really do with a
better name).

This gives us a way to characterize clusters under several
parameters.  Hopefully only a smallish number of cluster formations
will be common, so we don't need an excessive number of terms in our
summations.

The main idea is that we have two problems to solve in order to
understand clusters: the rigid cluster problem (relatively easy to
approximate, I hope), and how to combine rigid clusters together to
get flexible clusters.  The rigid clusters will have very little
entropy, and quite a high energy (relatively speaking, not counting
the single-molecule rigid cluster), while connecting them together
will give a looser higher-entropy configuration.  Once we understand
the energy and entropy of each cluster configuration, we are most of
the way towards an equation of state for the tetrahedrally-coordinated
fluid.

The missing term will be the interaction between different clusters.
This interaction is purely entropic (since the only energy term shows
up within clusters, by their definition), and we will need a plausible
way to approximate it.  A start for this is to work out the volume of
each cluster.  Then we could use scaled-particle theory to handle
their interactions, assuming each cluster is a sphere, which seems
like a reasonable starting point.  It does require that we come up
with a reasonable way of defining the volume of a cluster.  We would
also need to define their area and mean curvature, if we wanted to go
beyond assuming these things are spheres.

\subsection{Free energy of a rigid cluster}

We can find the entropy of a given cluster configuration pretty easily
by computing its free energy.  The process is as follows.  The free
energy can be written as
\newcommand\dr{d^3r}
\begin{align}
  F &= -k_BT \ln Z \\
  Z &=
  \frac{1}{N!}
  \left(\frac{2m}{\beta\hbar^2}\right)^N
  \left(\frac{2I}{\beta\hbar^2}\right)^N
  \int \dr_1d\Omega_1\dr_2d\Omega_2\cdots e^{-\beta
    \Phi(\rr_1,\Omega_1,\cdots)}
\end{align}
where $\Omega_i$ is the orientation of molecule $i$, and $\Phi$ is the
potential energy.  If the molecules had internal degrees of freedom,
we would need to integrate over those as well.

\subsubsection{Rotational and motional free energy}

We consider a single cluster by setting $N$ to the number of molecules
in the cluster, and enforcing a constraint that they be attached in
the desired topology.  We can envision this as a potential that is
infinite when the cluster is broken.

The first integral $\dr_1$ is unconstrained, and corresponds to the
translational motion of the cluster.  For a rigid cluster, each of the
remaining integrals is constrained to a smallish volume.  This volume
is related to, but not identical to, the ``mutual volume of
interaction'' $\kappa$, which determines the probability at low
density of a molecule being bound on a particular site:
\begin{align}
  P &= \kappa n
\end{align}
The volume in the integrals above may be smaller than this, because
the location of a given molecule is constrained by that of all the
molecules it is bound to.

Anyhow, the final result is that we get something that looks
approximately like (leaving out the $N!$, which we'll include when we
combine together a bunch of clusters:
\begin{align}
  Z &\approx
  \left(\frac{2m}{\beta\hbar^2}\right)^{3N}
  \left(\frac{2I}{\beta\hbar^2}\right)^{3N}
  (V 8\pi^2) (\delta V \delta \Omega)^{N-1}
  \\
  &= \frac{V}{\Lambda^3}\frac{8\pi^2}{\Xi}
  \left(\frac{\delta V \delta \Omega}{\Lambda^3\Xi}\right)^{N-1}
\end{align}
where I have defined $\Xi \equiv
\left(\frac{2I}{\beta\hbar^2}\right)^3$ as an $SO(3)$ ``themal solid
solid angle''\footnote{I've been very sloppy here, and this $\Xi$
  needs to be carefully derived.}, which defines the angular
constriction needed to reduce the entropy of orientation to zero at a
given temperature.  This partition function does not include the
energy of interaction, so we can easily relate it to the entropy of
the cluster:
\begin{align}
  F_N &= -k_BT \ln Z_N \\
  &= -k_BT \ln\left(
     \frac{V}{\Lambda^3}\frac{8\pi^2}{\Xi}
     \left(\frac{\delta V \delta \Omega}{\Lambda^3\Xi}\right)^{N-1}
  \right)
  \\
  &= -k_BT \left(
    \ln\left(
     \frac{V}{\Lambda^3}\frac{8\pi^2}{\Xi}
    \right)
    +
    (N-1)
    \ln\left(
     \frac{\delta V \delta \Omega}{\Lambda^3\Xi}
    \right)
  \right)
\end{align}
where we can see that the first log is the ideal gas free energy
corresponding to center-of-mass motion and rotation of the cluster,
and the second is the free energy of the cluster itself.  We have made
a simplifying approximation with regard to the amount of volume
available for each molecule added to the cluster, but this seems like
a reasonably small approximation (which we can also test).

I again note here that I have omitted the fixed energy of bonding from
this free energy.

\subsubsection{Configurational free energy}
To the above free energy, we will need to add the free energy from
configurational entropy, corresponding to the number of ways we can
arrange the molecules with a given number of bonds (and thus bonding
energy) and a given number of molecules.

\subsection{Free energy of a complete connected cluster}

We will now consider a complete connected cluster, which is composed
of a number of connected rigid clusters, which could themselves be
composed of a single molecule.  There are bascially two sources of
entropy for such a cluster.  The first is the configurational entropy,
which describes the number of distinct ways we could put together such
a cluster.  The second is the motional and rotational entropy
corresponding to the wiggle room the rigid clusters have, and their
ability to rotate around the bonds connecting them.  I note that there
is an interaction between the two contributions, since the
configuration (as in, which cluster is bonded to which other cluster)
will affect how much wiggle room there is, since there could be some
steric hindrance (i.e. things could bump).  As a first approximation,
we will neglect interactions between the two contributions, and assume
that if there is no steric hindrance preventing a given configuration
from forming, then there will be no steric hindrance preventing each
bond from rotating freely.

\subsubsection{Rotational and motional entropy}

I will begin with the rotational and motional entropy, since it is
more closely related to the previously computed entropy (which is
purely motional) of the rigid cluster.  We will imagine building up a
complete cluster by connecting together rigid clusters one at a time.
Each time a cluster is connected, it loses most of its ``center of mass and
rotation'' degrees of freedom.  Its motion will be restricted to
essentially the same volume $\delta V$ that the molecules within a
cluster are restricted to.  The second cluster, however, may freely
rotate about the bond between the two clusters, which means that it
will have considerably greater rotational entropy than is present in
the case of a rigid cluster.  We just need to figure out the amount of
``solid solid angle'' that is allowed, which will replace the $\delta
\Omega$ in our formula.  The final result should look something like
\begin{widetext}
\begin{align}
  F &= -k_BT\left(
    \ln\left(
     \frac{V}{\Lambda^3}\frac{8\pi^2}{\Xi}
    \right)
    +
    (N_1-1)
    \ln\left(
     \frac{\delta V \delta \Omega}{\Lambda^3\Xi}
    \right)
    +
    \sum_{i=2}^{M} \left(
        \ln\left(\frac{\delta V
          \delta\Omega_{\textit{rot}}}{\Lambda^3\Xi}\right)
        +
        (N_i-1)
        \ln\left(
         \frac{\delta V \delta \Omega}{\Lambda^3\Xi}
        \right)
    \right)
  \right)
\end{align}
\end{widetext}
where it is immediately clear that there is considerable
simplification to be performed.  I have introduced a new variable
$\delta\Omega_{\textit{rot}}$, which is the amount of ``solid solid
angle'' that is available to a rigid cluster that is rotating freely
about a bond.  When we do the obvious simplification ($N = \sum N_i$),
we get something that looks like:
\begin{widetext}
\begin{align}
  F &= -k_BT\left(
    \ln\left(
     \frac{V}{\Lambda^3}\frac{8\pi^2}{\Xi}
    \right)
    +
    (M-1)
     \ln\left(\frac{\delta V
       \delta\Omega_{\textit{rot}}}{\Lambda^3\Xi}\right)
    +
    (N-M-1)
    \ln\left(
      \frac{\delta V \delta \Omega}{\Lambda^3\Xi}
    \right)
  \right)
\end{align}
\end{widetext}
which itself could be simplified a bit more, in that the motional and
rotational bits could be treated separately.  Anyhow, it is
interesting that when we ignore steric hindrances, the entropy of a
connected cluster is independent of how it is connected, and in fact
only depends on the total number of molecules $N$ and the total number
of rigid clusters $M$.

\subsubsection{Configurational entropy}

The configurational multiplicity will always have a factor of $N!$
that will come about because of the various permutations that could
describe the same configuration.  We won't discuss this $N!$ below,
but must include it in our final answer.

When looking at the configuration entropy, we cannot avoid considering
steric hindrances, since they prevent many potential configurations
from forming.  One effect of this steric hindrance is that the number
of molecules in a cluster $N$ places a constraint on the number of
rigid sub-clusters in that cluster $M$.For instance, if $N = 6$, we
can conclude that it must be made of either one rigid cluster (of 6)
or 6 rigid clusters (each composed of a single molecule).
Unfortunately, the constraint isn't very limiting, since (almost?) any
given cluster size (greater than $N=5$) could always be made of either
a single rigid cluster (e.g. a loop) or a maximum of $M=N$
single-molecule rigid clusters, arranged in an open-ended chain.

Let's consider the linear chain clusters.  In this case $M=N$, and
this is a very high entropy configuration, primarily because of its
maximum value of rotational entropy.  However, the linear chain has no
configurational entropy.  But there is also a whole family of branched
chain clusters, still composed of single-molecule rigid clusters, so
we can view the $M=N$ family as having a resultant configurational
entropy, which we will estimate here.  When we add a molecule to a
branched chain made of $N-1$ molecules, we have roughly $2N$ places
we could put it.  Thus
\begin{align}
  g_N &\approx 2N g_{N-1} \\
  \ln g_N &\approx \ln g_{N-1} + \ln 2 + \ln N \\
  \ln g_N &\approx N\ln 2 + N\ln N - N \\
  S_N &\approx k_B\left(N\ln N - (1-\ln 2)N \right)
\end{align}
I note that this is a very crude approximation.  It ignores steric
constraints, which are admittedly smaller for more flexible
configurations.  In the limit of large $N$, the assumption that the
number of binding sites is proportional to $N$ is wrong.  In this
limit there are guaranteed to be major steric effects, since the
random walk of even a linear chain is likely to overlap with itself.
Since the mean radius of a random walk in three dimensions scales as
$\sqrt{N}$, the volume occupied by most of the sites in the random
walk scales as $N^{3/2}$.

As it turns out, there is a large literature on self-avoiding random
walks.  According to one paper studying a 3D cubic
lattice\cite{grassberger1993monte},
\begin{align}
  g_N^{\text{cubic}} &= 4.68^N N^{0.16}
\end{align}
Actually, it looks like for a diamond lattice it should
be\cite{chen2002universal}
\begin{align}
  g_N^{\text{cubic}} &= 1.24 2.88^N N^{0.16}
\end{align}
which gives us a slightly reasonable starting point.  This is the
degeneracy of a linear chain with length $N$.  The total degeneracy
including branching configurations will of course be smaller.  The
radius of gyration is given by
\begin{align}
  G_N &= C N^{2\nu} \\
  &= 0.48 N^{0.59}
\end{align}
from which you can see that self-avoidance has some effect, but not a
dramatic effect.  I'm thinking that we might like to have a
differently defined self-avoiding random walk, in which the path can't
come adjascent to another point on the path, which would keep the
cluster ``flexible'' to a greater degree.

\section{Various other notes to self}

TODO: look at paper from segura et al 1987? for MD and interesting
results on just the 4-site associating hard spheres.

\bibliography{paper}% Produces the bibliography via BibTeX.

\end{document}
