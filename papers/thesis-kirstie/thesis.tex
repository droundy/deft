\documentclass[double,12pt]{beavtex}
\usepackage{graphicx}
\usepackage{rotating} %Package added to allow the rotation of figures and chart on a page, {sidewaysfigure} command
\usepackage{tablefootnote} %Packaged added to allow footnotes in the tabular environment, use \tablefootnote command
\title{Freezing of a Weeks-Chandler-Anderson Fluid using classical Density Functional Theory}
\author{Kirstie L. Finster}
\degree{Master of Science}
\doctype{Thesis}
\department{Physics}
\depttype{Department}
\depthead{Head}
\major{Physics}
\advisor{David Roundy}
\submitdate{December 10, 2020}
\commencementyear{2021}

\newcommand\davidsays[1]{\textcolor{red}{[\it D: #1]}}

\usepackage{color}
\usepackage{amsmath}
\usepackage{subcaption}

\date{December 2020}

\abstract{
  We use Classical Density Functional Theory (cDFT) to predict the 
  freezing of a Weeks-Chandler-Anderson (WCA) fluid, and construct
  pressure-temperature and temperature-density phase diagrams. 
  A WCA fluid consists of soft spheres that exhibit the repulsive 
  portion of the Lennard-Jones potential widely used to model the 
  interaction of two neutral atoms. Our cDFT functional is constructed 
  using Soft Fundamental Theory (SFMT) with Schmidt's error function 
  model in which we have incorporated temperature dependent parameters 
  to make the second virial coefficient correct. Our functional was 
  successful in predicting the freezing of the WCA fluid with a fairly 
  good match to Monte-Carlo simulations. This result suggests that SFMT 
  may provide a strong foundation for future density functionals that 
  will model real fluids.
}

\begin{document}
   \maketitle
   \mainmatter

\chapter{Introduction}

The goal of this project is to generate temperature-density 
and pressure-temperature phase diagrams for a Weeks-Chandler-Anderson 
(WCA) model fluid using classical Density Functional Theory (cDFT). 
To do this, the Helmholtz free energies of the fluid
in a liquid state and in a crystalline solid state must be found.
In general, the Helmholtz free energy, which consists of an ideal part 
and an excess part, can be written as a functional of the number density 
profile $n(\vec{r})$ which describes the way the average number 
density varies spatially. 
\begin{equation}{F[n(\vec{r})]=F_{ideal}[n(\vec{r})] + F_{excess}[n(\vec{r})]}\end{equation} 

Soft Fundamental Measure Theory (SFMT) will be used to form the 
functional for the excess Helmholtz free energy. 
SFMT is a method that makes the task of 
forming an expression for the excess Helmholtz free energy of a particular 
model fluid much easier since the interaction that each sphere, or atom, 
has with every other sphere does not have to be taken explicitly into account.
Once the functional for the Helmholtz free energy is formed, then 
according to cDFT, the equilibrium
Helmholtz free energy of the system can be found for a particular 
temperature and density by varying the number 
density profile $n(\vec{r})$ until the free energy is minimized. 
Phase diagrams can then be constructed using Maxwell's Double Tangent Construction. 
Results will be compared to those from Monte-Carlo simulations.

\chapter{Theory}

\section{Model Fluids and the Weeks-Chandler-Anderson Fluid (WCA)}

Real fluids and their behavior are very complex from a physicist's point 
of view. Although a water molecule, for example, is composed of only three
atoms, the ever-changing redistribution of charges within the molecules, 
and their interaction with charges from other molecules make it difficult 
to study a fluid composed of water molecules directly. The task becomes 
even more daunting considering the complicated molecular motion of each 
water molecule, and the sheer number of molecules that must be taken 
into account. 
The problem of understanding the behavior of real liquids is better addressed 
by using models which reduce a complex picture to a simple one. Modifications, 
or perturbations, can then be incorporated into the model making it 
progressively more complex until it mimics the behavior of a real 
liquid as closely as possible.

In developing a model for a fluid, simplifications can be made regarding 
the composition, type, shape, and interatomic interaction of the molecules. 
Rather than dealing with any particular molecule or atom, the model only 
deals with generic ``particles", which may also be referred to as ``atoms''. 
The shape of the particles, or atoms will be the simplest shape possible - 
a sphere. The spheres in the simplest models do not exhibit quantum 
interference effects as the mass of the spheres, and the temperature 
are taken to be large enough that the de Broglie wavelength is smaller 
than the mean distance between the spheres. The fluid can then be treated 
as a classical fluid where the spheres interact with each other according 
to their electrostatic interatomic potential. 

There are many interatomic potentials which can be used to model a fluid 
each of which describes the interaction between two real atoms to some 
degree of accuracy. The simplest, and oldest, model used to describe a 
fluid is the hard-sphere model. The hard-sphere model envisions a fluid 
to be composed of hard spheres like, for example, a collection of marbles. 
This model came about by considering the near incompressibility of liquids 
indicating that the particles making up the liquid strongly resist being 
pushed closer together. This behavior can be described by an interatomic 
potential energy that is zero when two spheres are not in contact, but 
goes instantly to infinity when contact is made, as shown in 
Figure~\ref{fig:HardSphere_potential}. The inability to push 
real atoms closer together is due to the repulsion of the electrons in 
each atom, and is the dominating interparticle electrostatic interaction. 
Thus, the hard-sphere model - which is the simplest, but still effective 
representation of this repulsive behavior that dominates the interaction 
between real atoms, is currently the most widely used model for fluids.

\begin{figure}
    \centering
    \includegraphics[height=7cm]{plot_HardSphere_Potential.pdf}
    \caption{A hard-sphere potential is zero for $r>2R$, and infinity 
    for $r<{2R}$ where R is the radius of a hard sphere 
    (R$=\frac{r}{2}\approx$1.22/2=0.61 in this example).}
    \label{fig:HardSphere_potential}
  \end{figure}

Another fairly simple, but relatively accurate interatomic potential widely 
used to represent generic, neutral atoms is the Lennard-Jones potential. 
The Lennard-Jones potential describes both the long-range attraction, 
and the short-range repulsion that occurs between two neutral atoms. 
The attraction between two neutral atoms is due to fleeting dipoles that 
form spontaneously when electrons are redistributed as atoms come into 
close proximity. These are often characterized as Van der Walls, or London 
dispersion forces. The attraction grows stronger as two atoms move closer 
together until their atomic orbitals begin to overlap. 
At this point the electrons in each atom begin to oppose the overlap of 
the two atoms through their repulsion. The repulsion increases rapidly and 
approaches infinity as the distance between the atoms continues to decrease.  
 
\begin{figure}
    \centering
    \includegraphics[height=7cm]{LJ_Potential.pdf}
    \caption{Lennard-Jones potential for $\sigma=1$, $\epsilon=1$}
    \label{fig:LJ_potential}
  \end{figure}

The combined attractive and repulsive effects produce the overall shape of 
the Lennard-Jones interatomic potential shown in Figure \ref{fig:LJ_potential}. 
The potential is zero when the distance r between the centers of two atoms 
is equal to $\sigma$. At this point 
the attraction between the atoms just cancels their repulsion. As the distance 
between the centers of the two atoms decreases below $\sigma$, the potential 
rapidly approaches positive infinity corresponding to infinite repulsion. 
Above $\sigma$ the potential becomes negative indicating a net attraction 
between the atoms. The net attraction is maximized at a distance of 
r=$2^\frac{1}{6}\sigma=2R\approx{1.122}\sigma$ where R is the radius of 
the sphere given by $R={2^{-\frac{5}{6}}}\sigma$. At this point the depth of
the potential well, given by epsilon $\epsilon$, is a maximum. As r increases 
further, the net attraction decreases and the potential approaches zero
asymptotically. The attractive and repulsive effects can be separated into two
potentials, shown with different colors in Figure \ref{fig:LJ_potential_2parts}, 
which when added together regenerate the full Lennard-Jones potential. 

\begin{figure}
    \centering
    \includegraphics[height=7cm]{LJ_Potential_2part.pdf}
    \caption{Lennard-Jones potential broken into two parts for $\sigma=1$, $\epsilon=1$}
    \label{fig:LJ_potential_2parts}
  \end{figure}

\begin{figure}
    \centering
    \includegraphics[height=7cm]{WCA_Potential.pdf}
    \caption{WCA potential for $\sigma=1$, $\epsilon=1$}
    \label{fig:WCA_potential}
  \end{figure}

The repulsive portion of the Lennard-Jones potential forms the potential for a 
Weeks-Chandler-Anderson (WCA) model fluid~\cite{andersen1971relationship},
shown in Figure \ref{fig:WCA_potential}. 
The WCA potential is described by the equation 
\begin{align} \label{eq:VWCA}
    V_{WCA}=\left\{\begin{array}{rcl} {4\epsilon{\left[\left(\frac{\sigma}{r}\right)^{12} - \left(\frac{\sigma}{r}\right)^6 \right]}+\epsilon} & \mbox{for} & 0<r<{2R} \\ 0 & \mbox{for} & r>2R \end{array}\right.
\end{align} 
The WCA potential decreases from infinity as the distance between the 
centers of the atoms r increases, and becomes, and remains, essentially 
zero when r=$2^\frac{1}{6}\sigma=2R\approx{1.122}\sigma$ where R is the 
radius of the sphere. 
As with the Lenard-Jones potential, the repulsive WCA potential increases 
rapidly and approaches infinity as the two spheres begin to overlap. This 
is a more realistic repulsive potential than that of the hard-sphere model 
which goes instantly to infinity when the two spheres make contact (at r=2R), 
and is zero otherwise. Because the WCA spheres have no hard boundary in 
their interatomic potential, they are called ``soft spheres'' and their 
gradually fading potential makes them slightly ``squishy'', as illustrated in
Figure~\ref{fig:TwoSpheres}.

\begin{figure}
    \centering
    \includegraphics[height=3.5cm]{figs/TwoSpheres.pdf} 
    \caption{(Left) The WCA potential is a minimum at r=2R. 
             (Right) The WCA potential increases rapidly for $r<2R$.}
    \label{fig:TwoSpheres}
\end{figure} 
    
\section{Freezing and Maxwell's Double Tangent Construction}

Freezing is governed by entropy, which the universe, or any closed system, 
seeks to maximize according to the second law of Thermodynamics. When the 
entropy of a pure substance in its liquid state is lower than the entropy 
that it would have in its solid state at the same temperature (necessary 
for thermal equilibrium), the same pressure (necessary for mechanical 
equilibrium), and the same chemical potential (necessary for diffusive 
equilibrium), an abrupt phase transition occurs, that is, the liquid 
freezes. The particular structure and density that result are affected 
by the interatomic potential. When repulsive, soft sphere fluids 
(that are not too soft) freeze, they form lattice structures composed of 
Face Centered Cubic (FCC) crystal cells~\cite{Hansen}, as do hard spheres. 
This structure is what one would see by filling a box with marbles. It 
results in the highest density achievable for hard, and nearly-hard, spheres 
that are purely repulsive. 

\begin{figure}
    \centering
    \includegraphics[height=6cm]{figs/P-T_Diagram.pdf}
    \caption{Conceptual Pressure-Temperature Phase Diagram}
    \label{fig:P-T_Diagram}
\end{figure}

For pure substances, a phase transition occurs along a line on a 
Pressure-Temperature (P-T) phase diagram, such as can be seen in the phase 
diagram shown in Figure~\ref{fig:P-T_Diagram}. This phase diagram is 
representative of some arbitrary, unknown substance composed of atoms, 
or molecules which exhibit both attraction and repulsion. However, since 
the atoms in a WCA fluid do not exhibit attraction there will not be a 
dividing line between the gas and liquid phases, there will just simply
be a fluid, which we will refer to as a liquid in this paper.
%which may be referred to as a "liquid", since at the phase boundary the fluid is dense. 
Points along the line that separates the solid and the liquid phases 
indicate a phase transition where the solid and liquid phases coexist, 
and show that both the solid and the liquid have the same temperature 
and the same pressure at a phase transition.

The pressure can be related to the Helmholtz free energy F, which is minimized 
for a system at fixed temperature, volume and number of particles, as the 
entropy of the system and its surroundings is maximized. The negative of 
the partial derivative of the Helmholtz free energy with respect to volume V, 
and at a fixed temperature T and number of particles N, equals the pressure. 
\begin{equation}{P=-\frac{\partial{F}}{\partial{V}}\bigg|_{T,N}}\end{equation}
\noindent Dividing both the Helmholtz free energy and the volume by the 
number of particles, the partial derivative of the Helmholtz free energy 
with respect to volume becomes the partial derivative of the Helmholtz 
free energy per atom, $f$,  with respect to the inverse number density. 
\begin{equation}{P=-\frac{\partial{\frac{F}{N}}}{\partial{\frac{V}{N}}}\bigg|_{T,N} = -\frac{\partial{f}}{\partial{\frac{1}{n}}}\bigg|_{T,N}}\end{equation} 
Thus, the pressure corresponds to the negative of the slope of the curve 
on a plot of the Helmholtz free energy per atom versus the inverse 
density at fixed temperature. 
 
 \begin{figure}
    \centering
     \includegraphics[height=8cm]{figs/MaxwellDTC-Fig1.pdf}
    \caption{Liquid and solid densities at the solid-liquid phase 
    transition are identified using Maxwell's Double Tangent Construction}
    \label{fig:MaxwellDT}
  \end{figure}

Two such curves are shown in Figure~\ref{fig:MaxwellDT}, one for the liquid 
and one for the solid at a fixed temperature. Each curve consists of a 
collection of data points $(\frac{F}{N}, \frac{1}{n})$ for the liquid and
the solid respectively, which will be obtained by using cDFT to find the 
Helmholtz free energies of the liquid and the solid for various 
temperatures and number densities. 
Since the pressure of the 
liquid is the same as the pressure of the solid at the point of transition, 
the density of the liquid and the density of the solid at the point of 
freezing can be found from identifying the points at which the 
tangents to each curve have the same slope, as shown by the black line. 
This process is called Maxwell's Double Tangent Construction. 

To facilitate finding the pressure at the phase transition computationally, 
it is helpful to consider the Gibbs free energy given by G=F+PV. 
Rearranging the terms and dividing by N gives an  expression for the 
Helmholtz free energy per atom $f$ in terms of the Gibbs free energy 
per atom  $g$ (which is also the chemical potential)  
\begin{align}\frac{F}{N} = -P\frac{V}{N}+\frac{G}{N} \end{align}
\begin{align}f=-P\frac{1}{n}+g\end{align}
This equation has the form of a line
\begin{equation}
    y=mx+y_o
\end{equation}
and describes the black line shown in Figure~\ref{fig:MaxwellDT} with the 
pressure given by the negative of the slope of the line, and the Gibb's 
free energy per atom, or chemical potential given by the y-intercept. 
At the point of phase transition the chemical potential of the liquid is 
the same as the chemical potential of the solid, as is the pressure.
\begin{equation}
    \mu_L=\mu_S\mbox{~~~~and~~~~}P_L=P_S\mbox{~~~~at a phase transition}
\end{equation} 

 \begin{figure}
    \centering
     \includegraphics[height=8cm]{figs/MaxwellDTC-Fig2.pdf}
    \caption{Pressure is found at point of intersection which occurs 
    when $\mu_L=\mu_S$.}
    \label{fig:GibbsvsP}
  \end{figure}
  
\begin{figure}
    \centering
    \includegraphics[height=8cm]{figs/MaxwellDTC-Fig3.pdf}
    \caption{The liquid and solid densities at the phase transition are 
    found from a plot of pressure vs inverse density for both the liquid and the solid.}
    \label{fig:Pvsinvn}
  \end{figure}

The pressure at the phase transition can be obtained from the point of 
intersection that occurs when the chemical potential 
is plotted against the pressure for both the liquid and the 
solid at fixed temperature, as shown in Figure~\ref{fig:GibbsvsP}. 
This plot is constructed from data collected from lines tangent 
along the curves on the Helmholtz free energy per atom verses inverse 
density plots (like that of Figure~\ref{fig:MaxwellDT}) for both the 
liquid and the solid at a given temperature. 
The slope of the tangent line at a given point gives the negative of the 
pressure at that point, and the corresponding y-intercept of the tangent 
line gives the chemical potential at that point. Collecting this data 
(P,~$\mu$) at each point for the liquid and the solid respectively, 
produces the curves in Figure~\ref{fig:GibbsvsP}. 

\begin{figure}
    \centering
    \includegraphics[height=6cm]{figs/T-n_Diagram.pdf}
    \caption{Conceptual Temperature-Density Phase Diagram used to identify 
    the number densities of the liquid and the solid during a phase 
    transition. The gray region is where the liquid and the solid coexist, 
    and are therefore not pure states.}
    \label{fig:T-n_Diagram}
  \end{figure} 
  
Once the pressure is known at the point of phase transition, for a given 
temperature and number density, the liquid and solid number densities at 
the phase transition can be found from plots of the pressure versus the 
inverse of the number density for both the solid and the liquid, 
as shown in Figure \ref{fig:Pvsinvn}. By repeating this procedure to 
find the liquid and solid number densities at the phase transition 
for various temperatures, a Temperature-Density (T-n) phase diagram 
can be generated, similar to that shown in Figure \ref{fig:T-n_Diagram}, 
but without the gray hump that is characteristic of fluids that have 
both attractive and repulsive interatomic potentials. 

\section{Classical Density Functional Theory (cDFT)}

Classical Density Functional Theory will be used to find the 
Helmholtz free energies of the liquid and the solid for various 
temperatures and number densities, so that phase diagrams can be 
constructed using Maxwell's Double Tangent Construction. 
The key idea behind classical Density Functional Theory (cDFT) is that 
free energy can be written as a functional (a function of a function) of 
a number density profile $n(\vec{r})$ which describes the way the number 
density, or atoms per volume, varies spatially on average. 
\begin{displaymath}{f[n(}\vec{r}{)]}\end{displaymath}
This, coupled with the fact that the free energy of a system tends toward 
a minimum at equilibrium means that the equilibrium free energy of the 
system ${f_{eq}}{[}\rho{(}\vec{r}{)]}$, and its corresponding equilibrium 
number density profile $n_{eq}(\vec r)$, can be found simply by varying 
$n(\vec{r})$ through all possible spatial profiles until the free energy 
is minimized~\cite{MoritaDFT}. 
\begin{displaymath}f[n(\vec r)]\rightarrow f_{eq}[n_{eq}(\vec r)]  \mbox{ as $f$ is minimized} \end{displaymath}
\begin{displaymath}{\mbox{where }  n(\vec{r})=n_{eq}(\vec r)  \mbox{ at equilibrium}}\end{displaymath}
The reason the minimization works is because DFT satisfies the conditions 
of the variational principle. 
%the same mathematical principle used to determine 
%that the shortest distance between two points is a straight line by 
%varying the path between the two points until the distance is minimized.

A description of cDFT often begins with considering the Grand Free 
Energy defined as
\begin{equation}\Phi=F-\mu{N}\end{equation}
which can then be expressed in terms of a varying number density 
profile $n(\vec r)$
\begin{equation}\label{GrandFE}\Phi= F_{int} +\int n(\vec{r})\phi{(\vec r)}d\vec{r}-\mu\int n(\vec r)d\vec{r}\end{equation}
where $\phi=\frac{V_{ext}}{N}$ is an externally applied potential energy 
$V_{ext}(\vec r)$ per particle, and an intrinsic Helmholtz free energy 
is defined as
\begin{equation}F_{int} = F - \int n(\vec{r})\phi{(\vec r)}d\vec{r}\end{equation}  

The Grand free energy is applicable to systems with a varying number of 
particles and fixed chemical potential, temperature and volume. But in 
this paper a system with a fixed temperature, volume, and number of 
particles will be of interest in which case
\begin{equation}\int n(\vec r)d\vec{r}=N\end{equation}
and so the Grand free energy is constrained
\begin{equation}\label{GrandFE}\Phi_{constrained}= F[n(\vec r)]-\mu N\end{equation}
The dependence on $n(\vec r)$ is now only present in the expression for 
the Helmholtz free energy, and so it is the Helmholtz energy as a functional 
of the number density profile $F[n(\vec r)]$ that is minimized at equilibrium. 
More specifically, in this paper, the Helmholtz free energy per particle, 
or atom,  $f_N=\frac{F}{N}$ will be minimized which is given by
\begin{equation}f_N=\frac{U}{N}-\frac{TS}{N}\end{equation}
with differential
\begin{equation}\label{usetoshowmin}df_N=d\left(\frac{U}{N}\right)-d\left(\frac{TS}{N}\right)\end{equation}
\begin{equation}df_N=-\frac{S}{N}dT-\frac{P}{N}dV-PVd\left(\frac{1}{N}\right)\end{equation}
This result shows that the natural variables of the Helmholtz free energy 
per particle $f_N$ are temperature, volume and one over the number of 
particles, which in turn also holds the number density $n=\frac{V}{N}$ fixed. 
\begin{equation}f_N(T,V, \frac{1}{N})\end{equation}

The minimization can be illustrated for a system near 
equilibrium by considering that according to the 
second law of thermodynamics, a system that is not in equilibrium will 
experience spontaneous processes that increase the total entropy of a 
system and its surroundings until equilibrium is reached, 
$\Delta{S}_{System + Surroundings} \geq 0$. Here ``surroundings" means 
the entire universe minus the system. This, together with the conservation 
of energy (the first law of thermodynamics) which says that energy lost 
(or gained) by the system is gained (or lost) by the surroundings 
$\Delta{U}_{System}=-\Delta{U}_{Surroundings}$, gives the result that 
changes in the free energy 
will be negative for spontaneous processes 
when the natural variables of the free energy are held fixed. 
For our case, with a fixed temperature, volume, and number of particles
we get
%Rewriting Eq~\ref{usetoshowmin} to show the system terms explicitly
%\begin{equation}
  %df_{System}=d\left(\frac{U}{N}\right)_{System}-Td\left(\frac{S}{N}\right)_{System}+\left(\frac{S}{N}\right)dT_{System}
%\end{equation}
%and holding the temperature fixed ($T_{System}=T_{Surroundings}$) so that 
%$dT=0$ gives
%\begin{equation}
  %f_{System}=-d\left(\frac{U}{N}\right)_{Surroundings}-Td\left(\frac{S}{N}\right)_{System}
%\end{equation}
%where conservation of energy was also applied. Fixing the volume and the 
%number of particles, so that $dV=0$ and $d\left(\frac{1}{N}\right)=0$, 
%results in
\begin{equation}
  df_{System}=-\left(\frac{T}{N}\right)dS_{System+Surroundings}
\end{equation}
%which, for fixed N, is the same as
%\begin{equation}dF_{System}=-TdS_{System+Surroundings} \end{equation} 
%Since $dS_{System+Surroundings} \geq{0}$, and the temperature and number 
%of particles are fixed, the change in the free energy of the system will 
%always be negative, and thus the free energy tends toward a minimum.
%~\cite{schroeder}. 
and so the free energy tends toward a minimum. 
Details of this procedure be found in Thermal Physics by Schroeder~\cite{schroeder}, 
and a proof of minimization in general can be found in 
Theory of Simple Fluids by Hansen and McDonald~\cite{Hansen}.

The Helmholtz free energy can be written as
\begin{equation}{F=-k_{B}T\ln(Z)}\end{equation}
which results from equating the expressions for the Helmholtz free energy 
from thermodynamics to those of statistical mechanics. The paritian function 
Z is integrated over all of phase space for N identical particles each with 
a momentum $\vec{p}$ and a position $\vec{r}$, and is given by
\begin{equation}{Z=}\frac{1}{N!}\frac{1}{h^{3N}}\int{...}\int{dPdQ}~e^\frac{-H(\vec{p}_1,\vec{p}_2,...\vec{p}_N,\vec{r}_1, \vec{r}_2,...\vec{r}_N)}{k_BT}\end{equation}
where
\begin{displaymath}{dP=d\vec{p}_1d\vec{p}_2...d\vec{p}_N \mbox{~~~~and~~~~} d\vec{p}=dp_xdp_ydp_z}\end{displaymath}
\begin{displaymath}{dQ=d\vec{r}_1d\vec{r}_2...d\vec{r}_N \mbox{~~~~and~~~~} d\vec{r}=dxdydz\mbox{~~~~}}\end{displaymath}
Writing the Hamiltonian, H in terms of its component kinetic and potential energies gives
\begin{equation}{H = KE + V + V_{ext} = \sum_{i=1}^N\left(\frac{|\vec{p}_i|^2}{2m}\right)+V(\vec{r}_1,\vec{r}_2,{...},\vec{r}_N)+V_{ext}(\vec{r}_1,\vec{r}_2,{...},\vec{r}_N)}\end{equation}
where the translational kinetic energy of the N atoms is given by 
\begin{equation}{\text{KE}=\sum_{i=1}^N\left(\frac{|\vec{p}_i|^2}{2m}\right)}\end{equation}
and the potential energy of the total interaction of each atom with the other 
N-1 atoms is given by
\begin{displaymath}{V(\vec{r}_1,\vec{r}_2,{...},\vec{r}_N)}\end{displaymath} 
and where \begin{displaymath}{V_{ext}(\vec{r}_1,\vec{r}_2,{...},\vec{r}_N)}\end{displaymath} 
is an externally applied potential that varies spatially, such that its 
effects on each atom depend on the atom's position. Often $V_{ext}$ is 
set to zero, as it is in this paper. 

Noting that the kinetic energy depends only on momentum while the total 
interatomic and external potential energies depend only on position, 
the integral in equation (8) can be separated into two parts 
\begin{equation}{Z=\frac{1}{N!}\frac{1}{h^{3N}}\int...\int{dP}~e^\frac{-(\frac{|\vec{p}_1|^2}{2m}+\frac{|\vec{p}_2|^2}{2m}+...\frac{|\vec{p}_N|^2}{2m})}{k_BT}\int...\int{dQ}~e^\frac{-V(\vec{r}_1,\vec{r}_2,{...},\vec{r}_N)-V_{ext}(\vec{r}_1,\vec{r}_2,{...},\vec{r}_N)}{k_BT}}\end{equation}  
Multiplying by $\frac{V^N}{V^N}$ where V is the real space volume, and 
\begin{equation}{V^N=}\int{...}\int{dQ}\end{equation} 
the partitian function can be expressed as
\begin{equation}\label{Zmultiply}{Z=Z_{ideal}Z_{excess}}\end{equation}
where
\begin{align} \label{eq:Zideal}
    Z_{ideal}=V^N\frac{1}{N!}\frac{1}{h^{3N}}\int{...}\int{dP}~e^\frac{-(\frac{|\vec{p}_1|^2}{2m}+ \frac{|\vec{p}_2|^2}{2m}+...\frac{|\vec{p}_N|^2}{2m})}{k_BT}
\end{align}
\begin{align} \label{eq:Zexcess}
    Z_{excess}=\frac{1}{V^N}\int{...}\int{dQ}~e^\frac{-V(\vec{r}_1,\vec{r}_2,{...},\vec{r}_N)-V_{ext}(\vec{r}_1,\vec{r}_2,{...},\vec{r}_N )}{k_BT}
\end{align} 
The ideal partitian function $Z_{ideal}$ is associated with an ideal gas 
which exhibits no interaction between the atoms - a condition approached 
by a gas at low density where the atoms rarely encounter one another. 
The excess partitian function $Z_{excess}$ accounts for the interaction 
of all the atoms according to their interatomic potential. Using 
Equation~\ref{Zmultiply}, the free energy can also be broken into two parts.
\begin{equation}{F=-k_{B}T\ln(Z_{ideal}Z_{excess})}\end{equation}
\begin{align} \label{eq:Finparts}
    F=-k_{B}T\ln(Z_{ideal})-k_{B}T\ln(Z_{excess})
\end{align}
\begin{equation}{F=F_{ideal} + F_{excess}}\end{equation} 

Putting Equation~\ref{eq:Zideal} into Equation~\ref{eq:Finparts}, 
the ideal Helmholtz free energy is
\begin{equation}{F_{ideal}=-k_BT\ln{\left(V^N\frac{1}{N!}\frac{1}{h^{3N}}\int{...}\int{dP}~e^\frac{-(\frac{|\vec{p}_1|^2}{2m}+ \frac{|\vec{p}_2|^2}{2m}+...\frac{|\vec{p}_N|^2}{2m})}{k_BT}\right)}}\end{equation}  

\begin{equation}{\label{Fidealmultipleint}}{=-k_BT\ln\left(V^N\frac{1}{N!}\frac{1}{h^{3}}\int{d\vec{p}_1}~e^\frac{-\frac{|\vec{p}_1|^2}{2m}}{k_BT}\frac{1}{h^{3}}\int{d\vec{p}_2}~e^\frac{-\frac{|\vec{p}_2|^2}{2m}}{k_BT}{...}\frac{1}{h^{3}}\int{d\vec{p}_N}~e^\frac{-\frac{|\vec{p}_N|^2}{2m}}{k_BT}\right)}\end{equation}
These integrals can conveniently be written in terms of the de Broglie 
wavelength $\Lambda$, or the quantum volume $\text{n}_\text{Q}$
\begin{equation}{\frac{1}{h^{3}}\int{d\vec{p}}~e^\frac{-\frac{|\vec{p}|^2}{2m}}{k_BT}=\left(\sqrt{\frac{k_BTm}{2\pi\hbar^2}}\right)^3=\frac{1}{\Lambda^{3}}}=\text{n}_\text{Q}\end{equation} 
where \begin{equation}{\Lambda =\sqrt{\frac{2\pi\hbar^2}{k_BTm}}}\end{equation} 
Rewriting Equation~\ref{Fidealmultipleint} in terms of $\Lambda$ gives
\begin{equation}{F_{ideal}= -k_BT[\ln(V^N)-\ln(N!) - \ln(\Lambda^{3N})]}\end{equation}
Using Stirling's approximation \begin{equation}{\ln(N!)=N\ln{N}-N}\end{equation} 
this becomes
\begin{equation}{F_{ideal}= k_BT[N\ln(\frac{N\Lambda^{3}}{V}{)-N}]}\end{equation} 
Rewriting the equation in terms of the number density $n=\frac{N}{V}$, 
and the quantum volume, $\text{n}_\text{Q}$ gives
\begin{equation}{F_{ideal}= k_BT\left[\frac{N}{V}\ln{\left(\frac{N}{V}\frac{1}{\text{n}_\text{Q}}\right)}-\frac{N}{V}\right]}V\end{equation}
\begin{equation}\label{eq:Fideal}{F_{ideal}= k_BT[n\ln(n/\text{n}_\text{Q})-n]V}\end{equation}   
For a spatially varying number density, the ideal free energy in 
Equation~\ref{eq:Fideal} becomes
\begin{equation}\label{eq:Fideal-n(r)}{F_{ideal}[n(\vec{r})]= k_BT\int_{Vol}[n(\vec{r})\ln(n(\vec{r})/\text{n}_\text{Q})-n(\vec{r})]d\vec{r}}\end{equation} 

Putting Equation~\ref{eq:Zexcess} into Equation~\ref{eq:Finparts}, 
the excess Helmholtz free energy is written most generally as
\begin{equation}\label{Fexcess-mostgeneral}{F_{excess}= -k_BT\ln{\left(\frac{1}{V^N}\int{...}\int{dQ}~e^\frac{-V(\vec{r}_1, \vec{r}_2,...\vec{r}_N)-V_{ext}(\vec{r}_1, \vec{r}_2,...\vec{r}_N)}{k_BT}\right)}}\end{equation} 
This integral can be greatly simplified if it can be assumed that the 
potential energy $V(\vec{r_1},\vec{r_2},...\vec{r_N})$ is a sum of the 
potential energies between pairs of atoms, where each atom is paired in 
turn with each of the other atoms in the system. In this case 
$V(\vec{r}_1, \vec{r}_2,...\vec{r}_N)$ becomes 
\begin{equation}{V(\vec{r_1},\vec{r_2},...\vec{r_N})=\frac{1}{2}\sum^N_i\sum^N_{j\neq{i}}V_{ij}(\vec{r_i},\vec{r_j})}\end{equation} 
Using this potential, and setting the external potential 
$V_{ext}(\vec{r}_1, \vec{r}_2,...\vec{r}_N)$ to zero, the exponent in 
Equation~\ref{Fexcess-mostgeneral} becomes

\begin{equation}{e^{-\frac{\frac{1}{2}\sum^N_i\sum^N_{j\neq{i}}V_{ij}(\vec{r_i},\vec{r_j})}{k_BT}}=e^{\frac{-V_{12}(\vec{r_1},\vec{r_2})}{k_BT}}e^{-\frac{V_{13}(\vec{r_1},\vec{r_3})}{k_BT}}e^{-\frac{V_{14}(\vec{r_1},\vec{r_4})}{k_BT}}...}\end{equation}
\begin{equation}{\approx-(1+f_{12})(1+f_{13})(1+f_{14})...}\end{equation}
where each exponential term is simplified by a Taylor expansion $e^x\approx{1+x}$.  
The function $f_{ij}$ the Mayer function given by
\begin{align}\label{eq:mayerfunction}
     f_{ij}(\vec{r_i},\vec{r_j})=e^{-\frac{V_{ij}(\vec{r_i},\vec{r_j})}{k_BT}}-1
\end{align} 
Multiplying the Mayer function terms, the excess free energy can be 
written as~\cite{schroeder}
\begin{align} \label{eq:Fexcess-simplified}
    F_{excess}=-k_BT\ln{\left(\frac{1}{V^N}\int{...}\int{dQ}\left[1 + \sum{f_{ij}} + \sum{f_{ij}f_{kl}} +\sum{f_{ij}f_{kl}f_{mn}} +... \right]\right)}
\end{align}

  \begin{figure}
    \centering
    \includegraphics[height=7cm]{figs/diagrammic.pdf}
    \caption{Two-pair interactions $f_{ij}f_{kl}$ are shown for N=4 atoms. 
    Interactions between individual pairs of atoms are shown in green. 
    Interactions between pairs of atoms with an atom shared in common 
    are shown in red. At low densities only the green interactions
    are considered, since atoms are not likely to be clustered.}
    \label{fig:diagrammic}
  \end{figure}

Equation~\ref{eq:Fexcess-simplified} is difficult to solve in general, 
but it can be solved for low densities, and this provides a way to check
a more general solution.
At low densities, it is not likely to find two or more atoms in close 
proximity, and so only interactions between individual pairs of spheres, 
such as those shown in green in Figure~\ref{fig:diagrammic}, are considered.
Each pair can then be described by a function of the distance between the 
two atoms in the pair. 
With a careful counting of the resulting number of such pairs, the excess 
free energy can then be written~\cite{schroeder} as
\begin{equation}\label{eq:Fexcess}{\frac{F_{excess}}{N}=k_BTn\left(-\frac{1}{2}\int{f(r)}{d\vec{r}}\right)}\end{equation} 
in the thermodynamic limit of large N. 
The quantity in parenthesis can be recognized 
as the second virial expansion coefficient typically called $B_2$. 
It is important that an expression for the excess Helmholtz free energy 
correctly reduces to this result in the low-density limit. 
Outside of the low density limit, it is quite difficult to evaluate the 
excess Helmholtz free energy given in Equation~\ref{eq:Fexcess-simplified}. 
SFMT, discussed in the next section, provides a way around this problem
for some types of fluids.

Once an expression for the Helmholtz free energy as a functional of the
number density profile $n(\vec r)$ is formed then, according to classical 
Density Functional Theory, the equilibrium value of the 
Helmholtz energy, and its corresponding equilibrium density profile can be
found by varying the number density profile
through all possible configurations at some fixed temperature
until the free energy is minimized.  
%In this section 
%a functional for the Helmholtz free energy was formed by finding 
%general expressions for the ideal and the excess portions of the free 
%energy, and adding them together.
%\begin{equation}{F[n(\vec{r})]=F_{ideal}[n(\vec{r})] + F_{excess}[n(\vec{r})]}\end{equation}
%The functional for the ideal portion of the Helmholtz free energy is given by 
%Equation~\ref{eq:Fideal-n(r)}, and a functional for the excess portion of the Helmholtz 
%free energy is given in Equation~\ref{eq:Fexcess-simplified}. The specific 
%form for the excess Helmholtz free energy depends on the particular 
%fluid being modeled as defined by the  
%interatomic potential energy V(r) between the spheres, which can make
%forming an expression difficult. In this paper, Soft Fundamental Measure
%Theory will be used to form an expression for the excess Helmholtz free 
%energy, as explained in the next section.

\section{Soft Fundamental Measure Theory (SFMT)}
Soft Fundamental Measure Theory (SFMT) is a method for forming an 
expression for the excess Helmholtz free energy of a soft-sphere fluid.
In the next chapter, 
we will adapt an SFMT functional to fit the soft-sphere WCA fluid.
Recall that in order to apply classical Density Functional Theory 
to a system that has a fixed temperature, volume, and number of particles, 
a functional for the Helmholtz free energy must first be formed. 
The ideal portion of the Helmholtz free energy is given by 
Equation~\ref{eq:Fideal-n(r)}, but forming an expression for the excess 
Helmholtz free energy from Equation~\ref{eq:Fexcess-simplified} for the 
particular fluid being modeled can be challenging. 
SFMT is a method for forming an 
expression for the excess Helmholtz free energy
that does not depend on explicitly taking into account the interaction 
that each sphere with every other sphere in the fluid (as required by 
Equation~\ref{eq:Fexcess-simplified}).
%In the next chapter on methods, we will use an SFMT functional that we 
%adapt to fit the soft-sphere WCA fluid.
SFMT was developed by Schmidt~\cite{schmidt1999density, schmidt2000fluid} 
for soft-spheres, and is an extension of Fundamental Measure Theory (FMT) 
originally developed by 
Rosenfeld~\cite{rosenfeld1989, rosenfeld1994, rosenfeld1996, fmtfromsptandsignerror} 
for the hard-sphere fluid. FMT in turn extented  
Scaled Particle Theory (SPT)~\cite{ReissSPT, santos2012phi3}, which was 
developed for homogeneous hard-sphere fluids, 
to inhomogeneous, or non-uniform hard-sphere fluids.

%In order to apply classical Density Functional Theory to a system that has
%a fixed temperature, volume, and number of particles, a functional for 
%the Helmholtz free energy must first be formed. An expression for the ideal 
%portion of the Helmholtz free energy is given by Equation~\ref{eq:Fideal-n(r)}, 
%but forming an expression for the excess Helmholtz free
%energy from Equation~\ref{eq:Fexcess-simplified} for the particular fluid 
%being modeled can be challenging. 
%Soft Fundamental Measure Theory (SFMT) is a method for forming an 
%expression for the excess Helmholtz free energy of a soft-sphere fluid
%that does not depend on explicitly taking into account the interaction 
%that each sphere with every other sphere in the fluid (as required by 
%Equation~\ref{eq:Fexcess-simplified}). In the next chapter on methods, we
%will use an SFMT functional that we adapt to fit the soft-sphere WCA fluid.
%SFMT was developed by Schmidt~\cite{schmidt1999density, schmidt2000fluid} 
%for soft-spheres. It is an extension of Fundamental Measure Theory (FMT) 
%originally developed by 
%Rosenfeld~\cite{rosenfeld1989, rosenfeld1994, rosenfeld1996, fmtfromsptandsignerror} 
%for the hard-sphere fluid, and which expanded 
%Scaled Particle Theory (SPT)~\cite{santos2012phi3} 
%to inhomogeneous, or non-uniform hard-sphere fluids.

Scaled Particle Theory
seeks a general expression for the excess chemical 
potential of a homogeneous hard-sphere fluid which can then be integrated 
to find the excess Helmholtz free energy according to the relation
\begin{equation}\mu_{ex}=\frac{\partial{F_{ex}}}{\partial{N}}\bigg|_{T,V}{~~~~~}\rightarrow{~~~~~}F_{ex}=\int{\mu_{ex}dN}\end{equation}
It is an extension of the work done by Widom who showed that the excess 
chemical potential of a homogeneous fluid corresponds to the reversible 
work of inserting a sphere of radius R into a fluid consisting of hard 
spheres of radius R. 
\begin{align}
	\mu_{ex}=W_{rev}(R){~~~~}\text{work to insert a hard sphere of radius R}
\end{align}
Widom's result was obtained by considering the difference 
in the excess Helmholtz free energy when one atom is added to an otherwise 
closed system. He showed that the excess chemical potential is related to 
the ensemble average of the Boltzman factor for the potential energy 
difference $V_{diff}$ between a system with N atoms and one with N+1 atoms. 
The resulting equation, shown below, is known as Widom's Insertion 
Formula~\cite{Hansen}.
\begin{align} \label{mu-der-int}
    \mu_{ex}=\frac{F_{ex}(N+1)-F_{ex}(N)}{\Delta{N}}\bigg|_{T,V}
\end{align}
\begin{equation}\label{widoms-insertion-formula}{~}=-k_BT\ln\left(<e^{-\beta{V_{diff}}}>\right)\end{equation}
where 
\begin{align}
    <e^{-\beta{V_{diff}}}>=\sum_s{P_se^{-\beta V_s}}
\end{align}
and
\begin{align}
    Z=\sum_s{e^{-\beta V_s}}
\end{align}

The quantity $<e^{-\beta{V_{diff}}}>$ can be evaluated by considering that
there will be a change in the potential energy of $V_{diff}$ if one sphere
is added, but no change in potential energy if no sphere is added ($V_{diff}=0$). 
In this case, $P_s=\frac{e^{-\beta V_s}}{Z}$.
%at each point in the fluid, there is a probability that 
%a hard sphere of radius R has space available to exist at that point
%without overlapping with other spheres, or it does not, as shown in 
Widom's Insertion formula Equation~\ref{widoms-insertion-formula}, 
then becomes~\cite{Hansen}
\begin{align} 
	\mu_{ex} &= -k_BT\ln\left(e^{-\beta{V_{diff}}}+1\right)  \\
	         &= V_{diff} \\
             &= W_{rev}(R)  \label{workrev}
\end{align}
\begin{figure}
    \centering
    \includegraphics[height=7cm]{figs/P_overlap.pdf}
    \caption{No energy is needed to place a sphere at a position where 
    there is no overlap, such as that shown by the black dot near the 
    top. But it is not possible to place a sphere at a position where 
    an overlap occurs.}
    \label{fig:p_overlap}
  \end{figure}
The quantity $<e^{-\beta{V_{diff}}}>$ can also be evaluated by considering
that it takes no energy for a hard sphere to be placed at a point
in the fluid where it does not overlap with any other spheres 
($V_{diff}=0$), but it would require infinite energy ($V_{diff}=\infty$)
to place a sphere where it would overlap another sphere, as described 
in Figure~\ref{fig:p_overlap}. 
In this case Equation~\ref{widoms-insertion-formula} becomes~\cite{Hansen}
\begin{align} \label{muexcess}
  \mu_{excess}(R_{cavity}=R)=-k_BT\ln(1-n\frac{4\pi}{3}R^3) 
\end{align}
where the radius of the cavity $R_{cavity}$ equals the radius of a 
hard sphere R, and $nV_{Sphere}$, which is equal to the total volume 
occupied by the spheres divided by the total volume available, 
is the probability that a sphere cannot be placed at some random
point in the homogeneous fluid of number density, $n$, without
overlapping with another sphere. $P_s$ would then have values of
$nV_{Sphere}$ and $1-nV_{Sphere}$.
Combining  Equations~\ref{workrev} 
and \ref{muexcess} gives
\begin{align}
  %W_{rev}(R_{cavity}=R)=-k_BT\ln(1-n\frac{4\pi}{3}R^3)
  W_{rev}(R_{cavity}=R)=-k_BT\ln(1-\eta) 
\end{align}
where $\eta$ is the packing fraction defined as
\begin{equation}{\eta = \frac{\mbox{Volume occupied by the spheres}}{\mbox{Total Volume}}}\end{equation}
\begin{equation}{\eta = \frac{N\frac{4}{3}\pi{R^3}}{V}}\end{equation}

Scaled Particle Theory forms a more general result for the reversible work 
as a function of the radius of the cavity by considering additional 
cases where the cavity is larger, and much larger than the radius of 
a hard sphere, and requiring continuity of the functions for the 
different cases and their derivatives (first and second). 
By setting the radius of the cavity to the hard sphere radius, the 
more general work equation becomes equal to the chemical potential. 
Details of this procedure can be found 
in Theory of Simple Fluids by Hansen and McDonald~\cite{Hansen}.  
An expression for the excess Helmholtz free energy in terms of the
packing fraction $\eta$ can then be obtained through 
Equation~\ref{mu-der-int}. 
%
%By requiring continuity of the functions valid for different cavity 
%sizes, along with their first and second derivatives, an equation for 
%the work is obtained which can then be used to form 
%set equal to the excess chemical 
%potential by setting the cavity radius to the radius of hard sphere. 
%
%Forming a more general result by requiring continuity of the functions, 
%along with their first and second derivatives, for the reversible work
%written for fluids with a cavity radius equal to, larger than, or very 
%much larger than the hard sphere radius R, 
%
%Scaled Particle Theory arrives at 
%an expression for the excess Helmholtz free energy of 
%a homogeneous hard-sphere fluid in terms of the packing fraction $\eta$.
%For details, see Theory of Simple Fluids by Hansen and McDonald~\cite{Hansen}. 
\begin{align}\label{Fexcess-SPT}
    F_{excess}=k_BT\left(-ln(1-\eta)+\frac{3\eta}{1-\eta}+\frac{3{\eta}^2}{2(1-\eta)^2}\right)N
\end{align} 

Fundamental Measure Theory can be constructed from Equation~\ref{Fexcess-SPT}  
%and provides an expression for the excess Helmholtz free energy of an 
%inhomogeneous 
%hard-sphere fluid as a functional of the number density profile $n(\vec r)$. 
%
%Soft Fundamental Measure Theory (SFMT), which will be used in this paper, 
%is an extension of FMT and provides an expression for the excess Helmholtz
%free energy of fluids with soft potentials like that of the WCA fluid. 
%It will be helpful to continue looking at FMT before addressing SFMT.
%FMT begins 
by noticing that a quantity $G_{3}$ can be defined such that
\begin{equation}{\eta =\left(\frac{N}{V}\right)\frac{4}{3}\pi{R^3}=nG_{3}}\end{equation}
where $G_{3}$ is the volume of a sphere. 
Likewise, by defining quantities
\begin{equation}{G_{2}=4\pi{R^2}}\end{equation}
\begin{equation}{G_{1}=R}\end{equation}
\begin{equation}{G_{0}=1}\end{equation}
where $G_{2}$ is the surface area of a sphere, and $G_{1}$ is the 
radius of a sphere, Equation~\ref{Fexcess-SPT} can be written as
\begin{equation}\label{FexfromSPTinthermsofG}{F_{excess}=k_{B}T\left(-nG_{0}ln(1-nG_{3})+\frac{nG_{1}nG_{2}}{1-nG_{3}}+\frac{nG_{2}^3}{24\pi(1-Gn_{3})^2}\right)V}\end{equation}
The quantities $G_{3}, G_{2}, G_{1},$ and $G_{0}$ are fundamental 
geometric measures of a sphere, and are what give rise to the name
Fundamental Measure Theory. Introducing a set of variables
%weighted densities $n_{i}=nG_{i}$ 
given by
\begin{equation}\label{n3}{n_{3}=nG_{3}=\left(\frac{N}{V}\right)\frac{4}{3}\pi{R^3}=\eta}\end{equation}
\begin{equation}\label{n2}{n_{2}=nG_{2}=\left(\frac{N}{V}\right)4\pi{R^2}=\frac{3\eta}{R}}\end{equation}
\begin{equation}\label{n1}{n_{1}=nG_{1}=\left(\frac{N}{V}\right)R}\end{equation}
\begin{equation}\label{n0}{n_{0}=nG_{0}=\left(\frac{N}{V}\right)}\end{equation}
Equation~\ref{FexfromSPTinthermsofG} can be written as
\begin{align}\label{FexfromSPT}
    F_{excess}=k_{B}T\left(-n_{0}ln(1-n_{3})+\frac{n_{1}n_{2}}{1-n_{3}}+\frac{n_{2}^3}{24\pi(1-n_{3})^2}\right)V
\end{align}
which has the form 
\begin{align}
    F_{excess} = k_BT(\Phi_1+\Phi_2+\Phi_3)V
\end{align}
where 
\begin{align}
   \Phi_1 &= -n_{0}ln(1-n_{3}) \\
   \Phi_2 &= \frac{n_{1}n_{2}}{1-n_{3}} \\
   \Phi_3 &= \frac{n_{2}^3}{24\pi(1-n_{3})^2}   \label{oldPhi3equation}
\end{align}

The quantities $n_{0},n_{1},n_{2},n_{3}$ can be considered as weighted 
densities where the weights are given by $G_{0},G_{1},G_{2},G_{3}$ respectively.
\begin{align}
    n_{i}=nG_{i}
\end{align}
These weights are valid for determining the excess free energy of a system of 
hard spheres with a homogeneous number density n. But these weights, and 
their related  weighted densities, can be written in a more general form 
allowing for their extension to a system with a spatially varying number 
density $n(\vec{r})$, that is, an inhomogeneous fluid, in which case, the
excess Helmholtz free energy is written as
\begin{equation}\label{eq:F_ex-n-of-r}{F_{excess}[n(\vec{r})]= k_BT\int(\Phi_1(\vec{r})+\Phi_2(\vec{r})+\Phi_3(\vec{r}{)) d}\vec{r}}\end{equation} 
Using a backwards Heavyside step function defined as
\begin{equation}{\Theta(|\vec{r}|-R)=\left\{ \begin{array}{rc} 1 & 0<r \leq R \\ 0  & r>R \end{array}\right.}\end{equation}
the following expressions are obtained
\begin{equation}{G_{3}=\int_{Volume}{\Theta(|\vec{r}|-R)d{\vec{r}}} = \frac{4}{3}\pi{R^3}}\end{equation}
\begin{equation}{G_{2}=\int_{Volume}{\delta(|\vec{r}|-R)d{\vec{r}}} = 4\pi{R^2}}\end{equation}
\begin{equation}{G_{1}=\int_{Volume}{\frac{\delta(|\vec{r}|-R)}{r}d{\vec{r}}} = R}\end{equation}
\begin{equation}{G_{0}=\int_{Volume}{\frac{\delta(|\vec{r}|-R)}{r^2}d{\vec{r}}} = 1}\end{equation}
where the weighted densities can now be expressed as
\begin{align}
    n_{i}=\int_{Volume}{nw_{i}(r)}d{\vec{r}}
\end{align} 
or for a spatially varying number density $n(\vec{r})$,
\begin{equation}{n_i(\vec{r})=\int_{Volume}{n(\vec{r'})w_i(|\vec{r}-\vec{r'}|)d{\vec{r'}}}}\end{equation}
with weight functions
\begin{equation}\label{eq:w3}{w_{3}=\Theta(|\vec{r}|-R)}\end{equation}
\begin{equation}\label{eq:w2}{w_{2}=\delta(|\vec{r}|-R)}\end{equation}
\begin{equation}\label{eq:w1}{w_{1}=\frac{w_{2}}{4\pi{r}}}\end{equation}
\begin{equation}\label{eq:w0}{w_{0}=\frac{w_{2}}{4\pi{r}^2}}\end{equation}
The weight functions $w_{2}$ and $w_{3}$ have the relation
\begin{equation}\label{w2_w3_relation}{w_{3}=\int_{r}^{\infty}{w_{2}(\vec{r})dr}\mbox{~~~~or equivilantly,~~~~}-\frac{\partial{w_3(r)}}{\partial{r}}=w_2(r)}\end{equation}
which will be important to keep in tact when moving to SFMT~\cite{schmidt1999density}.

The weight functions can be related to the Mayer function 
\begin{align}
     f(r)=\exp^{-\frac{V(r)}{k_{B}T}}-1
\end{align} 
where r is the distance between the centers of two hard spheres, and 
$V(r)$ is the interatomic potential between two hard spheres given by 
\begin{align}
    V(r)=\left\{ \begin{array}{rc} \infty & 0<r \leq 2R \\ 0  & r>2R \end{array}\right.
\end{align}
Evaluating Equation with $V(r)$ given by Equation,  
the Mayer function for a hard-sphere fluid is found to be
\begin{equation}\label{f(r)step}{f(r)=-\Theta(|\vec{r}|-2R)}\end{equation} 
which can be written in terms of the weight functions if two 
additional vector weight functions $\vec{w_2}$ and $\vec{w_1}$ are introduced
\begin{equation}\label{eq:w_v2}{\vec{w_{2}}=w_{2}\frac{\vec{r}}{r}}\end{equation}
\begin{equation}\label{eq:w_v1}{\vec{w_{1}}=w_{1}\frac{\vec{r}}{r}}\end{equation} 
The mayor function can then be written in terms of the weight functions~\cite{Hansen}
\begin{equation}\label{mayer_deconvolution}{f(r)=-\Theta(|\vec{r}|-2R)= -2(w_3 \otimes w_0 + w_2 \otimes w_1 + \vec{w_2} \otimes \vec{w_1})}\end{equation}  

Taking the derivative of the Mayer function in Eq~\ref{f(r)step} provides 
another relation that will be important to keep in tact when moving to 
SFMT~\cite{schmidt2000fluid}.
\begin{align}\label{mayerder=w2convolution}
     \frac{df(r)}{dr} = w_2 \otimes w_2 = \int{w_2(r')w_2(r-r')dr'}
\end{align} 
This relation can easily be checked for a hard-sphere fluid where
\begin{equation}{\frac{df(r)}{dr}=\left\{ \begin{array}{rc} \frac{d}{dr}(-1)=0 & r<2R \\\frac{d}{dr}(\text{vertical line})=+\infty & r=2R \\ \frac{d}{dr}(0)=0  & r>2R \end{array}\right.}\end{equation}
\begin{equation}\label{mayerder}{\frac{df(r)}{dr} = \delta(r-2R)}\end{equation} 
The right side can be evaluated by using the general relation for 
convolution which is equal to the Inverse Fourier Transform 
of a product of Fourier Transforms. 
\begin{equation}{f \otimes g = \text{I.F.T.[ F.T.}[f]\text{ x F.T.[g] }]}\end{equation} 
Taking the Fourier Transform of the weight function $w_2(r)=\delta(r-R)$ 
gives
\begin{equation}{\text{F.T.}[w_2]=\int_{-\infty}^{\infty}\delta(r-R)e^{-ikr}dr=e^{-ikR}}\end{equation} 
and the right side of Equation~\ref{mayerder=w2convolution} becomes
\begin{equation}{w_2 \otimes w_2 = \text{I.F.T.}[e^{-ikR}e^{-ikR}]}\end{equation} 
\begin{equation}\label{w2convolution}{=\int_{-\infty}^{\infty}e^{-ik2R}e^{ikr}dk=\int_{-\infty}^{\infty}e^{ik(r-2R)}dk=\delta(r-2R)}\end{equation} 
which matches Equation~\ref{mayerder}.

Tarazona later improved upon the original FMT equations by replacing the
expression for $\Phi_3$ given by Equation~\ref{oldPhi3equation} with the
more accurate expression~\cite{tarazonaphi3, santos2012phi3} 
\begin{align}
\Phi_3 &= \frac{{n_2}^3-3n_2\vec{n}_{v2}\cdot\vec{n}_{v2}+\frac{9}{2}[\vec{n}_{v2}\cdot{\overleftrightarrow{n}_{m2}}\cdot{\vec{n}_{v2}}-\operatorname{Tr}({\overleftrightarrow{n}^3_{m2}})]}{24\pi(1-n_3)^2}  \label{eq:Phi3}
\end{align}
in which a tensor weight is incorporated given by
\begin{equation}\label{eq:tensorweight}{\overleftrightarrow{w}_{m2}(\vec{r})=w_2(r)\left(\frac{\vec{r}\vec{r}}{r^2}-\frac{I}{3}\right)}\end{equation} 
All the other FMT equations remain the same. 

The FMT equations shown so far have been for a hard-sphere fluid~\cite{Hansen}.
All of the weight functions, the relations between them, and the equation 
for the derivative of the Mayer function, can be expressed in terms of 
$w_{2}(r)$ which, for a hard-sphere fluid, consists of a delta function.
The weight function $w_{2}(r)$ is in turn related to $w_{3}(r)$ through 
Equation~\ref{w2_w3_relation}.
Soft Fundamental Measure Theory extends FMT discussed so far to fluids 
with soft potentials by keeping the relationships between the weight 
functions, and the equation for the derivative of the Mayer function 
the same (Equations~\ref{eq:w0},~\ref{eq:w1},~\ref{eq:w_v1},~\ref{eq:w_v2},
~\ref{w2_w3_relation},~\ref{eq:tensorweight},~\ref{mayerder=w2convolution}), 
but replacing $w_{3}(r)$ with a weight function appropriate 
for a soft-sphere potential~\cite{schmidt1999density, schmidt2000fluid}. 
This is what we do in forming a functional for the Weeks-Chandler-Anderson 
fluid as detailed in the next chapter on Methods.

\chapter{Methods}

\section{The Overall Approach}
  
Our goal is to generate temperature-density and pressure-temperature 
phase diagrams for a Weeks-Chandler-Anderson model fluid. 
A temperature-density phase diagram for a fluid that does not
have attraction between the particles that make up the fluid will show
two distinct regions - a "liquid" region at lower densities, and a solid 
region at higher densities. 
A phase transition between a liquid and a solid is accompanied by an 
abrupt change in the number density, and so there is a gap between the 
purely liquid and the purely solid regions. Within the gap, or coexistence
region, is found mixtures of various amounts of liquid and solid each of
which have number densities that are fixed at the values they had at the 
phase transition. 

To construct the phase diagram, the liquid and solid phase transition number densities
must be found. Maxwell's double tangent construction can be used to find 
these number densities given plots of the liquid and the solid Helmholtz 
free energies verses the inverse density at a given temperature, as
discussed in Section 2.2. We use classical Density Functional Theory 
to find the Helmholtz free energies of the liquid and the solid. 
A functional for the ideal portion of the
Helmholtz free energy is given by Equation~\ref{eq:Fideal-n(r)}, and a 
functional for 
the excess Helmholtz free energy, which depends on the interatomic potential
between particles, will be developed in the next section using SFMT. 
Once a functional is constructed, the Helmholtz free energy can be found - 
but not directly. The equilibrium number density profile must first be 
identified. According to cDFT, the equilibrium number density profile, 
and its corresponding equilibrium Helmholtz free energy can be found 
by trying many different number densities, and identifying the one that 
minimizes the Helmholtz free energy. In this section, we look at how to 
generate various number density profiles that can be used to implement cDFT. 

\begin{figure}
  \centering
  \includegraphics[height=3.5cm]{PrimitiveCellLightBlue.png}
  \caption{A primitive crystal cell is shown by the shaded parallelepiped.}
  \label{fig:primitivecell}
\end{figure}

It is known that a fluid with a soft potential like the WCA fluid will 
form a Face Centered Cubic (FCC) crystal in its solid state~\cite{Hansen}. 
Although an entire 
crystal lattice can be constructed with FCC cells, 
they are not the smallest cells with which the lattice can be built. 
The shaded parallelepiped shown within the FCC cell in Figure \ref{fig:primitivecell} 
is the smallest structure that, when repeated, re-creates the entire crystal 
lattice. This parallelepiped primitive cell takes up one-fourth the volume 
of an FCC cell, and contains the equivalent of one atom.
It includes portions of eight atoms each located on a lattice site conveniently 
reachable by moving along one of three different lattice vectors. 
The three lattice vectors will be useful in 
creating the lattice structure computationally, and will also make our 
program run faster.

\begin{figure}
  \centering
  \includegraphics[height=2.5cm]{Ensemble_Gaussian.png}
  \caption{On the left are shown four examples of possible positions 
  that an atom (blue sphere) might have within a cell that has one 
  lattice site (black dot) at the center. A large number of such 
  possibilities would give rise to a Gaussian distribution, 
  as shown on the right.}
  \label{fig:Ensemble_Gaus}
\end{figure} 

An atom will not realistically reside exactly at a lattice point. 
Statistically there will typically be a normal, or Gaussian distribution 
of positions about the lattice point. 
Atoms are always in motion, so when talking about "the location" 
of an atom, we are talking about a time averaged position, so that it 
does not change with time.
Figure \ref{fig:Ensemble_Gaus} shows examples 
of four possible positions an atom (shown in blue) might have in a crystal 
cell that has one lattice site at its center (represented by a black dot). 
The atom is slightly off the lattice site in each example, and 
a sampling of many possibilities - more than just the four given 
as examples - would create an average distribution profile that looks 
like a Gaussian distribution, as shown on the right.
A plot of a Gaussian distribution in 2-dimensions is shown in 
Figure~\ref{fig:Gaus_plot} where the width of the Gaussian is described 
by $\sigma$ (not to be confused with the $\sigma$ related to the 
Lennard-Jones model). 
Three-dimensional Gaussian distributions centered at lattice points can be 
used to represent a number density profile $n(\vec{r})$  which describes 
how the atoms are distributed on average over a collection of cells. 

\begin{figure}
  \centering
  \includegraphics[height=6cm]{Gaussian}
  \caption{A 2D Gaussian Distribution of width 2$\sigma$}
  \label{fig:Gaus_plot}
\end{figure}  
  
There can be many different number density profiles for a given number
density. The number density is the number of atoms per volume, whereas
the number density profile describes how much "atom" will be found at various 
points in the cell. 
It can be thought of as an average over a large collection of actual cells
of various possible spatial distributions of the atoms within a cell, or
more precisely, as an ensemble average over all possible microstates, 
each of which consists of a unique set of positions for the atoms in 
that occupy the cell. Mathematically, the microstates are represented 
by a set of delta functions.
\begin{align}
    n(\vec r)=~<\sum_{i=1}^N\delta(\vec r - \vec r_i)>~
\end{align} 
By varying the width of the Gaussian distributions, a number of different 
number density profiles can be generated for a given number density.
It is important to note that the number density does not change when 
introducing a number density profile.  
There are still the same number of atoms over some fixed volume. 
Taking the average of the number density profile over some volume 
gives the number density. 
\begin{align}
    n=<n(\vec{r})>=\frac{1}{V}\int{n(\vec{r})}{d\vec{r}}=\frac{N}{V}
\end{align}

\begin{figure}
   \centering
   \includegraphics[height=2.5cm]{Ensemble_Smallcells.png}
   \caption{On the left are shown four examples of possible positions 
   that an atom (blue sphere) might have within a fixed volume 
   containing eight cells. Each cell has one lattice site at the center 
   represented by a black dot. A large number of such possibilities 
   would give rise to the Gaussian distributions shown on the right.}
   \label{fig:Ensemble_Smallcells}
\end{figure} 

\begin{figure}
   \centering
   \includegraphics[height=2.5cm]{SameStatPic.png}
   \caption{Front views of a cube with eight smaller cells. The two views 
   show two ways of picturing a fraction of vacancies, $f_v$ of 7/8 which 
   give rise to the same Gaussian distribution statistically, 
   and the same number density.}
   \label{fig:SameStatPic}
\end{figure} 

Besides changing the width of the Gaussians, there is another way to 
generate different number density profiles for a given number density.
It may be that not all lattice sites are occupied. Vacancies can be taken 
into account while keeping the overall number density the same by 
making the cells smaller. A large collection of smaller cells creates a 
new number density profile shown in Figure~\ref{fig:Ensemble_Smallcells} 
with eight smaller Gaussians each of which is now one-eighth the height. 
Instead of thinking of an atom in one of the eight small cells, the same 
Gaussian distribution can be obtained from thinking of one-eighth of an 
atom in each small cell as shown in Figure~\ref{fig:SameStatPic}. 
This makes it convenient to describe the new 
number density in terms of the fraction of vacancies, $f_v$. 
The resulting number density is then given by
\begin{align}
    n = (1-f_v){n_{no.vacancies}}
\end{align} 

\begin{figure}
   \centering
   \includegraphics[height=8cm]{VaryWidthandVacancies.png}
   \caption{Shows an example set of Gaussian distribution number density 
   profiles $n(vec r)$ all with the same number density.
   with the same number density of one atom over the fixed volume.}
   \label{fig:Ensemble_vary}
\end{figure}  
  
\begin{figure}
   \centering
   \includegraphics[height=4cm]{figs/homogeneous_bold-box.pdf}
   \caption{A homogeneous number density profile $n(\vec{r})$.}
   \label{fig:homogen_denisty}
\end{figure} 

To implement classical Density Functional Theory, 
a set of number density profiles is generated by varying the Gaussian width 
and the fraction of vacancies at a given combination of temperature and 
number density, as shown by the example in Figure~\ref{fig:Ensemble_vary}.
A homogeneous number density profile is used for a liquid, as shown in 
Figure \ref{fig:homogen_denisty}.
Once again, it is important to note that the number density is the same 
for all these number density profiles $n(\vec{r})$.  
The Helmholtz free energy $F[n(\vec{r})]$ is calculated for each number 
density profile in the set, and also for the homogeneous number density profile 
characteristic of a liquid. Both the ideal and excess free energies are 
found, and added together to get the total Helmholtz free energy.
According to cDFT, the actual equilibrium Helmholtz free energy at
the given temperature and number density will be the one
that corresponds to the number density profile that minimizes the functional. 
This process is repeated to find the free energies at other combinations 
of temperatures and number densities. 

Once the Helmholtz free energy for both the crystal and liquid states of 
the fluid are found for a given temperature and density, they can be 
used to construct phase diagrams.
Outside the region of coexistence which exists between the liquid and 
crystal states, the fluid will be completely in a crystalline state if the 
crystal free energy is lower than the liquid free energy, and it will be 
completely in a liquid state if the liquid free energy is lower than 
the crystal free energy. Maxwell's Double Tangent Construction is used 
to determine the actual number densities of the crystal and liquid states
at the boundaries of the coexistence region for a given temperature, 
and construct phase diagrams as described in Section 2.2.

\section{Our Functional for the Excess Helmholtz free energy}
In order to apply classical Density Functional Theory, an expression 
for the Helmholtz free energy as a functional of a number density profile 
must be found. Computing the ideal portion of the 
Helmholtz free energy is straightforward, but computing the excess portion 
of the Helmholtz free energy is difficult since it depends on the interatomic 
potential energy $V(r)$ that exists between each sphere with all the other spheres, 
as is shown by Equations~\ref{eq:mayerfunction} and \ref{eq:Fexcess-simplified}.
Soft Fundamental Measure Theory is a method
developed to form an expression for the excess Helmholtz free energy without
explicitly taking into account the interaction of each sphere with all the others, 
but it must be tailored to fit a particular fluid\cite{schmidt1999density}.  

Our SFMT functional is tailored to fit a WCA fluid, and 
is written as 
\begin{align}\label{eq:Fexfunctional}
  F_{excess}[n(\vec{r})]=k_BT\int(\Phi_1(\vec{r})+\Phi_2(\vec{r})+\Phi_3(\vec{r}{)) d}\vec{r}
\end{align}
%where 
\begin{align}
\Phi_1 &= -n_{0}\ln(1-n_{3}) \\
\Phi_2 &= \frac{n_{1}n_{2}-\vec{n_{1}}\cdot\vec{n_{2}}}{1-n_{3}} \\
\Phi_3 &= \frac{{n_2}^3-3n_2\vec{n}_{v2}\cdot\vec{n}_{v2}+\frac{9}{2}[\vec{n}_{v2}\cdot{\overleftrightarrow{n}_{m2}}\cdot{\vec{n}_{v2}}-\operatorname{Tr}({\overleftrightarrow{n}^3_{m2}})]}{24\pi(1-n_3)^2}  
\end{align} 
The weighted densities $n_0, n_1, n_2,$ and $n_3$ are found from
\begin{align}\label{eq:numdenprofile}
	n_i(\vec{r})=\int_{Volume}{n(\vec{r'})w_i(|\vec{r}-\vec{r'}|)d{\vec{r'}}}
\end{align}
where the number density profile $n(\vec r)$ within the integral is given 
by Gaussian functions centered at each lattice point.
There are four scalar weight functions given by
\begin{align}\label{eq:weights}
  w_{0}(r) &=\frac{w_{2}}{4\pi{r}^2} \\
  w_{1}(r) &=\frac{w_{2}}{4\pi{r}} \\
  w_2(r) &=-\frac{\partial{w_3(r)}}{\partial{r}} \\
  &= \frac{\sqrt{2}}{\Xi\sqrt\pi}\exp^{-\left(\frac{r-\frac{\alpha}{2}}{\Xi/\sqrt{2}}\right)^2}  \\
  w_3(r) &=\frac{1}{2}\left[1-\operatorname{erf}\left(\frac{r-\frac{\alpha}{2}}{\frac{\Xi}{\sqrt{2}}}\right)\right]  
\end{align}
two vector weight functions given by 
\begin{equation}\label{eq:w_v2}{  \vec{w_{2}}=w_{2}\frac{\vec{r}}{r}  }\end{equation}
\begin{equation}\label{eq:w_v1}{  \vec{w_{1}}=w_{1}\frac{\vec{r}}{r}  }\end{equation} 
and one tensor weight function which is given by
\begin{equation}\label{eq:tensorweight-methods}\overleftrightarrow{w}_{m2}(\vec{r}) = w_2(r)\left(\frac{\vec{r}\vec{r}}{r^2}-\frac{I}{3}\right)\end{equation}
All of the weight functions depend on the expression for $w_3(r)$ 
which in turn depends on the interatomic potential $V(r)$ that exists 
between the spheres, and which defines the particular fluid being modeled.

The weight functions $w_{2}(r)$ and $w_{3}(r)$ are constructed using 
Schmidt's proposed error function model~\cite{schmidt2000fluid} designed 
to model a steep, soft potential in which he set
\begin{equation}{w_3(r)=\frac{1}{2}}\left[1-\operatorname{erf}\left(\frac{r-\frac{\sigma}{2}}{\frac{\text{a}}{\sqrt{2}}}\right)\right]\end{equation} 
subject to the boundary conditions $w_3(0)\approx{1}$ 
and $w_3(\infty)=0$. 
Our functional is  created by replacing Schmidt's parameters $\sigma$ 
and $a$ with our temperature dependent parameters $\alpha$ and $\Xi$.
The two parameters $\alpha$ and $\Xi$ make the temperature-dependent error 
function potential $V_{erf}(r)$ approximate, in the region of interest, 
the temperature-independent WCA potential which is difficult 
to implement computationally. 
%$\Xi(T)$ is determined by setting the second virial coefficient $B_2$ of 
%the error function potential $V_{erf}(r)$ equal to that of the
%WCA potential, $B_{2erf}(T)=B_{2wca}(T)$ for a given temperature.  
The error function potential emerges from $w_{3}(r)$, and the relationships 
in Equations~\ref{w2_w3_relation} and~\ref{mayerder=w2convolution}.
Using Equation~\ref{w2_w3_relation} repeated here,
\begin{displaymath}
     w_{3}=\int_{r}^{\infty}{w_{2}(\vec{r})dr}\mbox{~~~~or equivilantly,~~~~}-\frac{\partial{w_3(r)}}{\partial{r}}=w_2(r)
\end{displaymath}
and solving for $w_2(r)$
\begin{align}{\frac{dw_3(r)}{dr}\bigg|_{\infty}-\frac{dw_3(r)}{dr}\bigg|_{r}=w_2(r)}\end{align} 
we get \begin{equation}{ w_2(r)=\frac{\sqrt{2}}{\Xi\sqrt\pi}\exp^{-\left(\frac{r-\frac{\alpha}{2}}{\Xi/\sqrt{2}}\right)^2} }\end{equation} 
Putting $w_2(r)$ into Equation~\ref{mayerder=w2convolution}, shown again below, 
\begin{displaymath}
    \frac{df(r)}{dr}=\int_{-\infty}^{\infty}{w_2(r')w_2(r-r')dr'}
\end{displaymath}
gives
\begin{equation}{\frac{df(r)}{dr}=\frac{1}{\Xi\sqrt{\pi}}e^{-\left(\frac{r-\alpha}{\Xi}\right)^2}}\end{equation} 
which can be integrated to obtain the Mayer function
\begin{equation}{f(r)=\int_{\infty}^r{ \frac{1}{\Xi\sqrt{\pi}}e^{-\left(\frac{r'-\alpha}{\Xi}\right)^2}{dr'}}}\end{equation} 
The Mayer function for the error function potential $V_{erf}(r)$ is thus 
given by
\begin{align}  \label{eq:VerfMayerfunction}
     f(r)=-\frac{1}{2}\left[1-\operatorname{erf}\left(\frac{r-\alpha}{\Xi}\right)\right]
\end{align} 
Combining this result with the definition of the Mayer function 
\begin{align}f(r)=e^{-\frac{V(r)}{K_BT}}-1\end{align} 
gives an expression for the for the error function potential  $V(r)=V_{erf}(r)$ 
\begin{equation} \label{eq:Verf}
	V_{erf}(r)=-k_BT\ln\left[\frac{1}{2}\left(\operatorname{erf}\left(\frac{r-\alpha}{\Xi}\right)+1\right)\right]
\end{equation}  

\begin{figure}
  \centering
  \includegraphics[height=8cm]{plot_alpha.pdf}
  \caption{Shows the temperature dependence of $\alpha$ vs $k_BT$ 
  for $\sigma=1$ and $\epsilon=1$.}
  \label{fig:alphaXivsT}
\end{figure}

The parameter $\alpha(T)$ is derived by setting 
the error function potential $V_{erf}(r)$ 
equal to the WCA potential given by Equation~\ref{eq:VWCA} at $r=\alpha$. 
\begin{align}
    V_{erf}(\alpha)=V_{WCA}(\alpha)
\end{align}
This gives
\begin{equation}\label{alphaT}\alpha(T)=\sigma\left(\frac{2}{1+\sqrt{\frac{k_BT}{\epsilon}\ln(2)}}\right)^\frac{1}{6}\end{equation} 
Figure \ref{fig:alphaXivsT} shows how $\alpha(T)$ varies with temperature 
when $\sigma=1$ and $\epsilon=1$.    

\begin{figure}
  \centering
  \includegraphics[height=8cm]{xi_fromB2.pdf}
  \caption{Shows the temperature dependence of $\Xi$ generated from 
  matching the second virial coefficients for $V_{\operatorname{erf}}(r)$ and $V_{WCA}(r)$ 
  at each temperature.}
  \label{fig:xi_fromB2vsT}
\end{figure}

We determine the value of $\Xi(T)$ at a given temperature $T$,
by setting the second virial coefficient evaluated for the error 
function potential $V_{\operatorname{erf}}(r)$ equal to the second virial 
coefficient evaluated for the WCA potential $V_{WCA}(r)$.
\begin{equation}B_{2\operatorname{erf}}(T) =B_{2WCA}(T)\end{equation}
The second virial coefficient is given by
\begin{equation}B_2=-\frac{1}{2}\int_{Volume}f(\vec{r})d\vec r\end{equation}
Using the Mayer function for the error function potential given by 
Equation~\ref{eq:VerfMayerfunction} we found (see Appendix) the second 
virial coefficient $B_{2}$ for the error function potential $V_{erf}(r)$ 
to be
\begin{equation} \label{eq:B2erf}
	B_{2\operatorname{erf}}(T) = \frac{\pi}{3}\Xi^3\left[\left(\frac{\alpha^3}{\Xi^3}+\frac{3\alpha}{2\Xi}\right)\left(1+\operatorname{\operatorname{erf}}\left(\frac{\alpha}{\Xi}\right)\right)+\frac{1}{\sqrt{\pi}}\left(\frac{\alpha^2}{\Xi^2}+1\right)e^{-\left(\frac{\alpha}{\Xi}\right)^2}\right]
\end{equation}
where $\alpha(T)$ is given by Equation~\ref{alphaT}. Similarly, 
$B_{2WCA}$ can be found computationally from 
\begin{equation} \label{eq:B2wca}
	B_{2WCA}=-\frac{1}{2}\int_{Volume}\left(e^{-\beta{V}_{WCA}(r)}-1\right)d\vec{r} 
\end{equation}
The parameter $\Xi(T)$ is found to be the value that makes Equations~\ref{eq:B2erf} 
and \ref{eq:B2wca} equal each other for a given temperature T. 
A plot of $\Xi(T)$ obtained through this method is shown in 
Figure~\ref{fig:xi_fromB2vsT}.

\begin{figure}
  \centering
  \includegraphics[height=8cm]{weight_functions.pdf}
  \caption{Shows how the weight functions $w_0$, $w_1$, $w_2$, $w_3$ 
  vary along the z-axis for a reduced temperature of $K_BT=2$.}
  \label{fig:weight_functions}
\end{figure}

Figure \ref{fig:weight_functions} shows a one dimensional view of the 
weight functions $w_0$, $w_1$, $w_2$, $w_3$ as they vary with distance 
along the z-axis. In three dimensions, the weighting functions 
$w_0$, $w_1$, $w_2$ are spherical shells centered at $\vec{r}$, 
and $w_3$ is a solid sphere with an edge that tapers off.
$\Xi$ is related to the width of the weight functions which 
become thicker, or taper off longer as the temperature increases. 
$\alpha$ is related to the radius of the shell, or sphere, and decreases 
as the temperature increases.

\section{Computing the Liquid and Crystal Helmholtz free energies}

The liquid ideal Helmholtz free energy is computed analytically 
using Equation~\ref{eq:Fideal} 
for a given homogeneous reduced density $n^*$, and reduced temperature $t^*$. 
The crystal ideal Helmholtz free energy is computed computationally 
from Equation~\ref{eq:Fideal-n(r)} with the varying number density profile given by
\begin{align}
  n(|\vec{r}-\vec{R_j}|)=\sum_j{(1-f_v)\frac{e^{-\frac{{|\vec{r}-\vec{R_j}|}^2}{2{\sigma}^2}}}{\left(\sqrt{2\pi}\sigma\right)^3}}
\end{align}
where $\vec R_j$ is the position vector of the jth lattice site.
The liquid excess Helmholtz free energy is computed analytically  
using SFMT Equations~\ref{eq:Fexfunctional} through \ref{eq:tensorweight-methods} 
with a uniform number density profile, n that does not depend on $\vec r$. 

\begin{figure}
  \centering
  (a) \includegraphics[height=7cm]{figs/weightandGauss1.pdf}\\
  (b) \includegraphics[height=7cm]{figs/weightandGauss2.pdf}
  \caption{(a) A Gaussian distribution about a lattice site located at 
  $\vec R$ is shown in blue, and a cross-sectional 
  view of $w_2$ centered at $\vec r~'$ is shown in purple.
  (b) An alternative way of viewing the top figure, where 
  the weight function, shown by the purple dashed lines, is centered at $\vec{r}$,  
  and fixed in place as $\vec r~' - \vec R$
  varies randomly with a probability according to a Gaussian distribution.}
\label{fig:GaussandW2_actual}
\end{figure} 

% \begin{figure}
%   \centering
%   \includegraphics[height=7cm]{figs/weightandGauss2.pdf}
%   \caption{An alternative way of viewing Figure~\ref{fig:GaussandW2_actual} where 
%   the weight function, shown by the purple dashed lines, is centered at $\vec{r}$,  
%   and fixed in place as $\vec r~' - \vec R$
%   varies randomly with a probability according to a Gaussian distribution.} 
% \label{fig:GaussandW2_thinkas}
% \end{figure}

\begin{figure}
   \centering
   \includegraphics[height=5cm]{InclusionRadius.png}
   \caption{Gaussians are centered at each lattice point. After a radius 
   of a few Gaussian widths about each lattice point, the Gaussians go to 
   zero and can be neglected.}
   \label{fig:InclusionRadius}
\end{figure} 

The crystal excess Helmholtz free energy is computed  
using SFMT with a number density profile 
that consists of Gaussian distributions 
centered at lattice sites specified by position vectors $\vec R$. 
The weight functions $w_i(|\vec r - \vec r~'|)$ are centered at $\vec r'$,
as shown for $w_2$ in Figure~\ref{fig:GaussandW2_actual}a. 
But because the weight function is a function of the magnitude of the 
difference between $\vec r$ and $\vec r~'$, it can be thought of as 
centered at $\vec r$, as shown in Figure~\ref{fig:GaussandW2_actual}b.
Keeping the weight function centered at $\vec r$, 
Monte-Carlo integration can be used to compute the weighted densities $n_i(\vec r)$
by randomly generating vectors
$\vec r~' -\vec R$ according to a Gaussian distribution. 
Only lengths for $\vec r~'- \vec R$ that are within an inclusion radius
about a lattice site, like that shown in Figure \ref{fig:InclusionRadius}, 
will be taken into account as Gaussians 
farther away will not contribute much.
The equation for the contribution to the weighted density from a 
Gaussian centered at the jth lattice site located at $\vec R_j$ is given by
\begin{equation}{n_i(\vec r)= \int_{r'}{(1-f_v)n(\vec r~' - \vec R_j)w_i(|\vec{r}-\vec{r}~'|)} {d}\vec{r}~'}\end{equation} 
and is implemented computationally in the form
\begin{equation}{n_i(\vec r)= \sum_{r'}{(1-f_v)n(\vec r~' -\vec R_j)w_i(|\vec{r}-\vec{r}~'|)dV}}\end{equation} 
which can be written as
\begin{equation}{n_i(\vec r)= \sum_{NumPoints}(1-f_v)P_{r'}w_i(|\vec{r}-\vec{r}~'|)}\end{equation} 
NumPoints is the number of vectors $\vec r~' - \vec R_j$ randomly 
generated about the jth lattice site, and $n(\vec r~' - \vec R_j)$
is equal to a probability density which when multiplied by dV gives 
the probability $P_{r'}$ at which the value of $w_i(|\vec{r}-\vec{r}~'|)$ 
will occur. Using the relation
\begin{equation}{f_{average}=\sum_i^N{P_if_i}=\sum_i{\frac{f_i}{N}}}\end{equation} 
$n_i(\vec{r})$ can be expressed as the average value of $(1-f_v)w_i(|\vec{r}-\vec{r}~'|)$:
\begin{equation}{n_i(\vec{r})=\sum_{\text{NumPoints}}\frac{(1-f_v)w_i(|\vec{r}-\vec{r}~'|)}{\text{NumPoints}}}\end{equation}
This process is repeated to get the contributions to the weighted density 
$n_i(\vec r)$ from the Gaussians around each lattice point which begin to 
merge together as the width of the Gaussians increases.
All the contributions are then added together 
to get the final value of the weighted density at some point $\vec r$.

The result can be made more accurate by increasing NumPoints (the 
number of vectors randomly generated), and by using antithetic variates.
To use antithetic variates, each randomly generated  vector is paired 
with a vector of equal magnitude, but opposite in direction. 
First order error is reduced by averaging the two results that depend on 
the vectors, one of which may be erroneously higher than it should be 
while the other is erroneously lower than it should be.

The standard error of the mean (SEM) is given in general by
\begin{equation}{SD_{mean}=\frac{SD}{\sqrt{N}}}\end{equation} 
To compute the error, the standard deviation of weight function $w_3$ is used 
\begin{equation}SD=\sqrt{\overline{w_3^2}-(\overline{w}_3)^2}\end{equation} 
\begin{equation}{Error=\sqrt{\frac{\overline{w_3^2}-(\overline{w}_3)^2}{\text{NumPoints}}}}\end{equation} 
NumPoints, the number of vectors randomly generated, is increased until
\begin{equation}{Error<\frac{|1-n_3|}{4}}\end{equation}
where the maximum value that the weighted density $n_3$ 
can properly have is 1.

\chapter{Results and Discussion}

Phase diagrams for the WCA fluid are shown in Figures~\ref{fig:Phase_Diagram_of_T_vs_n} 
and \ref{fig:Phase_Diagram_P_vs_T}. In all cases $\sigma=1$ and $\epsilon=1$. 
In Figure~\ref{fig:Phase_Diagram_of_T_vs_n} the reduced temperature 
$T^*=\frac{kT}{\epsilon}$ is plotted against the reduced number density 
$n^*=n\sigma^3$ for temperatures of kT = 0.5 to 3.0. 
It is seen that in general, a liquid exists at lower number densities, 
and a solid exists at higher number densities. 
For a given temperature, it is seen that as the density of the 
liquid increases, a maximum number density is reached for the liquid. 
Looking at the coexistence region as a whole for all temperatures, 
it is seen to extend over a narrow range of number densities which 
gradually shifts towards higher densities as the temperature increases. 
This is expected since at higher temperatures the fluid would have to 
be under greater compression before forming a solid. 

White lines superimposed on the phase diagram show the boundaries of the 
coexistence region as predicted by Monte-Carlo simulations~\cite{May} 
of the WCA fluid. 
Overall, the general shape, and position of the two coexistence regions 
correspond fairly closely, but the Monte-Carlo region is a little wider
and shifted a bit toward smaller densities. For example, at a temperature 
of kT=1, the cDFT liquid transition reduced number density is 0.96, 
and the solid transition reduced number density is 1.02, giving a spread 
of 0.06, whereas the Monte-Carlo results give a liquid transition 
reduced number density of 0.92, and a solid transition reduced number 
density of 1.00, giving a spread of 0.08. 

\begin{figure}
  \centering
  \includegraphics[height=10cm]{figs/Phase_Diagram_of_T_vs_n}
  \caption{Phase Diagram of the reduced temperature verses the reduced 
  density for the WCA fluid. The red region is liquid, and the blue region is 
  solid. The gray area that separates the liquid and solid regions 
  is the region of coexistence where some liquid and solid exist together 
  at varying proportions. White lines show the boundaries of the coexistence 
  region generated from Monte-Carlo simulations of the WCA fluid.}
  \label{fig:Phase_Diagram_of_T_vs_n}
\end{figure}

%The solid phase transition number densities predicted by each method 
%are very nearly the same along all temperatures, but the liquid phase 
%transition number densities predicted by cDFT are notably higher than 
%those predicted by the Monte-Carlo simulations, making the width of the 
%Monte-Carlo region about one-third larger for most temperatures above KT=1. 
%Figure~\ref{fig:Phase_Diagram_of_T_vs_n_AS} shows our results for a wider 
%range of reduced temperatures from kT=0.5 to 38. The white lines show the results
%of molecular simulations of the WCA fluid made by Ahmed and Sadus~\cite{ahmedsadus}.
%As can be seen in the figure, the region of coexistence in our simulations 
%falls within the region of coexistence predicted by Ahmed and Sadus at almost 
%all temperatures from approximately kT=0.5 to 28. At the low temperature of kT=0.5,
%the width of the coexistence regions are nearly the same, while moving to 
%higher temperatures,the width of our coexistence region ranges from about 
%one-half to one-forth the size. Although, our coexistence region is only 
%about one-forth the size at kT=28, it is centered within the coexistence 
%region predicted by Ahmed and Sadus. 
%
%
%
%\begin{figure}
  %\centering
  %%\includegraphics[width=\columnwidth]{figs/Phase_Diagram_of_T_vs_n_AS}
  %%\includegraphics[height=10cm]{figs/Phase_Diagram_of_T_vs_n_AS}
  %%\caption{Phase Diagram of the reduced temperature verses the reduced density for the WCA fluid. 
  %%White lines show the liquid and solid transition densities at each temperature from 
  %%simulations of a WCA fluid by Ahmed and Sadus~\cite{ahmedsadus}. The gray region is the region of coexistence.}
  %\label{fig:Phase_Diagram_of_T_vs_n_AS}
%\end{figure}

Figure~\ref{fig:Phase_Diagram_P_vs_T} shows the reduced pressure 
$p^*=\frac{P\sigma^3}{\epsilon}$ plotted against the reduced temperature 
for temperatures of kT = 0.5 to 3.0. 
The black line marks the phase boundary between the solid and liquid regions. 
It is seen that the higher the temperature, the higher the pressure must be
to undergo a phase transition from a liquid to a solid, as would be expected. 
Ideally, the slope of the phase boundary $\frac{dP}{dT}$ at some point (P,T) is given by
\begin{align} 
    \frac{dP}{dT}=\frac{S_l-S_s}{V_l-V_s}
\end{align}
Since the entropy and the volume of the liquid is greater than the entropy
and the volume of the solid, the slope should be positive. 

Comparing the slope of the phase boundary predicted by cDFT to that
from Monte-Carlo simulations shown by the white line, it is seen that
the cDFT transition temperatures are consistently higher, and the 
discrepancy grows as the temperature increases. For example, at a 
temperature of kT=1, the phase transition reduced pressure predicted by 
cDFT is 15.24, whereas the Monte-Carlo simulation predicted a value of 9.99,
and at a temperature of kT=3, cDFT gives a value of 77.27 whereas the 
Monte-Carlo simulation predicts 43.80.
This may indicate that either the difference between the liquid 
and solid phase transition entropies predicted by cDFT is too high, or 
the difference in the volumes is too low. 

\begin{figure}
  \centering
  \includegraphics[height=10cm]{figs/Phase_Diagram_of_P_vs_T}
  \caption{Phase Diagram of the reduced pressure verses the reduced 
  temperature. The blue region is solid, the red region is liquid,
  and the black line shows the phase boundary where the solid and liquid
  phases coexist. The white line marks the solid-liquid phase boundary  
  as predicted by Monte-Carlo simulations.}
  \label{fig:Phase_Diagram_P_vs_T}
\end{figure}
%
%Figure~\ref{fig:Phase_Diagram_P_vs_T_AS} shows our results for a larger 
%range of reduced temperatures from kT=0.5 to 38, compared with the results
%from Ahmed and Sadus's molecular simulations. Here again, our transition 
%pressures and temperatures are consistently higher.
%
%
\begin{figure}
  \centering
  %\includegraphics[width=\columnwidth]{figs/Phase_Diagram_of_P_vs_T_AS}
  %\includegraphics[height=10cm]{figs/Phase_Diagram_of_P_vs_T_AS}
  %\caption{Phase Diagram of the reduced pressure verses the reduced temperature. 
  %The white line marks the boundary from simulations by Ahmed and Sadus~\cite{ahmedsadus}.}
  \label{fig:Phase_Diagram_P_vs_T_AS}
\end{figure}
%
%Figure~\ref{fig:p-vs-T_at_fixed_density} shows individual plots of the 
%reduced pressure versus the reduced temperature at fixed reduced densities 
%from $n^*$ = 0.7 to 1.2. As can be seen by comparison with Figure 
%\ref{fig:Phase_Diagram_P_vs_T_AS}, the plotted lines display kinks when 
%the WCA fluid transitions into a solid. These downward bends align to 
%form the curved interface separating the blue and red areas in 
%Figures~\ref{fig:Phase_Diagram_P_vs_T} and \ref{fig:Phase_Diagram_P_vs_T_AS}.
%
\begin{figure}
  \centering
  %\includegraphics[width=\columnwidth]{figs/p-vs-T_at_fixed_n_noMC}
  %\caption{Plot of the reduced pressure verses the reduced temperature. 
  %The lines bending downward indicate freezing.}
  \label{fig:p-vs-T_at_fixed_density}
\end{figure}


\chapter{Conclusion}
We were successful in predicting the freezing of a Weeks-Chandler-Anderson 
fluid using classical Density Functional Theory. To construct a Helmholtz 
free energy functional $F[n(r)]$, we used SFMT with Schmidt's error 
function model, and defined two temperature dependent parameters, $\alpha$ 
and $\Xi$, to model a WCA fluid. 
$\alpha$ was found by setting the error function potential equal to the 
WCA potential at $r=\alpha$, and $\Xi$ was found for each temperature by 
setting the second virial coefficients for the error function potential 
and the WCA potential equal to each other, $B_{2erf}(T)=B_{2wca}(T)$.
To implement cDFT we used Gaussian distribution functions centered at 
lattice points to represent the number density profile which was then varied
by changing the width of the Gaussians, and the fraction of vacant lattice
sites until the Helmholtz free energy was minimized. Once the equilibrium 
Helmholtz free energy for the liquid and solid states were found for a given
temperature and number density, Maxwell's Double Tangent Construction 
was used to find the number densities of the liquid and the solid at 
the phase transition.
Using this information, we generated a temperature-density 
phase diagram that yielded coexistence regions that approximately 
matched the Monte-Carlo
% and molecular simulation 
results, though shifted a bit toward higher densities. 
%for values of kT=0.5 to 3. 
%kT=0.5 to 38
We also generated a pressure-temperature
phase diagram, but the slope was notably steeper than that
predicted from Monte-Carlo simulations, giving transition 
temperatures and pressures that were too high. 



\backmatter

\chapter{Appendix}

\section{Fourier Transform of Weight Function $w_0(r)$}
\begin{equation}{w_0(r)=\frac{w_2(r)}{4{\pi}r^2}}\end{equation}
where
\begin{equation}{w_2(r)=\frac{\sqrt{2}}{\Xi\sqrt{\pi}}e^{-\left(\frac{r-\frac{\alpha}{2}}{\frac{\Xi}{\sqrt{2}}}\right)^2}}\end{equation}
temporarily set 
\begin{equation}{a=\frac{\Xi}{\sqrt{2}}}\end{equation}
\begin{equation}{w_0(r)=\frac{1}{4{\pi}\sqrt{\pi}a}\left(\frac{1}{r^2}\right)e^{-\left(\frac{r-\frac{\alpha}{2}}{a}\right)^2}}\end{equation}

\begin{equation}{\widetilde{w}_0(\vec{k})=\int_{0}^{\infty}\int_{-1}^{1}\int_{0}^{2\pi}w_0(r)e^{-i\vec{k}\cdot{\vec{r}}}r^2d{r}d{\cos\theta}d{\phi}}\end{equation}
Set $\vec{k}$ in the direction of $\hat{z}$ 
\begin{equation}{\widetilde{w}_0(k)=\int_{0}^{\infty}\int_{-1}^{1}\int_{0}^{2\pi}w_0(r)e^{-ikr\cos\theta}r^2d{r}d{\cos\theta}d{\phi}}\end{equation}

Calculate $\widetilde{w}_0(k)$ 
\begin{equation}{\widetilde{w}_0(k)=\frac{1}{4{\pi}\sqrt{\pi}a}\int_{0}^{\infty}\left(\frac{1}{r^2}\right)e^{-\left(\frac{r-\frac{\alpha}{2}}{a}\right)^2}r^2\left[\int_{-1}^{1}e^{-ikr\cos\theta}d{\cos\theta}\left(\int_{0}^{2\pi}d{\phi}\right)\right]d{r}}\end{equation}

\begin{equation}{\widetilde{w}_0(k)=\frac{2\pi}{4{\pi}\sqrt{\pi}a}\int_{0}^{\infty}e^{-\left(\frac{r-\frac{\alpha}{2}}{a}\right)^2}\left(\int_{-1}^{1}e^{-ikr\cos\theta}d{\cos\theta}\right)d{r}}\end{equation}

Set the lower limit to $-\infty$  since the integrand is nearly zero by the time r is zero. 
\begin{equation}{\widetilde{w}_0(k)=\frac{1}{\sqrt{\pi}ka}\int_{-\infty}^{\infty}e^{-\left(\frac{r-\frac{\alpha}{2}}{a}\right)^2}\frac{\sin(kr)}{r}d{r}}\end{equation}

\begin{equation}{\widetilde{w}_0(k)=\frac{\sqrt{\pi}}{2ka}e^{-\left(\frac{\alpha}{2a}\right)^2}\left[\operatorname{erf}\left(\frac{ka}{2}+i\frac{\alpha}{2a}\right)+\operatorname{erf}\left(\frac{ka}{2}-i\frac{\alpha}{2a}\right)\right]}\end{equation}
 
Putting in $a=\frac{\Xi}{\sqrt{2}}$ the final equation for $\widetilde{w}_0(k)$ becomes
\begin{equation}{\widetilde{w}_0(k)=\frac{\sqrt{\pi}}{\sqrt{2}k\Xi}e^{-\left(\frac{\alpha^2}{2\Xi^2}\right)}\left[\operatorname{erf}\left(\frac{k\Xi}{2\sqrt{2}}+i\frac{\alpha}{\sqrt{2}\Xi}\right)+\operatorname{erf}\left(\frac{k\Xi}{2\sqrt{2}}-i\frac{\alpha}{\sqrt{2}\Xi}\right)\right]}\end{equation}

\section{Fourier Transform of Weight Function $w_{1}(r)$}
\begin{equation}{w_1(r)=\frac{w_2(r)}{4{\pi}r}}\end{equation}
where
\begin{equation}{w_2(r)=\frac{\sqrt{2}}{\Xi\sqrt{\pi}}e^{-\left(\frac{r-\frac{\alpha}{2}}{\frac{\Xi}{\sqrt{2}}}\right)^2}}\end{equation}
temporarily set 
\begin{equation}{a=\frac{\Xi}{\sqrt{2}}}\end{equation}
\begin{equation}{w_1(r)=\frac{1}{4{\pi}\sqrt{\pi}a}\left(\frac{1}{r}\right)e^{-\left(\frac{r-\frac{\alpha}{2}}{a}\right)^2}}\end{equation}

\begin{equation}{\widetilde{w}_1(\vec{k})=\int_{0}^{\infty}\int_{-1}^{1}\int_{0}^{2\pi}w_1(r)e^{-i\vec{k}\cdot{\vec{r}}}r^2d{r}d{\cos\theta}d{\phi}}\end{equation}
Set $\vec{k}$ in the direction of $\hat{z}$ 
\begin{equation}{\widetilde{w}_1(k)=\int_{0}^{\infty}\int_{-1}^{1}\int_{0}^{2\pi}w_1(r)e^{-ikr\cos\theta}r^2d{r}d{\cos\theta}d{\phi}}\end{equation}

\noindent Calculate $\widetilde{w}_1(k)$ 
\begin{equation}{\widetilde{w}_1(k)=\frac{1}{4{\pi}\sqrt{\pi}a}\int_{0}^{\infty}\left(\frac{1}{r}\right)e^{-\left(\frac{r-\frac{\alpha}{2}}{a}\right)^2}r^2\left[\int_{-1}^{1}e^{-ikr\cos\theta}d{\cos\theta}\left(\int_{0}^{2\pi}d{\phi}\right)\right]d{r}}\end{equation}

\begin{equation}{\widetilde{w}_1(k)=\frac{2\pi}{4{\pi}\sqrt{\pi}a}\int_{0}^{\infty}e^{-\left(\frac{r-\frac{\alpha}{2}}{a}\right)^2}r\left(\int_{-1}^{1}e^{-ikr\cos\theta}d{\cos\theta}\right)d{r}}\end{equation}

\begin{equation}{\widetilde{w}_1(k)=\frac{2}{4\sqrt{\pi}a}\int_{0}^{\infty}e^{-\left(\frac{r-\frac{\alpha}{2}}{a}\right)^2}r\frac{2\sin(kr)}{kr}d{r}}\end{equation}
Set the lower limit to $-\infty$ since the integrand is nearly zero by the time r is zero. 
\begin{equation}{\widetilde{w}_1(k)=\frac{1}{\sqrt{\pi}ka}\int_{-\infty}^{\infty}e^{-\left(\frac{r-\frac{\alpha}{2}}{a}\right)^2}\sin(kr)d{r}}\end{equation}
After doing a u-substitution with $u=\frac{r-\frac{\alpha}{2}}{a}$ this becomes
\begin{equation}{\widetilde{w}_1(k)=\frac{1}{k}\sin\left(\frac{k\alpha}{2}\right)e^{-\left(\frac{k^2a^2}{4}\right)}}\end{equation}
Putting $a=\frac{\Xi}{\sqrt{2}}$ back in, the final equation for $\widetilde{w}_1(k)$ becomes
\begin{equation}
    \widetilde{w}_1(k)=\frac{1}{k}e^{-\frac{k^2\Xi^2}{8}}\sin\left(\frac{k\alpha}{2}\right)
\end{equation}

\section{Fourier Transform of Weight Function $w_{2}(r)$}
\begin{equation}{w_2(r)=\frac{\sqrt{2}}{\Xi\sqrt{\pi}}e^{-\left(\frac{r-\frac{\alpha}{2}}{\frac{\Xi}{\sqrt{2}}}\right)^2}}\end{equation}
set 
\begin{equation}{a=\frac{\Xi}{\sqrt{2}}}\end{equation}
\begin{equation}{w_2(r)=\frac{1}{a\sqrt{\pi}}e^{-\left(\frac{r-\frac{\alpha}{2}}{a}\right)^2}}\end{equation}

\begin{equation}{\widetilde{w}_2(\vec{k})=\int_{0}^{\infty}\int_{-1}^{1}\int_{0}^{2\pi}w_2(r)e^{-i\vec{k}\cdot{\vec{r}}}r^2d{r}d{\cos\theta}d{\phi}}\end{equation}
Set $\vec{k}$ in the direction of $\hat{z}$ 
\begin{equation}{\widetilde{w}_2(k)=\int_{0}^{\infty}\int_{-1}^{1}\int_{0}^{2\pi}w_2(r)e^{-ikr\cos\theta}r^2d{r}d{\cos\theta}d{\phi}}\end{equation}

\noindent Calculate $\widetilde{w}_2(k)$ 
\begin{equation}{\widetilde{w}_2(k)=\frac{1}{a\sqrt{\pi}}\int_{0}^{\infty}e^{-\left(\frac{r-\frac{\alpha}{2}}{a}\right)^2}r^2\left(\int_{-1}^{1}e^{-ikr\cos\theta}d{\cos\theta}\left(\int_{0}^{2\pi}d{\phi}\right)\right)d{r}}\end{equation}

\begin{equation}{\widetilde{w}_2(k)=\frac{2\pi}{a\sqrt{\pi}}\int_{0}^{\infty}e^{-\left(\frac{r-\frac{\alpha}{2}}{a}\right)^2}r^2\left(\int_{-1}^{1}e^{-ikr\cos\theta}d{\cos\theta}\right)d{r}}\end{equation}
\begin{equation}{\widetilde{w}_2(k)=\frac{2\sqrt{\pi}}{a}\int_{0}^{\infty}e^{-\left(\frac{r-\frac{\alpha}{2}}{a}\right)^2}r^2\frac{2\sin(kr)}{kr}d{r}}\end{equation}
Set the lower limit to $-\infty$  since the integrand is nearly zero by the time r is zero.
\begin{equation}{\widetilde{w}_2(k)=\frac{4\sqrt{\pi}}{ka}\int_{-\infty}^{\infty}e^{-\left(\frac{r-\frac{\alpha}{2}}{a}\right)^2}r\sin(kr)d{r}}\end{equation}
After doing a u-substitution with $u=\frac{r-\frac{\alpha}{2}}{a}$ this becomes
\begin{equation}{\widetilde{w}_2(k)=\frac{2\pi}{k}e^{-\left(\frac{k^2a^2}{4}\right)}\left(ka^2\cos\left(\frac{k\alpha}{2}\right)+\alpha\sin\left(\frac{k\alpha}{2}\right)\right)}\end{equation}
Putting $a=\frac{\Xi}{\sqrt{2}}$ back in, the final equation for $\widetilde{w}_2(k)$ becomes
\begin{equation}{\widetilde{w}_2(k)=\frac{2\pi}{k}e^{-\frac{k^2\Xi^2}{8}}\left[\frac{k\Xi^2}{2}\cos\left(\frac{k\alpha}{2}\right)+\alpha\sin\left(\frac{k\alpha}{2}\right)\right]}\end{equation}

\section{Fourier Transform of Weight Function $w_{3}(r)$}
\begin{equation}{w_3(r)=\frac{1}{2}\left[1-\operatorname{erf}\left(\frac{r-\frac{\alpha}{2}}{\frac{\Xi}{\sqrt{2}}}\right)\right]}\end{equation}
temporarily set 
\begin{equation}{a=\frac{\Xi}{\sqrt{2}}}\end{equation}
\begin{equation}{w_3(r)=\frac{1}{2}\left[1-\operatorname{erf}\left(\frac{r-\frac{\alpha}{2}}{a}\right)\right]}\end{equation}

\begin{equation}{\widetilde{w}_3(\vec{k})=\int_{0}^{\infty}\int_{-1}^{1}\int_{0}^{2\pi}w_3(r)e^{-i\vec{k}\cdot{\vec{r}}}r^2d{r}d{\cos\theta}d{\phi}}\end{equation}
Set $\vec{k}$ in the direction of $\hat{z}$ 
\begin{equation}{\widetilde{w}_3(k)=\int_{0}^{\infty}\int_{-1}^{1}\int_{0}^{2\pi}w_3(r)e^{-ikr\cos\theta}r^2d{r}d{\cos\theta}d{\phi}}\end{equation}

\noindent Calculate $\widetilde{w}_3(k)$ 
\begin{equation}{\widetilde{w}_3(k)=\frac{1}{2}\int_{0}^{\infty}\left[1-\operatorname{erf}\left(\frac{r-\frac{\alpha}{2}}{a}\right)\right]r^2\left(\int_{-1}^{1}e^{-ikr\cos\theta}d{\cos\theta}\left(\int_{0}^{2\pi}d{\phi}\right)\right)d{r}}\end{equation}

\begin{equation}{\widetilde{w}_3(k)=\frac{2\pi}{2}\int_{0}^{\infty}\left[1-\operatorname{erf}\left(\frac{r-\frac{\alpha}{2}}{a}\right)\right]r^2\left(\int_{-1}^{1}e^{-ikr\cos\theta}d{\cos\theta}\right)d{r}}\end{equation}

\begin{equation}{\widetilde{w}_3(k)=\pi\int_{0}^{\infty}\left[1-\operatorname{erf}\left(\frac{r-\frac{\alpha}{2}}{a}\right)\right]r^2\frac{2\sin(kr)}{kr}d{r}}\end{equation}

\begin{equation}{\widetilde{w}_3(k)=\frac{2\pi}{k}\int_{0}^{\infty}\left[1-\operatorname{erf}\left(\frac{r-\frac{\alpha}{2}}{a}\right)\right]r\sin(kr)d{r}}\end{equation}
Integrating by parts with 
\begin{displaymath}{f=1-\operatorname{erf}\left(\frac{r-\frac{\alpha}{2}}{a}\right)}\end{displaymath}
\begin{displaymath}{g=\frac{\sin(kr)-kr\cos(kr)}{k^2}}\end{displaymath}

\begin{multline}
  \widetilde{w}_3(k)=\frac{2\pi}{k}\Bigg[\left(1-\operatorname{erf}\left(\frac{r-\frac{\alpha}{2}}{a}\right)\right)\left(\frac{\sin(kr)-kr\cos(kr)}{k^2}\right)\bigg|^{\infty}_0
   \\
  -\int_{0}^{\infty}\left(\frac{\sin(kr)-kr\cos(kr)}{k^2}\right)\left(\frac{2}{a\sqrt{\pi}}e^{-\left(\frac{r-\frac{\alpha}{2}}{a}\right)^2}\right)dr\Bigg]
\end{multline}
Evaluating the integral by doing a u-substitution with $u=\frac{r-\frac{\alpha}{2}}{a}$, 
and setting the lower limit to $-\infty$  since the integrand is 
nearly zero by the time r is zero this becomes
\begin{equation}{\widetilde{w}_3(k)=\frac{4\pi}{k^3}e^{-\frac{k^2a^2}{4}}\left[\left(1+\frac{k^2a^2}{2}\right)\sin\left(\frac{k\alpha}{2}\right)-\frac{k\alpha}{2}\cos\left(\frac{k\alpha}{2}\right)\right]}\end{equation}
Putting $a=\frac{\Xi}{\sqrt{2}}$ back in, the final equation for 
$\widetilde{w}_3(k)$ becomes
\begin{equation}{\widetilde{w}_3(k)=\frac{4\pi}{k^3}e^{-\frac{k^2\Xi^2}{8}}\left[\left(1+\frac{k^2\Xi^2}{4}\right)\sin\left(\frac{k\alpha}{2}\right)-\frac{k\alpha}{2}\cos\left(\frac{k\alpha}{2}\right)\right]}\end{equation}

\section{Fourier Transform of Vector Weight Function $\vec{w}_{1}(r)$}
\begin{equation}{\vec{w}_1(\vec{r})=w_1(r)\frac{\vec{r}}{r}}\end{equation}
where
\begin{equation}{w_1(r)=\frac{w_2(r)}{4{\pi}r}}\end{equation}
\begin{equation}{w_2(r)=\frac{\sqrt{2}}{\Xi\sqrt{\pi}}e^{-\left(\frac{r-\frac{\alpha}{2}}{\frac{\Xi}{\sqrt{2}}}\right)^2}}\end{equation}
temporarily set 
\begin{equation}{a=\frac{\Xi}{\sqrt{2}}}\end{equation}
\begin{equation}{w_1(r)=\frac{1}{4{\pi}\sqrt{\pi}a}\left(\frac{1}{r}\right)e^{-\left(\frac{r-\frac{\alpha}{2}}{a}\right)^2}}\end{equation}

\begin{equation}{\widetilde{\vec{w}}_1(\vec{k})=\int_{0}^{\infty}\int_{-1}^{1}\int_{0}^{2\pi}\vec{w}_1(\vec{r})e^{-i\vec{k}\cdot{\vec{r}}}r^2d{r}d{\cos\theta}d{\phi}}\end{equation}

\begin{equation}{\widetilde{\vec{w}}_1(\vec{k})=\int_{0}^{\infty}\int_{-1}^{1}\int_{0}^{2\pi}w_1(r){~}\frac{\vec{r}}{r}{~}e^{-i\vec{k}\cdot{\vec{r}}}r^2d{r}d{\cos\theta}d{\phi}}\end{equation}
Set $\vec{k}$ in the direction of $\hat{z}'$ of a rotated 
coordinate system $(x',y'z')$ 
\begin{equation}{\widetilde{\vec{w}}_1(\vec{k})=\int_{0}^{\infty}\int_{-1}^{1}\int_{0}^{2\pi}w_1(r)\frac{(r_{x'}\hat{x}'+r_{y'}\hat{y}'+r_{z'}\hat{z}')}{r}e^{-ikr\cos\theta'}r^2d{r}d{\cos\theta'}d{\phi'}}\end{equation}
where
\begin{displaymath}{r_{x'}=r\sin\theta'\cos\phi'}\end{displaymath}
\begin{displaymath}{r_{y'}=r\sin\theta'\sin\phi'}\end{displaymath}
\begin{displaymath}{r_{z'}=r\cos\theta'}\end{displaymath} 
The components in the $x'$ and $y'$ directions vanish due to symmetry 
about $z'$. That leaves
\begin{equation}{\widetilde{\vec{w}}_1(\vec{k})=\int_{0}^{\infty}\int_{-1}^{1}\int_{0}^{2\pi}w_1(r)\cos{\theta}'e^{-ikr\cos\theta'}r^2d{r}d{\cos\theta'}d{\phi'}{~~}\hat{z}'}\end{equation}

\begin{equation}{\widetilde{\vec{w}}_1(\vec{k})=\frac{1}{4\pi\sqrt{\pi}a}\int_{0}^{\infty}\int_{-1}^{1}\int_{0}^{2\pi}e^{-\left(\frac{r-\frac{\alpha}{2}}{a}\right)^2}r\cos{\theta}'e^{-ikr\cos\theta'}d{r}d{\cos\theta'}d{\phi'}{~~}\hat{z}'}\end{equation}

\begin{equation}{\widetilde{\vec{w}}_1(\vec{k})=\frac{2}{4\sqrt{\pi}a}\int_{0}^{\infty}e^{-\left(\frac{r-\frac{\alpha}{2}}{a}\right)^2}r\left(\int_{-1}^{1}e^{-ikr\cos\theta'}\cos{\theta}'d{\cos\theta'}\right)d{r}{~~}\hat{z}'}\end{equation}

\begin{equation}{\widetilde{\vec{w}}_1(\vec{k})=\frac{2}{4\sqrt{\pi}a}\int_{0}^{\infty}e^{-\left(\frac{r-\frac{\alpha}{2}}{a}\right)^2}r\left(-\frac{2i}{kr}\cos(kr)+\frac{2i}{k^2r^2}\sin(kr)\right)d{r}{~~}\hat{z}'}\end{equation}

\begin{equation}{\widetilde{\vec{w}}_1(\vec{k})=\frac{i}{k\sqrt{\pi}a}\left[-\int_{0}^{\infty}e^{-\left(\frac{r-\frac{\alpha}{2}}{a}\right)^2}\cos(kr)d{r}+\frac{1}{k}\int_{0}^{\infty}e^{-\left(\frac{r-\frac{\alpha}{2}}{a}\right)^2}\frac{\sin(kr)}{r}d{r}\right]{~~}\hat{z}'}\end{equation}

\begin{equation}{\widetilde{\vec{w}}_1(\vec{k})=\frac{i}{k\sqrt{\pi}a}\left[-a\sqrt{\pi}e^{-\frac{k^2a^2}{4}}\cos\left(\frac{k\alpha}{2}\right)+\frac{\pi}{2k}e^{-\left(\frac{\alpha}{2a}\right)^2}\left[\operatorname{erf}\left(\frac{ka}{2}+\frac{i\alpha}{2a}\right)+\operatorname{erf}\left(\frac{ka}{2}-\frac{i\alpha}{2a}\right)\right]\right]{~~}\hat{z}'}\end{equation}

\begin{equation}{\widetilde{\vec{w}}_1(\vec{k})=\frac{i}{k^2}\left[\frac{\sqrt{\pi}}{2ka}e^{-\left(\frac{\alpha}{2a}\right)^2}\left[\operatorname{erf}\left(\frac{ka}{2}+\frac{i\alpha}{2a}\right)+\operatorname{erf}\left(\frac{ka}{2}-\frac{i\alpha}{2a}\right)\right]-e^{-\frac{k^2a^2}{4}}\cos\left(\frac{k\alpha}{2}\right)\right]{~~}\vec{k}}\end{equation}
Putting $a=\frac{\Xi}{\sqrt{2}}$ back in, the final equation for 
$\widetilde{\vec{w}}_1(k)$ becomes
\begin{equation}{\widetilde{\vec{w}}_1(\vec{k})=\frac{i}{k^2}\left[\frac{\sqrt{\pi}}{\sqrt{2}k\Xi}e^{-\left(\frac{\alpha}{\sqrt{2}\Xi}\right)^2}\left[\operatorname{erf}\left(\frac{k\Xi}{(\sqrt{2})^3}+\frac{i\alpha}{\sqrt{2}\Xi}\right)+\operatorname{erf}\left(\frac{k\Xi}{(\sqrt{2})^3}-\frac{i\alpha}{\sqrt{2}\Xi}\right)\right]-e^{-\frac{k^2\Xi^2}{8}}\cos\left(\frac{k\alpha}{2}\right)\right]{~~}\vec{k}}\end{equation}

\section{Fourier Transform of Vector Weight Function $\vec{w}_{2}(r)$}
\begin{equation}{\vec{w}_2(\vec{r})=w_2(r)\frac{\vec{r}}{r}}\end{equation}
where
\begin{equation}{w_2(r)=\frac{\sqrt{2}}{\Xi\sqrt{\pi}}e^{-\left(\frac{r-\frac{\alpha}{2}}{\frac{\Xi}{\sqrt{2}}}\right)^2}}\end{equation}
temporarily set 
\begin{equation}{a=\frac{\Xi}{\sqrt{2}}}\end{equation}
\begin{equation}{w_2(r)=\frac{1}{a\sqrt{\pi}}e^{-\left(\frac{r-\frac{\alpha}{2}}{a}\right)^2}}\end{equation}

\begin{equation}{\widetilde{\vec{w}}_2(\vec{k})=\int_{0}^{\infty}\int_{-1}^{1}\int_{0}^{2\pi}w_2(r){~}\frac{\vec{r}}{r}{~}e^{-i\vec{k}\cdot{\vec{r}}}r^2d{r}d{\cos\theta}d{\phi}}\end{equation}
Set $\vec{k}$ in the direction of $\hat{z}'$ of a rotated coordinate 
system $(x',y'z')$ 
\begin{equation}{\widetilde{\vec{w}}_2(\vec{k})=\int_{0}^{\infty}\int_{-1}^{1}\int_{0}^{2\pi}w_2(r)\frac{(r_{x'}\hat{x}'+r_{y'}\hat{y}'+r_{z'}\hat{z}')}{r}e^{-ikr\cos\theta'}r^2d{r}d{\cos\theta'}d{\phi'}}\end{equation}
where 
\begin{displaymath}{r_{x'}=r\sin\theta'\cos\phi'}\end{displaymath}
\begin{displaymath}{r_{y'}=r\sin\theta'\sin\phi'}\end{displaymath}
\begin{displaymath}{r_{z'}=r\cos\theta'}\end{displaymath} 
The components in the $x'$ and $y'$ directions vanish due to symmetry 
about $z'$. That leaves
\begin{equation}{\widetilde{\vec{w}}_2(\vec{k})=\int_{0}^{\infty}\int_{-1}^{1}\int_{0}^{2\pi}w_2(r)\cos{\theta}'e^{-ikr\cos\theta'}r^2d{r}d{\cos\theta'}d{\phi'}{~~}\hat{z}'}\end{equation}

\begin{equation}{\widetilde{\vec{w}}_2(\vec{k})=\frac{2\sqrt{\pi}}{a}\int_{0}^{\infty}e^{-\left(\frac{r-\frac{\alpha}{2}}{a}\right)^2}r^2\left(\int_{-1}^{1}e^{-ikr\cos\theta'}\cos{\theta}'d{\cos\theta'}\right)d{r}{~~}\hat{z}'}\end{equation}

\begin{equation}{\widetilde{w}_2(\vec{k})=\frac{2\sqrt{\pi}}{a}\int_{0}^{\infty}e^{-\left(\frac{r-\frac{\alpha}{2}}{a}\right)^2}r^2\left(-\frac{2i}{kr}\cos(kr) + \frac{2i}{k^2r^2}\sin(kr)\right)d{r}{~~}\hat{z}'}\end{equation}

\begin{equation}{\widetilde{\vec{w}}_2(\vec{k})=\frac{-4i\sqrt{\pi}}{ak}\int_{0}^{\infty}e^{-\left(\frac{r-\frac{\alpha}{2}}{a}\right)^2}r\cos(kr)d{r} + \frac{4i\sqrt{\pi}}{ak^2}\int_{0}^{\infty}e^{-\left(\frac{r-\frac{\alpha}{2}}{a}\right)^2}\sin(kr)d{r}{~~}\hat{z}'}\end{equation}

\begin{equation}{\widetilde{\vec{w}}_2(\vec{k})=\frac{-4\sqrt{\pi}i}{ak}\left[\frac{a\alpha\sqrt{\pi}}{2}e^{-\frac{k^2a^2}{4}}\cos\left(\frac{k\alpha}{2}\right)-\frac{a^3k\sqrt{\pi}}{2}e^{-\frac{k^2a^2}{4}}\sin\left(\frac{k\alpha}{2}\right)\right]}\end{equation} 
\begin{displaymath}{+{~~}\frac{4\sqrt{\pi}i}{ak^2}\left[a\sqrt{\pi}e^{-\frac{k^2a^2}{4}}\sin\left(\frac{k\alpha}{2}\right)\right]{~~}\hat{z}'}\end{displaymath}

\begin{equation}{\widetilde{\vec{w}}_2(\vec{k})=\frac{4\pi{i}}{k^2}e^{-\frac{k^2a^2}{4}}\left[\left(\frac{a^2k}{2}+\frac{1}{k}\right)\sin\left(\frac{k\alpha}{2}\right)-\frac{\alpha}{2}\cos\left(\frac{k\alpha}{2}\right)\right]{~~}\vec{k}}\end{equation} 
Putting $a=\frac{\Xi}{\sqrt{2}}$ back in, the final equation for 
$\widetilde{\vec{w}}_2(k)$ becomes
\begin{equation}{\widetilde{\vec{w}}_2(\vec{k})=\frac{4\pi{i}}{k^2}e^{-\frac{k^2\Xi^2}{8}}\left[\left(\frac{\Xi^2k}{4}+\frac{1}{k}\right)\sin\left(\frac{k\alpha}{2}\right)-\frac{\alpha}{2}\cos\left(\frac{k\alpha}{2}\right)\right]{~~}\vec{k}}\end{equation} 

\section{Fourier Transform of Tensor Weight $w_{m2}$}
\begin{equation}{\overleftrightarrow{w}_{m2}(\vec{r})=w_2(r)\left(\frac{\vec{r}\vec{r}}{r^2}-\frac{I}{3}\right)}\end{equation}
where
\begin{equation}{w_2(r)=\frac{\sqrt{2}}{\Xi\sqrt{\pi}}e^{-\left(\frac{r-\frac{\alpha}{2}}{\frac{\Xi}{\sqrt{2}}}\right)^2}}\end{equation}
Form the outer product
\begin{equation}{\vec{r}\vec{r}=\left(\begin{array}{c} r_x \\ r_y \\ r_z \end{array} \right) \left(\begin{array}{rrr} r_x & r_y & r_z \end{array} \right)=\left(\begin{array}{ccc} {r^2}_x & r_xr_y & r_xr_z \\ r_yr_x & {r_y}^2 & r_yr_z \\ r_zr_x & r_zr_y & {r_z}^2 \end{array}\right)}\end{equation}
and identity matrix
\begin{equation}{I=\left(\begin{array}{ccc} 1 & 0 & 0 \\ 0 & 1 & 0 \\ 0 & 0 & 1 \end{array}\right)}\end{equation}
The tensor weight matrix elements are then given by
\begin{equation}{w_{m2_{ij}}(\vec{r})=\frac{1}{a\sqrt{\pi}}e^{-\left(\frac{r-\frac{\alpha}{2}}{a}\right)^2}\left(\frac{r_ir_j}{r^2}-\frac{\delta_{ij}}{3}\right)}\end{equation}
with temporarily setting
\begin{equation}{a=\frac{\Xi}{\sqrt{2}}}\end{equation}
The Fourier transform of the tensor weight matrix elements are given by
\begin{equation}{\widetilde{w}_{m2_{ij}}(\vec{k})=\int_{allspace}{w_{{m2}_{ij}}}(\vec{r})e^{-i\vec{k}\cdot\vec{r}}d{\vec{r}}}\end{equation}
Since the integration is over all space, the integral can be simplified 
by setting 
\begin{equation}{\vec{k}=k\hat{z}}\end{equation}
\begin{equation}{\vec{k}\cdot\vec{r}=kr\cos\theta\hat{z}}\end{equation}
Instead of constraining vector $\vec{k}$ to be in the z direction, 
however, a rotated coordinate system $(x',y',z')$ is used which is 
rotated from the (x,y,z) coordinate system in such a way that $\hat{z}'$ 
points in the direction of vector $\vec{k}$. After solving for the 
elements of the transformed weight function in coordinate system 
$(x',y',z')$, they will be expressed again in terms of vector $\vec{k}$ 
as it appears in coordinate system (x,y,z) to get the elements of the 
transformed weight function in coordinate system (x,y,z). 
\begin{equation}{\widetilde{w}_{m2_{ij}}(k)=\frac{1}{a\sqrt{\pi}}\int_{0}^{\infty}\int_{-1}^{1}\int_{0}^{2\pi}e^{-\left(\frac{r-\frac{\alpha}{2}}{a}\right)^2}\left(\frac{r_ir_j}{r^2}-\frac{\delta_{ij}}{3}\right)e^{-ikr\cos\theta'}r^2d{r}d{\cos\theta'}d{\phi'}}\end{equation}
where $r=r'$ and
\begin{displaymath}{r_{x'}=r\sin\theta\cos\phi}\end{displaymath}
\begin{displaymath}{r_{y'}=r\sin\theta\sin\phi}\end{displaymath}
\begin{displaymath}{r_{z'}=r\cos\theta}\end{displaymath} 
with
\begin{displaymath}{\delta_{ij}=\left\{ \begin{array}{rc} 1 & i = j \\ 0  & i\neq j \end{array}\right.}\end{displaymath}

Calculate $\widetilde{w}_{{m2}_{x'y'}}(k)$ 
\begin{equation}{\widetilde{w}_{{m2}_{x'y'}}(k)=\frac{1}{a\sqrt{\pi}}\int_{0}^{\infty}\int_{-1}^{1}\int_{0}^{2\pi}e^{-\left(\frac{r-\frac{\alpha}{2}}{a}\right)^2}\left(\frac{(r_{x'})(r_{y'})}{r^2}-\frac{\delta_{x'y'}}{3}\right)e^{-ikr\cos\theta'}r^2d{r}d{\cos\theta'}d{\phi'}}\end{equation}
\begin{equation}{\widetilde{w}_{{m2}_{x'y'}}(k)=\frac{1}{a\sqrt{\pi}}\int_{0}^{\infty}\int_{-1}^{1}e^{-\left(\frac{r-\frac{\alpha}{2}}{a}\right)^2}r^2e^{-ikr\cos\theta'}\sin^2\theta'\underbrace{\left(\int_{0}^{2\pi}\cos\phi'\sin{\phi'}~d{\phi'}\right)}d{r}d{\cos\theta'}=0}\end{equation}
$~~~~~~~~~~~~~~~~~~~~~~~~~~~~~~~~~~~~~~~~~~~~~~~~~~~~~~~~~~~~~~~~~~~~~~~~~~~~~~~~0$

Calculate $\widetilde{w}_{{m2}_{x'z'}}(k)$ 
\begin{equation}{\widetilde{w}_{{m2}_{x'z'}}(k)=\frac{1}{a\sqrt{\pi}}\int_{0}^{\infty}\int_{-1}^{1}\int_{0}^{2\pi}e^{-\left(\frac{r-\frac{\alpha}{2}}{a}\right)^2}\left(\frac{(r_{x'})(r_{z'})}{r^2}-\frac{\delta_{x'z'}}{3}\right)e^{-ikr\cos\theta'}r^2d{r}d{\cos\theta'}d{\phi'}}\end{equation}
\begin{equation}{\widetilde{w}_{{m2}_{x'z'}}(k)=\frac{1}{a\sqrt{\pi}}\int_{0}^{\infty}\int_{-1}^{1}e^{-\left(\frac{r-\frac{\alpha}{2}}{a}\right)^2}r^2e^{-ikr\cos\theta'}\sin\theta'\cos\theta'\underbrace{\left(\int_{0}^{2\pi}\cos\phi'~d{\phi'}\right)}d{r}d{\cos\theta'}=0}\end{equation}
$~~~~~~~~~~~~~~~~~~~~~~~~~~~~~~~~~~~~~~~~~~~~~~~~~~~~~~~~~~~~~~~~~~~~~~~~~~~~~~~~~~0$

Calculate $\widetilde{w}_{{m2}_{y'z'}}(k)$ 
\begin{equation}{\widetilde{w}_{{m2}_{y'z'}}(\vec{k})=\frac{1}{a\sqrt{\pi}}\int_{0}^{\infty}\int_{-1}^{1}\int_{0}^{2\pi}e^{-\left(\frac{r-\frac{\alpha}{2}}{a}\right)^2}\left(\frac{(r_{y'})(r_{z'})}{r^2}-\frac{\delta_{yz}}{3}\right)e^{-ikr\cos\theta'}r^2d{r}d{\cos\theta'}d{\phi'}}\end{equation}
\begin{equation}{\widetilde{w}_{{m2}_{y'z'}}(\vec{k})=\frac{1}{a\sqrt{\pi}}\int_{0}^{\infty}\int_{-1}^{1}e^{-\left(\frac{r-\frac{\alpha}{2}}{a}\right)^2}r^2e^{-ikr\cos\theta'}\sin\theta'\cos\theta'\underbrace{\left(\int_{0}^{2\pi}\sin\phi'~d{\phi'}\right)}d{r}d{\cos\theta'}=0}\end{equation}
$~~~~~~~~~~~~~~~~~~~~~~~~~~~~~~~~~~~~~~~~~~~~~~~~~~~~~~~~~~~~~~~~~~~~~~~~~~~~~~~~~~0$

Calculate $\widetilde{w}_{{m2}_{x'x'}}(k)$ 
\begin{equation}{\widetilde{w}_{{m2}_{x'x'}}(k)=\frac{1}{a\sqrt{\pi}}\int_{0}^{\infty}\int_{-1}^{1}\int_{0}^{2\pi}e^{-\left(\frac{r-\frac{\alpha}{2}}{a}\right)^2}\left(\frac{(r_{x'})(r_{x'})}{r^2}-\frac{\delta_{x'x'}}{3}\right)e^{-ikr\cos\theta'}r^2d{r}d{\cos\theta'}d{\phi'}}\end{equation}
\begin{equation}{\widetilde{w}_{{m2}_{x'x'}}(k)=\frac{1}{a\sqrt{\pi}}\int_{0}^{\infty}\int_{-1}^{1}\int_{0}^{2\pi}e^{-\left(\frac{r-\frac{\alpha}{2}}{a}\right)^2}\left(\sin^2\theta'\cos^2\phi'-\frac{1}{3}\right)e^{-ikr\cos\theta'}r^2d{r}d{\cos\theta'}d{\phi'}}\end{equation}

\begin{displaymath}{\widetilde{w}_{{m2}_{x'x'}}(k)=\frac{1}{a\sqrt{\pi}}\int_{0}^{\infty}e^{-\left(\frac{r-\frac{\alpha}{2}}{a}\right)^2}r^2\left[\int_{-1}^{1}\left(1-\cos^2\theta'\right)e^{-ikr\cos\theta'}\left(\int_{0}^{2\pi}\cos^2\phi'~d{\phi'}\right)d{\cos\theta'}\right]d{r}}\end{displaymath} 
\begin{equation}{-\frac{1}{3a\sqrt{\pi}}\int_{0}^{\infty}e^{-\left(\frac{r-\frac{\alpha}{2}}{a}\right)^2}r^2\left[\int_{-1}^{1}e^{-ikr\cos\theta'}\left(\int_{0}^{2\pi}d{\phi'}\right)d{\cos\theta'}\right]d{r}}\end{equation}

%\color{blue}
\begin{displaymath}{\widetilde{w}_{{m2}_{x'x'}}(k)=-\frac{\sqrt{\pi}}{a}\int_{0}^{\infty}e^{-\left(\frac{r-\frac{\alpha}{2}}{a}\right)^2}r^2\left[\int_{-1}^{1}\cos^2\theta'~e^{-ikr\cos\theta'}~d{\cos\theta'}\right]d{r}}\end{displaymath} 
\begin{equation}\label{compare_equation}{+\frac{\sqrt{\pi}}{3a}\int_{0}^{\infty}e^{-\left(\frac{r-\frac{\alpha}{2}}{a}\right)^2}r^2\left[\int_{-1}^{1}e^{-ikr\cos\theta'}~d{\cos\theta'}\right]d{r}}\end{equation}
%\color{black}

\begin{displaymath}{\widetilde{w}_{{m2}_{x'x'}}(k)=-\frac{\sqrt{\pi}}{a}\int_{0}^{\infty}e^{-\left(\frac{r-\frac{\alpha}{2}}{a}\right)^2}r^2\left(\frac{2\sin(kr)}{kr}+\frac{4\cos(kr)}{k^2r^2}-\frac{4\sin(kr)}{k^3r^3}\right)d{r}}\end{displaymath} 
\begin{equation}{+\frac{\sqrt{\pi}}{3a}\int_{0}^{\infty}e^{-\left(\frac{r-\frac{\alpha}{2}}{a}\right)^2}r^2\left(\frac{2\sin(kr)}{kr}\right)d{r}}\end{equation}

Set the lower limit to -infinity since the integrand is nearly zero by the time r is zero. 
\begin{displaymath}{\widetilde{w}_{{m2}_{x'x'}}(k)=-\frac{4\sqrt{\pi}}{3ka}\left(\int_{-\infty}^{\infty}e^{-\left(\frac{r-\frac{\alpha}{2}}{a}\right)^2}r\sin(kr)d{r}\right)}\end{displaymath} 
\begin{displaymath}{-\frac{4\sqrt{\pi}}{k^2a}\left(\int_{-\infty}^{\infty}e^{-\left(\frac{r-\frac{\alpha}{2}}{a}\right)^2}\cos(kr)d{r}\right)}\end{displaymath} 
\begin{equation}{+\frac{4\sqrt{\pi}}{k^3a}\left(\int_{-\infty}^{\infty}e^{-\left(\frac{r-\frac{\alpha}{2}}{a}\right)^2}\frac{\sin(kr)}{r}d{r}\right)}\end{equation} 

\begin{displaymath}{\widetilde{w}_{{m2}_{x'x'}}(k)=-\frac{4\sqrt{\pi}}{3ka}\left(\frac{ka^3}{2}\sqrt{\pi}e^{-\left(\frac{ka}{2}\right)^2}\cos(\frac{k\alpha}{2})+\frac{a\alpha}{2}\sqrt{\pi}e^{-\left(\frac{ka}{2}\right)^2}\sin(\frac{k\alpha}{2})\right)}\end{displaymath} 
\begin{displaymath}{-\frac{4\sqrt{\pi}}{k^2a}\left(a\sqrt{\pi}e^{-\left(\frac{ka}{2}\right)^2}\cos(\frac{k\alpha}{2})\right)}\end{displaymath} 
\begin{equation}{+\frac{4\sqrt{\pi}}{k^3a}\left(\frac{\pi}{2}e^{-\left(\frac{\alpha}{2a}\right)^2}\left[\operatorname{erf}\left(\frac{ka}{2}+i\frac{\alpha}{2a}\right)+\operatorname{erf}\left(\frac{ka}{2}-i\frac{\alpha}{2a}\right)\right]\right)}\end{equation} 

\begin{displaymath}{\widetilde{w}_{{m2}_{x'x'}}(k)=\left(\frac{-2\pi{a}^2}{3}-\frac{4\pi}{k^2}\right)e^{-\left(\frac{ka}{2}\right)^2}\cos(\frac{k\alpha}{2})-\frac{2\pi\alpha}{3k}e^{-\left(\frac{ka}{2}\right)^2}\sin(\frac{k\alpha}{2})}\end{displaymath} 
\begin{equation}{+\frac{2\pi\sqrt{\pi}}{k^3a}e^{-\left(\frac{\alpha}{2a}\right)^2}\left[\operatorname{erf}\left(\frac{ka}{2}+i\frac{\alpha}{2a}\right)+\operatorname{erf}\left(\frac{ka}{2}-i\frac{\alpha}{2a}\right)\right]}\end{equation} 
Putting in $a=\frac{\Xi}{\sqrt{2}}$ the final equation for 
$\widetilde{w}_{{m2}_{xx}}(k)$ becomes

\begin{displaymath}{\widetilde{w}_{{m2}_{x'x'}}(k)=\left(\frac{-\pi{\Xi}^2}{3}-\frac{4\pi}{k^2}\right)e^{-\frac{k^2\Xi^2}{8}}\cos(\frac{k\alpha}{2})-\frac{2\pi\alpha}{3k}e^{-\frac{k^2\Xi^2}{8}}\sin(\frac{k\alpha}{2})}\end{displaymath} 
\begin{equation}{+\frac{2\pi\sqrt{2\pi}}{k^3\Xi}e^{-\left(\frac{\alpha^2}{2\Xi^2}\right)}\left[\operatorname{erf}\left(\frac{k\Xi}{2*\sqrt{2}}+i\frac{\alpha}{\sqrt{2}\Xi}\right)+\operatorname{erf}\left(\frac{k\Xi}{2\sqrt{2}}-i\frac{\alpha}{\sqrt{2}\Xi}\right)\right]}\end{equation} 


Calculate $\widetilde{w}_{{m2}_{y'y'}}(k)$ 
\begin{equation}{\widetilde{w}_{{m2}_{y'y'}}(k)=\frac{1}{a\sqrt{\pi}}\int_{0}^{\infty}\int_{-1}^{1}\int_{0}^{2\pi}e^{-\left(\frac{r-\frac{\alpha}{2}}{a}\right)^2}\left(\frac{(r_{y'})(r_{y'})}{r^2}-\frac{\delta_{y'y'}}{3}\right)e^{-ikr\cos\theta'}r^2d{r}d{\cos\theta'}d{\phi'}}\end{equation}
\begin{equation}{\widetilde{w}_{{m2}_{y'y'}}(k)=\frac{1}{a\sqrt{\pi}}\int_{0}^{\infty}\int_{-1}^{1}\int_{0}^{2\pi}e^{-\left(\frac{r-\frac{\alpha}{2}}{a}\right)^2}\left(\sin^2\theta'\sin^2\phi'-\frac{1}{3}\right)e^{-ikr\cos\theta'}r^2d{r}d{\cos\theta'}d{\phi'}}\end{equation}

\begin{displaymath}{\widetilde{w}_{{m2}_{y'y'}}(k)=\frac{1}{a\sqrt{\pi}}\int_{0}^{\infty}e^{-\left(\frac{r-\frac{\alpha}{2}}{a}\right)^2}r^2\left[\int_{-1}^{1}\left(1-\cos^2\theta\right)e^{-ikr\cos\theta'}\left(\int_{0}^{2\pi}\sin^2\phi~d{\phi}\right)d{\cos\theta}\right]d{r}}\end{displaymath} 
\begin{equation}{-\frac{1}{3a\sqrt{\pi}}\int_{0}^{\infty}e^{-\left(\frac{r-\frac{\alpha}{2}}{a}\right)^2}r^2\left[\int_{-1}^{1}e^{-ikr\cos\theta'}\left(\int_{0}^{2\pi}d{\phi'}\right)d{\cos\theta'}\right]d{r}}\end{equation}


\begin{displaymath}{\widetilde{w}_{{m2}_{y'y'}}(k)=-\frac{\sqrt{\pi}}{a}\int_{0}^{\infty}e^{-\left(\frac{r-\frac{\alpha}{2}}{a}\right)^2}r^2\left[\int_{-1}^{1}\cos^2\theta~e^{-ikr\cos\theta'}~d{\cos\theta'}\right]d{r}}\end{displaymath} 
\begin{equation}{+\frac{\sqrt{\pi}}{3a}\int_{0}^{\infty}e^{-\left(\frac{r-\frac{\alpha}{2}}{a}\right)^2}r^2\left[\int_{-1}^{1}e^{-ikr\cos\theta'}~d{\cos\theta'}\right]d{r}}\end{equation}

This is the same as equation \ref{compare_equation}, and so \begin{equation}{\widetilde{w}_{{m2}_{y'y'}}(k)=\widetilde{w}_{{m2}_{x'x'}}(k)}\end{equation}
Calculate $\widetilde{w}_{{m2}_{z'z'}}(k)$ 
\begin{equation}{\widetilde{w}_{{m2}_{z'z'}}(k)=\frac{1}{a\sqrt{\pi}}\int_{0}^{\infty}\int_{-1}^{1}\int_{0}^{2\pi}e^{\left(\frac{r-\frac{\alpha}{2}}{a}\right)^2}\left(\frac{(r_{y'})(r_{y'})}{r^2}-\frac{\delta_{z'z'}}{3}\right)e^{-ikr\cos\theta'}r^2d{r}d{\cos\theta'}d{\phi'}}\end{equation}
\begin{equation}{\widetilde{w}_{{m2}_{z'z'}}(k)=\frac{1}{a\sqrt{\pi}}\int_{0}^{\infty}\int_{-1}^{1}\int_{0}^{2\pi}e^{\left(\frac{r-\frac{\alpha}{2}}{a}\right)^2}\left(\cos^2\theta'-\frac{1}{3}\right)e^{-ikr\cos\theta'}r^2d{r}d{\cos\theta'}d{\phi'}}\end{equation}

\begin{displaymath}{\widetilde{w}_{{m2}_{z'z'}}(k)=\frac{1}{a\sqrt{\pi}}\int_{0}^{\infty}e^{-\left(\frac{r-\frac{\alpha}{2}}{a}\right)^2}r^2\left[\int_{-1}^{1}\cos^2\theta'~e^{-ikr\cos\theta'}\left(\int_{0}^{2\pi}d{\phi'}\right)d{\cos\theta'}\right]d{r}}\end{displaymath} 
\begin{equation}{-\frac{1}{3a\sqrt{\pi}}\int_{0}^{\infty}e^{-\left(\frac{r-\frac{\alpha}{2}}{a}\right)^2}r^2\left[\int_{-1}^{1}e^{-ikr\cos\theta'}\left(\int_{0}^{2\pi}d{\phi'}\right)d{\cos\theta'}\right]d{r}}\end{equation}

\begin{displaymath}{\widetilde{w}_{{m2}_{z'z'}}(k)=2\frac{\sqrt{\pi}}{a}\int_{0}^{\infty}e^{-\left(\frac{r-\frac{\alpha}{2}}{a}\right)^2}r^2\left[\int_{-1}^{1}\cos^2\theta'~e^{-ikr\cos\theta'}d{\cos\theta'}\right]d{r}}\end{displaymath} 

\begin{equation}{-2\frac{\sqrt{\pi}}{3a}\int_{0}^{\infty}e^{-\left(\frac{r-\frac{\alpha}{2}}{a}\right)^2}r^2\left[\int_{-1}^{1}e^{-ikr\cos\theta'}d{\cos\theta'}\right]d{r}}\end{equation} 

This is the same as -2 times equation \ref{compare_equation}, and so \begin{equation}{\widetilde{w}_{{m2}_{z'z'}}(k)=-2\widetilde{w}_{{m2}_{x'x'}}(k)}\end{equation}

%\subsection{The tensor weight components in the coordinate system (x,y,z)}
\noindent The tensor weight components in the coordinate system 
$(x',y',z')$ are:
\begin{equation}\label{tensorcomp}{\widetilde{w'}_{m2}=\left(\begin{array}{ccc} \widetilde{w}_{{m2}_{x'x'}} & 0 & 0 \\ 0 & \widetilde{w}_{{m2}_{y'y'}} & 0 \\ 0 & 0 & \widetilde{w}_{{m2}_{z'z'}} \end{array}\right)'}\end{equation}

\begin{displaymath}{=\widetilde{w}_{{m2}_{x'x'}}\left(\begin{array}{ccc} 1 & 0 & 0 \\ 0 & 0 & 0 \\ 0 & 0 & 0 \end{array}\right)'+ \widetilde{w}_{{m2}_{y'y'}}\left(\begin{array}{ccc} 0 & 0 & 0 \\ 0 & 1 & 0 \\ 0 & 0 & 0 \end{array}\right)' + \widetilde{w}_{{m2}_{z'z'}}\left(\begin{array}{ccc} 0 & 0 & 0 \\ 0 & 0 & 0 \\ 0 & 0 & 1 \end{array}\right)'}\end{displaymath}

\begin{equation}{=\widetilde{w}_{{m2}_{x'x'}}\hat{x'}\hat{x'}+\widetilde{w}_{{m2}_{y'y'}}\hat{y'}\hat{y'}+\widetilde{w}_{{m2}_{z'z'}}\hat{z'}\hat{z'}}\end{equation}
written in terms of the outer products

\begin{displaymath}{\hat{x}'\hat{x}'= \left(\begin{array}{c} 1 \\ 0 \\ 0 \end{array}\right)'\left(\begin{array}{ccc} 1 & 0 & 0 \end{array}\right)'=\left(\begin{array}{ccc} 1 & 0 & 0 \\ 0 & 0 & 0 \\ 0 & 0 & 0 \end{array}\right)'}\end{displaymath}

\begin{displaymath}{\hat{y}'\hat{y}'= \left(\begin{array}{c} 0 \\ 1 \\ 0 \end{array}\right)'\left(\begin{array}{ccc} 0 & 1 & 0 \end{array}\right)'=\left(\begin{array}{ccc} 0 & 0 & 0 \\ 0 & 1 & 0 \\ 0 & 0 & 0 \end{array}\right)'}\end{displaymath}

\begin{displaymath}{\hat{z}'\hat{z}'= \left(\begin{array}{c} 0 \\ 0 \\ 1 \end{array}\right)'\left(\begin{array}{ccc} 0 & 0 & 1 \end{array}\right)'=\left(\begin{array}{ccc} 0 & 0 & 0 \\ 0 & 0 & 0 \\ 0 & 0 & 1 \end{array}\right)'}\end{displaymath}

\noindent Using the relation
\begin{displaymath}{\text{I} = \hat{x'}\hat{x'} + \hat{y'}\hat{y'} + \hat{z'}\hat{z'}}\end{displaymath}
\begin{displaymath}{\text{I} -\hat{z'}\hat{z'} = \hat{x'}\hat{x'} + \hat{y'}\hat{y'}}\end{displaymath}
and the results from earlier
\begin{equation}{\widetilde{w}_{{m2}_{y'y'}}=\widetilde{w}_{{m2}_{x'x'}}}\end{equation}
\begin{equation}{\widetilde{w}_{{m2}_{z'z'}}=-2\widetilde{w}_{{m2}_{x'x'}}}\end{equation}
Equation \ref{tensorcomp} then becomes 
\begin{equation}{\widetilde{w}_{m2}(\vec{k})= (\text{I}-3\hat{z}'\hat{z}')\widetilde{w}_{{m2}_{x'x'}}(\vec{k})}\end{equation}
\noindent In the $(x',y',z')$ coordinate system, vector $\vec{k}$ had 
been aligned with the $\hat{z}'$ direction so that 
\begin{displaymath}{\hat{k}=\hat{z'}}\end{displaymath}
Replacing $\hat{z}'$ with $\hat{k}$ gives
\begin{equation}{\widetilde{w}_{m2}(\vec{k})= (\text{I}-3\hat{k}\hat{k})\widetilde{w}_{{m2}_{x'x'}}(\vec{k})}\end{equation}
Now  $\vec{k}$ can be expressed in the $(x,y,z)$ coordinate system, 
$\vec{k}=k_x\hat{x} + k_y\hat{y} + k_z\hat{z}$. The components of the 
tensor weight in the $(x,y,z)$ coordinate system are then given by
\begin{equation}{\widetilde{w}_{m2_{ij}}(\vec{k})= (\delta{ij}-3\frac{k_ik_j}{k^2})\widetilde{w}_{{m2}_{x'x'}}(\vec{k})}\end{equation}

%\subsection{Transform of tensor weight ${w}_{m2}(r)$ goes to zero as k goes to zero}
\noindent Check that the transform of tensor weight ${w}_{m2}(r)$ 
goes to zero as k goes to zero:
\begin{displaymath}{\widetilde{w}_{{m2}_{x'x'}}(k)=\left(\frac{-\pi\Xi^2}{3}-\frac{4\pi}{k^2}\right)e^{-\left(\frac{k^2\Xi^2}{8}\right)}\cos(\frac{k\alpha}{2})-\frac{2\pi\alpha}{3k}e^{-\left(\frac{k^2\Xi^2}{8}\right)}\sin(\frac{k\alpha}{2})}\end{displaymath} 
\begin{equation}{+\frac{2\pi\sqrt{2\pi}}{k^3\Xi}e^{-\left(\frac{\alpha^2}{2\Xi^2}\right)}\left[\operatorname{erf}\left(\frac{k\Xi}{2*\sqrt{2}}+i\frac{\alpha}{\sqrt{2}\Xi}\right)+\operatorname{erf}\left(\frac{k\Xi}{2\sqrt{2}}-i\frac{\alpha}{\sqrt{2}\Xi}\right)\right]}\end{equation}

Term 1: 
\begin{equation}{\left(\frac{-\pi\Xi^2}{3}-\frac{4\pi}{k^2}\right)e^{-\left(\frac{k^2\Xi^2}{8}\right)}\cos(\frac{k\alpha}{2})}\end{equation}
\begin{displaymath}{\approx\left(-\frac{\pi\Xi^2}{3}-\frac{4\pi}{k^2}\right)\left(1-\frac{k^2\Xi^2}{8}\right)\left(1-\left(\frac{1}{2}\right)\frac{k^2\alpha^2}{4}\right)}\end{displaymath} 
\begin{displaymath}{\approx-\frac{4\pi}{k^2}+\frac{\pi\alpha^2}{2}+\frac{\pi\Xi^2}{6}}\end{displaymath} 

Term 2:
\begin{equation}{-\frac{2\pi\alpha}{3k}e^{-\left(\frac{k^2\Xi^2}{8}\right)}\sin(\frac{k\alpha}{2})}\end{equation} 
\begin{displaymath}{\approx\left(-\frac{2\pi\alpha}{3k}\right)\left(1-\frac{k^2\Xi^2}{8}\right)\left(\frac{k\alpha}{2}\right)}\end{displaymath} 
\begin{displaymath}{\approx}-\frac{\pi\alpha^2}{3}\end{displaymath}
 
Term 3:
\begin{equation}{\frac{2\pi\sqrt{2\pi}}{k^3\Xi}e^{-\left(\frac{\alpha^2}{2\Xi^2}\right)}\left[\operatorname{erf}\left(\frac{k\Xi}{2\sqrt{2}}+i\frac{\alpha}{\sqrt{2}\Xi}\right)+\operatorname{erf}\left(\frac{k\Xi}{2\sqrt{2}}-i\frac{\alpha}{\sqrt{2}\Xi}\right)\right]}\end{equation}
\begin{displaymath}{\approx\frac{\left(2\pi\right)^\frac{2}{3}}{\Xi{k}^3}e^{-\left(\frac{\alpha^2}{2\Xi^2}\right)}\left[2\operatorname{Re}\left(\operatorname{erf}\left(\frac{k\Xi}{2\sqrt{2}}+i\frac{\alpha}{\sqrt{2}\Xi}\right)\right)\right]}\end{displaymath} 
\begin{displaymath}{\approx\frac{\left(2\pi\right)^\frac{2}{3}}{\Xi{k}^3}e^{-\left(\frac{\alpha^2}{2\Xi^2}\right)}\left[2e^{\left(\frac{\alpha^2}{2\Xi^2}\right)}\frac{2}{\sqrt{\pi}}   \left(\frac{k\Xi}{2\sqrt{2}}   -\frac{2\left(\frac{\alpha^2}{2\Xi^2}\right)+1}{3}\left(\frac{k\Xi}{2\sqrt{2}}\right)^3\right)\right]}\end{displaymath}
\begin{displaymath}{\approx\frac{4\pi}{k^2}-\frac{\pi}{6}\left(\alpha^2-\Xi^2\right)}\end{displaymath}  

Adding the three terms gives:
\begin{displaymath}{\approx-\frac{4\pi}{k^2}+\frac{\pi\alpha^2}{2}+\frac{\pi\Xi^2}{6}-\frac{\pi\alpha^2}{3}+\frac{4\pi}{k^2}-\frac{\pi\alpha^2}{6}-\frac{\pi\Xi^2}{6}}\end{displaymath} 
\begin{displaymath}{\approx0}\end{displaymath} 

\textbf{4. Integrals used to Compute Fourier Transforms of Weight Functions}
\begin{equation}{\int_{-1}^{1}{e^{-ikr\cos{\theta}}d{\cos{\theta}}}=\frac{2\sin(kr)}{kr}}\end{equation} 
\begin{equation}{\int_{-1}^{1}{e^{-ikr\cos{\theta}}\cos{\theta}{~}d{\cos{\theta}}}=-\frac{2i}{kr}\cos(kr)+\frac{2i}{k^2r^2}\sin(kr)}\end{equation} 

\begin{equation}{\int_{-1}^{1}{e^{-ikr\cos{\theta}}\cos^2(\theta)~d{\cos{\theta}}}=\frac{2\sin(kr)}{kr}+\frac{4\cos(kr)}{k^2r^2}+\frac{4\sin(kr)}{k^3r^3}}\end{equation} 

\begin{equation}{\int_{-\infty}^{\infty}{e^{-\left(\frac{r-\frac{\alpha}{2}}{a}\right)^2}\cos(kr)d{r}}}\end{equation}
\begin{displaymath}{=a\int_{-\infty}^{\infty}{e^{-u^2}\left[\cos(kau)\cos(\frac{k\alpha}{2})-\sin(kau)\sin(\frac{k\alpha}{2})\right]d{u}}}\end{displaymath}  
\begin{equation}{=a\sqrt{\pi}e^{-\left(\frac{ka}{2}\right)^2}\cos(\frac{k\alpha}{2})}\end{equation} 

\begin{equation}{\int_{-\infty}^{\infty}{e^{-\left(\frac{r-\frac{\alpha}{2}}{a}\right)^2}r\cos(kr)d{r}}}\end{equation}
\begin{equation}{=\frac{a\alpha}{2}\cos\left(\frac{k\alpha}{2}\right)\int_{-\infty}^{\infty}{e^{-u^2}\cos(kau)d{u}} -a^2\sin\left(\frac{k\alpha}{2}\right)\int_{-\infty}^{\infty}{e^{-u^2}u\sin(kau)d{u}}}\end{equation}
\begin{equation}{=\frac{a\alpha\sqrt{\pi}}{2}e^{-\left(\frac{ka}{2}\right)^2}\cos\left(\frac{k\alpha}{2}\right)-\frac{a^3k\sqrt{\pi}}{2}e^{-\left(\frac{ka}{2}\right)^2}\sin\left(\frac{k\alpha}{2}\right)}\end{equation}

\begin{equation}{\int_{-\infty}^{\infty}{e^{-\left(\frac{r-\frac{\alpha}{2}}{a}\right)^2}\sin(kr)d{r}}}\end{equation} 
\begin{displaymath}{=a\int_{-\infty}^{\infty}{e^{-u^2}\left[\sin(kau)\cos(\frac{k\alpha}{2})+\cos(kau)\sin(\frac{k\alpha}{2})\right]d{u}}}\end{displaymath}  
\begin{equation}{=a\sqrt{\pi}e^{-\left(\frac{ka}{2}\right)^2}\sin(\frac{k\alpha}{2})}\end{equation} 

\begin{equation}{\int_{-\infty}^{\infty}{e^{-\left(\frac{r-\frac{\alpha}{2}}{a}\right)^2}r\sin(kr)d{r}}}\end{equation} 
\begin{displaymath}{=a\int_{-\infty}^{\infty}{e^{-u^2}(au+\frac{\alpha}{2})\sin(k(au+\frac{\alpha}{2}))d{u}}}\end{displaymath} 
\begin{displaymath}{=a^2\cos(\frac{k\alpha}{2})\int_{-\infty}^{\infty}{e^{-u^2}u\sin(kau)d{u} +\frac{a\alpha}{2}\sin(\frac{k\alpha}{2})\int_{-\infty}^{\infty}e^{-u^2}\cos(kau)d{u}}}\end{displaymath} 
\begin{equation}{=\frac{ka^3}{2}\sqrt{\pi}e^{-\left(\frac{ka}{2}\right)^2}\cos(\frac{k\alpha}{2})+\frac{a\alpha}{2}\sqrt{\pi}e^{-\left(\frac{ka}{2}\right)^2}\sin(\frac{k\alpha}{2})}\end{equation} 

\begin{equation}{\int_{-\infty}^{\infty}{e^{-\left(\frac{r-\frac{\alpha}{2}}{a}\right)^2}\frac{\sin(kr)}{r}d{r}}}\end{equation} 
%Noting that \begin{equation}{\frac{\sin(kr)}{r}=\int_{0}^{k}{\cos(kr)d{k}}}\end{equation} 
%this becomes
\begin{displaymath}{=\int_{-\infty}^{\infty}{e^{-\left(\frac{r-\frac{\alpha}{2}}{a}\right)^2}\left(\int_{0}^{k}{\cos(kr)d{k}}\right)d{r}}}\end{displaymath} 
\begin{displaymath}{=\int_{0}^{k}\int_{-\infty}^{\infty}{e^{-\left(\frac{r-\frac{\alpha}{2}}{a}\right)^2}{\cos(kr)d{r}}d{k}}}\end{displaymath} 
\begin{displaymath}{=a\sqrt{\pi}\int_{0}^{k}e^{-\left(\frac{ka}{2}\right)^2}\cos\left(\frac{k\alpha}{2}\right)d{k} }\end{displaymath} 
\begin{displaymath}{=\frac{a\sqrt{\pi}}{2}\left[\int_{-k}^{k}e^{-\left(\frac{ka}{2}\right)^2 + i\frac{k\alpha}{2}}d{k}\right]}\end{displaymath} 
\begin{displaymath}{=\sqrt{\pi}e^{-\left(\frac{\alpha}{2a}\right)^2}\int_{-\frac{ka}{2}+i\frac{\alpha}{2a}}^{\frac{ka}{2}+i\frac{\alpha}{2a}}e^{-u^2}d{u}}\end{displaymath} 
\begin{equation}{=\frac{\pi}{2}e^{-\left(\frac{\alpha}{2a}\right)^2}\left[\operatorname{erf}\left(\frac{ka}{2}+i\frac{\alpha}{2a}\right)+\operatorname{erf}\left(\frac{ka}{2}-i\frac{\alpha}{2a}\right)\right]}\end{equation} 

\section{Derivation of Tensor Density $n_{m2ij}(\vec{r})$}
\begin{equation}{\overleftrightarrow{n}_{m2}(\vec{r})=\int_{allspace}n(\vec{r'})\overleftrightarrow{w}_{m2}(\vec{r}-\vec{r'})d{\vec{r'}}}\end{equation} 
\begin{equation}{n_{m2ij}(\vec{r})=\int_{allspace}{n(\vec{r}')w_{m2ij}(\vec{r}-\vec{r}'){}d{\vec{r}'}}}\end{equation}
Using the definition of the delta function
\begin{equation}{\delta}((\vec{r}-\vec{r}')-\vec{r}'')={ \frac{1}{\left(2\pi\right)^3}\int e^{i\vec k\cdot ((\vec r-\vec r')-\vec{r}'')}d\vec{k}}\end{equation} 
$w_{m2ij}(\vec{r}-\vec{r}')$ can be expressed as
\begin{align}
    w_{m2ij}(\vec{r}-\vec{r}') &= \int{\delta((\vec{r}-\vec{r}')-\vec{r}'')w_{m2ij}(\vec{r}''){}d{\vec{r}''}} \\
    &= \int{\left(\frac{1}{\left(2\pi\right)^3}\int e^{i\vec k\cdot((\vec r-\vec r')-\vec{r}'')}d\vec{k}\right){~}w_{m2ij}(\vec{r}''){}d{\vec{r}''}} \\
    &= \frac{1}{\left(2\pi\right)^3}\int{\left(\int w_{m2ij}(\vec{r}'')e^{-i\vec k\cdot\vec{r}''}d\vec{r}''\right)e^{i\vec k\cdot(\vec r-\vec r')}{~}{}d{\vec{k}}} \\
    &= \frac{1}{\left(2\pi\right)^3}\int{w_{m2ij}(\vec k)e^{i\vec k\cdot(\vec r-\vec r')}{~}{}d{\vec{k}}} 
  \end{align} 
Putting this into the expression for $n_{m2ij}(\vec{r})$ gives 
\begin{align}
    n_{m2ij}(\vec r) &= \int n(\vec r') \left(\frac{1}{\left(2\pi\right)^3}\int w_{m2ij}(\vec k)e^{i\vec k\cdot (\vec r-\vec r')}d\vec{k}\right)d\vec r' \\
    &= \int d\vec r' \left(\frac{1}{\left(2\pi\right)^3}\int d\vec k' n(\vec k')e^{i\vec k'\cdot \vec r'}\right) \left(\frac{1}{\left(2\pi\right)^3}\int d\vec k w_{m2ij}(\vec k)e^{i\vec k\cdot (\vec r-\vec r')}\right) \\
    &=  \frac{1}{\left(2\pi\right)^3}\int d\vec k' n(\vec k') \int d\vec k w_{m2ij}(\vec k)
    e^{i\vec k\cdot \vec r}\left(\frac{1}{\left(2\pi\right)^3}\int d\vec r'e^{i(\vec k'-\vec k)\cdot \vec r'}\right)
    \\
   &= \frac{1}{\left(2\pi\right)^3}\int d\vec k' n(\vec k') \int d\vec k w_{m2ij}(\vec k)e^{i\vec k\cdot \vec r}\delta(\vec k'-\vec k) \\
    &= \frac{1}{\left(2\pi\right)^3}\int d\vec k \left(\int d\vec k' n(\vec k')\delta(\vec k'-\vec k)\right) w_{m2ij}(\vec k)e^{i\vec k\cdot \vec r}
    \\
    &= \frac{1}{\left(2\pi\right)^3}\int d\vec k\, n(\vec k) w_{m2ij}(\vec k)e^{i\vec k\cdot \vec r}
  \end{align} 

\section{Derivation of the Tensor version of $\Phi_3$}

\begin{equation}{\Phi_3=\frac{12\pi^2R^6}{(1-\eta)^2}\left[\overrightarrow{\operatorname{V}}\cdot\overleftrightarrow{\operatorname{T}}\cdot\overrightarrow{\operatorname{V}}-n\overrightarrow{\operatorname{V}}\cdot\overrightarrow{\operatorname{V}}-\operatorname{Tr}\left(\overleftrightarrow{\operatorname{T}}^3\right)+n\operatorname{Tr}\left(\overleftrightarrow{\operatorname{T}}^2\right)\right]}\end{equation}

\begin{equation}{\overleftrightarrow{\operatorname{T}}=\frac{{\overleftrightarrow{n}_{m2}}+\frac{n_2}{3}\hat{\operatorname{I}}}{4\pi{R}^2}}\end{equation}

\begin{equation}{\overrightarrow{\operatorname{V}}=\frac{\vec{n}_2}{4\pi{R}^2}}\end{equation}

\begin{displaymath}{\Phi_3=\frac{12\pi^2R^6}{(1-\eta)^2}\left(\frac{1}{4\pi{R}^2}\right)^3\left[\vec{n}_{2v}\cdot\left({\overleftrightarrow{n}_{m2}}+\frac{n_2}{3}\hat{\operatorname{I}}\right)\cdot\vec{n}_{2v}-n_2|\vec{n}_{2v}|^2-\operatorname{Tr}\left(({\overleftrightarrow{n}_{m2}}+\frac{n_2}{3}\hat{\operatorname{I}})^3\right)\color{white}\right]\color{black}}\end{displaymath}
\begin{equation}{\color{white}\left[\color{black}+n_2\operatorname{Tr}\left(({\overleftrightarrow{n}_{m2}}+\frac{n_2}{3}\hat{\operatorname{I}})^2\right)\right]}\end{equation}

\begin{displaymath}{\Phi_3=\frac{3}{16\pi(1-\eta)^2}\left[\vec{n}_{v2}\cdot{\overleftrightarrow{n}_{m2}}\cdot{\vec{n}_{v2}}-\frac{2}{3}n_2\vec{n}_{v2}\cdot\vec{n}_{v2}-\operatorname{Tr}\left(({\overleftrightarrow{n}_{m2}}+\frac{n_2}{3}\hat{\operatorname{I}})^3\right)\color{white}\right]\color{black}}\end{displaymath}
\begin{equation}{\color{white}\left[\color{black}+n_2\operatorname{Tr}\left((\overleftrightarrow{n}_{m2}+\frac{n_2}{3}\hat{\operatorname{I}})^2\right)\right]}\end{equation}

\begin{displaymath}{\Phi_3=\frac{3}{16\pi(1-\eta)^2}\left(\vec{n}_{v2}\cdot{\overleftrightarrow{n}_{m2}}\cdot{\vec{n}_{v2}}-\frac{2}{3}n_2|\vec{n}_{v2}|^2-\operatorname{Tr}({\overleftrightarrow{n}_{m2}^3})
-n_2\operatorname{Tr}(\overleftrightarrow{n}_{m2}^2)\color{white}\right)\color{black}}\end{displaymath} 
\begin{equation}{\color{white}\left(\color{black}-\frac{1}{3}{n_2}^2\operatorname{Tr}(\overleftrightarrow{n}_{m2})-\frac{1}{9}{n_2}^3+n_2\operatorname{Tr}(\overleftrightarrow{n}_{m2}^2)+\frac{1}{3}{n_2}^3+\frac{2}{3}n^2_2\operatorname{Tr}(\overleftrightarrow{n}_{m2})\right)}\end{equation} 

\begin{equation}{\Phi_3=\frac{3}{16\pi(1-\eta)^2}\left(\frac{2}{9}{n_2}^3-\frac{2}{3}n_2|\vec{n}_{v2}|^2+\vec{n}_{v2}\cdot{\overleftrightarrow{n}_{m2}}\cdot{\vec{n}_{v2}}-\operatorname{Tr}({\overleftrightarrow{n}_{m2}}^3)+\frac{1}{3}n^2_2\operatorname{Tr}(\overleftrightarrow{n}_{m2})\right)}\end{equation} 

\begin{equation}{\operatorname{Tr}(\overleftrightarrow{n}_{m2})=0}\end{equation} 

\begin{equation}{\Phi_3=\frac{{n_2}^3-3n_2|\vec{n}_{v2}|^2+\frac{9}{2}(\vec{n}_{v2}\cdot{\overleftrightarrow{n}_{m2}}\cdot{\vec{n}_{v2}})-\frac{9}{2}\operatorname{Tr}({\overleftrightarrow{n}^3_{m2}})}{24\pi(1-\eta)^2}}\end{equation} 

\begin{equation}{\eta=n_3}\end{equation} 

\begin{equation}{\Phi_3=\frac{{n_2}^3-3n_2\vec{n}_{v2}\cdot\vec{n}_{v2}+\frac{9}{2}[\vec{n}_{v2}\cdot{\overleftrightarrow{n}_{m2}}\cdot{\vec{n}_{v2}}-\operatorname{Tr}({\overleftrightarrow{n}^3_{m2}})]}{24\pi(1-n_3)^2}}\end{equation} 


\section{Second Virial Coefficient $B_{2}$ for the Error Function Potential}
\begin{equation}B_2=-\frac{1}{2}\int_{allspace}f(\vec{r})d\vec r\end{equation}
The Mayer function for the Error Function potential $V_{\operatorname{erf}}$ is
\begin{equation}f(r)=\frac{1}{2}\left[\operatorname{erf}\left(\frac{r-\alpha}{\Xi}\right)-1\right]\end{equation} 
from which it can be seen that $f(r)$ is approximately 0 for $r>r_{max}$ 
where $r_{max}$ is some maximum value of $r$.
\begin{align}
 B_2 &= -\frac{1}{2}\int_0^{\pi}\int_0^{2\pi}\int_0^\infty\frac{1}{2}\left[\operatorname{erf}\left(\frac{r-\alpha}{\Xi}\right)-1\right]r^2\sin{\theta}d{\theta}d{\phi}dr \\
     &= -\frac{1}{2}4\pi\frac{1}{2}\int_{0}^{r_{max}}\left[\operatorname{erf}\left(\frac{r-\alpha}{\Xi}\right)-1\right]r^2dr \\
     &= -\pi\int_{0}^{r_{max}}\operatorname{erf}\left(\frac{r-\alpha}{\Xi}\right)r^2dr {~~}+{~~} \pi\int_0^{r_{max}}r^2dr   \\
     &= -\pi\int_{0}^{r_{max}}\operatorname{erf}\left(\frac{r-\alpha}{\Xi}\right)r^2dr {~~}+{~~} \frac{\pi}{3}r_{max}^3   
\end{align}

\begin{align}
  B_2 &= \left(-\frac{\pi}{3}\alpha^3-\frac{\pi}{2}\alpha\Xi^2+\frac{\pi}{3}r_{max}^3\right)\operatorname{erf}\left(\frac{\alpha-r_{max}}{\Xi}\right) \\
      &+ \left(\frac{\pi}{3}\alpha^3+\frac{\pi}{2}\alpha\Xi^2\right)\operatorname{erf}\left(\frac{\alpha}{\Xi}\right) \\
      &+ \frac{\sqrt{\pi}}{3}\left(-\Xi\alpha^2-\Xi\alpha r_{max}-\Xi^3-\Xi r_{max}^2\right)e^{-{\left(\frac{\alpha-r_{max}}{\Xi}\right)^2}} \\
      &+ \frac{\sqrt{\pi}}{3}\left(\Xi\alpha^2+\Xi^3\right)e^{-\left(\frac{\alpha}{\Xi}\right)^2}+\frac{\pi}{3}r_{max}^3 
\end{align}
Letting $r_{max}\rightarrow\infty$ this becomes
\begin{align}
  B_2 &= \left(-\frac{\pi}{3}\alpha^3-\frac{\pi}{2}\alpha\Xi^2+\frac{\pi}{3}\left(\infty\right)^3\right)\left(-1\right) \\
      &+ \left(\frac{\pi}{3}\alpha^3+\frac{\pi}{2}\alpha\Xi^2\right)\operatorname{erf}\left(\frac{\alpha}{\Xi}\right) \\
      &+ \frac{\sqrt{\pi}}{3}\left(-\Xi\alpha^2-\Xi\alpha\left(\infty\right)-\Xi^3-\Xi r_{max}^2\right)\left(0\right) \\
      &+ \frac{\sqrt{\pi}}{3}\left(\Xi\alpha^2+\Xi^3\right)e^{-\left(\frac{\alpha}{\Xi}\right)^2}+\frac{\pi}{3}\left(\infty\right)^3 
\end{align}
\begin{equation}B_2 = \frac{\pi}{3}\Xi^3\left[\left(\frac{\alpha^3}{\Xi^3}+\frac{3\alpha}{2\Xi}\right)\left(1+\operatorname{erf}\left(\frac{\alpha}{\Xi}\right)\right)+\frac{1}{\sqrt{\pi}}\left(\frac{\alpha^2}{\Xi^2}+1\right)e^{-\left(\frac{\alpha}{\Xi}\right)^2}\right]\end{equation}

\section{Third Virial Coefficient $B_{3}$ General Approach}

\begin{equation}\label{B3}B_3=-\frac{1}{3}\int_{allspace}\int_{allspace}f(\vec{r})f(\vec{r'})f(\vec{r}-\vec{r'})d\vec rd\vec r'\end{equation}

First find
\begin{equation}f(\vec{r}-\vec{r'})=\int_{allspace}f(\vec{r''})\delta(\vec{r}''-(\vec{r}-\vec{r}'))d\vec r'' \end{equation}
where
\begin{equation}{\delta}(\vec{r}''-(\vec{r}-\vec{r}'))={ \frac{1}{\left(2\pi\right)^3}\int_{allspace} e^{i\vec k\cdot (\vec{r}''-(\vec r-\vec r'))}d\vec{k}}\end{equation} 

\begin{equation}f(\vec{r}-\vec{r'})=\int{f(\vec{r''})\left( \frac{1}{\left(2\pi\right)^3}\int e^{i\vec k\cdot (\vec{r}''-(\vec r-\vec r'))}d\vec{k} \right) d\vec r''} \end{equation}

\begin{equation}f(\vec{r}-\vec{r'})=\int{ e^{-i\vec k\cdot (\vec r-\vec r')}\left(\frac{1}{\left(2\pi\right)^3}\int{f(\vec{r''}) e^{i\vec k\cdot (\vec{r}'')}d\vec{r''}} \right) d\vec k} \end{equation}

\begin{equation}f(\vec{r}-\vec{r'})=\int{ e^{-i\vec k\cdot \vec r}e^{i\vec k\cdot \vec r}\widetilde{f}(-\vec k) d\vec k} \end{equation}

Putting this result into Eq~\ref{B3} gives
\begin{equation}B_3=-\frac{1}{3}\int{\int{f(\vec{r})f(\vec{r'})\left(\int{ e^{-i\vec k\cdot \vec r}e^{i\vec k\cdot \vec r}\widetilde{f}(-\vec k) d\vec k}\right)d\vec rd\vec r'}}\end{equation}

\begin{equation}B_3=-\frac{1}{3}\int{\widetilde{f}(-\vec k)\left(\int{f(\vec{r})e^{-i\vec k\cdot \vec r}}d\vec r\int{f(\vec{r'})e^{i\vec k\cdot \vec r'} d\vec r'}\right)d\vec k}\end{equation}

\begin{equation}B_3=-\frac{1}{3}\int{\widetilde{f}(-\vec k)\widetilde{f}(\vec k)\widetilde{f}(-\vec k)d\vec k}\end{equation}

\begin{equation}B_3=-\frac{1}{3}\int_0^{2\pi}\int_0^{\pi}\int_0^{\infty}{|\widetilde{f}(\vec k)|^3d\vec k}\end{equation}

Since the Mayer function does not depend on $\phi$ and $\theta$, this reduced to
\begin{equation}B_3=-\frac{4\pi}{3}\int{|\widetilde{f}(\vec k)|^3k^2dk}\end{equation}

The Mayer function for the Error Function potential $V_{\operatorname{erf}}$, 
and its Fourier Transform are
\begin{equation}f_{\operatorname{erf}}(r)=\frac{1}{2}\left[\operatorname{erf}\left(\frac{r-\alpha}{\Xi}\right)-1\right]\end{equation} 

\begin{equation}\widetilde{f}_{\operatorname{erf}}(k)=-\frac{4\pi}{k^3}e^{-\frac{k^2\Xi^2}{4}}\left[\left(1+\frac{k^2\Xi^2}{2}\right)\sin(k\alpha)-k\alpha\cos(k\alpha)\right]\end{equation} 

The Mayer function for the WCA potential $V_{WCA}$, and its Fourier Transform are
 
\begin{equation}f_{WCA}(r)=\operatorname{exp}\left(-\frac{1}{k_BT}\left[4\epsilon\left(\sigma^{12}r^{-12}-\sigma^{6}r^{-6}\right)+\epsilon\right]\color{white}\frac{1}{1}\color{black}\right)-1\end{equation} 

\begin{equation}\widetilde{f}_{WCA}(k)=\frac{4\pi}{k}\int_0^{\infty}{\left(e^{-\frac{1}{k_BT}\left[4\epsilon\left(\sigma^{12}r^{-12}-\sigma^{6}r^{-6}\right)+\epsilon\right]}-1\right) r\sin(kr)dr}\end{equation} 


\section{General Derivation of the Virial Equation}
The second virial coefficient can be derived from Euler's equation for 
the internal energy of a system.
\begin{equation}PV=TS-(U-\mu{N})\end{equation}
\begin{equation}\label{Euler_rearranged}\frac{PV}{k_BT}=\frac{S}{k_B}-\frac{1}{k_BT}(U-\mu{N})\end{equation}
Setting $\beta=\frac{1}{k_BT}$, and putting in the relations from statistical 
thermophysics which views the thermodynamic quantities as ensemble averages
\begin{equation}S=-k_B\sum_{i=0}^\infty{P_i\ln{P_i}}\end{equation}
\begin{equation}U=<E>=\sum_{i=0}^\infty{P_iE_i}\end{equation}
\begin{equation}N=<n>=\sum_{i=0}^\infty{P_in_i}\end{equation}
Equation~\ref{Euler_rearranged} then becomes
\begin{equation}\frac{PV}{k_BT}=-\sum{P_i\ln{P_i}}-\sum{P_i\beta(E_i-\mu
{n_i}})\end{equation}
\begin{equation}\frac{PV}{k_BT}=-\sum{P_i}\left[\ln{P_i}+\beta(E_i-\mu
{n_i})\right]\end{equation}
For a Grand Canonical Ensemble the probability of a state given by
\begin{equation}P_{n,s(n)}=\frac{e^{-\beta(E_{s(n)}-\mu{n})}}{\sum_{n=0}^\infty\sum_{s(n)}e^{-\beta(E_{s(n)}-\mu{n})}}\end{equation}
results in 
\begin{equation}\frac{PV}{k_BT}=-\sum_{n=0}^\infty\sum_{s(n)}P_{n,s(n)}\left[\ln{\left(\frac{ e^{-\beta(E_{s(n)}-\mu{n})}}{\sum_{n=0}^\infty\sum_{s(n)}e^{-\beta(E_{s(n)}-\mu{n})}}\right)}+\beta(E_{s(n)}-\mu{n})\right]\end{equation}

\begin{equation}\frac{PV}{k_BT}=-\sum_{n=0}^\infty\sum_{s(n)}P_{n,s(n)}\left[-\beta(E_{s(n)}-\mu{n})-\ln{\left(\sum_{n=0}^\infty\sum_{s(n)}e^{-\beta(E_{s(n)}-\mu{n})}\right)}+\beta(E_{s(n)}-\mu{n})\right]\end{equation}

\begin{equation}\frac{PV}{k_BT}=-\sum_{n=0}^\infty\sum_{s(n)}\left(\frac{e^{-\beta(E_{s(n)}-\mu{n})}}{\sum_{n=0}^\infty\sum_{s(n)}e^{-\beta(E_{s(n)}-\mu{n})}} \left[-\ln{\left(\sum_{n=0}^\infty\sum_{s(n)}e^{-\beta(E_{s(n)}-\mu{n})}\right)}\right]\right)\end{equation}

\begin{equation}\frac{PV}{k_BT}=\left[\ln{\left(\sum_{n=0}^\infty\sum_{s(n)}e^{-\beta(E_{s(n)}-\mu{n})}\right)}\right]\left(\frac{\sum_{n=0}^\infty\sum_{s(n)}e^{-\beta(E_{s(n)}-\mu{n})}}{\sum_{n=0}^\infty\sum_{s(n)}e^{-\beta(E_{s(n)}-\mu{n})}}\right) \end{equation}

\begin{equation}\frac{PV}{k_BT}=\ln{\left(\sum_{n=0}^\infty\sum_{s(n)}e^{-\beta(E_{s(n)}-\mu{n})}\right)} \end{equation}

\begin{equation}\frac{PV}{k_BT}=\ln{\left(\sum_{n=0}^\infty \left(e^{\beta\mu}\right)^n\sum_{s(n)}e^{-\beta{E}_{s(n)}}\right)} \end{equation}

\begin{equation}\frac{PV}{k_BT}=\ln{\left(\sum_{n=0}^\infty \lambda^nZ_n\right)} \end{equation}

\begin{equation}\frac{PV}{k_BT}=\ln{\left(\sum_{n=0}^\infty \lambda^nZ_{on}Q_n\right)} \end{equation}
where $Q_n$ is the configuration integral.
For a system of N particles only, the sum collapses to one term, 
that for n=N, which gives 
\begin{equation}\label{PV/kBT_reduces_for_N}\frac{PV}{k_BT}=\ln{\left(\lambda^NZ_{oN}Q_N\right)}
=\beta\mu{N} + \ln{Z}_{oN} +\ln{Q}_N\end{equation}
\begin{equation}\label{PV/kBT_reduces_for_N}\frac{PV}{k_BT}=\beta\mu_{ideal}{N} + \ln{Z}_{oN} + \beta\mu_{excess}{N} + \ln{Q}_N\end{equation}
Noting that $Q_N=1$ for an ideal gas since $V(r)=0$ for an ideal gas
\begin{equation}\frac{P_{ideal}V}{k_BT}=\beta\mu_{ideal}{N} + \ln{Z}_{oN}=N\end{equation}
\begin{equation}\frac{P_{excess}}{k_BT}= \beta\mu_{excess}{N} + \ln{Q}_N\end{equation}
and Equation~\ref{PV/kBT_reduces_for_N} becomes
\begin{equation}\frac{PV}{k_BT}=N + \beta\mu_{excess}{N} + \ln{Q}_N\end{equation}
\begin{equation}\frac{PV}{k_BT}=N + \beta\frac{\partial{F}_{excess}}{\partial{N}}{N} - \frac{F_{excess}}{k_BT}\end{equation}
Plugging in the expression for $F_{excess}$ in terms of the Mayer function 
for low density fluids
\begin{equation}F_{excess}=k_BT\frac{N^2}{V}\left(-\frac{1}{2}\int{f(r)d\vec{r}}\right)\end{equation}
gives
\begin{equation}\frac{PV}{k_BT}=N + \left(-\frac{N}{V}\int{f(r)d\vec{r}}\right){N} + \frac{N^2}{V}\frac{1}{2}\int{f(r)d\vec{r}}\end{equation}
\begin{equation}\frac{PV}{k_BT}=N\left(1-\frac{N}{V}\frac{1}{2}\int{f(r)d\vec{r}}\right)\end{equation}
\begin{equation}\frac{PV}{k_BT}=N\left(1+\frac{N}{V}B_2(T)\right) \end{equation}
where $B_2(T)$ is given by
\begin{equation}B_2=-\frac{1}{2}\int_{Volume}f(r)d\vec{r} \end{equation}

This result was for low densities where higher terms in the 
expression for the excess free energy were neglected. The full virial 
equation is given by
\begin{equation}\frac{PV}{k_BT}=N\left(1+\frac{N}{V}B_2(T)+\left(\frac{N}{V}\right)^2B_3(T)+ \cdot\cdot\cdot\right) \end{equation}


\section{Derivation of $F = -k_BTln(Z)$ for Continuous Systems}
The canonical equilibrium phase-space probability density is given by
\begin{equation}P_{eq}(\vec r^N, \vec p^N)=\frac{1}{N!}\frac{1}{h^{3N}}\frac{e^{-\beta H}}{partian function}\end{equation}
\begin{equation}{partian function}=\frac{1}{N!}\frac{1}{h^{3N}}\int\dot~\dot~\dot~\int e^{-\beta H} d\vec r^{N} d\vec p^{N}\end{equation}
The Helmholtz free energy is given by 
\begin{equation}\label{F=U-TS}F = U - TS\end{equation}
where
\begin{equation}U = <E> = \frac{1}{N!}\frac{1}{h^{3N}}\int\dot~\dot~\dot~\int P_{eq}~H~d\vec r^{N} d\vec p^{N}\end{equation}
\begin{align}
 TS &= T\left[-k_B\frac{1}{N!}\frac{1}{h^{3N}}\int\dot~\dot~\dot~\int P_{eq}\operatorname{ln}\left(N!~h^{3N}P_{eq}\right) d\vec r^{N} d\vec p^{N}\right]   \\
	&= -k_BT\frac{1}{N!}\frac{1}{h^{3N}}\int\dot~\dot~\dot~\int P_{eq}\operatorname{ln}\left(\frac{e^{-\beta H}}{\frac{1}{N!}\frac{1}{h^{3N}}\int\dot~\dot~\dot~\int e^{-\beta H} d\vec r^{N} d\vec p^{N}}\right) d\vec r^{N} d\vec p^{N}   \\
	&= -k_BT\frac{1}{N!}\frac{1}{h^{3N}}\int\dot~\dot~\dot~\int \left[P_{eq}\left(\frac{-1}{k_BT}H\right)-ln\left(\frac{1}{N!}\frac{1}{h^{3N}}\int\dot~\dot~\dot~\int e^{-\beta H} d\vec r^{N} d\vec p^{N}\right) d\vec r^{N} d\vec p^{N}\right]  \\ 
    &= -k_BT\frac{1}{N!}\frac{1}{h^{3N}}\int\dot~\dot~\dot~\int \left[P_{eq}\left(\frac{-1}{k_BT}H\right)-ln\left(\frac{1}{N!}\frac{1}{h^{3N}}\int\dot~\dot~\dot~\int e^{-\beta H} d\vec r^{N} d\vec p^{N}\right) d\vec r^{N} d\vec p^{N}\right]   
\end{align}  
\begin{align}  
    &= \frac{1}{N!}\frac{1}{h^{3N}}\int\dot~\dot~\dot~\int P_{eq}~H~d\vec r^{N} d\vec p^{N} +k_BT\frac{1}{N!}\frac{1}{h^{3N}}\int\dot~\dot~\dot~\int P_{eq}ln(Z)d\vec r^{N} d\vec p^{N}     \\
    &= \frac{1}{N!}\frac{1}{h^{3N}}\int\dot~\dot~\dot~\int P_{eq}~H~d\vec r^{N} d\vec p^{N} +k_BTln(Z)\left[\frac{1}{N!}\frac{1}{h^{3N}}\int\dot~\dot~\dot~\int P_{eq}d\vec r^{N} d\vec p^{N}\right]   \\
    &= \frac{1}{N!}\frac{1}{h^{3N}}\int\dot~\dot~\dot~\int P_{eq}~H~d\vec r^{N} d\vec p^{N} +k_BTln(Z)\left[1\right]   
\end{align}
Plugging these into Equation~\ref{F=U-TS} gives
\begin{equation} F = \frac{1}{N!}\frac{1}{h^{3N}}\int\dot~\dot~\dot~\int P_{eq}~H~d\vec r^{N} d\vec p^{N}-\frac{1}{N!}\frac{1}{h^{3N}}\int\dot~\dot~\dot~\int P_{eq}~H~d\vec r^{N} d\vec p^{N} -k_BTln(Z)\end{equation}
\begin{equation}F = -k_BTln(Z)\end{equation}


\section{Program Outline}

\begin{displaymath}\text{Input:~~~~~~~~~~~~~~~~~~~~~~~~~~~T,~~ n,~~Range of fv,~~Range of gw~~~~~~~~~~~~~~~~~~~~~~~~~~~~~~~~~~~}\end{displaymath} 
\begin{displaymath}\text{Set:~~~~~~~~~~~~~~~~~~~~~~~~~~~~~~~~~~~~~~~~~~~~~~N~=~1~~~~~~~~~~~~~~~~~~~~~~~~~~~~~~~~~~~~~~~~~~~~~~~~~~~~}\end{displaymath} 
\begin{displaymath}\text{Best}\left(\frac{F}{N}\right)_{diff}= 1e100\end{displaymath}
Loop: 
\begin{displaymath}\text{fv = one of the fv values from input}\end{displaymath}
\begin{equation}N_{primitive.cell}=(1-\text{fv})\text{N}\end{equation}
\begin{equation}V_{primitive.cell}=\frac{\text{1}}{\text{n}}(1-\text{fv})\text{N}\end{equation}
\begin{displaymath}\sigma = \text{one of the gw values from input}\end{displaymath}
\begin{equation}{n(r)=\frac{1}{\left(\sqrt{2\pi}\sigma\right)^3}\int\exp^{-\frac{|\vec{r}-\vec{R}|^2}{2\sigma^2}}d\vec{r}}\end{equation} 
\begin{equation}\left(\frac{F}{N}\right)_{crystal}=\frac{cFideal_{primitive.cell} + cFexcess_{primitive.cell}}{N_{primitive.cell}}\end{equation}
\begin{equation}\left(\frac{F}{N}\right)_{liquid}=\frac{lFideal_{primitive.cell} + lFexcess_{primitive.cell}}{N_{primitive.cell}}\end{equation}
\begin{equation}\left(\frac{F}{N}\right)_{diff}=\left(\frac{F}{N}\right)_{crystal}-{~~}\left(\frac{F}{N}\right)_{liquid}\end{equation}
\begin{displaymath}\text{If~~~~~~}\left(\frac{F}{N}\right)_{diff}<\text{~~~~Best}\left(\frac{F}{N}\right)_{diff}\end{displaymath}
\begin{displaymath}\text{Then~~~~~~}\text{Best}\left(\frac{F}{N}\right)_{diff}=\left(\frac{F}{N}\right)_{diff}\end{displaymath}
\begin{displaymath}\text{Output:~~~~~~~~~~~~ Best}\left(\frac{F}{N}\right)_{crystal}\text{~~and~~~~~~}\left(\frac{F}{N}\right)_{liquid}\text{~~~~for the given T, n ~~~~~~~~~~~~~~~~~~~~~~~~~}\end{displaymath} 

%lattice_constant= pow(4*(1-fv)/reduced_density, 1.0/3)
%The number of atoms in one primitive cell is one. 

\bibliography{thesis}
\bibliographystyle{unsrt}

\end{document}




