\documentclass[12pt]{article}
\usepackage{setspace}
\doublespacing
%\usepackage[top=1.2in, bottom=1.2in]{geometry}
%\usepackage[top=1.5in]{geometry}
\usepackage[bottom=1.2in]{geometry}
\usepackage{fancyhdr}
\usepackage{yhmath}
\pagestyle{fancy}
\fancyhf{}
\renewcommand{\headrulewidth}{0pt}
\fancyhead[R]{\thepage}
\usepackage{graphicx}
\usepackage{subcaption}
\usepackage{color}
\usepackage{enumitem}
\title{Freezing of a Weeks-Chandler-Anderson Fluid using classical Density Functional Theory}
\author{by Kirstie Finster}
\date{August 2020}
\begin{document}
  \maketitle
  \thispagestyle{fancy}

\noindent IA. Model Fluids and the Weeks-Chandler-Anderson Fluid (WCA)

Real fluids and their behavior are very complex from a physicist's point of view. Although a water molecule, for example, is composed of only three atoms, the ever-changing redistribution of charges within the molecules, and their interaction with charges from other molecules make it difficult to study a fluid composed of water molecules directly. The task becomes even more daunting considering the complicated molecular motion of each water molecule, and the sheer number of molecules that must be taken into account. 
%Although a water molecule, for example, is composed of only three atoms, the ever-changing redistribution of charges within the molecules and their interaction with charges from other molecules, the complicated molecular motion of a collection of water molecules, and the sheer number of molecules that must be taken into account make it difficult to study a fluid composed of water molecules directly.
The problem of understanding the behavoir of real liquids is better addressed by using models which reduce a complex picture to a simple one. Modifications, or perturbations, can then be incorporated into the model making it progressively more complex until it mimics the behavior of a real liquid as closely as possible.

In developing a model for a fluid, simplifications can be made regarding the composition, type, shape, and interatomic interaction of the molecules. Rather than dealing with any particular molecule or atom, the model only deals with generic "particles", which may also be referred to as "atoms". The shape of the particles, or atoms will be the simplest shape possible - a sphere. The spheres in the simplest models do not exhibit quantum interference effects as the mass of the spheres, and the temperature are taken to be large enough that the de Broglie wavelength is smaller than the mean distance between the spheres. The fluid can then be treated as a classical fluid where the spheres interact with each other according to their electrostatic interatomic potential. 

There are many interatomic potentials which can be used to model a fluid each of which describes the interaction between two real atoms to some degree of accuracy. The simplest, and oldest, model used to describe a fluid is the hard-sphere model. The hard-sphere model envisions a fluid to be composed of hard spheres like, for example, a collection of marbles. This model came about by considering the near incompressibilty of liquids indicating that the particles making up the liquid strongly resist being pushed closer together. This behavior can be described by an interatomic potential energy that is zero when two spheres are not in contact, but goes instantly to infinity when contact is made. The inablilty to push real atoms closer together is due to the repulsion of the electrons in each atom, and is the dominating interparticle electrostatic interaction. Thus, the hard-sphere model - which is the simplest, but still effective representation of this replusive behavior that dominates the interaction between real atoms, is currently the most widely used model for fluids.

\begin{figure}[h!]
    \centering
    \includegraphics[height=7cm]{plot_HardSphere_Potential.pdf}
    \caption{A Hard Sphere Potential is zero for $r>2R$, and infinity for $r\leq{2R}$ which is approximately 1.225 in this example.}
    \label{fig:HardSphere_potential}
  \end{figure}
\[{}\]
\[{}\]
\[{}\]
\[{}\]

Another fairly simple, but relatively accurate interatomic potential widely used to represent generic, neutral atoms is the Leonard-Jones potential. The Leonard-Jones potential describes both the long-range attraction, and the short-range replusion that occurs between two nuetral atoms. The attraction between two nuetral atoms is due to fleeting dipoles that form spontaneously when electrons are redistributed as atoms come into close proximty. These are often characterized as Van der Walls, or London dispersion forces. The attraction grows stronger as two atoms move closer together until their atomic orbitals begin to overlap. %At this point the electrons in each atom begin to repel the electrons in the other atom and oppose the overlap of the two atoms.
 At this point the electrons in each atom begin to oppose the overlap of the two atoms through their replusion. The repulsion increases rapidly and approaches infinity as the distance between the atoms conintues to decrease.  
 
\begin{figure}[h!]
    \centering
    \includegraphics[height=7cm]{LJ_Potential.pdf}
    \caption{Leonard-Jones Potential for $\sigma=1$, $\epsilon=1$}
    \label{fig:LJ_potential}
  \end{figure}

The combined attractive and repulsive effects produce the overall shape of the Leonard-Jones interatomic potential shown in Figure \ref{fig:LJ_potential}. The potential is zero when the distance r between the centers of two atoms is equal to $\sigma$ (shown in the figure at r/$\sigma$=1). At this point the attraction bewteen the atoms just cancels their repulsion. As the distance between the centers of the two atoms decreases below $\sigma$, the potential rapidly approaches positive infinity corresponding to infinite repulsion. Above $\sigma$ the potential becomes negative indicating a net attraction between the atoms. The net attraction is maximized at a distance of r=$2^\frac{1}{6}\sigma=2R\approx{1.122}\sigma$ where R is the radius of the sphere given by $R={2^{-\frac{5}{6}}}\sigma$. At this point the depth of the potential well, given by epsilon $\epsilon$, is a maximum. As r increases further, the net attraction decreases and the potential approaches zero asymptotically. The attractive and repulsive effects can be separated into two potentials, shown with different colors in Figure \ref{fig:LJ_potential_2parts}, which when added together regenerate the full Leonard-Jones potential. 

\begin{figure}[h!]
    \centering
    \includegraphics[height=7cm]{LJ_Potential_2part.pdf}
    \caption{Leanard-Jones Potential broken into two parts for $\sigma=1$, $\epsilon=1$}
    \label{fig:LJ_potential_2parts}
  \end{figure}
 
%\[{}\]
%\[{}\]

\begin{figure}[h!]
    \centering
    \includegraphics[height=7cm]{WCA_Potential.pdf}
    \caption{WCA Potential for $\sigma=1$, $\epsilon=1$}
    \label{fig:WCA_potential}
  \end{figure}

%\[{}\]
%\[{}\]

The repulsive portion of the Leonard-Jones potential forms the potential for a Weeks-Chandler-Anderson (WCA) model fluid, shown in Figure \ref{fig:WCA_potential}. The WCA potential is described by the equation: \begin{equation}{V_{WCA}=\left\{\begin{array}{rcl} {4\epsilon{\left[\left(\frac{\sigma}{r}\right)^{12} - \left(\frac{\sigma}{r}\right)^6 \right]}+\epsilon} & \mbox{for} & 0<r<{2R} \\ 0 & \mbox{for} & r>2R \end{array}\right.}\end{equation} 

\noindent As seen in Figure \ref{fig:WCA_potential}, the WCA potential decreases from infinity as the distance between the centers of the atoms r increases, and becomes, and remains, essentially zero when r=$2^\frac{1}{6}\sigma=2R\approx{1.122}\sigma$ where R is the radius of the sphere as shown in Figure \ref{fig:TwoSpheres}. As with the Lenard-Jones potential, the repulsive WCA potential increases rapidly and approches infinity as the two spheres begin to overlap. This is a more realistic repulsive potential than that of the Hard Sphere model which goes instantly to infinity when the two spheres make contact (at r=2R), and is zero otherwise. Because the WCA spheres have no hard boundary in their interatomic potential, they are called "soft spheres" and their gradually fading potential makes them slightly "squishy".

\begin{figure}[h!]
    \centering
      % FIXME \includegraphics[height=3.5cm]{TwoSpheres.pdf} 
    \caption{(Left) The WCA potential is a minimum at r=2R. 
             (Right) The WCA potential increases rapidly for $r<2R$.}
    \label{fig:TwoSpheres}
    \end{figure} 

The WCA potential is not well suited for implementing Density Functional Theory.  Therefore, an erf potential which closely approximates the WCA potential at temperatures of interest will be used instead of the WCA potential. The erf potential is given by 
\begin{equation}
	{V_{erf}=-k_BT\ln\left[\frac{1}{2}\left(\operatorname{erf}\left(\frac{r-\alpha}{\Xi}\right)+1\right)\right]}
\end{equation} 
where the parameter $\alpha$ is derived by setting the erf potential equal to the WCA potentail when r equals $\alpha$ 
\begin{displaymath}{V_{erf}(r=\alpha)=V_{WCA}(r=\alpha)}\end{displaymath} 
and is given by 
\begin{equation}\label{alphaT}
	{\alpha=\sigma\left(\frac{2}{1+\sqrt{\frac{k_BT}{\epsilon}\ln(2)}}\right)^\frac{1}{6}}
\end{equation} 
The parameter $\Xi$ is derived by matching the slope of the erf potential to the slope of the WCA potential at r=$\alpha$:
\begin{displaymath}{\frac{dV_{erf}(r=\alpha)}{dr}=\frac{dV_{WCA}(r=\alpha)}{dr}}\end{displaymath} 
and is given by: 
\begin{equation}
	{\Xi=\frac{\alpha}{6\sqrt{\pi}\left(\sqrt{\frac{\epsilon}{k_BT}\ln(2)}+\ln(2)\right)}}
\end{equation} 

\noindent These conditions, along with V$_{erf}=$V$_{WCA}$ at $r=\alpha$, help fit the curve of the erf potential to the curve of the WCA potential.

Unlike the WCA potential, the erf potential depends on temperature through the parameters $\alpha$ and $\Xi$. Figure \ref{fig:alphaXivsT} shows how $\alpha$ and $\Xi$ vary with temperature when $\sigma=1$ and $\epsilon=1$. % and Figure \ref{fig:erf_Potential} shows the erf potential at reduced temperatures $\frac{k_BT}{\epsilon}=1$, $\frac{k_BT}{\epsilon}=2$, and $\frac{k_BT}{\epsilon}=3$ alongside the WCA potential. %As seen in Figure \ref{fig:erf_potential}, the erf potentials closely match the WCA potential for reduced temperatures below $\frac{k_BT}{\epsilon}=2$, and for r/$\sigma$ less than ???. % so only values of $V_{erf}$ that are much less than $k_BT$/$\epsilon$ ??? will be considered. 
% In these regions $\alpha$ decreases with increasing temperature, and $\Xi$ increases with increasing temperature.

 \begin{figure}[h!]
    \centering
    \begin{subfigure}[b]{0.4\linewidth}
      \includegraphics[height=5cm]{plot_alpha.pdf}
      \caption{$\alpha$ versus $k_BT/\epsilon$}
    \end{subfigure} 
    \begin{subfigure}[b]{0.4\linewidth}
      \includegraphics[height=5cm]{plot_xi.pdf}
      \caption{$\Xi$ versus $k_BT/\epsilon$}
    \end{subfigure} 
    \caption{These plots, generated with $\sigma=1$ and $\epsilon=1$, show how the parameters $\alpha$ and $\Xi$ vary with temperature. (a) $\alpha$ decreases with increasing temperature. (b) $\Xi$ rapidly increases for temperatures below $\frac{k_BT}{\epsilon}=52$ and then gradually decreases.}
    \label{fig:alphaXivsT}
    \end{figure} 

%\begin{figure}[h!]
%    \centering
    %\includegraphics[height=7cm]{erf_Potential.pdf}
%    \caption{Error Function Potential $V_{erf}(r)$ at several temperatures with $\sigma=1$, $\epsilon=1$, compared to the WCA potential $V_{WCA}(r)$ shown in black.}
%    \label{fig:erf_Potential}
%  \end{figure}

%\color{red}\noindent!ADD PLOT of df/dr=w2*w2 !!! BUT not here!\color{black}

Another way to find values for $\Xi(T)$ is to use the second virial coefficent $B_{2}$ from the equation of state expressed in terms of the virial expansion. 
\begin{equation}\label{virialequation}
    \frac{PV}{k_BT}=N\left(1+\frac{N}{V}B_2(T)+\left(\frac{N}{V}\right)^2B_3(T)+ \cdot\cdot\cdot\right) 
\end{equation}
where 
\begin{equation}
	B_{2}=-\frac{1}{2}\int_{Volume}\left(e^{-\beta{V}(r)}-1\right)d\vec{r} 
\end{equation}
The second virial coefficient can be derived from Euler's equation for the internal energy of a system, 
\begin{equation}U=TS-PV+\mu{N}\end{equation}
using the relations from statistical thermophysics which views the thermodynamic quantities as ensemble averages
\begin{equation}S=-k_B\sum_{i=0}^\infty{P_i\ln{P_i}}\end{equation}
\begin{equation}U=<E>=\sum_{i=0}^\infty{P_iE_i}\end{equation}
\begin{equation}N=<n>=\sum_{i=0}^\infty{P_in_i}\end{equation}
which, for a canonical ensemble, gives
\begin{equation}\frac{PV}{k_BT}=N + \beta\frac{\partial{F}_{excess}}{\partial{N}}{N} - \frac{F_{excess}}{k_BT}\end{equation}
Expressing the Helmholtz free energy in terms of the pair potential $V(r)$ results in the equation of state expressed in terms of the virial expansion as given by Equation~\ref{virialequation}.

The value of $\Xi$ at a given temperature, T, can be found computationally by equating the second virial coefficient evaluated for the error function potential $V_{erf}(r)$, and the second virial coefficent evalued for the WCA potential $V_{WCA}$.
\begin{equation}B_{2erf}(T) =B_{2WCA}(T)\end{equation}
Evaluating $B_{2}$ for the erf potential gives
%\begin{equation}B_{2erf}=-\frac{1}{2}\int_{Volume}\left(e^{-\beta{V}_{erf}(r)}-1\right)d\vec{r} \end{equation}
\begin{equation}
	B_{2erf}(T) = \frac{\pi}{3}\Xi^3\left[\left(\frac{\alpha^3}{\Xi^3}+\frac{3\alpha}{2\Xi}\right)\left(1+\operatorname{erf}\left(\frac{\alpha}{\Xi}\right)\right)+\frac{1}{\sqrt{\pi}}\left(\frac{\alpha^2}{\Xi^2}+1\right)e^{-\left(\frac{\alpha}{\Xi}\right)^2}\right]
\end{equation}
where $\alpha(T)$ is given by Equation~\ref{alphaT} as before. Similarly, $B_{2WCA}$ can be found computationally from 
\begin{equation}
	B_{2WCA}=-\frac{1}{2}\int_{Volume}\left(e^{-\beta{V}_{WCA}(r)}-1\right)d\vec{r} 
\end{equation}
 
A plot of $\Xi(T)$ obtained through this method is shown in Figure \ref{fig:xi_fromB2vsT}, and Figure \ref{fig:erf_Potential_xifromB2} shows the erf potential at a few different temperatures alongside the WCA potential. 

\begin{figure}[h!]
    \centering
    \includegraphics[height=6cm]{xi_fromB2.pdf}
    \caption{Shows the temperature dependence of $\Xi$ generated from matching the second virial coeficients for $V_{erf}(r)$ and $V_{WCA}(r)$ at each temperature.}
    \label{fig:xi_fromB2vsT}
  \end{figure}


\begin{figure}[h!]
    \centering
    \includegraphics[height=6cm]{erf_Potential_xifromB2.pdf}
    \caption{Error Function Potential $V_{erf}(r)$ at several temperatures with $\sigma=1$, $\epsilon=1$, compared to the WCA potential $V_{WCA}(r)$ shown in black.}
    \label{fig:erf_Potential_xifromB2}
  \end{figure}

%\color{red}
% FIXME \noindent!ADD PLOT of df/dr=w2*w2 !!! BUT not here!
%\color{black}

%The spheres in the WCA model fluid do not exhibit quantum interference effects as the mass of the spheres, and the temperatures are taken to be large enough that the de Broglie wavelength is smaller than the mean distance between the spheres. Thus, the WCA fluid can be treated as a classical fluid.

\[\]
\noindent IB. The Theory of Freezing and Maxwell's Double Tangent Construction

Freezing is governed by entropy, which the universe, or any closed system, seeks to maximize according to the second law of Thermodynamics. When the entropy of a pure substance in its liquid state is lower than the entropy that it would have in its solid state at the same temperature (necessary for thermal equilibrium), the same pressure (necessary for mechanical equilibrium), and the same chemical potential (necessary for diffusive equilibrium), an abrupt phase transition occurs, that is, the liquid freezes forming a crystalline structure. The particular crystalline structure and density that result are affected by the interatomic potential. When repulsive, soft sphere fluids (that are not too soft) freeze, they form lattice structures composed of Face Centered Cubic (FCC) crystal cells, as do hard spheres. %This structure is the three-dimensional equivilant to the two-dimensional structure that one would see by filling the bottom of a box with a single layer of marbles. 
This structure is what one would see by filling a cube with marbles. It results in the highest density achievable for hard, and nearly-hard, spheres that are purely repulsive. 

%move this paragraph to methods?
Although the entire crystal lattice can be constructed with FCC cells, FCC cells are not the smallest cells with which the lattice can be built. The shaded parallepiped shown within a FCC cell in Figure \ref{fig:primitivecell_firsttime} is the smallest structure that, when repeated, re-creates the entire crystal lattice. The parallepiped primitive cell takes up one-fourth the volume of an FCC cell, includes portions of eight atoms for a total of one full atom, and has eight lattice sites convenietly reachable by moving along one of three different lattice vectors to step from one atom to the next. The three lattice vectors will be useful in creating the lattice structure computationally, and will also make the program run faster.

  \begin{figure}[h!]
    \centering
    % FIXME \includegraphics[height=3.5cm]{PrimitiveCellLightBlue.png}
    \caption{The Primivitive Crystal Cell is shown by the shaded parallelpiped}
    \label{fig:primitivecell_firsttime}
  \end{figure}
%-------------------------------

For pure substances, a phase transition occurs along a line on a Pressure-Temperature (P-T) phase diagram, such as can be seen in the phase diagram shown in Figure \ref{fig:P-T_Diagram}. This phase diagram is representative of some arbitrary, unknown substance composed of atoms, or molecules which exibit both attraction and repulsion. However, since the atoms in a WCA fliud do not exhibit attraction there will not be a dividing line between the gas and liquid phases, and so the term "liquid" or "gas" is not used, but simply "fluid". Points along the line that separates the solid and the fluid phases indicate a phase transition, and show that the solid and the fluid both have the same temperature and the same pressure at a phase transistion.

%For each particular temperature, the pressure of the liquid and the solid are the same at the point of transistion. 
\begin{figure}[h!]
    \centering
    % FIXME \includegraphics[height=5cm]{P-T_Diagram.png}
    \caption{Conceptual Pressure-Temperature Phase Diagram}
    \label{fig:P-T_Diagram}
  \end{figure}
The pressure can be related to the Helmholtz free energy F, which is minimized for a system at fixed temperature, volume and number of particles as the entropy of the system and its surroundings is maximized. The negative of the partial derivative of the Helmholtz free energy with respect to volume V at a fixed temperature T and number of particles N equals the pressure. 
\begin{equation}{P=-\frac{\partial{F}}{\partial{V}}\bigg|_{T,N}}\end{equation}
\noindent Dividing both the Helmholtz free energy and the volume by the number of particles, the partial derivative of the Helmholtz free energy with respect to volume becomes the partial derivative of the Helmholtz free energy per atom with respect to the inverse number density. \begin{equation}{P=-\frac{\partial{\frac{F}{N}}}{\partial{\frac{V}{N}}}\bigg|_{T,N} = -\frac{\partial{f}}{\partial{\frac{1}{n}}}\bigg|_{T,N}}\end{equation} Thus, the pressure corresponds to the slope of the curve on a plot of the Helmholtz free energy per atom versus the inverse density at fixed temperature. 

%In a process called Maxwell's Double Tangent construction, two curves are used to find the densities of the liquid and the solid at the point of transistion. Figure \ref{fig:MaxwellDT} shows a curve for the liquid, and a curve for the solid both at a fixed temperature. Since the pressure of the liquid is the same as the pressure of the solid at the point of transistion, the density of the liquid and the density of the solid at the point of freezing can be found from identifying the inverse densties at which the tangents to each curve have the same slope.%
Two such curves are shown in Figure \ref{fig:MaxwellDT}, one for the liquid and one for the solid at a fixed temperature. Since the pressure of the liquid is the same as the pressure of the solid at the point of transistion, the density of the liquid and the density of the solid at the point of freezing can be found from identifying the inverse densties at which the tangents to each curve have the same slope. This process is called Maxwell's Double Tangent Construction. 

\begin{figure}[h!]
    \centering
     \includegraphics[height=7cm]{figs/MaxwellDTC-Fig1.pdf}
     % FIXME\includegraphics[height=7cm]{plot-pressure_Fig1.pdf}
    \caption{Identifying densities at the solid-liquid transition using Maxwell's Double Tangent Construction}
    \label{fig:MaxwellDT}
  \end{figure}

To facilitate finding the pressure at the phase transition, %, which occurs at the points which have the same slope,
it is helpful to consider the Gibbs free energy given by G=F+PV. Rearranging the terms and dividing by N gives an  expression for the Helmholtz free energy per atom $f$ in terms of the Gibbs free energy per atom  $g$ (which is also the chemical potential):  %From the Gibbs free energy an expression can be derived which relates  ....: 

%\begin{displaymath}{F=-PV+G}\end{displaymath} 
\begin{displaymath}{\frac{F}{N}=-P\frac{V}{N}+\frac{G}{N}}\end{displaymath} 
\begin{displaymath}{f=-P\frac{1}{n}+g}\end{displaymath}
\noindent This equation has the form of a line:
\begin{displaymath}{y=mx+y_o}\end{displaymath}
%\begin{displaymath}{G_1=G_2}\end{displaymath} 
%\begin{displaymath}{F_1+PV_1=F_2+PV_2}\end{displaymath} 
%\begin{displaymath}{\frac{F_1}{N}+P\frac{V_1}{N}=\frac{F_2}{N}+P\frac{V_2}{N}}\end{displaymath} 
\noindent and describes the red line shown in Figure \ref{fig:MaxwellDT} with the pressure given by the negative of the slope of the line, and the Gibb's free energy per atom given by the y-intercept. At the point of phase transition:
\begin{displaymath}{g_L=g_S}\end{displaymath} 
\begin{displaymath}{f_L+P\frac{1}{n_L}=f_S+P\frac{1}{n_S}}\end{displaymath}

The pressure at the transistion point can be obtained from the point of intersection that occurs when the Gibbs free energy per atom is plotted against the pressure for both the liquid and the solid at fixed temperature, as shown in Figure \ref{fig:GibbsvsP}. These curves are constructed from the slope and cooresponding y-intercept data collected from lines tangent along the curves on the the Helmholtz free energy per atom verses inverse density plots for both the liquid and the solid. 
\begin{figure}[h!]
    \centering
     %\includegraphics[height=7cm]{MaxwellDTC-Fig2.pdf}
     %FIXME\includegraphics[height=7cm]{plot-pressure_Fig2.pdf}
    \caption{Pressure is found at point of intersection which occurs when $g_L=g_S$}
    \label{fig:GibbsvsP}
  \end{figure}
%CURVE FITTING! - put in methods/analysis


Once the pressure is known for the transition point, the density of the liquid at the transition point can be found from a plot of the pressure versus the inverse of the density for the liquid. The density of the solid at the transition point can be found the same way, as shown in Figure \ref{fig:Pvsinvn}.
%\[\]

\begin{figure}[h!]
    \centering
    %\includegraphics[height=7cm]{MaxwellDTC-Fig3.pdf}
    % FIXME\includegraphics[height=7cm]{plot-pressure_Fig3.pdf}
    \caption{The liquid and solid densities at the point of transistion are found from a plot of pressure vs inverse denisty for both the liquid and the solid.}
    \label{fig:Pvsinvn}
  \end{figure}

A Temperature-Density (T-n) phase diagram, such as that in Figure \ref{fig:T-n_Diagram}, can be generated by repeating this procedure for various temperatures. %The lowest density the crystal exhibts and the highest density the fluid exhibits at the phase transistion can then be identified for each temperature.  

\begin{figure}[h!]
    \centering
    % FIXME \includegraphics[height=4cm]{T-n_Diagram.png}
    \caption{Conceptual Temperature-Density Phase Diagram used to identify the number densities of the liquid and the solid during a phase transistion. The gray region is where the liquid and the solid coexist, and are therefore not pure states.}
    \label{fig:T-n_Diagram}
  \end{figure} 
%One way of implementing this practically is to determine what the Helmholtz free energy would be for both the liquid and solid phases at a particular temperature and density. Whichever phase has the lower Helmholtz free energy will be the phase that the substance will be in. 
%\[\]
\[\]
IC. Classical Density Functional Theory (cDFT)

The key idea behind classical Density Functional Theory (cDFT) is that free energy can be written as a functional (a function of a function) of a number density profile $n(\vec{r})$ which describes the way the number density, or atoms per volume, varies spatially. 
\begin{displaymath}{f[n(}\vec{r}{)]}\end{displaymath}
This, coupled with the fact that the free energy of a system tends toward a minimum at equilibrium means that the equilibrium free energy of the system ${f_{equil}}{[}\rho{(}\vec{r}{)]}$, and its corresponding equilibrium number density profile $\rho{(}\vec{r}{)}$, can be found simply by varying $n(\vec{r})$ through all possible spatial profiles until the free energy is minimized. 

\begin{displaymath}{f[n(}\vec{r}{)]}\rightarrow{f_{equil}}{[}\rho{(}\vec{r}{)]}  \mbox{ as $f$ is minimized} \end{displaymath}
\begin{displaymath}{\mbox{where }  n(\vec{r})=\rho{(}\vec{r})  \mbox{ at equilibrium}}\end{displaymath}

%The reason this works is because cDFT satisfies the conditions of the variational principle, the same mathematical principle used to determine that the shortest distance between two points is a straight line by varying the path between the two points until the distance is minimized.

In this paper a system with a fixed temperature, volume, and number of particles will be examined in which case it is the Helmholtz energy as a functional of the number density profile $F[n(\vec r)]$ that is of interest. The Helmholtz free energy F is defined as 
\begin{equation}F=U-TS\end{equation}
\begin{equation}\frac{F}{N}=\frac{U}{N}-\frac{TS}{N}\end{equation}
\begin{equation}\label{usetoshowmin}d\left(\frac{F}{N}\right)=d\left(\frac{U}{N}\right)-d\left(\frac{TS}{N}\right)\end{equation}
\begin{equation}d\left(\frac{F}{N}\right)=d\left(\frac{U}{N}\right)-\frac{T}{N}dS-\frac{S}{N}dT-TS d\left(\frac{1}{N}\right)\end{equation}
Setting $f_N=\frac{F}{N}$ this becomes
\begin{equation}df_N=-\frac{S}{N}dT-\frac{P}{N}dV-PVd\left(\frac{1}{N}\right)\end{equation}
This result shows that the natural variables of the Helmholtz free energy per particle $f_N$ are temperature, volume and one over the number of particles, which in turn also holds the number density $n=\frac{V}{N}$ fixed. It is the Helmholtz free energy will be minimized in a system that has a temperature, volume, and number of particles that are fixed.
%\begin{equation}f_N(T,V, \frac{1}{N})\mbox{~~~~or~~~~}f_N(T,n)\end{equation}
\begin{equation}f_N(T,V, \frac{1}{N})\end{equation}
 
The Helmholtz free energy can be written as
\begin{equation}{F=-k_{B}T\ln(Z)}\end{equation}
where the paritian function Z is integrated over all of phase space for N identical particles each with a momentum $\vec{p}$ and a position $\vec{r}$
\begin{equation}{Z=}\frac{1}{N!}\frac{1}{h^{3N}}\int{...}\int{dPdQ}~e^\frac{-H(\vec{p}_1,\vec{p}_2,...\vec{p}_N,\vec{r}_1, \vec{r}_2,...\vec{r}_N)}{k_BT}\end{equation}
\begin{displaymath}{dP=d\vec{p}_1d\vec{p}_2...d\vec{p}_N \mbox{~~~~where~~~~} d\vec{p}=dp_xdp_ydp_z}\end{displaymath}
\begin{displaymath}{dQ=d\vec{r}_1d\vec{r}_2...d\vec{r}_N \mbox{~~~~where~~~~} d\vec{r}=dxdydz\mbox{~~~~}}\end{displaymath}

Breaking the Hamiltonian, H into its component energies
\begin{equation}{H = KE + V + V_{ext} = \sum_{i=1}^N\left(\frac{|\vec{p}_i|^2}{2m}\right)+V(\vec{r}_1,\vec{r}_2,{...},\vec{r}_N)+V_{ext}(\vec{r}_1,\vec{r}_2,{...},\vec{r}_N)}\end{equation}
where the translational kinetic energy of the N atoms is given by \begin{displaymath}{\sum_{i=1}^N\left(\frac{|\vec{p}_i|^2}{2m}\right)}\end{displaymath}
the potential energy of the total interaction of each atom with the other N-1 atoms is given by
\begin{displaymath}{V(\vec{r}_1,\vec{r}_2,{...},\vec{r}_N)}\end{displaymath} and \begin{displaymath}{V_{ext}(\vec{r}_1,\vec{r}_2,{...},\vec{r}_N)}\end{displaymath} is an externally applied potential that varies spatially, such that its effects on each atom depend on the atom's position. Often $V_{ext}$ is set to zero, as it is in this paper. 

Noting that the kinetic energy depends only on momentum while the total interatomic and external potential energies depend only on position, the integral in equation (8) can be separated into two parts 
\begin{equation}{Z=\frac{1}{N!}\frac{1}{h^{3N}}\int...\int{dP}~e^\frac{-(\frac{|\vec{p}_1|^2}{2m}+\frac{|\vec{p}_2|^2}{2m}+...\frac{|\vec{p}_N|^2}{2m})}{k_BT}\int...\int{dQ}~e^\frac{-V(\vec{r}_1,\vec{r}_2,{...},\vec{r}_N)-V_{ext}(\vec{r}_1,\vec{r}_2,{...},\vec{r}_N)}{k_BT}}\end{equation}  Multiplying by $\frac{V^N}{V^N}$ where V is the real space volume satisfying \begin{equation}{V^N=}\int{...}\int{dQ}\end{equation} 
%the equation can be written
%\begin{equation}{Z=\left(V^N\frac{1}{N!}\frac{1}{h^{3N}}\int{...}\int{dP}e^\frac{-(\frac{|\vec{p}_2|^2}{2m}+ \frac{|\vec{p}_2|^2}{2m}+...\frac{|\vec{p}_N|^2}{2m})}{k_BT}\right)\left(\frac{1}{V^N}\int{...}\int{dQ}e^\frac{-V(\vec{r}_1,\vec{r}_2,{...},\vec{r}_N)-V_{ext}(\vec{r}_1,\vec{r}_2,{...},\vec{r}_N)}{k_BT}\right)}\end{equation}  
the partitian function can be expressed as
\begin{equation}\label{Zmultiply}{Z=Z_{ideal}Z_{excess}}\end{equation}
where
\begin{equation}{Z_{ideal}=V^N\frac{1}{N!}\frac{1}{h^{3N}}\int{...}\int{dP}~e^\frac{-(\frac{|\vec{p}_1|^2}{2m}+ \frac{|\vec{p}_2|^2}{2m}+...\frac{|\vec{p}_N|^2}{2m})}{k_BT}}\end{equation}
\begin{equation}{Z_{excess}=\frac{1}{V^N}\int{...}\int{dQ}~e^\frac{-V(\vec{r}_1,\vec{r}_2,{...},\vec{r}_N)-V_{ext}(\vec{r}_1,\vec{r}_2,{...},\vec{r}_N )}{k_BT}}\end{equation} The ideal partitian function $Z_{ideal}$ is associated with an ideal gas which exhibits no interaction between the atoms - a condition approached by a gas at low density where the atoms rarely encounter one another. The excess partitian function $Z_{excess}$ accounts for the interaction of all the atoms according to their interatomic potential. The free energy can also be broken into two parts using Equation~\ref{Zmultiply} giving
%\begin{equation}{F=-k_{B}T\ln(Z)}\end{equation}
\begin{equation}{F=-k_{B}T\ln(Z_{ideal}Z_{excess})}\end{equation}
\begin{equation}{F=-k_{B}T\ln(Z_{ideal})-k_{B}T\ln(Z_{excess})}\end{equation}
\begin{equation}{F=F_{ideal} + F_{excess}}\end{equation} 

The ideal Helmholtz free energy is
\begin{displaymath}{F_{ideal}=-k_BT\ln{\left(V^N\frac{1}{N!}\frac{1}{h^{3N}}\int{...}\int{dP}~e^\frac{-(\frac{|\vec{p}_1|^2}{2m}+ \frac{|\vec{p}_2|^2}{2m}+...\frac{|\vec{p}_N|^2}{2m})}{k_BT}\right)}}\end{displaymath}  

\begin{equation}{\label{Fidealmultipleint}}{=-k_BT\ln\left(V^N\frac{1}{N!}\frac{1}{h^{3}}\int{d\vec{p}_1}~e^\frac{-\frac{|\vec{p}_1|^2}{2m}}{k_BT}\frac{1}{h^{3}}\int{d\vec{p}_2}~e^\frac{-\frac{|\vec{p}_2|^2}{2m}}{k_BT}{...}\frac{1}{h^{3}}\int{d\vec{p}_N}~e^\frac{-\frac{|\vec{p}_N|^2}{2m}}{k_BT}\right)}\end{equation}
These integrals can conveniently be written in terms of the DeBroglie wavelength $\Lambda$, or the quantum volume $\text{n}_\text{Q}$
\begin{equation}{\frac{1}{h^{3}}\int{d\vec{p}}~e^\frac{-\frac{|\vec{p}|^2}{2m}}{k_BT}=\left(\sqrt{\frac{k_BTm}{2\pi\hbar^2}}\right)^3=\frac{1}{\Lambda^{3}}}=\text{n}_\text{Q}\end{equation} 
where \begin{equation}{\Lambda =\sqrt{\frac{2\pi\hbar^2}{k_BTm}}}\end{equation} 
Rewriting Equation~\ref{Fidealmultipleint} in terms of $\Lambda$ gives
\begin{equation}{F_{ideal}= -k_BT[\ln(V^N)-\ln(N!) - \ln(\Lambda^{3N})]}\end{equation}Using Stirling's approximation \begin{displaymath}{\ln(N!)=N\ln{N}-N}\end{displaymath} this becomes
\begin{equation}{F_{ideal}= k_BT[N\ln(\frac{N\Lambda^{3}}{V}{)-N}]}\end{equation} 
Rewriting the equation in terms of the number density $n=\frac{N}{V}$, and the quantum volume, $\text{n}_\text{Q}$ gives
\begin{equation}{F_{ideal}= k_BT\left[\frac{N}{V}\ln{\left(\frac{N}{V}\frac{1}{\text{n}_\text{Q}}\right)}-\frac{N}{V}\right]}V\end{equation}
\begin{equation}{F_{ideal}= k_BT[n\ln(n/\text{n}_\text{Q})-n]V}\label{eq:Fideal}\end{equation}   
For a spatially varying number density, the ideal free energy in Equation~\ref{eq:Fideal} becomes
\begin{equation}{F_{ideal}[n(\vec{r})]= k_BT\int_{Vol}[n(\vec{r})\ln(n(\vec{r})/\text{n}_\text{Q})-n(\vec{r})]d\vec{r}}\end{equation} 

Now consider the excess free energy which is written most generally as
\begin{equation}\label{Fexcess-mostgeneral}{F_{excess}= -k_BT\ln{\left(\frac{1}{V^N}\int{...}\int{dQ}~e^\frac{-V(\vec{r}_1, \vec{r}_2,...\vec{r}_N)-V_{ext}(\vec{r}_1, \vec{r}_2,...\vec{r}_N)}{k_BT}\right)}}\end{equation} 
This integral can be greatly simplified if it can be assumed that the potential energy $V(\vec{r_1},\vec{r_2},...\vec{r_N})$ is a sum of the potential energies between pairs of atoms, where each atom is paired in turn with each of the other atoms in the system. In this case $V(\vec{r}_1, \vec{r}_2,...\vec{r}_N)$ becomes 
\begin{equation}{V(\vec{r_1},\vec{r_2},...\vec{r_N})=\frac{1}{2}\sum^N_i\sum^N_{j\neq{i}}V_{ij}(\vec{r_i},\vec{r_j})}\end{equation} 
%\begin{displaymath}{=V_{12}(\vec{r_1},\vec{r_2})+V_{13}(\vec{r_1},\vec{r_3}) + V_{14}(\vec{r_1},\vec{r_4}) + ...}\end{displaymath}
%\begin{displaymath}{+ V_{23}(\vec{r_2},\vec{r_3})+V_{24}(\vec{r_2},\vec{r_4})+V_{25}(\vec{r_2},\vec{r_5})+ ...}\end{displaymath} 
%\begin{displaymath}{+ V_{34}(\vec{r_3},\vec{r_4})+V_{35}(\vec{r_3},\vec{r_5})+V_{36}(\vec{r_3},\vec{r_6})+...}\end{displaymath}
Using this potential, and setting the external potential $V_{ext}(\vec{r}_1, \vec{r}_2,...\vec{r}_N)$ to zero, the exponent in Equation~\ref{Fexcess} becomes

\begin{displaymath}{e^{-\frac{\frac{1}{2}\sum^N_i\sum^N_{j\neq{i}}V_{ij}(\vec{r_i},\vec{r_j})}{k_BT}}=e^{\frac{-V_{12}(\vec{r_1},\vec{r_2})}{k_BT}}e^{-\frac{V_{13}(\vec{r_1},\vec{r_3})}{k_BT}}e^{-\frac{V_{14}(\vec{r_1},\vec{r_4})}{k_BT}}...}\end{displaymath}    \begin{equation}{\approx-(1+f_{12})(1+f_{13})(1+f_{14})...}\end{equation}
where each exponential term is simplified by a Taylor expansion $e^x\approx{1+x}$. %\begin{equation}{\exp^{-V_{ij}(\vec{r_i},\vec{r_j})}\approx 1+f_{ij}(\vec{r_i},\vec{r_j})}\end{equation}%    %$\exp^x=1+x+\frac{x^2}{2!}+\frac{x^3}{3!}...$.%  
The function $f_{ij}$ is called a Mayer function, and is given by
\begin{equation}{f_{ij}(\vec{r_i},\vec{r_j})=e^{-\frac{V_{ij}(\vec{r_i},\vec{r_j})}{k_BT}}-1}\end{equation} 
Multiplying the Mayer function terms, the excess free energy can be written as
\begin{equation}\label{Fexcess-simplified}{F_{excess}=-k_BT\ln{\left(\frac{1}{V^N}\int{...}\int{dQ}\left[1 + \sum{f_{ij}} + \sum{f_{ij}f_{kl}} +\sum{f_{ij}f_{kl}f_{mn}} +... \right]\right)  }}\end{equation}

  \begin{figure}[h!]
    \centering
    %\includegraphics[height=7cm]{diagrammic.pdf}
    \caption{Two-pair Interactions $f_{ij}f_{kl}$ for N=4 atoms. Interations between individual pairs of atoms are shown in green. 
             Interactions between pairs of atoms with an atom shared in common are shown in red. 
             At low densities only the interactions shown in green are considered, since atoms are not likely to be clustered.}
    \label{fig:diagrammic}
  \end{figure}

At low densities, it is not likely to find two or more atoms in close proximity, and so only interactions between individual pairs of spheres, such as those shown in green in Figure~\ref{fig:diagrammic}, are considered.
Each pair can then be described by a function of the distance between the two atoms in the pair. Making a change of variables $\vec{r}=\vec{r}_i-\vec{r}_j $ for each pair we get 
\begin{equation}f_{ij}(\vec{r_i}-\vec{r_j})=f(\vec{r})=e^{-\frac{V(\vec{r})}{k_BT}}-1\end{equation} 
\color{black}
With a careful counting of the resulting number of such pairs, the excess free energy can then be written as 
\begin{equation}{F_{excess}=-k_BT\ln\left(1+\frac{1}{2}\frac{N^2}{V}\int{f(r)}{d\vec{r}}+\frac{1}{2}\left(\frac{1}{2}\frac{N^2}{V}\int{f(r)}{d\vec{r}}\right)^2+ ...\right)}\end{equation}
where in the thermodynamic limit (large N), $N-1, N-2, ...\approx{N}$. 
Written in terms of an exponential $e^x=1+x+\frac{x^2}{2!}+\frac{x^3}{3!}+ ...$ for a homogeneous density n=$\frac{N}{V}$ this becomes
\begin{equation}{F_{excess}=-k_BT\ln{\left(e^{\frac{1}{2}\frac{N^2}{V}\int{f(r)}{d\vec{r}}}\right)}}\end{equation} 
Thus the free excess energy per atom in the low-density limit is given by 
\begin{equation}{\frac{F_{excess}}{N}=k_BTn\left(-\frac{1}{2}\int{f(r)}{d\vec{r}}\right)}\end{equation} where the quantity in parenthesis can be recognized 
as the second virial expanision coefficient typically called $B_2$. %where $x=\frac{1}{2}\frac{N(N-1)}{V^{2}}\int{...}\int{dQ}{f_{12}}$.% 
It is important that the expression for the excess Helmholtz free energy correctly reduces to this result in the low-density limit. 
%Although we will not be restricting ourselves to the low-density limit, it is important that the theory we use correctly reduces to this same result in the low-density limit. 
Outside of the low density limit, it is quite difficult to evaluate the excess Helmholtz free energy given in Equation~\ref{Fexcess-simplified}. 

Another way to form an expression for the excess Helmholtz free energy in terms of a homogeneous number density is to use Scaled Particle Theory (SPT).
Scaled particle theory seeks a general expression for the excess chemical potential of a homogeneous Hard Sphere fluid which can then be integrated to find the excess Helmholtz free energy.
\begin{equation}F_{ex}=\int{\mu_{ex}dN}\end{equation}
It is an extension of the work done by Widom who showed that the excess chemical potential of a homogenesous fluid corredsponds to the reversible work 
of inserting a sphere of radius R into a fluid consisting of hard spheres of radius R. 
Widom's result was obtained by considering the difference in the excess Helmholtz free energy when one atom is added to an otherwise closed system. 
He showed that the excess chemcial potential is related to the ensemble average of the Boltzman factor for the potential energy difference $V_{diff}$ 
between a system with N atoms and one with N+1 atoms. The resulting equation is known as Widom's Insertion Formula.
%\begin{displaymath}\mu_{ex}=\frac{\partial{F_{ex}}}{\partial{N}}\bigg|_{T,V}{~~~~~~}\end{displaymath}
\begin{equation}\mu_{ex}=\frac{F_{ex}(N+1)-F_{ex}(N)}{\Delta{N}}\bigg|_{T,V}\end{equation}
\begin{equation}\label{widoms-insertion-formula}{~}=-k_BT\ln\left(<e^{-\beta{V_{diff}}}>\right)\end{equation}
\begin{equation}=-k_BT(\ln(1-n\frac{4\pi}{3}(R_{cavity}+R)^3))\end{equation}
where the radius of the cavity $R_{cavity}$ is equal to the radius of a hard sphere, R. Forming a more general result for different sizes of cavities in the fluid, Scaled Partical Theory arrives at an expression for the excess Helmholtz free energy of a homogeneous fluid in terms of the packing fraction $\eta$ 
\begin{equation}\label{Fexcess-SPT}{\frac{\beta{F_{excess}}}{N}=-ln(1-\eta)+\frac{3\eta}{1-\eta}+\frac{3{\eta}^2}{2(1-\eta)^2}}\end{equation} 
where
\begin{equation}{\eta = \frac{\mbox{Volume occupied by the spheres}}{\mbox{Total Volume}}}\end{equation}
\begin{equation}{\eta = \frac{N\frac{4}{3}\pi{R^3}}{V}}\end{equation}

An expression for the excess Helmholtz free energy as a functional of the number density profile $n(\vec r)$ can be found using Fundamental Measure Theory (FMT).
FMT was developed for hard sphere fluids, and has been very sucessful at predicting the thermodynamic properties of inhomogeneous hard sphere fluids starting from an expression for the excess Helmholtz free energy obtained from Scaled Partical Theory. 
Soft Fundamental Measure Theory (SFMT), which will be addressed later, extends FMT to fluids with soft potentials like that of the WCA fluid.

%\begin{equation}{\eta = \frac{NR_{3}}{V}}\end{equation}
%\begin{equation}{\eta = \left(\frac{N}{V}\right)\frac{4}{3}\pi{R^3}}\end{equation}
%\begin{equation}{\eta = n\frac{4}{3}\pi{R^3}=nR_{3}=n_{3}}\end{equation}
From Equation~\ref{Fexcess-SPT}, a quantity $\zeta_{3}$ can be defined as follows
\begin{equation}{\eta =\left(\frac{N}{V}\right)\frac{4}{3}\pi{R^3}=n\zeta_{3}}\end{equation}
where $\zeta_{3}$ is the volume of a sphere. Likewise,
\begin{equation}{\zeta_{2}=4\pi{R^2}}\end{equation}
\begin{equation}{\zeta_{1}=R}\end{equation}
\begin{equation}{\zeta_{0}=1}\end{equation}
where $\zeta_{2}$ is the surface area of a sphere, and $\zeta_{1}$ is the radius of a sphere. The quantities $\zeta_{3}, \zeta_{2}, \zeta_{1},$ and $\zeta_{0}$ are fundamental geometric measures of a sphere, and they will prove useful by defining a set of weighted densities $n_{i}=n\zeta_{i}$ where $n=\frac{N}{V}$. 
\begin{equation}\label{n3}{n_{3}=n\zeta_{3}=\left(\frac{N}{V}\right)\frac{4}{3}\pi{R^3}=\eta}\end{equation}
\begin{equation}\label{n2}{n_{2}=n\zeta_{2}=\left(\frac{N}{V}\right)4\pi{R^2}=\frac{3\eta}{R}}\end{equation}
\begin{equation}\label{n1}{n_{1}=n\zeta_{1}=\left(\frac{N}{V}\right)R}\end{equation}
\begin{equation}\label{n0}{n_{0}=n\zeta_{0}=\left(\frac{N}{V}\right)}\end{equation}

\noindent The excess free energy in Eq~\ref{Fexcess-SPT} can be written in terms of the number density $\frac{N}{V}$
\begin{equation}{F_{excess}=k_{B}T\left(-\frac{N}{V}ln(1-\eta)+\frac{N}{V}\frac{3\eta}{1-\eta}+\frac{N}{V}\frac{3{\eta}^2}{2(1-\eta)^2}\right)V}\end{equation}
and using the expressions in Equations~\ref{n3} through~\ref{n0}, this becomes
\begin{equation}\label{FexfromSPT}{F_{excess}=k_{B}T\left(-n_{0}ln(1-n_{3})+\frac{n_{1}n_{2}}{1-n_{3}}+\frac{n_{2}^3}{24\pi(1-n_{3})^2}\right)V}\end{equation}
which has the form 
\begin{equation}{F_{excess} = k_BT(\Phi_1+\Phi_2+\Phi_3)V}\end{equation}
where 
\begin{align}
   \Phi_1 &= -n_{0}ln(1-n_{3}) \\
   \Phi_2 &= \frac{n_{1}n_{2}}{1-n_{3}} \\
   \Phi_3 &= \frac{n_{2}^3}{24\pi(1-n_{3})^2} 
\end{align}

The quantities $n_{0},n_{1},n_{2},n_{3}$ can be considered as weighted densities given the relation $n_{i}=n\zeta_{i}$, where the weights are given by $\zeta_{0},\zeta_{1},\zeta_{2},\zeta_{3}$ respectively. These weights are valid for determining the excess free energy of a system of hard spheres with a homogeneous number density n. But these weights, and their related  weighted densities, can be written in a more general form allowing for their extention to a system with a spacially varying number density, that is, an inhomogeneous fluid. 

Using a backwards Heavyside step function defined as
\begin{displaymath}{\Theta(|\vec{r}|-R)=\left\{ \begin{array}{rc} 1 & 0<r \leq R \\ 0  & r>R \end{array}\right.}\end{displaymath}
the following expressions are obtained
\begin{equation}{\zeta_{3}=\int_{Volume}{\Theta(|\vec{r}|-R)d{\vec{r}}} = \frac{4}{3}\pi{R^3}}\end{equation}
\begin{equation}{\zeta_{2}=\int_{Volume}{\delta(|\vec{r}|-R)d{\vec{r}}} = 4\pi{R^2}}\end{equation}

%\begin{equation}{\zeta_{1}=\frac{R_{2}}{4\pi{R}}=R}\end{equation}
\begin{equation}{\zeta_{1}=\int_{Volume}{\frac{\delta(|\vec{r}|-R)}{r}d{\vec{r}}} = R}\end{equation}
%\begin{equation}{\zeta_{0}=\frac{R_{2}}{4\pi{R}^2}=1}\end{equation}
\begin{equation}{\zeta_{0}=\int_{Volume}{\frac{\delta(|\vec{r}|-R)}{r^2}d{\vec{r}}} = 1}\end{equation}

The weighted densities can now be expressed as
\begin{equation}{\color{red} CHECK!~~\color{black}n_{i}=\int_{Volume}{nw_{i}(r)}d{\vec{r}}}\end{equation} 
with weight functions
\begin{equation}{w_{3}=\Theta(|\vec{r}|-R)}\end{equation}
\begin{equation}{w_{2}=\delta(|\vec{r}|-R)}\end{equation}

%\begin{equation}{w_{1}=\frac{w_{2}}{4\pi{R}}}\end{equation}
\begin{equation}{w_{1}=\frac{w_{2}}{4\pi{r}}}\end{equation}
%\begin{equation}{w_{0}=\frac{w_{2}}{4\pi{R}^2}}\end{equation}
\begin{equation}{w_{0}=\frac{w_{2}}{4\pi{r}^2}}\end{equation}
where the weight functions $w_{2}$ and $w_{3}$ have the relation
\begin{equation}\label{w2_w3_relation}{w_{3}=\int_{r}^{\infty}{w_{2}(\vec{r})dr}\mbox{~~~~or equivilantly,~~~~}-\frac{\partial{w_3(r)}}{\partial{r}}=w_2(r)}\end{equation}

The weight functions can be related to the mayer function \begin{equation}{f(r)=\exp^{-\frac{V(r)}{k_{B}T}}-1}\end{equation} where r is the distance between the centers of two hard spheres. For a hard sphere fluid the interatomic potential $V(r)$ is given by \begin{displaymath}{V(r)=\left\{ \begin{array}{rc} \infty & 0<r \leq 2R \\ 0  & r>2R \end{array}\right.}\end{displaymath}
and so the mayer function for a hard sphere is given by \begin{equation}\label{f(r)step}{f(r)=-\Theta(|\vec{r}|-2R)}\end{equation} 
The mayer function can be written in terms of weight functions if two additional vector weight functions $\vec{w_2}$ and $\vec{w_1}$ are introduced
\begin{equation}\label{mayer_deconvolution}{f(r)=-\Theta(|\vec{r}|-2R)= -2(w_3 \otimes w_0 + w_2 \otimes w_1 + \vec{w_2} \otimes \vec{w_1})}\end{equation}  where the vector weights are defined as \begin{equation}{\vec{w_{2}}=w_{2}\frac{\vec{r}}{r}}\end{equation}
\begin{equation}{\vec{w_{1}}=w_{1}\frac{\vec{r}}{r}}\end{equation} 

Taking the derivative of the mayer function in Eq~\ref{f(r)step} will prove useful, and is given by
\begin{equation}\label{mayerder=w2convolution}{\frac{df(r)}{dr} = w_2 \otimes w_2 = \int{w_2(r')w_2(r-r')dr'}}\end{equation} 
where, for a hard sphere fluid the left side of Equation~\ref{mayerder=w2convolution} gives
\begin{displaymath}{\frac{df(r)}{dr}=\left\{ \begin{array}{rc} \frac{d}{dr}(-1)=0 & r<2R \\\frac{d}{dr}(\text{vertical line})=+\infty & r=2R \\ \frac{d}{dr}(0)=0  & r>2R \end{array}\right.}\end{displaymath}
\begin{equation}\label{mayerder}{\frac{df(r)}{dr} = \delta(r-2R)}\end{equation} 
Using the general relation for convolution, which is equal to the inverse Fourier Transform of a product of Fourier Transforms \begin{equation}{f \otimes g = \text{IFT[ FT}[f]\text{ x FT[g] }]}\end{equation} with
\begin{equation}{w_2(r)=\delta(r-R)}\end{equation} 
\begin{equation}{\text{Fourier Transform}[w_2]=\int_{-\infty}^{\infty}\delta(r-R)e^{-ikr}dr=e^{-ikR}}\end{equation} 
the right side of Equation~\ref{mayerder=w2convolution} gives
\begin{equation}{w_2 \otimes w_2 = \text{Inverse Fourier Transform}[e^{-ikR}e^{-ikR}]}\end{equation} 
\begin{equation}\label{w2convolution}{=\int_{-\infty}^{\infty}e^{-ik2R}e^{ikr}dk=\int_{-\infty}^{\infty}e^{ik(r-2R)}dk=\delta(r-2R)}\end{equation} 
which matches Equation~\ref{mayerder}.

Incorporating into Equation~\ref{FexfromSPT} the two vector weights which were added to satisfy Equation~\ref{mayer_deconvolution}, and an updated expression for $\Phi_3$ developed by Tarazona, the excess free energy becomes
\begin{displaymath}{F_{excess}=k_{B}T\left(-n_{0}ln(1-n_{3})+\frac{n_{1}n_{2}-\vec{n_{1}}\cdot\vec{n_{2}}}{1-n_{3}}\color{white}\right)}\end{displaymath}
\begin{equation}{\color{white}\left(\color{black}+\frac{{n_2}^3-3n_2\vec{n}_{v2}\cdot\vec{n}_{v2}+\frac{9}{2}[\vec{n}_{v2}\cdot{\overleftrightarrow{n}_{m2}}\cdot{\vec{n}_{v2}}-\operatorname{Tr}({\overleftrightarrow{n}^3_{m2}})]}{24\pi(1-n_3)^2}\right)\color{black}V}\end{equation} 
where the tensor weight is given by
\begin{equation}{\overleftrightarrow{w}_{m2}(\vec{r})=w_2(r)\left(\frac{\vec{r}\vec{r}}{r^2}-\frac{I}{3}\right)}\end{equation} 
%and $\Phi_3$ is now
and
\begin{align}
\Phi_1 &= -n_{0}ln(1-n_{3}) \\
\Phi_2 &= \frac{n_{1}n_{2}-\vec{n_{1}}\cdot\vec{n_{2}}}{1-n_{3}} \\
\Phi_3 &= \frac{{n_2}^3-3n_2\vec{n}_{v2}\cdot\vec{n}_{v2}+\frac{9}{2}[\vec{n}_{v2}\cdot{\overleftrightarrow{n}_{m2}}\cdot{\vec{n}_{v2}}-\operatorname{Tr}({\overleftrightarrow{n}^3_{m2}})]}{24\pi(1-n_3)^2}  
\end{align}

The expression for the weighted densities can be written in a way that accomodates both homogenious fluids where the number denstiy n is constant, and inhomogeneous fluids where the number density becomes a function of the position $n(\vec{r})$.
\begin{equation}{n_i(\vec{r})=\int_{Volume}{n(\vec{r'})w_i(|\vec{r}-\vec{r'}|)d{\vec{r'}}}}\end{equation}
%The equatons above were derived for a uniform fluid where the number density n is constant, but they can be extended for inhomogenious fluids where the number density becomes a function of r, $n(\vec{r})$.
For a spatially varying number density $n(\vec{r})$, the excess Free Energy becomes a functional of the number density profile $n(\vec{r})$ :
\begin{equation}{F_{excess}[n(\vec{r})]= k_BT\int(\Phi_1(\vec{r})+\Phi_2(\vec{r})+\Phi_3(\vec{r}{)) d}\vec{r}}\end{equation} 

The equations developed so far have been for a hard sphere fluid. All of the weight function equations, as well as the equation for the deriviative of the mayer function, can be expressed in terms of $w_{2}(r)$ which consists of a delta function characteristic of hard spheres. Soft Fundamental Measure Theory (SFMT) extends the Fundamental Measure Theory discussed so far to fluids with soft potenials by keeping the relationships between the weight functions, and that the weight functions satisfy the same, but replacing $w_{2}(r)$ with a weight function appropriate for a soft sphere potential. A new set of weight functions can then be derived for a system of soft spheres. 

In this paper,  $w_{2}(r)$ will be constructed using the following expresssion for $w_3(r)$ proposed by Schmidt$^{(?)}$ 
\begin{equation}{w_3(r)=\frac{1}{2}}\left[1-\operatorname{erf}\left(\frac{r-\frac{\alpha}{2}}{\frac{\Xi}{\sqrt{2}}}\right)\right]\end{equation} 
 where $w_3(r)$ is subject to the boundary conditions $w_3(0)\approx{1}$ and $w_3(\infty)=0$. 
%\color{red}These boundary conditions are satisfied for $\Xi{\sqrt{2}<<\alpha}$, or $k_BT << 2.21$. \color{black} 
Using Equation~\ref{w2_w3_relation}, shown again below, 
\begin{displaymath}{w_{3}=\int_{r}^{\infty}{w_{2}(\vec{r})dr}\mbox{~~~~or equivilantly,~~~~}-\frac{\partial{w_3(r)}}{\partial{r}}=w_2(r)}\end{displaymath}
and solving for $w_2(r)$
\begin{displaymath}{\frac{dw_3(r)}{dr}\bigg|_{\infty}-\frac{dw_3(r)}{dr}\bigg|_{r}=w_2(r)}\end{displaymath} 
we get \begin{equation}{ w_2(r)=\frac{\sqrt{2}}{\Xi\sqrt\pi}\exp^{-\left(\frac{r-\frac{\alpha}{2}}{\Xi/\sqrt{2}}\right)^2} }\end{equation} %From the deconvolution equation \begin{equation}{-\frac{1}{2}f(r)=w_0*w_3 + w_1*w_2 -\vec{w}_{V1}*\vec{w}_{V2}}\end{equation} it follows that $w_2$ must satisfy the relation
Putting $w_2(r)$ into Equation~\ref{mayerder=w2convolution}, shown again below, 
\begin{displaymath}{\frac{df(r)}{dr}=\int_{-\infty}^{\infty}{w_2(r')w_2(r-r')dr'}}\end{displaymath}
gives
\begin{equation}{\frac{df(r)}{dr}=\frac{1}{\Xi\sqrt{\pi}}e^{-\left(\frac{r-\alpha}{\Xi}\right)^2}}\end{equation} 
which can be integrated to obtain the Mayer function
\begin{equation}{f(r)=\int_{\infty}^r{ \frac{1}{\Xi\sqrt{\pi}}e^{-\left(\frac{r'-\alpha}{\Xi}\right)^2}{dr'}}}\end{equation} 
The Mayer function for the error function potential $V_{erf}(r)$ is thus given by
\begin{equation}{f(r)=-\frac{1}{2}\left[1-\operatorname{erf}\left(\frac{r-\alpha}{\Xi}\right)\right]}\end{equation} Combining this result with the definition of the Mayer function \begin{displaymath}f(r)=e^{-\frac{V(r)}{K_BT}}-1\end{displaymath} %This gives an interatomic potential of \begin{equation}{V_{erf}=-k_BT\ln(f(r)-1)}\end{equation} 
gives an expression for the for the error function potential  $V(r)=V_{erf}(r)$ 
\begin{equation}{V_{erf}(r)=-k_BT\ln\left[\frac{1}{2}\left(\operatorname{erf}\left(\frac{r-\alpha}{\Xi}\right)+1\right)\right]}\end{equation} The two parameters $\alpha$ and $\Xi$ are given in equations (3) and (4), and are determined to make $V_{erf}(r)$ match $V_{WCA}(r)$ as closely as possible by compensating for the unwanted temperature dependence that shows up in the error function potential $V_{erf}(r)$.

In summary, classical Density Functional Theory (cDFT) is used in conjunction with Soft Fundamental Measure Theory (SFMT) to find the equilibrium free energy of a system, and the corresponding equilibrium number density profile. In particular, SFMT is used to form the expression for the excess free energy without having to take into account the interaction each sphere, or atom, has with every other sphere. The excess free energy is then added to the ideal free energy calculated analytically to find the resulting free energy for the system. 
\begin{equation}{F=F_{ideal} + F_{excess}}\end{equation} 
For a spatially varying number density, as would be expected for a crystal, the free energy is a functional of the number density profile $n(\vec{r})$  
\begin{equation}{F[n(\vec{r})]=F_{ideal}[n(\vec{r})] + F_{excess}[n(\vec{r})]}\end{equation} To implement cDFT, the number density profile $n(\vec{r})$ is varied with the temperature fixed until the free energy is minimized. The minimum free energy and its corresponding number density profile are then the equilibrium values for that temperature.  

Once the Helmholtz free energy for both the crystal and liquid states of the fluid are found, they can be compared to find the resulting state of the fluid. Outside the region of coexistence, the fluid will be completely in a crystalline state if the crystal free energy is lower than the liquid free energy, and it will be completely in a liquid state if the liquid free energy is lower than the crystal free energy. 

\[{}\]
II. METHODS

To implement cDFT computationally, we begin by picturing one single crystal cell. Although the crystal structure is composed of Face Centered Cubic (FCC) cells, the entire lattice can be generated with a cell one-fourth the size called a primitive cell shown by the shaded region in Figure \ref{fig:primitivecell}. The primitive cell has the shape of a parallelpiped, and contains the equivalent of one atom.
 
  \begin{figure}[h!]
    \centering
    % FIXME \includegraphics[height=4.5cm]{PrimitiveCellLightBlue.png}
    \caption{The Primivitive Crystal Cell is shown by the shaded parallelpiped}
    \label{fig:primitivecell}
  \end{figure}


  \begin{figure}[h!]
    \centering
    % FIXME \includegraphics[height=4.5cm]{Simple_Cell.png}
    \caption{A Simplified Crystal Cell}
    \label{fig:simplecell}
  \end{figure}


This cell is rather complicated for the purposes of this discussion, so consider instead a simplified crystal cell with one atom in its center as shown in Figure \ref{fig:simplecell}. 
The Helmholtz free energy is minimized for a fixed volume (and temperature and number of particles).  A fixed cubical volume called the "bold-box" volume, $V_{bold-box}$ will be used for illustration. The number density n for a collection of simplified cells is simply 1 atom per fixed bold-box volume.
\begin{displaymath}{n = \frac{N}{V}=\frac{1~\text{atom}}{V_{bold-box}}}\end{displaymath} It is easier to think of 1 atom per "box" (a reduced number density) than 1 atom per some length$^3$ (the actual number density). Reduced number density is given by n* and is equal to the number of atoms in a "box" of size $\sigma^3$.
\begin{displaymath}{n*=n\sigma^3}\end{displaymath} If $V_{bold-box}$  has a volume of $\sigma^3$, the reduced number density is just the number of atoms in the fixed bold-box volume. 
\begin{displaymath}\text{Reduced Number Density of the simple cell = 1 atom}\end{displaymath} 

In a large collection of simple cells, the atom at the center of each cell will not realistically reside exactly at the lattice point in the center of the cell. Statistically there will typically be a normal distribution of positions about the lattice point as pictured in the far right box in Figure \ref{fig:Ensemble_Gaus}. A plot of a 2-dimensional Gaussian distribution is shown in Figure \ref{fig:Gaus_plot}. A 3-dimenstional Gaussian distribution centered at a lattice point can be used to represent a number density profile $n(\vec{r})$  which describes how the atoms are statistically distrubuted over a collection of cells.

  \begin{figure}[h!]
    \centering
    % FIXME \includegraphics[height=2.5cm]{Ensemble_Gaussian.png}
    \caption{A Gaussian Distribution Number Density Profile}
    \label{fig:Ensemble_Gaus}
  \end{figure} 


 \begin{figure}[h!]
    \centering
    \includegraphics[height=6cm]{Gaussian}
    \caption{A 2D Gaussian Distribution of width 2$\sigma$}
    \label{fig:Gaus_plot}
  \end{figure}  

It is important to note that the number density does not change when introducing a number density profile. There is still a total of one atom per "bold box" volume.

\begin{equation}{n=<n(\vec{r})>=\frac{1}{V_{bold-box}}\overbrace{\int_{bold-box}{n(\vec{r})}{d\vec{r}}}=\frac{N}{V}=\frac{1~\text{atom}}{V_{bold-box}}}\end{equation}

In a large collection of cells, not all lattice sites will be occupied. Vacancies can be taken into account while keeping the volume of the bold box, and thus the overall number density, the same. This is done by making the cells smaller. A large collection of smaller cells creates a new number density profile shown in Figure \ref{fig:Ensemble_Smallcells} with eight smaller Gaussians of one-eigth the height.

 \begin{figure}[h!]
    \centering
    % FIXME \includegraphics[height=2.5cm]{Ensemble_Smallcells.png}
    \caption{A Gaussian Distribution Number Density Profile}
    \label{fig:Ensemble_Smallcells}
  \end{figure} 

Instead of thinking of an atom in one of the eight small cells, the same Gaussian distribution can be obtained from thinking of one-eigth of an atom in each small cell. This makes it convenient to describe the new reduced number density in terms of the fraction of vacancies, $f_v$: 

\begin{displaymath}{ n^* = (1-f_v){n_{no.vacancies}^*}}\end{displaymath} 

\begin{figure}[h!]
    \centering
    % FIXME \includegraphics[height=2.5cm]{SameStatPic.png}
    \caption{Front views of a cube with eight smaller cells. The two views show two ways of picturing a fraction of vacancies, $f_v$ of 7/8 which give rise to the same Gaussian distribution statistically}
    \label{fig:SameStatPic}
  \end{figure} 

By varying the width of the Gaussians and the fraction of vacancies, new number density profiles can be generated as shown in Figure \ref{fig:Ensemble_vary}.

\begin{figure}[h!]
    \centering
    % FIXME \includegraphics[height=8cm]{VaryWidthandVacancies.png}
    \caption{Several Gaussian Distribution Number Density Profiles all with a number density of 1 atom per $V_{bold box}$}
    \label{fig:Ensemble_vary}
  \end{figure}  

Once again, it is important to note that the number density is the same for all these number density profiles $n(\vec{r})$. There is still a total of one atom per "bold box" volume. This prodcedure is used only to find the crystal free energy as number density profile for a liquid is homogeneous, as shown in Figure \ref{fig:homogen_denisty}.  

 \begin{figure}[h!]
    \centering
    %\includegraphics[height=4cm]{Homogeneous_bold-box.pdf}
    % FIXME \includegraphics[height=4cm]{Homogeneous_bold-box.pdf}
    \caption{A homogeneouse number density profile $n(\vec{r})$}
    \label{fig:homogen_denisty}
  \end{figure}    
 
A simplified flowchart of the program created to investigate the freezing of a WCA fluid is shown on pages x-x. The program takes four inputs: reduced number density n*, reduced temperature t*, a range of Gaussian widths gw, and a range of fraction of vacancies fv.The number density profile is varied by varying gw and fv, and the free energy $F[n(\vec{r})]$ is calculated for each profile until the minimum free energy is found for that combination of n* and t*, which is known from cDFT to be unique. The process is then repeated to find the free energies at other number densities and temperatures. 

In general, if the crystal free energy at some n* and t* is lower than the homogeneous free energy of a liquid at that same n* and t*, then the crystaline state dominates. But if the crystal free energy is higher than the homogeneous free energy, the liquid state dominates. However, there is a region of densities between the liquid and crystal states that are forbidden, and so the actual state can not be determined just by which energy is lower at that particular n* and t*. Maxwell's Double Tangent Construction must be implemented using the date collected to determine the actual crystal and liquid states, and construct phase diagrams. 
\[{}\]
\textbf{Program Implementation}

The program computes both the crystal and the homogeneous Helmholtz free energies. The Helmholtz free energy is minimized for a system with a fixed temperature, volume, and number density. The temperature, T is an input to the program. The number density, which is also an input to the program, is used to calculate the volume of one primitive cell over which the free energy is computed. The number of atoms in one primitive cell is one. To compute the free energy, it is necessary to compute both the ideal and excess free energies and add them, $F = F_{ideal} + F_{excess}$.
%\begin{displaymath}F_{homogeneous} = F_{ideal.homogeneous} + F_{excess.homogeneous}\end{displaymath}
%\begin{displaymath}F_{crystal} = F_{ideal.crystal} + F_{excess.crystal}\end{displaymath}
%\newline
%\newline\noindent Inputs: 
%t*, n*, range of gw values , range of fv values
%\newline Outputs: Homogeneous (liquid)  Helmholtz free energy (at the given T,V,N) 
%\newline\indent\indent\indent Crystal Helmholtz free energy (at the given T,V,N)

%\noindent Input:
\begin{displaymath}\text{Input:~~~~~~~~~~~~~~~~~~~~~~~~~~~T,~~ n,~~Range of fv,~~Range of gw~~~~~~~~~~~~~~~~~~~~~~~~~~~~~~~~~~~}\end{displaymath} 
%\noindent Set:
%\begin{displaymath}\text{N = 1}\end{displaymath}
\begin{displaymath}\text{Set:~~~~~~~~~~~~~~~~~~~~~~~~~~~~~~~~~~~~~~~~~~~~~~N~=~1~~~~~~~~~~~~~~~~~~~~~~~~~~~~~~~~~~~~~~~~~~~~~~~~~~~~}\end{displaymath} 
\begin{displaymath}\text{Best}\left(\frac{F}{N}\right)_{diff}= 1e100\end{displaymath}
%Outer Loop:  %set fv to one of the input values
Loop: 
\begin{displaymath}\text{fv = one of the fv values from input}\end{displaymath}
\begin{equation}N_{primitive.cell}=(1-\text{fv})\text{N}\end{equation}
\begin{equation}V_{primitive.cell}=\frac{\text{1}}{\text{n}}(1-\text{fv})\text{N}\end{equation}
%\indent\indent\indent Inner Loop : %set gw to one of the input values
\begin{displaymath}\sigma = \text{one of the gw values from input}\end{displaymath}
\begin{equation}{n(r)=\frac{1}{\left(\sqrt{2\pi}\sigma\right)^3}\int\exp^{-\frac{|\vec{r}-\vec{R}|^2}{2\sigma^2}}d\vec{r}}\end{equation} 
\begin{equation}\left(\frac{F}{N}\right)_{crystal}=\frac{cFideal_{primitive.cell} + cFexcess_{primitive.cell}}{N_{primitive.cell}}\end{equation}
\begin{equation}\left(\frac{F}{N}\right)_{homogeneous}=\frac{hFideal_{primitive.cell} + hFexcess_{primitive.cell}}{N_{primitive.cell}}\end{equation}
\begin{equation}\left(\frac{F}{N}\right)_{diff}=\left(\frac{F}{N}\right)_{crystal}-{~~}\left(\frac{F}{N}\right)_{homogeneous}\end{equation}

%Compare:
\begin{displaymath}\text{If~~~~~~}\left(\frac{F}{N}\right)_{diff}<\text{~~~~Best}\left(\frac{F}{N}\right)_{diff}\end{displaymath}
\begin{displaymath}\text{Then~~~~~~}\text{Best}\left(\frac{F}{N}\right)_{diff}=\left(\frac{F}{N}\right)_{diff}\end{displaymath}
%\noindent Output:
\begin{displaymath}\text{Output:~~~~~~~~~~~~ Best}\left(\frac{F}{N}\right)_{crystal}\text{~~and~~~~~~}\left(\frac{F}{N}\right)_{homogeneous}\text{~~~~for the given T, n ~~~~~~~~~~~~~~~~~~~~~~~~~}\end{displaymath} 
%\begin{displaymath}\text{Best}\left(\frac{F}{N}\right)_{crystal}\end{displaymath}
%\begin{displaymath}\left(\frac{F}{N}\right)_{homogeneous}\end{displaymath}
%lattice_constant= pow(4*(1-fv)/reduced_density, 1.0/3)
\[\]

The Liquid (homogeneous) ideal free energy $hF_{ideal}$ is computed analytically for a given n* and t*, where $\sigma=1$ so that n*=n$\sigma^3$=n, and $\epsilon=1$ so that t*=$\frac{k_BT}{\epsilon}=k_BT$. 
%\begin{displaymath}{\frac{hF_{ideal}}{V}=k_BTn(\ln(\frac{n*2.646476976618268x10^{-6}}{\sqrt{(k_BT)^3}})-1.0)-DELETE-EQN!}\end{displaymath} 
\begin{displaymath}{\frac{hF_{ideal}}{V}=k_BTn(\ln(n\Lambda^3)-1.0)}\end{displaymath} where 

\begin{displaymath}{\Lambda =\sqrt{\frac{2\pi\hbar^2}{k_BTm}}}\end{displaymath}  and
\begin{displaymath}\left(\sqrt{\frac{2\hbar^2}{m}}\right)^3=4.752731840013864\text{x}10^{-7}\color{red}units!\color{black}\end{displaymath} 
%\begin{displaymath}{m=}\end{displaymath}  
%\begin{displaymath}{\hbar=}\end{displaymath} 
%\begin{displaymath}{k_B=}\end{displaymath} 

The Liquid (homogeneous) excess free energy $hF_{excess}$ is computed analytically for a given n* and t* using SFMT with a uniform number density profile that does not depend on $\vec r$. 
%done in hf.energy() note- does not depend on fv or gw but does use SFMT):
\begin{displaymath}{\frac{hF_{excess}}{V} = k_BT(\Phi_1+\Phi_2+\Phi_3)}\end{displaymath} 
For the case of a homogeneous number density n, $\Phi_1$, $\Phi_2$, and $\Phi_3$ simplify to:
\begin{displaymath}{\Phi_1= -n_0\ln(1-n_3)}\end{displaymath} 
\begin{displaymath}{\Phi_2= \frac{n_1n_2}{1-n_3}}\end{displaymath} 
\color{red}\begin{displaymath}{\Phi_3=\frac{n_2^3}{24\pi(1-n_3)^2}}\end{displaymath}\color{black}
where \begin{displaymath}{n_i=nw_i}\end{displaymath} 
%\begin{displaymath}{n_0=nw_0}\end{displaymath} 
%\begin{displaymath}{n_1=nw_1}\end{displaymath}
%\begin{displaymath}{n_2=nw_2}\end{displaymath}
%\begin{displaymath}{n_3=nw_3}\end{displaymath}
and the weights are as usual 
\begin{equation}{w_{0}=\frac{w_{2}}{4\pi{r}^2}}\end{equation}
\begin{equation}{w_{1}=\frac{w_{2}}{4\pi{r}}}\end{equation}
\begin{equation}{ w_2(r)=\frac{\sqrt{2}}{\Xi\sqrt\pi}\exp^{-\left(\frac{r-\frac{\alpha}{2}}{\Xi/\sqrt{2}}\right)^2} }\end{equation}
\begin{equation}{w_3(r)=\frac{1}{2}}\left[1-\operatorname{erf}\left(\frac{r-\frac{\alpha}{2}}{\frac{\Xi}{\sqrt{2}}}\right)\right]\end{equation} 
%\begin{displaymath}{w_0=?}\end{displaymath} 
%\begin{displaymath}{w_1=\frac{\alpha}{2}}\end{displaymath}
%\begin{displaymath}{w_2=6.2...\Xi^2 + \pi\alpha^2}\end{displaymath}
%\begin{displaymath}{w_3=\alpha(\pi\Xi^2 + 0.52...\alpha^2)}\end{displaymath} \[\]
\[\]

The Crystal ideal free energy $cF_{ideal}$ is computed in two different ways from the equation 
\begin{displaymath}{F_{ideal}[n(\vec{r})]= k_BT\int[n(\vec{r})\ln(n(\vec{r})\Lambda^3)-n(\vec{r})]dV}\end{displaymath}
with the varying number density profile given by 
\begin{displaymath}{n(|\vec{r}-\vec{R_i}|)=\sum_i{(1-f_v)\frac{\exp^{-\frac{{|\vec{r}-\vec{R_i}|}^2}{2{(gw)}^2}}}{\left(\sqrt{2\pi}gw\right)^3}}}\end{displaymath}
\newline $\bullet$ For small gw values it is assumed that the Guassians do not overlap, in which case the equation for the free energy can be simplified and solved analytically with the result
\begin{displaymath}{cF_{ideal}=(1-f_v)k_BT\left(\ln((1-f_v)\Lambda^3)-3\ln(\sqrt{2\pi{gw}})-\frac{5.0}{2}\right)}\end{displaymath}

%\begin{displaymath}{cF_{ideal}=(1-f_v)k_BT\left(\ln(\frac{(1-fv)2.646476976618268x10^{-6}}{\sqrt{(k_BT)^3}})-3\ln(\sqrt{2\pi{gw}})-\frac{5.0}{2}\right)-DELETE-EQN!}\end{displaymath} 

$\bullet$ For larger gw values it is assumed that the Gaussians do overlap, and so the free energy is caluculated computationally. First dx is scaled by gw
\begin{displaymath}{dx_{scaled}=(dx_{input})(gw)}\end{displaymath}
then $n(\vec{r})$ is computed at one position of $\vec{r}$ within the primitive cell taking into account the contribution of Gaussians from many (5x5x5) nearby cells whose centers are given by $R_i$.
%\begin{displaymath}{n(|\vec{r}-\vec{R_i}|)+=\frac{(1-f_v)\exp^{-\frac{{|\vec{r}-\vec{R_i}|}^2}{2{(gw)}^2}}}{\left(\sqrt{2\pi}gw\right)^3}}\end{displaymath}
The contribution to the free energy of $n(\vec{r})$ at the one position is then added to the accumulating sum.
%\begin{displaymath}{cF_{ideal}+=k_BTn(ln(\frac{(n)2.646476976618268x10^{-6}}{\sqrt{(k_BT)^3}})-1.0)dV-DELETE-EQN!}\end{displaymath}
%\begin{displaymath}{cF_{ideal}+=k_BTn(\ln(n\Lambda^3)-1.0)dV}\end{displaymath}
This process is repeated for all positions within one primitive cell to get the crystal ideal free energy for the primitive cell. This energy is then divided by the volume of the primitive cell to get the crystal ideal free energy per volume.
%\[\]

The crystal excess free energy $cF_{excess}$ is computed computationally using SFMT which gives
%First dx is scaled by $\Xi$ or gw, whichever is larger \begin{displaymath}{dx_{scaled}=(dx_{input})(\Xi + gw)}\end{displaymath}
%Then the crystal, or inhomogeneous, excess free energy is found from
\begin{displaymath}{F_{excess}[n(\vec{r})]= k_BT\int(\Phi_1(\vec{r})+\Phi_2(\vec{r})+\Phi_3(\vec{r}{)) d}\vec{r}}\end{displaymath} and using Monte-Carlo integration to compute the weighted densities $n_0(\vec{r})$, $n_1(\vec{r})$, $n_2(\vec{r})$, $n_3(\vec{r})$ used to form $\Phi_1(\vec{r})$, $\Phi_2(\vec{r})$, and $\Phi_3(\vec{r})$.

The center of with weighting function is placed at some vector $\vec{r}=\vec{r}_k$ within the primitive cell, and a vector $\vec{R}_j$ to the center of one of the Gaussians is selected that is within the primitive cell, or one of the neighboring cells within a prudently picked inclusion radius as Gaussians farther away will not contriubute much (see Figure \ref{fig:InclusionRadius}).

 \begin{figure}[h!]
    \centering
    % FIXME \includegraphics[height=5cm]{InclusionRadius.png}
    \caption{The Gaussians at each lattice point are of width gw. After a radius of few gw's the Gaussians go to zero and can be neglected.}
    \label{fig:InclusionRadius}
  \end{figure} 



The weighted densities $n_0(\vec{r}_k)$, $n_1(\vec{r}_k)$, $n_2(\vec{r}_k)$, $n_3(\vec{r}_k)$ at $\vec{r}_k$ are then computed using the weight functions $w_0(|\vec{r}_k-\vec{R_j}-\vec{dr}|)$, $w_1(|\vec{r}_k-\vec{R_j}-\vec{dr}|)$, $w_2(|\vec{r}_k-\vec{R_j}-\vec{dr}|)$, $w_3(|\vec{r}_k-\vec{R_j}-\vec{dr}|)$ where $\vec{dr}$ is a vector from the Gaussian centered at $\vec{R}_j$ to some point in space. Monte-Carlo integration works by randomly generating $\vec{dr}$ according to a Gaussian distribution. Setting $\vec{r}'=\vec{R}_j + \vec{dr}$, and $n(\vec{r'})=(1-f_v)n_g(\vec{r'})$ where $n_g(\vec{r'})$ is the gaussian number density profile for no vacancies, the equation:

\begin{equation}{n_i(\vec{r})= \int_{r'}{(1-f_v)n_g(\vec{r'})w_i(|\vec{r}-\vec{r}'|)} {d}\vec{r}'}\end{equation} implemented computationally in the form:
\begin{equation}{n_i(\vec{r})= \sum_{r'}{(1-f_v)n_g(\vec{r'})w_i(|\vec{r}-\vec{r}'|)dV}}\end{equation} becomes:
\begin{equation}{n_i(\vec{r})= \sum_{NumPoints}(1-f_v)P_{\vec{dr}}w_i(|\vec{r}-\vec{r}'|)}\end{equation} where NumPoints is the number of vectors $\vec{r'}$ randomly generated, and $n_g(\vec{r'})$ (the value of the Gaussian at at radius of $|\vec{r'}|$) is equal to a probability density which when multiplied by dV gives the probability $P_{\vec{dr}}$ at which the value of $w_i(|\vec{r}-\vec{r}'|)$ will occur. 
Using the relation:
\begin{displaymath}{f_{average}=\sum_i^N{P_if_i}=\sum_i{\frac{f_i}{N}}}\end{displaymath} $n_i(\vec{r})$ can be expressed as the average value of $(1-f_v)w_i(|\vec{r}-\vec{r}'|)$:
\begin{equation}{n_i(\vec{r})=\sum_{NumPoints}\frac{(1-f_v)w_i(|\vec{r}-\vec{r}'|)}{NumPoints}}\end{equation}

The result can be made more accurate by increasing NumPoints which is the number of vectors randomly generated. In order to further reduce the error in the average, antithetic variates are used. Rather than computing an average of a set of random points, each random point in the set is paired with a point along the same line as the first point, but opposite in direction, as shown in Figure \ref{fig:AntitheticVariate}. This reduces first order error in the average by filtering out an erroneous offset of values along the line connecting the two points by averaging the two results, one of which may be erroneously higher than it should be while the other is erronesously lower than it should be.

 \begin{figure}[h!]
    \centering
    % FIXME \includegraphics[height=4cm]{AntitheticVariate.png}
    \caption{Antithetic variates reduce first order error in an average. Each point in the average is paired with another point from along the same line as the first, and at an equal, but opposite distance.}
    \label{fig:AntitheticVariate}
  \end{figure} 
Incorporating antithetic variates $-\vec{dr}$ for every $\vec{dr}$ equation ?? becomes:
\begin{displaymath}{n_i(\vec{r}_k, \vec{R}_j)=\sum_{NumPoints}\frac{1}{2}\frac{(1-f_v)\left[w_i(|\vec{r}-\vec{R_j}-\vec{dr}|) +w_i(|\vec{r}-\vec{R_j}+\vec{dr}|)\right]}{NumPoints}}\end{displaymath}   The process is repeated for all the other $\vec{R}_j$'s to get the total weighted densities at vector $\vec{r}_k$ due to all the Gaussians. 
\begin{displaymath}{n_i(\vec{r}_k)=\sum_j{n_i(\vec{r}_k, \vec{R}_j)}}\end{displaymath}
Then the entire process repeated for each vector $\vec{r}_k$ to get the total weighted densities $n_i(\vec{r})$ as a function of the position $\vec{r}$. 

The standard error of the mean (SEM) is given by:
\begin{equation}{SD_{mean}=\frac{SD}{\sqrt{N}}}\end{equation} which gives a measure of the error in $n_3=\overline{w}_3$:
\begin{equation}{Error=\sqrt{\frac{\overline{w_3^2}-(\overline{w}_3)^2}{NumPoints}}}\end{equation} NumPoints, the number of vectors randomly generated, is increased until:
\begin{equation}{Error<\frac{|1-n_3|}{4}}\end{equation} where the maximum value that $n_3$ can be without error is 1.

Figure \ref{fig:weight_functions} shows a 1 dimensional view of the weight functions $w_0$, $w_1$, $w_2$, $w_3$ as they vary with distance along the z-axis. In three dimensions, the weighting functions $w_0$, $w_1$, $w_2$ are spherical shells centered at $\vec{r}$. As shown in the cross-sectional view in Figure \ref{fig:W0andW3}a, $w_2$ peaks at a radius of $|\vec{r}-\vec{r'}|=\frac{\alpha}{2}$, a width determined by $\frac{\Xi}{\sqrt{2}}$, and edges that tapers off on either side of the peak giving the shell. The weighting function $w_3$, shown in Figure \ref{fig:W0andW3}b is a solid sphere up to a radius of ?FIX? after which it tapers off over a distance of ?FIX?.

 \begin{figure}[h!]
    \centering
    \includegraphics[height=6cm]{weight_functions.pdf}
    \caption{Shows how the weight functions $w_0$, $w_1$, $w_2$, $w_3$ vary along the z-axis for a reduced temperature of 2}
    \label{fig:weight_functions}
  \end{figure}  


 \begin{figure}[h!]
    \centering
    \begin{subfigure}[b]{0.4\linewidth}
      % FIXME % FIXME \includegraphics[height=5cm]{W2.png}
      \caption{A crossectional view of $w_2$}
    \end{subfigure} 
    \begin{subfigure}[b]{0.4\linewidth}
      % FIXME \includegraphics[height=5cm]{W3.png}
      \caption{A crossectional view of $w_3$}
    \end{subfigure} 
    \caption{Cross-sectional views of the 3D weight functions. (a) $w_2$ is a spherical shell in 3D with soft edges. Weight functions $w_0$ and $w_1$ have a similar shape. (b) $w_3$ is a sphere with soft edges in 3D.}
    \label{fig:W0andW3}
    \end{figure} 


Figure \ref{fig:GaussandW2_actual} ......

 \begin{figure}[h!]
    \centering
    % FIXME \includegraphics[height=6cm]{GaussandW2_actual.png}
    \caption{A Gaussian centered at $\vec{R}$ is shown in blue, and a cross-sectional view of $w_2$ centered at $\vec{r}'$ is shown in purple. Here, the weight function is evaluated at $\vec{r}-\vec{r}'$ where $\vec{r}'=\vec{R}+\vec{dr}$ and $\vec{dr}$ is generated randomly according to a Gaussian distribution.} 
  \label{fig:GaussandW2_actual}
  \end{figure} 

 \begin{figure}[h!]
    \centering
    % FIXME \includegraphics[height=6cm]{GaussandW2_thinkas.png}
    \caption{An alternative way of viewing Figure \ref{fig:GaussandW2_actual} where the weight function, shown by the purple dashed lines, is centered at $\vec{r}$  and fixed in place as $\vec{dr}$ varies randomly.} 
  \label{fig:GaussandW2_thinkas}
  \end{figure} 


The parameters $\alpha$ and $\Xi$ are temperature dependent, as shown earlier in Figure \ref{fig:alphaXivsT}. $\Xi$ is related to the width of the functions which become wider in general as the temperature increases. The radial distance to the peak decreases as the temperature increases, that is, the shells shrink, as does the sphere. The functions can be treated as neglible outside of a distance of about $3(\frac{\Xi}{\sqrt{2}}) + \frac{\alpha}{2}$ CHECK!! from the center of the weight function.

\[{}\]
\[{}\]
\[{}\]
\[{}\]
\[{}\]
\[{}\]
III. RESULTS
\[{}\]
\[{}\]
\[{}\]
\[{}\]
\[{}\]
\[{}\]
\[{}\]
\[{}\]



APPENDIX
\begin{enumerate}

\item Fourier Transforms of Weight Functions $w_0(r), w_1(r), w_2(r), w_3(r)$
\item Fourier Transforms of Vector Weight Functions $\vec{w}_1(\vec{r}), \vec{w}_2(\vec{r})$  
\item Fourier Transform of Tensor Weight $\overleftrightarrow{w}_{m2}(r)$ in coordinate system (x',y',z')
%\item The tensor weight components in the coordinate system (x,y,z) 
%\item Show transform of tensor weight $\overleftrightarrow{w}_{m2}(r)$ goes to zero as k goes to 0
\item Integrals used to Compute Fourier Transforms of Weight Functions
\item Deriviation of Tensor Density $n_{m2ij}(\vec{r})$ 
\item Deriviation of the Tensor version of $\Phi_3$ 
\item Second Virial Coefficient $B_{2}$ for the Error Function Potential 
\item Third Virial Coefficient $B_{3}$
\item Derivation of the Virial Equation
\end{enumerate}

\textbf{1a. Fourier Transform of Weight Function $w_0(r)$:}
\begin{equation}{w_0(r)=\frac{w_2(r)}{4{\pi}r^2}}\end{equation}
where
\begin{equation}{w_2(r)=\frac{\sqrt{2}}{\Xi\sqrt{\pi}}e^{-\left(\frac{r-\frac{\alpha}{2}}{\frac{\Xi}{\sqrt{2}}}\right)^2}}\end{equation}
temporarily set 
\begin{equation}{a=\frac{\Xi}{\sqrt{2}}}\end{equation}
\begin{equation}{w_0(r)=\frac{1}{4{\pi}\sqrt{\pi}a}\left(\frac{1}{r^2}\right)e^{-\left(\frac{r-\frac{\alpha}{2}}{a}\right)^2}}\end{equation}
%\[{}\]
\begin{equation}{\widetilde{w}_0(\vec{k})=\int_{0}^{\infty}\int_{-1}^{1}\int_{0}^{2\pi}w_0(r)e^{-i\vec{k}\cdot{\vec{r}}}r^2d{r}d{\cos\theta}d{\phi}}\end{equation}
Set $\vec{k}$ in the direction of $\hat{z}$ 
\begin{equation}{\widetilde{w}_0(k)=\int_{0}^{\infty}\int_{-1}^{1}\int_{0}^{2\pi}w_0(r)e^{-ikr\cos\theta}r^2d{r}d{\cos\theta}d{\phi}}\end{equation}
\[{}\]
Calculate $\widetilde{w}_0(k)$ 
\begin{equation}{\widetilde{w}_0(k)=\frac{1}{4{\pi}\sqrt{\pi}a}\int_{0}^{\infty}\left(\frac{1}{r^2}\right)e^{-\left(\frac{r-\frac{\alpha}{2}}{a}\right)^2}r^2\left[\int_{-1}^{1}e^{-ikr\cos\theta}d{\cos\theta}\left(\int_{0}^{2\pi}d{\phi}\right)\right]d{r}}\end{equation}
\[{}\]
\begin{equation}{\widetilde{w}_0(k)=\frac{2\pi}{4{\pi}\sqrt{\pi}a}\int_{0}^{\infty}e^{-\left(\frac{r-\frac{\alpha}{2}}{a}\right)^2}\left(\int_{-1}^{1}e^{-ikr\cos\theta}d{\cos\theta}\right)d{r}}\end{equation}
\[{}\]
Set the lower limit to $-\infty$  since the integrand is nearly zero by the time r is zero. 
\begin{equation}{\widetilde{w}_0(k)=\frac{1}{\sqrt{\pi}ka}\int_{-\infty}^{\infty}e^{-\left(\frac{r-\frac{\alpha}{2}}{a}\right)^2}\frac{\sin(kr)}{r}d{r}}\end{equation}
\[{}\]
\begin{equation}{\widetilde{w}_0(k)=\frac{\sqrt{\pi}}{2ka}e^{-\left(\frac{\alpha}{2a}\right)^2}\left[\operatorname{erf}\left(\frac{ka}{2}+i\frac{\alpha}{2a}\right)+\operatorname{erf}\left(\frac{ka}{2}-i\frac{\alpha}{2a}\right)\right]}\end{equation}
\[{}\] 
Putting in $a=\frac{\Xi}{\sqrt{2}}$ the final equation for $\widetilde{w}_0(k)$ becomes
\begin{equation}{\widetilde{w}_0(k)=\frac{\sqrt{\pi}}{\sqrt{2}k\Xi}e^{-\left(\frac{\alpha^2}{2\Xi^2}\right)}\left[\operatorname{erf}\left(\frac{k\Xi}{2*\sqrt{2}}+i\frac{\alpha}{\sqrt{2}\Xi}\right)+\operatorname{erf}\left(\frac{k\Xi}{2*\sqrt{2}}-i\frac{\alpha}{\sqrt{2}\Xi}\right)\right]}\end{equation}


\[{}\]
%\[{}\]
\textbf{1b. Fourier Transform of Weight Function $w_{1}(r)$:}
\begin{equation}{w_1(r)=\frac{w_2(r)}{4{\pi}r}}\end{equation}
where
\begin{equation}{w_2(r)=\frac{\sqrt{2}}{\Xi\sqrt{\pi}}e^{-\left(\frac{r-\frac{\alpha}{2}}{\frac{\Xi}{\sqrt{2}}}\right)^2}}\end{equation}
temporarily set 
\begin{equation}{a=\frac{\Xi}{\sqrt{2}}}\end{equation}
\begin{equation}{w_1(r)=\frac{1}{4{\pi}\sqrt{\pi}a}\left(\frac{1}{r}\right)e^{-\left(\frac{r-\frac{\alpha}{2}}{a}\right)^2}}\end{equation}

\begin{equation}{\widetilde{w}_1(\vec{k})=\int_{0}^{\infty}\int_{-1}^{1}\int_{0}^{2\pi}w_1(r)e^{-i\vec{k}\cdot{\vec{r}}}r^2d{r}d{\cos\theta}d{\phi}}\end{equation}
Set $\vec{k}$ in the direction of $\hat{z}$ 
\begin{equation}{\widetilde{w}_1(k)=\int_{0}^{\infty}\int_{-1}^{1}\int_{0}^{2\pi}w_1(r)e^{-ikr\cos\theta}r^2d{r}d{\cos\theta}d{\phi}}\end{equation}

\noindent Calculate $\widetilde{w}_1(k)$ 
\begin{equation}{\widetilde{w}_1(k)=\frac{1}{4{\pi}\sqrt{\pi}a}\int_{0}^{\infty}\left(\frac{1}{r}\right)e^{-\left(\frac{r-\frac{\alpha}{2}}{a}\right)^2}r^2\left[\int_{-1}^{1}e^{-ikr\cos\theta}d{\cos\theta}\left(\int_{0}^{2\pi}d{\phi}\right)\right]d{r}}\end{equation}
\[{}\]
\begin{equation}{\widetilde{w}_1(k)=\frac{2\pi}{4{\pi}\sqrt{\pi}a}\int_{0}^{\infty}e^{-\left(\frac{r-\frac{\alpha}{2}}{a}\right)^2}r\left(\int_{-1}^{1}e^{-ikr\cos\theta}d{\cos\theta}\right)d{r}}\end{equation}
\[{}\]
\begin{equation}{\widetilde{w}_1(k)=\frac{2}{4\sqrt{\pi}a}\int_{0}^{\infty}e^{-\left(\frac{r-\frac{\alpha}{2}}{a}\right)^2}r\frac{2\sin(kr)}{kr}d{r}}\end{equation}
Set the lower limit to $-\infty$ since the integrand is nearly zero by the time r is zero. 
\begin{equation}{\widetilde{w}_1(k)=\frac{1}{\sqrt{\pi}ka}\int_{-\infty}^{\infty}e^{-\left(\frac{r-\frac{\alpha}{2}}{a}\right)^2}\sin(kr)d{r}}\end{equation}
After doing a u-substitution with $u=\frac{r-\frac{\alpha}{2}}{a}$ this becomes
\begin{equation}{\widetilde{w}_1(k)=\frac{1}{k}\sin\left(\frac{k\alpha}{2}\right)e^{-\left(\frac{k^2a^2}{4}\right)}}\end{equation}
Putting $a=\frac{\Xi}{\sqrt{2}}$ back in, the final equation for $\widetilde{w}_1(k)$ becomes
\begin{equation}
    \widetilde{w}_1(k)=\frac{1}{k}e^{-\frac{k^2\Xi^2}{8}}\sin\left(\frac{k\alpha}{2}\right)
\end{equation}

\[{}\]
%\[{}\]
\textbf{1c. Fourier Transform of Weight Function $w_{2}(r)$:}
\begin{equation}{w_2(r)=\frac{\sqrt{2}}{\Xi\sqrt{\pi}}e^{-\left(\frac{r-\frac{\alpha}{2}}{\frac{\Xi}{\sqrt{2}}}\right)^2}}\end{equation}
set 
\begin{equation}{a=\frac{\Xi}{\sqrt{2}}}\end{equation}
\begin{equation}{w_2(r)=\frac{1}{a\sqrt{\pi}}e^{-\left(\frac{r-\frac{\alpha}{2}}{a}\right)^2}}\end{equation}

\begin{equation}{\widetilde{w}_2(\vec{k})=\int_{0}^{\infty}\int_{-1}^{1}\int_{0}^{2\pi}w_2(r)e^{-i\vec{k}\cdot{\vec{r}}}r^2d{r}d{\cos\theta}d{\phi}}\end{equation}
Set $\vec{k}$ in the direction of $\hat{z}$ 
\begin{equation}{\widetilde{w}_2(k)=\int_{0}^{\infty}\int_{-1}^{1}\int_{0}^{2\pi}w_2(r)e^{-ikr\cos\theta}r^2d{r}d{\cos\theta}d{\phi}}\end{equation}

\noindent Calculate $\widetilde{w}_2(k)$ 
\begin{equation}{\widetilde{w}_2(k)=\frac{1}{a\sqrt{\pi}}\int_{0}^{\infty}e^{-\left(\frac{r-\frac{\alpha}{2}}{a}\right)^2}r^2\left(\int_{-1}^{1}e^{-ikr\cos\theta}d{\cos\theta}\left(\int_{0}^{2\pi}d{\phi}\right)\right)d{r}}\end{equation}
\[{}\]
\begin{equation}{\widetilde{w}_2(k)=\frac{2\pi}{a\sqrt{\pi}}\int_{0}^{\infty}e^{-\left(\frac{r-\frac{\alpha}{2}}{a}\right)^2}r^2\left(\int_{-1}^{1}e^{-ikr\cos\theta}d{\cos\theta}\right)d{r}}\end{equation}
\begin{equation}{\widetilde{w}_2(k)=\frac{2\sqrt{\pi}}{a}\int_{0}^{\infty}e^{-\left(\frac{r-\frac{\alpha}{2}}{a}\right)^2}r^2\frac{2\sin(kr)}{kr}d{r}}\end{equation}
Set the lower limit to $-\infty$  since the integrand is nearly zero by the time r is zero.
\begin{equation}{\widetilde{w}_2(k)=\frac{4\sqrt{\pi}}{ka}\int_{-\infty}^{\infty}e^{-\left(\frac{r-\frac{\alpha}{2}}{a}\right)^2}r\sin(kr)d{r}}\end{equation}
After doing a u-substitution with $u=\frac{r-\frac{\alpha}{2}}{a}$ this becomes
\begin{equation}{\widetilde{w}_2(k)=\frac{2\pi}{k}e^{-\left(\frac{k^2a^2}{4}\right)}\left(ka^2\cos\left(\frac{k\alpha}{2}\right)+\alpha\sin\left(\frac{k\alpha}{2}\right)\right)}\end{equation}
Putting $a=\frac{\Xi}{\sqrt{2}}$ back in, the final equation for $\widetilde{w}_2(k)$ becomes
\begin{equation}{\widetilde{w}_2(k)=\frac{2\pi}{k}e^{-\frac{k^2\Xi^2}{8}}\left[\frac{k\Xi^2}{2}\cos\left(\frac{k\alpha}{2}\right)+\alpha\sin\left(\frac{k\alpha}{2}\right)\right]}\end{equation}

\[{}\]
%\[{}\]
\textbf{1d. Fourier Transform of Weight Function $w_{3}(r)$:}
\begin{equation}{w_3(r)=\frac{1}{2}\left[1-\operatorname{erf}\left(\frac{r-\frac{\alpha}{2}}{\frac{\Xi}{\sqrt{2}}}\right)\right]}\end{equation}
temporarily set 
\begin{equation}{a=\frac{\Xi}{\sqrt{2}}}\end{equation}
\begin{equation}{w_3(r)=\frac{1}{2}\left[1-\operatorname{erf}\left(\frac{r-\frac{\alpha}{2}}{a}\right)\right]}\end{equation}

\begin{equation}{\widetilde{w}_3(\vec{k})=\int_{0}^{\infty}\int_{-1}^{1}\int_{0}^{2\pi}w_3(r)e^{-i\vec{k}\cdot{\vec{r}}}r^2d{r}d{\cos\theta}d{\phi}}\end{equation}
Set $\vec{k}$ in the direction of $\hat{z}$ 
\begin{equation}{\widetilde{w}_3(k)=\int_{0}^{\infty}\int_{-1}^{1}\int_{0}^{2\pi}w_3(r)e^{-ikr\cos\theta}r^2d{r}d{\cos\theta}d{\phi}}\end{equation}

\noindent Calculate $\widetilde{w}_3(k)$ 
\begin{equation}{\widetilde{w}_3(k)=\frac{1}{2}\int_{0}^{\infty}\left[1-\operatorname{erf}\left(\frac{r-\frac{\alpha}{2}}{a}\right)\right]r^2\left(\int_{-1}^{1}e^{-ikr\cos\theta}d{\cos\theta}\left(\int_{0}^{2\pi}d{\phi}\right)\right)d{r}}\end{equation}
\[{}\]

\begin{equation}{\widetilde{w}_3(k)=\frac{2\pi}{2}\int_{0}^{\infty}\left[1-\operatorname{erf}\left(\frac{r-\frac{\alpha}{2}}{a}\right)\right]r^2\left(\int_{-1}^{1}e^{-ikr\cos\theta}d{\cos\theta}\right)d{r}}\end{equation}

\begin{equation}{\widetilde{w}_3(k)=\pi\int_{0}^{\infty}\left[1-\operatorname{erf}\left(\frac{r-\frac{\alpha}{2}}{a}\right)\right]r^2\frac{2\sin(kr)}{kr}d{r}}\end{equation}

\begin{equation}{\widetilde{w}_3(k)=\frac{2\pi}{k}\int_{0}^{\infty}\left[1-\operatorname{erf}\left(\frac{r-\frac{\alpha}{2}}{a}\right)\right]r\sin(kr)d{r}}\end{equation}
Integrating by parts with 
\begin{displaymath}{f=1-\operatorname{erf}\left(\frac{r-\frac{\alpha}{2}}{a}\right)}\end{displaymath}
\begin{displaymath}{g=\frac{\sin(kr)-kr\cos(kr)}{k^2}}\end{displaymath}

\begin{equation}{\widetilde{w}_3(k)=\frac{2\pi}{k}\left[\left(1-\operatorname{erf}\left(\frac{r-\frac{\alpha}{2}}{a}\right)\right)\left(\frac{\sin(kr)-kr\cos(kr)}{k^2}\right)\bigg|^{\infty}_0\color{white}\right]}\end{equation} \color{black}
\begin{displaymath}{-\int_{0}^{\infty}\left(\frac{\sin(kr)-kr\cos(kr)}{k^2}\right)\left(\frac{2}{a\sqrt{\pi}}e^{-\left(\frac{r-\frac{\alpha}{2}}{a}\right)^2}\right)d{r}}\end{displaymath}
Evaluating the integral by doing a u-substitution with $u=\frac{r-\frac{\alpha}{2}}{a}$, and setting the lower limit to $-\infty$  since the integrand is nearly zero by the time r is zero this becomes
\begin{equation}{\widetilde{w}_3(k)=\frac{4\pi}{k^3}e^{-\frac{k^2a^2}{4}}\left[\left(1+\frac{k^2a^2}{2}\right)\sin\left(\frac{k\alpha}{2}\right)-\frac{k\alpha}{2}\cos\left(\frac{k\alpha}{2}\right)\right]}\end{equation}
Putting $a=\frac{\Xi}{\sqrt{2}}$ back in, the final equation for $\widetilde{w}_3(k)$ becomes
\begin{equation}{\widetilde{w}_3(k)=\frac{4\pi}{k^3}e^{-\frac{k^2\Xi^2}{8}}\left[\left(1+\frac{k^2\Xi^2}{4}\right)\sin\left(\frac{k\alpha}{2}\right)-\frac{k\alpha}{2}\cos\left(\frac{k\alpha}{2}\right)\right]}\end{equation}

\[{}\]
%\[{}\]
\textbf{2a. Fourier Transform of Vector Weight Function $\vec{w}_{1}(r)$:}
\begin{equation}{\vec{w}_1(\vec{r})=w_1(r)\frac{\vec{r}}{r}}\end{equation}
where
\begin{equation}{w_1(r)=\frac{w_2(r)}{4{\pi}r}}\end{equation}
\begin{equation}{w_2(r)=\frac{\sqrt{2}}{\Xi\sqrt{\pi}}e^{-\left(\frac{r-\frac{\alpha}{2}}{\frac{\Xi}{\sqrt{2}}}\right)^2}}\end{equation}
temporarily set 
\begin{equation}{a=\frac{\Xi}{\sqrt{2}}}\end{equation}
\begin{equation}{w_1(r)=\frac{1}{4{\pi}\sqrt{\pi}a}\left(\frac{1}{r}\right)e^{-\left(\frac{r-\frac{\alpha}{2}}{a}\right)^2}}\end{equation}

\begin{equation}{\widetilde{\vec{w}}_1(\vec{k})=\int_{0}^{\infty}\int_{-1}^{1}\int_{0}^{2\pi}\vec{w}_1(\vec{r})e^{-i\vec{k}\cdot{\vec{r}}}r^2d{r}d{\cos\theta}d{\phi}}\end{equation}

\begin{equation}{\widetilde{\vec{w}}_1(\vec{k})=\int_{0}^{\infty}\int_{-1}^{1}\int_{0}^{2\pi}w_1(r){~}\frac{\vec{r}}{r}{~}e^{-i\vec{k}\cdot{\vec{r}}}r^2d{r}d{\cos\theta}d{\phi}}\end{equation}
Set $\vec{k}$ in the direction of $\hat{z}'$ of a rotated coordinate system $(x',y'z')$ 
\begin{equation}{\widetilde{\vec{w}}_1(\vec{k})=\int_{0}^{\infty}\int_{-1}^{1}\int_{0}^{2\pi}w_1(r)\frac{(r_{x'}\hat{x}'+r_{y'}\hat{y}'+r_{z'}\hat{z}')}{r}e^{-ikr\cos\theta'}r^2d{r}d{\cos\theta'}d{\phi'}}\end{equation}
where 
\begin{displaymath}{r_{x'}=r\sin\theta'\cos\phi'}\end{displaymath}
\begin{displaymath}{r_{y'}=r\sin\theta'\sin\phi'}\end{displaymath}
\begin{displaymath}{r_{z'}=r\cos\theta'}\end{displaymath} 
The components in the $x'$ and $y'$ directions vanish due to symmetry about $z'$. That leaves
\begin{equation}{\widetilde{\vec{w}}_1(\vec{k})=\int_{0}^{\infty}\int_{-1}^{1}\int_{0}^{2\pi}w_1(r)\cos{\theta}'e^{-ikr\cos\theta'}r^2d{r}d{\cos\theta'}d{\phi'}{~~}\hat{z}'}\end{equation}

\begin{equation}{\widetilde{\vec{w}}_1(\vec{k})=\frac{1}{4\pi\sqrt{\pi}a}\int_{0}^{\infty}\int_{-1}^{1}\int_{0}^{2\pi}e^{-\left(\frac{r-\frac{\alpha}{2}}{a}\right)^2}r\cos{\theta}'e^{-ikr\cos\theta'}d{r}d{\cos\theta'}d{\phi'}{~~}\hat{z}'}\end{equation}

\begin{equation}{\widetilde{\vec{w}}_1(\vec{k})=\frac{2}{4\sqrt{\pi}a}\int_{0}^{\infty}e^{-\left(\frac{r-\frac{\alpha}{2}}{a}\right)^2}r\left(\int_{-1}^{1}e^{-ikr\cos\theta'}\cos{\theta}'d{\cos\theta'}\right)d{r}{~~}\hat{z}'}\end{equation}

\begin{equation}{\widetilde{\vec{w}}_1(\vec{k})=\frac{2}{4\sqrt{\pi}a}\int_{0}^{\infty}e^{-\left(\frac{r-\frac{\alpha}{2}}{a}\right)^2}r\left(-\frac{2i}{kr}\cos(kr)+\frac{2i}{k^2r^2}\sin(kr)\right)d{r}{~~}\hat{z}'}\end{equation}

\begin{equation}{\widetilde{\vec{w}}_1(\vec{k})=\frac{i}{k\sqrt{\pi}a}\left[-\int_{0}^{\infty}e^{-\left(\frac{r-\frac{\alpha}{2}}{a}\right)^2}\cos(kr)d{r}+\frac{1}{k}\int_{0}^{\infty}e^{-\left(\frac{r-\frac{\alpha}{2}}{a}\right)^2}\frac{\sin(kr)}{r}d{r}\right]{~~}\hat{z}'}\end{equation}

\begin{equation}{\widetilde{\vec{w}}_1(\vec{k})=\frac{i}{k\sqrt{\pi}a}\left[-a\sqrt{\pi}e^{-\frac{k^2a^2}{4}}\cos\left(\frac{k\alpha}{2}\right)+\frac{\pi}{2k}e^{-\left(\frac{\alpha}{2a}\right)^2}\left[\operatorname{erf}\left(\frac{ka}{2}+\frac{i\alpha}{2a}\right)+\operatorname{erf}\left(\frac{ka}{2}-\frac{i\alpha}{2a}\right)\right]\right]{~~}\hat{z}'}\end{equation}

\begin{equation}{\widetilde{\vec{w}}_1(\vec{k})=\frac{i}{k^2}\left[\frac{\sqrt{\pi}}{2ka}e^{-\left(\frac{\alpha}{2a}\right)^2}\left[\operatorname{erf}\left(\frac{ka}{2}+\frac{i\alpha}{2a}\right)+\operatorname{erf}\left(\frac{ka}{2}-\frac{i\alpha}{2a}\right)\right]-e^{-\frac{k^2a^2}{4}}\cos\left(\frac{k\alpha}{2}\right)\right]{~~}\vec{k}}\end{equation}
Putting $a=\frac{\Xi}{\sqrt{2}}$ back in, the final equation for $\widetilde{\vec{w}}_1(k)$ becomes
\begin{equation}{\widetilde{\vec{w}}_1(\vec{k})=\frac{i}{k^2}\left[\frac{\sqrt{\pi}}{\sqrt{2}k\Xi}e^{-\left(\frac{\alpha}{\sqrt{2}\Xi}\right)^2}\left[\operatorname{erf}\left(\frac{k\Xi}{(\sqrt{2})^3}+\frac{i\alpha}{\sqrt{2}\Xi}\right)+\operatorname{erf}\left(\frac{k\Xi}{(\sqrt{2})^3}-\frac{i\alpha}{\sqrt{2}\Xi}\right)\right]-e^{-\frac{k^2\Xi^2}{8}}\cos\left(\frac{k\alpha}{2}\right)\right]{~~}\vec{k}}\end{equation}

\[{}\]
%\[{}\]
\textbf{2b. Fourier Transform of Vector Weight Function $\vec{w}_{2}(r)$:}
\begin{equation}{\vec{w}_2(\vec{r})=w_2(r)\frac{\vec{r}}{r}}\end{equation}
where
\begin{equation}{w_2(r)=\frac{\sqrt{2}}{\Xi\sqrt{\pi}}e^{-\left(\frac{r-\frac{\alpha}{2}}{\frac{\Xi}{\sqrt{2}}}\right)^2}}\end{equation}
temporarily set 
\begin{equation}{a=\frac{\Xi}{\sqrt{2}}}\end{equation}
\begin{equation}{w_2(r)=\frac{1}{a\sqrt{\pi}}e^{-\left(\frac{r-\frac{\alpha}{2}}{a}\right)^2}}\end{equation}

\begin{equation}{\widetilde{\vec{w}}_2(\vec{k})=\int_{0}^{\infty}\int_{-1}^{1}\int_{0}^{2\pi}w_2(r){~}\frac{\vec{r}}{r}{~}e^{-i\vec{k}\cdot{\vec{r}}}r^2d{r}d{\cos\theta}d{\phi}}\end{equation}
Set $\vec{k}$ in the direction of $\hat{z}'$ of a rotated coordinate system $(x',y'z')$ 
\begin{equation}{\widetilde{\vec{w}}_2(\vec{k})=\int_{0}^{\infty}\int_{-1}^{1}\int_{0}^{2\pi}w_2(r)\frac{(r_{x'}\hat{x}'+r_{y'}\hat{y}'+r_{z'}\hat{z}')}{r}e^{-ikr\cos\theta'}r^2d{r}d{\cos\theta'}d{\phi'}}\end{equation}
where 
\begin{displaymath}{r_{x'}=r\sin\theta'\cos\phi'}\end{displaymath}
\begin{displaymath}{r_{y'}=r\sin\theta'\sin\phi'}\end{displaymath}
\begin{displaymath}{r_{z'}=r\cos\theta'}\end{displaymath} 
The components in the $x'$ and $y'$ directions vanish due to symmetry about $z'$. That leaves
\begin{equation}{\widetilde{\vec{w}}_2(\vec{k})=\int_{0}^{\infty}\int_{-1}^{1}\int_{0}^{2\pi}w_2(r)\cos{\theta}'e^{-ikr\cos\theta'}r^2d{r}d{\cos\theta'}d{\phi'}{~~}\hat{z}'}\end{equation}

\begin{equation}{\widetilde{\vec{w}}_2(\vec{k})=\frac{2\sqrt{\pi}}{a}\int_{0}^{\infty}e^{-\left(\frac{r-\frac{\alpha}{2}}{a}\right)^2}r^2\left(\int_{-1}^{1}e^{-ikr\cos\theta'}\cos{\theta}'d{\cos\theta'}\right)d{r}{~~}\hat{z}'}\end{equation}

\begin{equation}{\widetilde{w}_2(\vec{k})=\frac{2\sqrt{\pi}}{a}\int_{0}^{\infty}e^{-\left(\frac{r-\frac{\alpha}{2}}{a}\right)^2}r^2\left(-\frac{2i}{kr}\cos(kr) + \frac{2i}{k^2r^2}\sin(kr)\right)d{r}{~~}\hat{z}'}\end{equation}

\begin{equation}{\widetilde{\vec{w}}_2(\vec{k})=\frac{-4i\sqrt{\pi}}{ak}\int_{0}^{\infty}e^{-\left(\frac{r-\frac{\alpha}{2}}{a}\right)^2}r\cos(kr)d{r} + \frac{4i\sqrt{\pi}}{ak^2}\int_{0}^{\infty}e^{-\left(\frac{r-\frac{\alpha}{2}}{a}\right)^2}\sin(kr)d{r}{~~}\hat{z}'}\end{equation}

\begin{equation}{\widetilde{\vec{w}}_2(\vec{k})=\frac{-4\sqrt{\pi}i}{ak}\left[\frac{a\alpha\sqrt{\pi}}{2}e^{-\frac{k^2a^2}{4}}\cos\left(\frac{k\alpha}{2}\right)-\frac{a^3k\sqrt{\pi}}{2}e^{-\frac{k^2a^2}{4}}\sin\left(\frac{k\alpha}{2}\right)\right]}\end{equation} 
\begin{displaymath}{+{~~}\frac{4\sqrt{\pi}i}{ak^2}\left[a\sqrt{\pi}e^{-\frac{k^2a^2}{4}}\sin\left(\frac{k\alpha}{2}\right)\right]{~~}\hat{z}'}\end{displaymath}
\[{}\]
\begin{equation}{\widetilde{\vec{w}}_2(\vec{k})=\frac{4\pi{i}}{k^2}e^{-\frac{k^2a^2}{4}}\left[\left(\frac{a^2k}{2}+\frac{1}{k}\right)\sin\left(\frac{k\alpha}{2}\right)-\frac{\alpha}{2}\cos\left(\frac{k\alpha}{2}\right)\right]{~~}\vec{k}}\end{equation} 
Putting $a=\frac{\Xi}{\sqrt{2}}$ back in, the final equation for $\widetilde{\vec{w}}_2(k)$ becomes
\begin{equation}{\widetilde{\vec{w}}_2(\vec{k})=\frac{4\pi{i}}{k^2}e^{-\frac{k^2\Xi^2}{8}}\left[\left(\frac{\Xi^2k}{4}+\frac{1}{k}\right)\sin\left(\frac{k\alpha}{2}\right)-\frac{\alpha}{2}\cos\left(\frac{k\alpha}{2}\right)\right]{~~}\vec{k}}\end{equation} 
\[{}\]
\textbf{3. Fourier Transform of Tensor Weight $\overleftrightarrow{w}_{m2}(r)$:}
\begin{equation}{\overleftrightarrow{w}_{m2}(\vec{r})=w_2(r)\left(\frac{\vec{r}\vec{r}}{r^2}-\frac{I}{3}\right)}\end{equation}
where
\begin{equation}{w_2(r)=\frac{\sqrt{2}}{\Xi\sqrt{\pi}}e^{-\left(\frac{r-\frac{\alpha}{2}}{\frac{\Xi}{\sqrt{2}}}\right)^2}}\end{equation}
Form the outer product
\begin{equation}{\vec{r}\vec{r}=\left(\begin{array}{c} r_x \\ r_y \\ r_z \end{array} \right) \left(\begin{array}{rrr} r_x & r_y & r_z \end{array} \right)=\left(\begin{array}{ccc} {r^2}_x & r_xr_y & r_xr_z \\ r_yr_x & {r_y}^2 & r_yr_z \\ r_zr_x & r_zr_y & {r_z}^2 \end{array}\right)}\end{equation}
and identity matrix
\begin{equation}{I=\left(\begin{array}{ccc} 1 & 0 & 0 \\ 0 & 1 & 0 \\ 0 & 0 & 1 \end{array}\right)}\end{equation}
The tensor weight matrix elements are then given by
\begin{equation}{w_{m2_{ij}}(\vec{r})=\frac{1}{a\sqrt{\pi}}e^{-\left(\frac{r-\frac{\alpha}{2}}{a}\right)^2}\left(\frac{r_ir_j}{r^2}-\frac{\delta_{ij}}{3}\right)}\end{equation}
with temporarily setting
\begin{equation}{a=\frac{\Xi}{\sqrt{2}}}\end{equation}
The fourier transform of the tensor weight matrix elements are given by
\begin{equation}{\widetilde{w}_{m2_{ij}}(\vec{k})=\int_{allspace}{w_{{m2}_{ij}}}(\vec{r})e^{-i\vec{k}\cdot\vec{r}}d{\vec{r}}}\end{equation}
Since the integration is over all space, the integral can be simplified by setting 
\begin{equation}{\vec{k}=k\hat{z}}\end{equation}
\begin{equation}{\vec{k}\cdot\vec{r}=kr\cos\theta\hat{z}}\end{equation}
Instead of constraining vector $\vec{k}$ to be in the z direction, however, a rotated coordinate system $(x',y',z')$ is used which is rotated from the (x,y,z) coordinate system in such a way that $\hat{z}'$ points in the direction of vector $\vec{k}$. After solving for the elements of the transformed weight function in coordinate system $(x',y',z')$, they will be expressed again in terms of vector $\vec{k}$ as it appears in coordinate system (x,y,z) to get the elements of the transformed weight function in coordinate system (x,y,z).%The weight components of interest are those corresponding to the component of vector $\vec{r}$ parallel to vector $\vec{k}$ which will be called $r_{z'}$, and to the two components of $\vec{r}$ perpendicular to vector $\vec{k}$ (and to each other) which will be called $r_{x'}$ and $r_{y'}$. 
\begin{equation}{\widetilde{w}_{m2_{ij}}(k)=\frac{1}{a\sqrt{\pi}}\int_{0}^{\infty}\int_{-1}^{1}\int_{0}^{2\pi}e^{-\left(\frac{r-\frac{\alpha}{2}}{a}\right)^2}\left(\frac{r_ir_j}{r^2}-\frac{\delta_{ij}}{3}\right)e^{-ikr\cos\theta'}r^2d{r}d{\cos\theta'}d{\phi'}}\end{equation}
%\begin{displaymath}{r_x=r\sin(\theta)\cos(\phi) ~~~~~ r_y=r\sin(\theta)\sin(\phi)~~~~~ r_z=r\cos(\theta)}\end{displaymath}
where $r=r'$ and
\begin{displaymath}{r_{x'}=r\sin\theta\cos\phi}\end{displaymath}
\begin{displaymath}{r_{y'}=r\sin\theta\sin\phi}\end{displaymath}
\begin{displaymath}{r_{z'}=r\cos\theta}\end{displaymath} 
with
\begin{displaymath}{\delta_{ij}=\left\{ \begin{array}{rc} 1 & i = j \\ 0  & i\neq j \end{array}\right.}\end{displaymath}
\[{}\]

%\[{}\]
%\[{}\]
Calculate $\widetilde{w}_{{m2}_{x'y'}}(k)$ 
%where \begin{displaymath}{r_x=r\sin(\theta)\cos(\phi)}\end{displaymath}
%\begin{displaymath}{r_y=r\sin(\theta)\sin(\phi)}\end{displaymath}
\begin{equation}{\widetilde{w}_{{m2}_{x'y'}}(k)=\frac{1}{a\sqrt{\pi}}\int_{0}^{\infty}\int_{-1}^{1}\int_{0}^{2\pi}e^{-\left(\frac{r-\frac{\alpha}{2}}{a}\right)^2}\left(\frac{(r_{x'})(r_{y'})}{r^2}-\frac{\delta_{x'y'}}{3}\right)e^{-ikr\cos\theta'}r^2d{r}d{\cos\theta'}d{\phi'}}\end{equation}
\begin{equation}{\widetilde{w}_{{m2}_{x'y'}}(k)=\frac{1}{a\sqrt{\pi}}\int_{0}^{\infty}\int_{-1}^{1}e^{-\left(\frac{r-\frac{\alpha}{2}}{a}\right)^2}r^2e^{-ikr\cos\theta'}\sin^2\theta'\underbrace{\left(\int_{0}^{2\pi}\cos\phi'\sin{\phi'}~d{\phi'}\right)}d{r}d{\cos\theta'}=0}\end{equation}
$~~~~~~~~~~~~~~~~~~~~~~~~~~~~~~~~~~~~~~~~~~~~~~~~~~~~~~~~~~~~~~~~~~~~~~~~~~~~~~~~0$

Calculate $\widetilde{w}_{{m2}_{x'z'}}(k)$ 
\begin{equation}{\widetilde{w}_{{m2}_{x'z'}}(k)=\frac{1}{a\sqrt{\pi}}\int_{0}^{\infty}\int_{-1}^{1}\int_{0}^{2\pi}e^{-\left(\frac{r-\frac{\alpha}{2}}{a}\right)^2}\left(\frac{(r_{x'})(r_{z'})}{r^2}-\frac{\delta_{x'z'}}{3}\right)e^{-ikr\cos\theta'}r^2d{r}d{\cos\theta'}d{\phi'}}\end{equation}
\begin{equation}{\widetilde{w}_{{m2}_{x'z'}}(k)=\frac{1}{a\sqrt{\pi}}\int_{0}^{\infty}\int_{-1}^{1}e^{-\left(\frac{r-\frac{\alpha}{2}}{a}\right)^2}r^2e^{-ikr\cos\theta'}\sin\theta'\cos\theta'\underbrace{\left(\int_{0}^{2\pi}\cos\phi'~d{\phi'}\right)}d{r}d{\cos\theta'}=0}\end{equation}
$~~~~~~~~~~~~~~~~~~~~~~~~~~~~~~~~~~~~~~~~~~~~~~~~~~~~~~~~~~~~~~~~~~~~~~~~~~~~~~~~~~0$
\[{}\]
Calculate $\widetilde{w}_{{m2}_{y'z'}}(k)$ 
\begin{equation}{\widetilde{w}_{{m2}_{y'z'}}(\vec{k})=\frac{1}{a\sqrt{\pi}}\int_{0}^{\infty}\int_{-1}^{1}\int_{0}^{2\pi}e^{-\left(\frac{r-\frac{\alpha}{2}}{a}\right)^2}\left(\frac{(r_{y'})(r_{z'})}{r^2}-\frac{\delta_{yz}}{3}\right)e^{-ikr\cos\theta'}r^2d{r}d{\cos\theta'}d{\phi'}}\end{equation}
\begin{equation}{\widetilde{w}_{{m2}_{y'z'}}(\vec{k})=\frac{1}{a\sqrt{\pi}}\int_{0}^{\infty}\int_{-1}^{1}e^{-\left(\frac{r-\frac{\alpha}{2}}{a}\right)^2}r^2e^{-ikr\cos\theta'}\sin\theta'\cos\theta'\underbrace{\left(\int_{0}^{2\pi}\sin\phi'~d{\phi'}\right)}d{r}d{\cos\theta'}=0}\end{equation}
$~~~~~~~~~~~~~~~~~~~~~~~~~~~~~~~~~~~~~~~~~~~~~~~~~~~~~~~~~~~~~~~~~~~~~~~~~~~~~~~~~~0$
\[{}\]
Calculate $\widetilde{w}_{{m2}_{x'x'}}(k)$ 
\begin{equation}{\widetilde{w}_{{m2}_{x'x'}}(k)=\frac{1}{a\sqrt{\pi}}\int_{0}^{\infty}\int_{-1}^{1}\int_{0}^{2\pi}e^{-\left(\frac{r-\frac{\alpha}{2}}{a}\right)^2}\left(\frac{(r_{x'})(r_{x'})}{r^2}-\frac{\delta_{x'x'}}{3}\right)e^{-ikr\cos\theta'}r^2d{r}d{\cos\theta'}d{\phi'}}\end{equation}
\begin{equation}{\widetilde{w}_{{m2}_{x'x'}}(k)=\frac{1}{a\sqrt{\pi}}\int_{0}^{\infty}\int_{-1}^{1}\int_{0}^{2\pi}e^{-\left(\frac{r-\frac{\alpha}{2}}{a}\right)^2}\left(\sin^2\theta'\cos^2\phi'-\frac{1}{3}\right)e^{-ikr\cos\theta'}r^2d{r}d{\cos\theta'}d{\phi'}}\end{equation}
\[{}\]
%\[{}\]
%\begin{displaymath}{\widetilde{w}_{{m2}_{xx}}(\vec{k})=\frac{1}{a\sqrt{\pi}}\int_{0}^{\infty}e^{\left(\frac{r-\frac{\alpha}{2}}{a}\right)^2}r^2\left[\int_{-1}^{1}\sin^2(\theta)e^{-ikr\cos(\theta)}\left(\int_{0}^{2\pi}\cos^2(\phi)d{\phi}\right)d{\cos(\theta)}\right]d{r}}\end{displaymath} 
\begin{displaymath}{\widetilde{w}_{{m2}_{x'x'}}(k)=\frac{1}{a\sqrt{\pi}}\int_{0}^{\infty}e^{-\left(\frac{r-\frac{\alpha}{2}}{a}\right)^2}r^2\left[\int_{-1}^{1}\left(1-\cos^2\theta'\right)e^{-ikr\cos\theta'}\left(\int_{0}^{2\pi}\cos^2\phi'~d{\phi'}\right)d{\cos\theta'}\right]d{r}}\end{displaymath} 
\begin{equation}{-\frac{1}{3a\sqrt{\pi}}\int_{0}^{\infty}e^{-\left(\frac{r-\frac{\alpha}{2}}{a}\right)^2}r^2\left[\int_{-1}^{1}e^{-ikr\cos\theta'}\left(\int_{0}^{2\pi}d{\phi'}\right)d{\cos\theta'}\right]d{r}}\end{equation}
\[{}\]
\color{blue}
\begin{displaymath}{\widetilde{w}_{{m2}_{x'x'}}(k)=-\frac{\sqrt{\pi}}{a}\int_{0}^{\infty}e^{-\left(\frac{r-\frac{\alpha}{2}}{a}\right)^2}r^2\left[\int_{-1}^{1}\cos^2\theta'~e^{-ikr\cos\theta'}~d{\cos\theta'}\right]d{r}}\end{displaymath} 
\begin{equation}\label{compare_equation}{+\frac{\sqrt{\pi}}{3a}\int_{0}^{\infty}e^{-\left(\frac{r-\frac{\alpha}{2}}{a}\right)^2}r^2\left[\int_{-1}^{1}e^{-ikr\cos\theta'}~d{\cos\theta'}\right]d{r}}\end{equation}
\color{black}
\[{}\]
\begin{displaymath}{\widetilde{w}_{{m2}_{x'x'}}(k)=-\frac{\sqrt{\pi}}{a}\int_{0}^{\infty}e^{-\left(\frac{r-\frac{\alpha}{2}}{a}\right)^2}r^2\left(\frac{2\sin(kr)}{kr}+\frac{4\cos(kr)}{k^2r^2}-\frac{4\sin(kr)}{k^3r^3}\right)d{r}}\end{displaymath} 
\begin{equation}{+\frac{\sqrt{\pi}}{3a}\int_{0}^{\infty}e^{-\left(\frac{r-\frac{\alpha}{2}}{a}\right)^2}r^2\left(\frac{2\sin(kr)}{kr}\right)d{r}}\end{equation}
%\[{}\]
Set the lower limit to -infinity since the integrand is nearly zero by the time r is zero. 
\begin{displaymath}{\widetilde{w}_{{m2}_{x'x'}}(k)=-\frac{4\sqrt{\pi}}{3ka}\left(\int_{-\infty}^{\infty}e^{-\left(\frac{r-\frac{\alpha}{2}}{a}\right)^2}r\sin(kr)d{r}\right)}\end{displaymath} 
\begin{displaymath}{-\frac{4\sqrt{\pi}}{k^2a}\left(\int_{-\infty}^{\infty}e^{-\left(\frac{r-\frac{\alpha}{2}}{a}\right)^2}\cos(kr)d{r}\right)}\end{displaymath} 
\begin{equation}{+\frac{4\sqrt{\pi}}{k^3a}\left(\int_{-\infty}^{\infty}e^{-\left(\frac{r-\frac{\alpha}{2}}{a}\right)^2}\frac{\sin(kr)}{r}d{r}\right)}\end{equation} 
\[{}\]
\begin{displaymath}{\widetilde{w}_{{m2}_{x'x'}}(k)=-\frac{4\sqrt{\pi}}{3ka}\left(\frac{ka^3}{2}\sqrt{\pi}e^{-\left(\frac{ka}{2}\right)^2}\cos(\frac{k\alpha}{2})+\frac{a\alpha}{2}\sqrt{\pi}e^{-\left(\frac{ka}{2}\right)^2}\sin(\frac{k\alpha}{2})\right)}\end{displaymath} 
\begin{displaymath}{-\frac{4\sqrt{\pi}}{k^2a}\left(a\sqrt{\pi}e^{-\left(\frac{ka}{2}\right)^2}\cos(\frac{k\alpha}{2})\right)}\end{displaymath} 
\begin{equation}{+\frac{4\sqrt{\pi}}{k^3a}\left(\frac{\pi}{2}e^{-\left(\frac{\alpha}{2a}\right)^2}\left[\operatorname{erf}\left(\frac{ka}{2}+i\frac{\alpha}{2a}\right)+\operatorname{erf}\left(\frac{ka}{2}-i\frac{\alpha}{2a}\right)\right]\right)}\end{equation} 
\[{}\]
\begin{displaymath}{\widetilde{w}_{{m2}_{x'x'}}(k)=\left(\frac{-2\pi{a}^2}{3}-\frac{4\pi}{k^2}\right)e^{-\left(\frac{ka}{2}\right)^2}\cos(\frac{k\alpha}{2})-\frac{2\pi\alpha}{3k}e^{-\left(\frac{ka}{2}\right)^2}\sin(\frac{k\alpha}{2})}\end{displaymath} 
\begin{equation}{+\frac{2\pi\sqrt{\pi}}{k^3a}e^{-\left(\frac{\alpha}{2a}\right)^2}\left[\operatorname{erf}\left(\frac{ka}{2}+i\frac{\alpha}{2a}\right)+\operatorname{erf}\left(\frac{ka}{2}-i\frac{\alpha}{2a}\right)\right]}\end{equation} 
Putting in $a=\frac{\Xi}{\sqrt{2}}$ the final equation for $\widetilde{w}_{{m2}_{xx}}(k)$ becomes
\[{}\] \color{green}
\begin{displaymath}{\widetilde{w}_{{m2}_{x'x'}}(k)=\left(\frac{-\pi{\Xi}^2}{3}-\frac{4\pi}{k^2}\right)e^{-\frac{k^2\Xi^2}{8}}\cos(\frac{k\alpha}{2})-\frac{2\pi\alpha}{3k}e^{-\frac{k^2\Xi^2}{8}}\sin(\frac{k\alpha}{2})}\end{displaymath} 
\begin{equation}{+\frac{2\pi\sqrt{2\pi}}{k^3\Xi}e^{-\left(\frac{\alpha^2}{2\Xi^2}\right)}\left[\operatorname{erf}\left(\frac{k\Xi}{2*\sqrt{2}}+i\frac{\alpha}{\sqrt{2}\Xi}\right)+\operatorname{erf}\left(\frac{k\Xi}{2*\sqrt{2}}-i\frac{\alpha}{\sqrt{2}\Xi}\right)\right]}\end{equation} 
\color{black}
\[{}\]
Calculate $\widetilde{w}_{{m2}_{y'y'}}(k)$ 
\begin{equation}{\widetilde{w}_{{m2}_{y'y'}}(k)=\frac{1}{a\sqrt{\pi}}\int_{0}^{\infty}\int_{-1}^{1}\int_{0}^{2\pi}e^{-\left(\frac{r-\frac{\alpha}{2}}{a}\right)^2}\left(\frac{(r_{y'})(r_{y'})}{r^2}-\frac{\delta_{y'y'}}{3}\right)e^{-ikr\cos\theta'}r^2d{r}d{\cos\theta'}d{\phi'}}\end{equation}
\begin{equation}{\widetilde{w}_{{m2}_{y'y'}}(k)=\frac{1}{a\sqrt{\pi}}\int_{0}^{\infty}\int_{-1}^{1}\int_{0}^{2\pi}e^{-\left(\frac{r-\frac{\alpha}{2}}{a}\right)^2}\left(\sin^2\theta'\sin^2\phi'-\frac{1}{3}\right)e^{-ikr\cos\theta'}r^2d{r}d{\cos\theta'}d{\phi'}}\end{equation}
\[{}\]
\begin{displaymath}{\widetilde{w}_{{m2}_{y'y'}}(k)=\frac{1}{a\sqrt{\pi}}\int_{0}^{\infty}e^{-\left(\frac{r-\frac{\alpha}{2}}{a}\right)^2}r^2\left[\int_{-1}^{1}\left(1-\cos^2\theta\right)e^{-ikr\cos\theta'}\left(\int_{0}^{2\pi}\sin^2\phi~d{\phi}\right)d{\cos\theta}\right]d{r}}\end{displaymath} 
\begin{equation}{-\frac{1}{3a\sqrt{\pi}}\int_{0}^{\infty}e^{-\left(\frac{r-\frac{\alpha}{2}}{a}\right)^2}r^2\left[\int_{-1}^{1}e^{-ikr\cos\theta'}\left(\int_{0}^{2\pi}d{\phi'}\right)d{\cos\theta'}\right]d{r}}\end{equation}

\color{blue}
\begin{displaymath}{\widetilde{w}_{{m2}_{y'y'}}(k)=-\frac{\sqrt{\pi}}{a}\int_{0}^{\infty}e^{-\left(\frac{r-\frac{\alpha}{2}}{a}\right)^2}r^2\left[\int_{-1}^{1}\cos^2\theta~e^{-ikr\cos\theta'}~d{\cos\theta'}\right]d{r}}\end{displaymath} 
\begin{equation}{+\frac{\sqrt{\pi}}{3a}\int_{0}^{\infty}e^{-\left(\frac{r-\frac{\alpha}{2}}{a}\right)^2}r^2\left[\int_{-1}^{1}e^{-ikr\cos\theta'}~d{\cos\theta'}\right]d{r}}\end{equation}
\color{black} 
\[{}\]
This is the same as equation \ref{compare_equation}, and so \begin{equation}{\widetilde{w}_{{m2}_{y'y'}}(k)=\widetilde{w}_{{m2}_{x'x'}}(k)}\end{equation}
Calculate $\widetilde{w}_{{m2}_{z'z'}}(k)$ 
\begin{equation}{\widetilde{w}_{{m2}_{z'z'}}(k)=\frac{1}{a\sqrt{\pi}}\int_{0}^{\infty}\int_{-1}^{1}\int_{0}^{2\pi}e^{\left(\frac{r-\frac{\alpha}{2}}{a}\right)^2}\left(\frac{(r_{y'})(r_{y'})}{r^2}-\frac{\delta_{z'z'}}{3}\right)e^{-ikr\cos\theta'}r^2d{r}d{\cos\theta'}d{\phi'}}\end{equation}
\begin{equation}{\widetilde{w}_{{m2}_{z'z'}}(k)=\frac{1}{a\sqrt{\pi}}\int_{0}^{\infty}\int_{-1}^{1}\int_{0}^{2\pi}e^{\left(\frac{r-\frac{\alpha}{2}}{a}\right)^2}\left(\cos^2\theta'-\frac{1}{3}\right)e^{-ikr\cos\theta'}r^2d{r}d{\cos\theta'}d{\phi'}}\end{equation}
\[{}\]
\begin{displaymath}{\widetilde{w}_{{m2}_{z'z'}}(k)=\frac{1}{a\sqrt{\pi}}\int_{0}^{\infty}e^{-\left(\frac{r-\frac{\alpha}{2}}{a}\right)^2}r^2\left[\int_{-1}^{1}\cos^2\theta'~e^{-ikr\cos\theta'}\left(\int_{0}^{2\pi}d{\phi'}\right)d{\cos\theta'}\right]d{r}}\end{displaymath} 
\begin{equation}{-\frac{1}{3a\sqrt{\pi}}\int_{0}^{\infty}e^{-\left(\frac{r-\frac{\alpha}{2}}{a}\right)^2}r^2\left[\int_{-1}^{1}e^{-ikr\cos\theta'}\left(\int_{0}^{2\pi}d{\phi'}\right)d{\cos\theta'}\right]d{r}}\end{equation}

\color{blue}
\begin{displaymath}{\widetilde{w}_{{m2}_{z'z'}}(k)=2\frac{\sqrt{\pi}}{a}\int_{0}^{\infty}e^{-\left(\frac{r-\frac{\alpha}{2}}{a}\right)^2}r^2\left[\int_{-1}^{1}\cos^2\theta'~e^{-ikr\cos\theta'}d{\cos\theta'}\right]d{r}}\end{displaymath} 

\begin{equation}{-2\frac{\sqrt{\pi}}{3a}\int_{0}^{\infty}e^{-\left(\frac{r-\frac{\alpha}{2}}{a}\right)^2}r^2\left[\int_{-1}^{1}e^{-ikr\cos\theta'}d{\cos\theta'}\right]d{r}}\end{equation}
\color{black} 

This is the same as -2 times equation \ref{compare_equation}, and so \begin{equation}{\widetilde{w}_{{m2}_{z'z'}}(k)=-2\widetilde{w}_{{m2}_{x'x'}}(k)}\end{equation}

%\subsection{The tensor weight components in the coordinate system (x,y,z)}
\noindent The tensor weight components in the coordinate system $(x',y',z')$ are:
\begin{equation}\label{tensorcomp}{\widetilde{w'}_{m2}=\left(\begin{array}{ccc} \widetilde{w}_{{m2}_{x'x'}} & 0 & 0 \\ 0 & \widetilde{w}_{{m2}_{y'y'}} & 0 \\ 0 & 0 & \widetilde{w}_{{m2}_{z'z'}} \end{array}\right)'}\end{equation}

\begin{displaymath}{=\widetilde{w}_{{m2}_{x'x'}}\left(\begin{array}{ccc} 1 & 0 & 0 \\ 0 & 0 & 0 \\ 0 & 0 & 0 \end{array}\right)'+ \widetilde{w}_{{m2}_{y'y'}}\left(\begin{array}{ccc} 0 & 0 & 0 \\ 0 & 1 & 0 \\ 0 & 0 & 0 \end{array}\right)' + \widetilde{w}_{{m2}_{z'z'}}\left(\begin{array}{ccc} 0 & 0 & 0 \\ 0 & 0 & 0 \\ 0 & 0 & 1 \end{array}\right)'}\end{displaymath}

\begin{equation}{=\widetilde{w}_{{m2}_{x'x'}}\hat{x'}\hat{x'}+\widetilde{w}_{{m2}_{y'y'}}\hat{y'}\hat{y'}+\widetilde{w}_{{m2}_{z'z'}}\hat{z'}\hat{z'}}\end{equation}
written in terms of the outer products

\begin{displaymath}{\hat{x}'\hat{x}'= \left(\begin{array}{c} 1 \\ 0 \\ 0 \end{array}\right)'\left(\begin{array}{ccc} 1 & 0 & 0 \end{array}\right)'=\left(\begin{array}{ccc} 1 & 0 & 0 \\ 0 & 0 & 0 \\ 0 & 0 & 0 \end{array}\right)'}\end{displaymath}

\begin{displaymath}{\hat{y}'\hat{y}'= \left(\begin{array}{c} 0 \\ 1 \\ 0 \end{array}\right)'\left(\begin{array}{ccc} 0 & 1 & 0 \end{array}\right)'=\left(\begin{array}{ccc} 0 & 0 & 0 \\ 0 & 1 & 0 \\ 0 & 0 & 0 \end{array}\right)'}\end{displaymath}

\begin{displaymath}{\hat{z}'\hat{z}'= \left(\begin{array}{c} 0 \\ 0 \\ 1 \end{array}\right)'\left(\begin{array}{ccc} 0 & 0 & 1 \end{array}\right)'=\left(\begin{array}{ccc} 0 & 0 & 0 \\ 0 & 0 & 0 \\ 0 & 0 & 1 \end{array}\right)'}\end{displaymath}

%\newline $k_{x'}= k_{x'_x} \hat{x} + k_{x'_y} \hat{y} + k_{x'_z} \hat{z}$
%\newline $k_{y'}= k_{y'_x} \hat{x} + k_{y'_y} \hat{y} + k_{y'_z} \hat{z}$
%\newline $k_{z'}= k_{z'_x} \hat{x} + k_{z'_y} \hat{y} + k_{z'_z} \hat{z}$
%\newline \begin{displaymath}{\widetilde{w}_{{m2}_{y'y'}}(\vec{k})=\widetilde{w}_{{m2}_{x'x'}}(\vec{k})}\end{displaymath}
\noindent Using the relation
%\begin{displaymath}{\hat{k_{x'}}\hat{k_{x'}} + \hat{k_{y'}}\hat{k_{y'}} + \hat{k_{z'}}\hat{k_{z'}} = 1}\end{displaymath} 
\begin{displaymath}{\text{I} = \hat{x'}\hat{x'} + \hat{y'}\hat{y'} + \hat{z'}\hat{z'}}\end{displaymath}
\begin{displaymath}{\text{I} -\hat{z'}\hat{z'} = \hat{x'}\hat{x'} + \hat{y'}\hat{y'}}\end{displaymath}
and the results from earlier
\begin{equation}{\widetilde{w}_{{m2}_{y'y'}}=\widetilde{w}_{{m2}_{x'x'}}}\end{equation}
\begin{equation}{\widetilde{w}_{{m2}_{z'z'}}=-2\widetilde{w}_{{m2}_{x'x'}}}\end{equation}
Equation \ref{tensorcomp} then becomes 
\begin{equation}{\widetilde{w}_{m2}(\vec{k})= (\text{I}-3\hat{z}'\hat{z}')\widetilde{w}_{{m2}_{x'x'}}(\vec{k})}\end{equation}
\noindent In the $(x',y',z')$ coordinate system, vector $\vec{k}$ had been aligned with the $\hat{z}'$ direction so that 
\begin{displaymath}{\hat{k}=\hat{z'}}\end{displaymath}
Replacing $\hat{z}'$ with $\hat{k}$ gives
\begin{equation}{\widetilde{w}_{m2}(\vec{k})= (\text{I}-3\hat{k}\hat{k})\widetilde{w}_{{m2}_{x'x'}}(\vec{k})}\end{equation}
Now  $\vec{k}$ can be expressed in the $(x,y,z)$ coordinate system, $\vec{k}=k_x\hat{x} + k_y\hat{y} + k_z\hat{z}$. The components of the tensor weight in the $(x,y,z)$ coordinate system are then given by
\begin{equation}{\widetilde{w}_{m2_{ij}}(\vec{k})= (\delta{ij}-3\frac{k_ik_j}{k^2})\widetilde{w}_{{m2}_{x'x'}}(\vec{k})}\end{equation}
\[{}\]
%\subsection{Transform of tensor weight ${w}_{m2}(r)$ goes to zero as k goes to zero}
\noindent Check that the transform of tensor weight ${w}_{m2}(r)$ goes to zero as k goes to zero:
\begin{displaymath}{\widetilde{w}_{{m2}_{x'x'}}(k)=\left(\frac{-\pi\Xi^2}{3}-\frac{4\pi}{k^2}\right)e^{-\left(\frac{k^2\Xi^2}{8}\right)}\cos(\frac{k\alpha}{2})-\frac{2\pi\alpha}{3k}e^{-\left(\frac{k^2\Xi^2}{8}\right)}\sin(\frac{k\alpha}{2})}\end{displaymath} 
\begin{equation}{+\frac{2\pi\sqrt{2\pi}}{k^3\Xi}e^{-\left(\frac{\alpha^2}{2\Xi^2}\right)}\left[\operatorname{erf}\left(\frac{k\Xi}{2*\sqrt{2}}+i\frac{\alpha}{\sqrt{2}\Xi}\right)+\operatorname{erf}\left(\frac{k\Xi}{2*\sqrt{2}}-i\frac{\alpha}{\sqrt{2}\Xi}\right)\right]}\end{equation}
\[{}\]
Term 1: 
%\color{green}
\begin{equation}{\left(\frac{-\pi\Xi^2}{3}-\frac{4\pi}{k^2}\right)e^{-\left(\frac{k^2\Xi^2}{8}\right)}\cos(\frac{k\alpha}{2})}\end{equation}
\begin{displaymath}{\approx\left(-\frac{\pi\Xi^2}{3}-\frac{4\pi}{k^2}\right)\left(1-\frac{k^2\Xi^2}{8}\right)\left(1-\left(\frac{1}{2}\right)\frac{k^2\alpha^2}{4}\right)}\end{displaymath} 
\begin{displaymath}{\approx-\frac{4\pi}{k^2}+\frac{\pi\alpha^2}{2}+\frac{\pi\Xi^2}{6}}\end{displaymath} 
\color{black}
Term 2:
%\color{blue}
\begin{equation}{-\frac{2\pi\alpha}{3k}e^{-\left(\frac{k^2\Xi^2}{8}\right)}\sin(\frac{k\alpha}{2})}\end{equation} 
\begin{displaymath}{\approx\left(-\frac{2\pi\alpha}{3k}\right)\left(1-\frac{k^2\Xi^2}{8}\right)\left(\frac{k\alpha}{2}\right)}\end{displaymath} 
\begin{displaymath}{\approx}-\frac{\pi\alpha^2}{3}\end{displaymath}
\color{black} 
Term 3:
%\color{red}
\begin{equation}{\frac{2\pi\sqrt{2\pi}}{k^3\Xi}e^{-\left(\frac{\alpha^2}{2\Xi^2}\right)}\left[\operatorname{erf}\left(\frac{k\Xi}{2*\sqrt{2}}+i\frac{\alpha}{\sqrt{2}\Xi}\right)+\operatorname{erf}\left(\frac{k\Xi}{2*\sqrt{2}}-i\frac{\alpha}{\sqrt{2}\Xi}\right)\right]}\end{equation}
\begin{displaymath}{\approx\frac{\left(2\pi\right)^\frac{2}{3}}{\Xi{k}^3}e^{-\left(\frac{\alpha^2}{2\Xi^2}\right)}\left[2\operatorname{Re}\left(\operatorname{erf}\left(\frac{k\Xi}{2\sqrt{2}}+i\frac{\alpha}{\sqrt{2}\Xi}\right)\right)\right]}\end{displaymath} 
\begin{displaymath}{\approx\frac{\left(2\pi\right)^\frac{2}{3}}{\Xi{k}^3}e^{-\left(\frac{\alpha^2}{2\Xi^2}\right)}\left[2e^{\left(\frac{\alpha^2}{2\Xi^2}\right)}\frac{2}{\sqrt{\pi}}   \left(\frac{k\Xi}{2\sqrt{2}}   -\frac{2\left(\frac{\alpha^2}{2\Xi^2}\right)+1}{3}\left(\frac{k\Xi}{2\sqrt{2}}\right)^3\right)\right]}\end{displaymath}
\begin{displaymath}{\approx\frac{4\pi}{k^2}-\frac{\pi}{6}\left(\alpha^2-\Xi^2\right)}\end{displaymath}  
\color{black}
Adding the three terms gives:
\begin{displaymath}{\approx-\frac{4\pi}{k^2}+\frac{\pi\alpha^2}{2}+\frac{\pi\Xi^2}{6}-\frac{\pi\alpha^2}{3}+\frac{4\pi}{k^2}-\frac{\pi\alpha^2}{6}-\frac{\pi\Xi^2}{6}}\end{displaymath} 
\begin{displaymath}{\approx0}\end{displaymath} 
\[{}\]

\textbf{4. Integrals used to Compute Fourier Transforms of Weight Functions}
\begin{equation}{\int_{-1}^{1}{e^{-ikr\cos{\theta}}d{\cos{\theta}}}=\frac{2\sin(kr)}{kr}}\end{equation} 
\begin{equation}{\int_{-1}^{1}{e^{-ikr\cos{\theta}}\cos{\theta}{~}d{\cos{\theta}}}=-\frac{2i}{kr}\cos(kr)+\frac{2i}{k^2r^2}\sin(kr)}\end{equation} 
\[{}\]
\begin{equation}{\int_{-1}^{1}{e^{-ikr\cos{\theta}}\cos^2(\theta)~d{\cos{\theta}}}=\frac{2\sin(kr)}{kr}+\frac{4\cos(kr)}{k^2r^2}+\frac{4\sin(kr)}{k^3r^3}}\end{equation} 
\[{}\]
\[{}\]
\begin{equation}{\int_{-\infty}^{\infty}{e^{-\left(\frac{r-\frac{\alpha}{2}}{a}\right)^2}\cos(kr)d{r}}}\end{equation}
\begin{displaymath}{=a\int_{-\infty}^{\infty}{e^{-u^2}\left[\cos(kau)\cos(\frac{k\alpha}{2})-\sin(kau)\sin(\frac{k\alpha}{2})\right]d{u}}}\end{displaymath}  
\begin{equation}{=a\sqrt{\pi}e^{-\left(\frac{ka}{2}\right)^2}\cos(\frac{k\alpha}{2})}\end{equation} 
\[{}\]
\[{}\]
\begin{equation}{\int_{-\infty}^{\infty}{e^{-\left(\frac{r-\frac{\alpha}{2}}{a}\right)^2}r\cos(kr)d{r}}}\end{equation}
\begin{equation}{=\frac{a\alpha}{2}\cos\left(\frac{k\alpha}{2}\right)\int_{-\infty}^{\infty}{e^{-u^2}\cos(kau)d{u}} -a^2\sin\left(\frac{k\alpha}{2}\right)\int_{-\infty}^{\infty}{e^{-u^2}u\sin(kau)d{u}}}\end{equation}
\begin{equation}{=\frac{a\alpha\sqrt{\pi}}{2}e^{-\left(\frac{ka}{2}\right)^2}\cos\left(\frac{k\alpha}{2}\right)-\frac{a^3k\sqrt{\pi}}{2}e^{-\left(\frac{ka}{2}\right)^2}\sin\left(\frac{k\alpha}{2}\right)}\end{equation}
\[{}\]
\begin{equation}{\int_{-\infty}^{\infty}{e^{-\left(\frac{r-\frac{\alpha}{2}}{a}\right)^2}\sin(kr)d{r}}}\end{equation} 
\begin{displaymath}{=a\int_{-\infty}^{\infty}{e^{-u^2}\left[\sin(kau)\cos(\frac{k\alpha}{2})+\cos(kau)\sin(\frac{k\alpha}{2})\right]d{u}}}\end{displaymath}  
\begin{equation}{=a\sqrt{\pi}e^{-\left(\frac{ka}{2}\right)^2}\sin(\frac{k\alpha}{2})}\end{equation} 
\[{}\]
\[{}\]
\begin{equation}{\int_{-\infty}^{\infty}{e^{-\left(\frac{r-\frac{\alpha}{2}}{a}\right)^2}r\sin(kr)d{r}}}\end{equation} 
\begin{displaymath}{=a\int_{-\infty}^{\infty}{e^{-u^2}(au+\frac{\alpha}{2})\sin(k(au+\frac{\alpha}{2}))d{u}}}\end{displaymath} 
\begin{displaymath}{=a^2\cos(\frac{k\alpha}{2})\int_{-\infty}^{\infty}{e^{-u^2}u\sin(kau)d{u} +\frac{a\alpha}{2}\sin(\frac{k\alpha}{2})\int_{-\infty}^{\infty}e^{-u^2}\cos(kau)d{u}}}\end{displaymath} 
\begin{equation}{=\frac{ka^3}{2}\sqrt{\pi}e^{-\left(\frac{ka}{2}\right)^2}\cos(\frac{k\alpha}{2})+\frac{a\alpha}{2}\sqrt{\pi}e^{-\left(\frac{ka}{2}\right)^2}\sin(\frac{k\alpha}{2})}\end{equation} 
\[{}\]
\begin{equation}{\int_{-\infty}^{\infty}{e^{-\left(\frac{r-\frac{\alpha}{2}}{a}\right)^2}\frac{\sin(kr)}{r}d{r}}}\end{equation} 
%Noting that \begin{equation}{\frac{\sin(kr)}{r}=\int_{0}^{k}{\cos(kr)d{k}}}\end{equation} 
%this becomes
\begin{displaymath}{=\int_{-\infty}^{\infty}{e^{-\left(\frac{r-\frac{\alpha}{2}}{a}\right)^2}\left(\int_{0}^{k}{\cos(kr)d{k}}\right)d{r}}}\end{displaymath} 
\begin{displaymath}{=\int_{0}^{k}\int_{-\infty}^{\infty}{e^{-\left(\frac{r-\frac{\alpha}{2}}{a}\right)^2}{\cos(kr)d{r}}d{k}}}\end{displaymath} 
\begin{displaymath}{=a\sqrt{\pi}\int_{0}^{k}e^{-\left(\frac{ka}{2}\right)^2}\cos\left(\frac{k\alpha}{2}\right)d{k} }\end{displaymath} 
%\begin{displaymath}{=a\sqrt{\pi}\int_{0}^{k}e^{-\left(\frac{ka}{2}\right)^2}\left(\frac{e^{i\frac{k\alpha}{2}}+e^{-i\frac{k\alpha}{2}}}{2}\right)d{k} }\end{displaymath} 
%\begin{displaymath}{=\frac{a\sqrt{\pi}}{2}\left[\int_{0}^{k}e^{-\left(\frac{ka}{2}\right)^2 + i\frac{k\alpha}{2}}d{k}+\int_{0}^{k}e^{-\left(\frac{ka}{2}\right)^2-i\frac{k\alpha}{2}}d{k}\right]}\end{displaymath} 
\begin{displaymath}{=\frac{a\sqrt{\pi}}{2}\left[\int_{-k}^{k}e^{-\left(\frac{ka}{2}\right)^2 + i\frac{k\alpha}{2}}d{k}\right]}\end{displaymath} 
%\begin{displaymath}{=\frac{a\sqrt{\pi}}{2}e^{-\left(\frac{\alpha}{2a}\right)^2}\left[\int_{-k}^{k}e^{-\left(\frac{ka}{2} + i\frac{\alpha}{2a}\right)^2}d{k}\right]}\end{displaymath} 
\begin{displaymath}{=\sqrt{\pi}e^{-\left(\frac{\alpha}{2a}\right)^2}\int_{-\frac{ka}{2}+i\frac{\alpha}{2a}}^{\frac{ka}{2}+i\frac{\alpha}{2a}}e^{-u^2}d{u}}\end{displaymath} 
\begin{equation}{=\frac{\pi}{2}e^{-\left(\frac{\alpha}{2a}\right)^2}\left[\operatorname{erf}\left(\frac{ka}{2}+i\frac{\alpha}{2a}\right)+\operatorname{erf}\left(\frac{ka}{2}-i\frac{\alpha}{2a}\right)\right]}\end{equation} 

\[{}\]
\[{}\]
\[{}\]
%\[{}\]

\textbf{5. Deriviation of Tensor Density $n_{m2ij}(\vec{r})$}
\begin{equation}{\overleftrightarrow{n}_{m2}(\vec{r})=\int_{allspace}n(\vec{r'})\overleftrightarrow{w}_{m2}(\vec{r}-\vec{r'})d{\vec{r'}}}\end{equation} 
\begin{equation}{n_{m2ij}(\vec{r})=\int_{allspace}{n(\vec{r}')w_{m2ij}(\vec{r}-\vec{r}'){}d{\vec{r}'}}}\end{equation}
Using the definintion of the delta function
\begin{equation}{\delta}((\vec{r}-\vec{r}')-\vec{r}'')={ \frac{1}{\left(2\pi\right)^3}\int e^{i\vec k\cdot ((\vec r-\vec r')-\vec{r}'')}d\vec{k}}\end{equation} 
$w_{m2ij}(\vec{r}-\vec{r}')$ can be expressed as
\begin{align}
    w_{m2ij}(\vec{r}-\vec{r}') &= \int{\delta((\vec{r}-\vec{r}')-\vec{r}'')w_{m2ij}(\vec{r}''){}d{\vec{r}''}} \\
    &= \int{\left(\frac{1}{\left(2\pi\right)^3}\int e^{i\vec k\cdot((\vec r-\vec r')-\vec{r}'')}d\vec{k}\right){~}w_{m2ij}(\vec{r}''){}d{\vec{r}''}} \\
    &= \frac{1}{\left(2\pi\right)^3}\int{\left(\int w_{m2ij}(\vec{r}'')e^{-i\vec k\cdot\vec{r}''}d\vec{r}''\right)e^{i\vec k\cdot(\vec r-\vec r')}{~}{}d{\vec{k}}} \\
    &= \frac{1}{\left(2\pi\right)^3}\int{w_{m2ij}(\vec k)e^{i\vec k\cdot(\vec r-\vec r')}{~}{}d{\vec{k}}} 
  \end{align} 
Putting this into the expression for $n_{m2ij}(\vec{r})$ gives 
\begin{align}
%    n_{m2ij}(\vec r) &= \int n(\vec r') w_{m2ij}(\vec r- \vec r')d\vec r' \\
%    w_{m2ij}(\vec r) &= \frac1{2\pi}\int d\vec k w_{m2ij}(\vec k)e^{i\vec k\cdot \vec r}\\
%    n_{m2ij}(\vec{r})=\int_{allspace{n(\vec{r}')w_{m2ij}(\vec{r}-\vec{r}'){}d{\vec{r}'}}\\
    n_{m2ij}(\vec r) &= \int n(\vec r') \left(\frac{1}{\left(2\pi\right)^3}\int w_{m2ij}(\vec k)e^{i\vec k\cdot (\vec r-\vec r')}d\vec{k}\right)d\vec r' \\
    &= \int d\vec r' \left(\frac{1}{\left(2\pi\right)^3}\int d\vec k' n(\vec k')e^{i\vec k'\cdot \vec r'}\right) \left(\frac{1}{\left(2\pi\right)^3}\int d\vec k w_{m2ij}(\vec k)e^{i\vec k\cdot (\vec r-\vec r')}\right) \\
    &=  \frac{1}{\left(2\pi\right)^3}\int d\vec k' n(\vec k') \int d\vec k w_{m2ij}(\vec k)
    e^{i\vec k\cdot \vec r}\left(\frac{1}{\left(2\pi\right)^3}\int d\vec r'e^{i(\vec k'-\vec k)\cdot \vec r'}\right)
    \\
   &= \frac{1}{\left(2\pi\right)^3}\int d\vec k' n(\vec k') \int d\vec k w_{m2ij}(\vec k)e^{i\vec k\cdot \vec r}\delta(\vec k'-\vec k) \\
    &= \frac{1}{\left(2\pi\right)^3}\int d\vec k \left(\int d\vec k' n(\vec k')\delta(\vec k'-\vec k)\right) w_{m2ij}(\vec k)e^{i\vec k\cdot \vec r}
    \\
    &= \frac{1}{\left(2\pi\right)^3}\int d\vec k\, n(\vec k) w_{m2ij}(\vec k)e^{i\vec k\cdot \vec r}
  \end{align} 
\[{}\]
\[{}\]
%\[{}\]

\textbf{6. Deriviation of the Tensor version of $\Phi_3$ }
\[{}\]
\begin{equation}{\Phi_3=\frac{12\pi^2R^6}{(1-\eta)^2}\left[\overrightarrow{\operatorname{V}}\cdot\overleftrightarrow{\operatorname{T}}\cdot\overrightarrow{\operatorname{V}}-n\overrightarrow{\operatorname{V}}\cdot\overrightarrow{\operatorname{V}}-\operatorname{Tr}\left(\overleftrightarrow{\operatorname{T}}^3\right)+n\operatorname{Tr}\left(\overleftrightarrow{\operatorname{T}}^2\right)\right]}\end{equation}
\[{}\]
\begin{equation}{\overleftrightarrow{\operatorname{T}}=\frac{{\overleftrightarrow{n}_{m2}}+\frac{n_2}{3}\hat{\operatorname{I}}}{4\pi{R}^2}}\end{equation}
\[{}\]
\begin{equation}{\overrightarrow{\operatorname{V}}=\frac{\vec{n}_2}{4\pi{R}^2}}\end{equation}
\[{}\]
\begin{displaymath}{\Phi_3=\frac{12\pi^2R^6}{(1-\eta)^2}\left(\frac{1}{4\pi{R}^2}\right)^3\left[\vec{n}_{2v}\cdot\left({\overleftrightarrow{n}_{m2}}+\frac{n_2}{3}\hat{\operatorname{I}}\right)\cdot\vec{n}_{2v}-n_2|\vec{n}_{2v}|^2-\operatorname{Tr}\left(({\overleftrightarrow{n}_{m2}}+\frac{n_2}{3}\hat{\operatorname{I}})^3\right)\color{white}\right]\color{black}}\end{displaymath}
\begin{equation}{\color{white}\left[\color{black}+n_2\operatorname{Tr}\left(({\overleftrightarrow{n}_{m2}}+\frac{n_2}{3}\hat{\operatorname{I}})^2\right)\right]}\end{equation}
\[{}\]
\begin{displaymath}{\Phi_3=\frac{3}{16\pi(1-\eta)^2}\left[\vec{n}_{v2}\cdot{\overleftrightarrow{n}_{m2}}\cdot{\vec{n}_{v2}}-\frac{2}{3}n_2\vec{n}_{v2}\cdot\vec{n}_{v2}-\operatorname{Tr}\left(({\overleftrightarrow{n}_{m2}}+\frac{n_2}{3}\hat{\operatorname{I}})^3\right)\color{white}\right]\color{black}}\end{displaymath}\begin{equation}{\color{white}\left[\color{black}+n_2\operatorname{Tr}\left((\overleftrightarrow{n}_{m2}+\frac{n_2}{3}\hat{\operatorname{I}})^2\right)\right]}\end{equation}
\[{}\]
\begin{displaymath}{\Phi_3=\frac{3}{16\pi(1-\eta)^2}\left(\vec{n}_{v2}\cdot{\overleftrightarrow{n}_{m2}}\cdot{\vec{n}_{v2}}-\frac{2}{3}n_2|\vec{n}_{v2}|^2-\operatorname{Tr}({\overleftrightarrow{n}_{m2}^3})
-n_2\operatorname{Tr}(\overleftrightarrow{n}_{m2}^2)\color{white}\right)\color{black}}\end{displaymath} \begin{equation}{\color{white}\left(\color{black}-\frac{1}{3}{n_2}^2\operatorname{Tr}(\overleftrightarrow{n}_{m2})-\frac{1}{9}{n_2}^3+n_2\operatorname{Tr}(\overleftrightarrow{n}_{m2}^2)+\frac{1}{3}{n_2}^3+\frac{2}{3}n^2_2\operatorname{Tr}(\overleftrightarrow{n}_{m2})\right)}\end{equation} 
\[{}\]
\begin{equation}{\Phi_3=\frac{3}{16\pi(1-\eta)^2}\left(\frac{2}{9}{n_2}^3-\frac{2}{3}n_2|\vec{n}_{v2}|^2+\vec{n}_{v2}\cdot{\overleftrightarrow{n}_{m2}}\cdot{\vec{n}_{v2}}-\operatorname{Tr}({\overleftrightarrow{n}_{m2}}^3)+\frac{1}{3}n^2_2\operatorname{Tr}(\overleftrightarrow{n}_{m2})\right)}\end{equation} 
\[{}\]
\begin{equation}{\operatorname{Tr}(\overleftrightarrow{n}_{m2})=0}\end{equation} 
\[{}\]
\begin{equation}{\Phi_3=\frac{{n_2}^3-3n_2|\vec{n}_{v2}|^2+\frac{9}{2}(\vec{n}_{v2}\cdot{\overleftrightarrow{n}_{m2}}\cdot{\vec{n}_{v2}})-\frac{9}{2}\operatorname{Tr}({\overleftrightarrow{n}^3_{m2}})}{24\pi(1-\eta)^2}}\end{equation} 
\[{}\]
\begin{equation}{\eta=n_3}\end{equation} 
\[{}\]
\begin{equation}{\Phi_3=\frac{{n_2}^3-3n_2\vec{n}_{v2}\cdot\vec{n}_{v2}+\frac{9}{2}[\vec{n}_{v2}\cdot{\overleftrightarrow{n}_{m2}}\cdot{\vec{n}_{v2}}-\operatorname{Tr}({\overleftrightarrow{n}^3_{m2}})]}{24\pi(1-n_3)^2}}\end{equation} 
\[{}\]
\[{}\]
%\[{}\]

\textbf{7. Second Virial Coefficient $B_{2}$ for the Error Function Potential $V_{erf}$:}
\begin{equation}B_2=-\frac{1}{2}\int_{allspace}f(\vec{r})d\vec r\end{equation}
The mayer function for the Error Function potential $V_{erf}$ is
\begin{equation}f(r)=\frac{1}{2}\left[\operatorname{erf}\left(\frac{r-\alpha}{\Xi}\right)-1\right]\end{equation} from which it can be seen that $f(r)$ is approximately 0 for $r>r_{max}$ where $r_{max}$ is some maximum value of $r$.
\begin{align}
 B_2 &= -\frac{1}{2}\int_0^{\pi}\int_0^{2\pi}\int_0^\infty\frac{1}{2}\left[\operatorname{erf}\left(\frac{r-\alpha}{\Xi}\right)-1\right]r^2\sin{\theta}d{\theta}d{\phi}dr \\
     &= -\frac{1}{2}4\pi\frac{1}{2}\int_{0}^{r_{max}}\left[\operatorname{erf}\left(\frac{r-\alpha}{\Xi}\right)-1\right]r^2dr \\
     &= -\pi\int_{0}^{r_{max}}\operatorname{erf}\left(\frac{r-\alpha}{\Xi}\right)r^2dr {~~}+{~~} \pi\int_0^{r_{max}}r^2dr   \\
     &= -\pi\int_{0}^{r_{max}}\operatorname{erf}\left(\frac{r-\alpha}{\Xi}\right)r^2dr {~~}+{~~} \frac{\pi}{3}r_{max}^3   
\end{align}

\begin{align}
  B_2 &= \left(-\frac{\pi}{3}\alpha^3-\frac{\pi}{2}\alpha\Xi^2+\frac{\pi}{3}r_{max}^3\right)\operatorname{erf}\left(\frac{\alpha-r_{max}}{\Xi}\right) \\
      &+ \left(\frac{\pi}{3}\alpha^3+\frac{\pi}{2}\alpha\Xi^2\right)\operatorname{erf}\left(\frac{\alpha}{\Xi}\right) \\
      &+ \frac{\sqrt{\pi}}{3}\left(-\Xi\alpha^2-\Xi\alpha r_{max}-\Xi^3-\Xi r_{max}^2\right)e^{-{\left(\frac{\alpha-r_{max}}{\Xi}\right)^2}} \\
      &+ \frac{\sqrt{\pi}}{3}\left(\Xi\alpha^2+\Xi^3\right)e^{-\left(\frac{\alpha}{\Xi}\right)^2}+\frac{\pi}{3}r_{max}^3 
\end{align}
Letting $r_{max}\rightarrow\infty$ this becomes
\begin{align}
  B_2 &= \left(-\frac{\pi}{3}\alpha^3-\frac{\pi}{2}\alpha\Xi^2+\frac{\pi}{3}\left(\infty\right)^3\right)\left(-1\right) \\
      &+ \left(\frac{\pi}{3}\alpha^3+\frac{\pi}{2}\alpha\Xi^2\right)\operatorname{erf}\left(\frac{\alpha}{\Xi}\right) \\
      &+ \frac{\sqrt{\pi}}{3}\left(-\Xi\alpha^2-\Xi\alpha\left(\infty\right)-\Xi^3-\Xi r_{max}^2\right)\left(0\right) \\
      &+ \frac{\sqrt{\pi}}{3}\left(\Xi\alpha^2+\Xi^3\right)e^{-\left(\frac{\alpha}{\Xi}\right)^2}+\frac{\pi}{3}\left(\infty\right)^3 
\end{align}
\begin{equation}B_2 = \frac{\pi}{3}\Xi^3\left[\left(\frac{\alpha^3}{\Xi^3}+\frac{3\alpha}{2\Xi}\right)\left(1+\operatorname{erf}\left(\frac{\alpha}{\Xi}\right)\right)+\frac{1}{\sqrt{\pi}}\left(\frac{\alpha^2}{\Xi^2}+1\right)e^{-\left(\frac{\alpha}{\Xi}\right)^2}\right]\end{equation}

\[{}\]
%\[{}\]

\textbf{8. Third Virial Coefficient $B_{3}$:}

\begin{equation}\label{B3}B_3=-\frac{1}{3}\int_{allspace}\int_{allspace}f(\vec{r})f(\vec{r'})f(\vec{r}-\vec{r'})d\vec rd\vec r'\end{equation}

First find
\begin{equation}f(\vec{r}-\vec{r'})=\int_{allspace}f(\vec{r''})\delta(\vec{r}''-(\vec{r}-\vec{r}'))d\vec r'' \end{equation}

where
\begin{equation}{\delta}(\vec{r}''-(\vec{r}-\vec{r}'))={ \frac{1}{\left(2\pi\right)^3}\int_{allspace} e^{i\vec k\cdot (\vec{r}''-(\vec r-\vec r'))}d\vec{k}}\end{equation} 

\begin{equation}f(\vec{r}-\vec{r'})=\int{f(\vec{r''})\left( \frac{1}{\left(2\pi\right)^3}\int e^{i\vec k\cdot (\vec{r}''-(\vec r-\vec r'))}d\vec{k} \right) d\vec r''} \end{equation}

\begin{equation}f(\vec{r}-\vec{r'})=\int{ e^{-i\vec k\cdot (\vec r-\vec r')}\left(\frac{1}{\left(2\pi\right)^3}\int{f(\vec{r''}) e^{i\vec k\cdot (\vec{r}'')}d\vec{r''}} \right) d\vec k} \end{equation}

\begin{equation}f(\vec{r}-\vec{r'})=\int{ e^{-i\vec k\cdot \vec r}e^{i\vec k\cdot \vec r}\widetilde{f}(-\vec k) d\vec k} \end{equation}

Putting this result into Eq~\ref{B3} gives
\begin{equation}B_3=-\frac{1}{3}\int{\int{f(\vec{r})f(\vec{r'})\left(\int{ e^{-i\vec k\cdot \vec r}e^{i\vec k\cdot \vec r}\widetilde{f}(-\vec k) d\vec k}\right)d\vec rd\vec r'}}\end{equation}

\begin{equation}B_3=-\frac{1}{3}\int{\widetilde{f}(-\vec k)\left(\int{f(\vec{r})e^{-i\vec k\cdot \vec r}}d\vec r\int{f(\vec{r'})e^{i\vec k\cdot \vec r'} d\vec r'}\right)d\vec k}\end{equation}

\begin{equation}B_3=-\frac{1}{3}\int{\widetilde{f}(-\vec k)\widetilde{f}(\vec k)\widetilde{f}(-\vec k)d\vec k}\end{equation}

\begin{equation}B_3=-\frac{1}{3}\int_0^{2\pi}\int_0^{\pi}\int_0^{\infty}{|\widetilde{f}(\vec k)|^3d\vec k}\end{equation}

Since the Mayer function does not depend on $\phi$ and $\theta$, this reduced to
\begin{equation}B_3=-\frac{4\pi}{3}\int{|\widetilde{f}(\vec k)|^3k^2dk}\end{equation}

The Mayer function for the Error Function potential $V_{erf}$, and its Fourier Transform are
\begin{equation}f_{erf}(r)=\frac{1}{2}\left[\operatorname{erf}\left(\frac{r-\alpha}{\Xi}\right)-1\right]\end{equation} 

\begin{equation}\widetilde{f}_{erf}(k)=-\frac{4\pi}{k^3}e^{-\frac{k^2\Xi^2}{4}}\left[\left(1+\frac{k^2\Xi^2}{2}\right)\sin(k\alpha)-k\alpha\cos(k\alpha)\right]\end{equation} 

The Mayer function for the WCA potential $V_{WCA}$, and its Fourier Transform are

%\begin{equation}f(r)=e^{-\frac{1}{k_BT}\left[4\epsilon\left(\left(\frac{\sigma}{r}\right)^{12}\left(-\frac{\sigma}{r}\right)^6\right)+\epsilon\right]}-1\end{equation} 

\begin{equation}f_{WCA}(r)=\operatorname{exp}\left(-\frac{1}{k_BT}\left[4\epsilon\left(\sigma^{12}r^{-12}-\sigma^{6}r^{-6}\right)+\epsilon\right]\color{white}\frac{1}{1}\color{black}\right)-1\end{equation} 

\begin{equation}\widetilde{f}_{WCA}(k)=\frac{4\pi}{k}\int_0^{\infty}{\left(e^{-\frac{1}{k_BT}\left[4\epsilon\left(\sigma^{12}r^{-12}-\sigma^{6}r^{-6}\right)+\epsilon\right]}-1\right) r\sin(kr)dr}\end{equation} 

%\begin{equation}B_3=-\frac{4\pi}{3}\int{|\widetilde{f}(-\vec k)|^3k^2dk}\end{equation}


\[{}\]
%\[{}\]

\noindent \textbf{9. Derivation of the Virial Equation}
The second virial coefficient can be derived from Euler's equation for the internal energy of a system.
\begin{equation}PV=TS-(U-\mu{N})\end{equation}
\begin{equation}\label{Euler_rearranged}\frac{PV}{k_BT}=\frac{S}{k_B}-\frac{1}{k_BT}(U-\mu{N})\end{equation}
Setting $\beta=\frac{1}{k_BT}$, and putting in the relations from statistical thermophysics which views the 
thermodynamic quantities as ensemble averages
\begin{equation}S=-k_B\sum_{i=0}^\infty{P_i\ln{P_i}}\end{equation}
\begin{equation}U=<E>=\sum_{i=0}^\infty{P_iE_i}\end{equation}
\begin{equation}N=<n>=\sum_{i=0}^\infty{P_in_i}\end{equation}
%\begin{equation}\beta=\frac{1}{k_BT}\end{equation}
%\begin{equation}\frac{PV}{k_BT}=-\sum{P_i\ln{P_i}}-\frac{1}{k_BT}(\sum{P_iE_i}-\mu\sum{P_iN})\end{equation}
Equation~\ref{Euler_rearranged} then becomes
\begin{equation}\frac{PV}{k_BT}=-\sum{P_i\ln{P_i}}-\sum{P_i\beta(E_i-\mu
{n_i}})\end{equation}
\begin{equation}\frac{PV}{k_BT}=-\sum{P_i}\left[\ln{P_i}+\beta(E_i-\mu
{n_i})\right]\end{equation}
For a Grand Canonical Ensemble the probablity of a state given by
\begin{equation}P_{n,s(n)}=\frac{e^{-\beta(E_{s(n)}-\mu{n})}}{\sum_{n=0}^\infty\sum_{s(n)}e^{-\beta(E_{s(n)}-\mu{n})}}\end{equation}
results in 
\begin{equation}\frac{PV}{k_BT}=-\sum_{n=0}^\infty\sum_{s(n)}P_{n,s(n)}\left[\ln{\left(\frac{ e^{-\beta(E_{s(n)}-\mu{n})}}{\sum_{n=0}^\infty\sum_{s(n)}e^{-\beta(E_{s(n)}-\mu{n})}}\right)}+\beta(E_{s(n)}-\mu{n})\right]\end{equation}

\begin{equation}\frac{PV}{k_BT}=-\sum_{n=0}^\infty\sum_{s(n)}P_{n,s(n)}\left[-\beta(E_{s(n)}-\mu{n})-\ln{\left(\sum_{n=0}^\infty\sum_{s(n)}e^{-\beta(E_{s(n)}-\mu{n})}\right)}+\beta(E_{s(n)}-\mu{n})\right]\end{equation}

\begin{equation}\frac{PV}{k_BT}=-\sum_{n=0}^\infty\sum_{s(n)}\left(\frac{e^{-\beta(E_{s(n)}-\mu{n})}}{\sum_{n=0}^\infty\sum_{s(n)}e^{-\beta(E_{s(n)}-\mu{n})}} \left[-\ln{\left(\sum_{n=0}^\infty\sum_{s(n)}e^{-\beta(E_{s(n)}-\mu{n})}\right)}\right]\right)\end{equation}

\begin{equation}\frac{PV}{k_BT}=\left[\ln{\left(\sum_{n=0}^\infty\sum_{s(n)}e^{-\beta(E_{s(n)}-\mu{n})}\right)}\right]\left(\frac{\sum_{n=0}^\infty\sum_{s(n)}e^{-\beta(E_{s(n)}-\mu{n})}}{\sum_{n=0}^\infty\sum_{s(n)}e^{-\beta(E_{s(n)}-\mu{n})}}\right) \end{equation}

\begin{equation}\frac{PV}{k_BT}=\ln{\left(\sum_{n=0}^\infty\sum_{s(n)}e^{-\beta(E_{s(n)}-\mu{n})}\right)} \end{equation}

\begin{equation}\frac{PV}{k_BT}=\ln{\left(\sum_{n=0}^\infty \left(e^{\beta\mu}\right)^n\sum_{s(n)}e^{-\beta{E}_{s(n)}}\right)} \end{equation}

\begin{equation}\frac{PV}{k_BT}=\ln{\left(\sum_{n=0}^\infty \lambda^nZ_n\right)} \end{equation}

\begin{equation}\frac{PV}{k_BT}=\ln{\left(\sum_{n=0}^\infty \lambda^nZ_{on}Q_n\right)} \end{equation}
where $Q_n$ is the configuration integral.
\color{red}For a system of N particles only, the sum collapses to one term, that for n=N, which gives \color{black}
\begin{equation}\label{PV/kBT_reduces_for_N}\frac{PV}{k_BT}=\ln{\left(\lambda^NZ_{oN}Q_N\right)}
%=\ln\lambda^n + \ln{Z}_{on}+\ln{Q}_n
=\beta\mu{N} + \ln{Z}_{oN} +\ln{Q}_N\end{equation}
\begin{equation}\label{PV/kBT_reduces_for_N}\frac{PV}{k_BT}=\beta\mu_{ideal}{N} + \ln{Z}_{oN} + \beta\mu_{excess}{N} + \ln{Q}_N\end{equation}
Noting that $Q_N=1$ for an ideal gas since $V(r)=0$ for an ideal gas
\begin{equation}\frac{P_{ideal}V}{k_BT}=\beta\mu_{ideal}{N} + \ln{Z}_{oN}=N\end{equation}
%\begin{equation}P_{ideal}=\mu{N} + k_BT\ln{Z}_{oN}\end{equation}
\begin{equation}\frac{P_{excess}}{k_BT}= \beta\mu_{excess}{N} + \ln{Q}_N\end{equation}
and Equation~\ref{PV/kBT_reduces_for_N} becomes
\begin{equation}\frac{PV}{k_BT}=N + \beta\mu_{excess}{N} + \ln{Q}_N\end{equation}
\begin{equation}\frac{PV}{k_BT}=N + \beta\frac{\partial{F}_{excess}}{\partial{N}}{N} - \frac{F_{excess}}{k_BT}\end{equation}
Plugging in the expression for $F_{excess}$ in terms of the mayer function for low density fluids
\begin{equation}F_{excess}=k_BT\frac{N^2}{V}\left(-\frac{1}{2}\int{f(r)d\vec{r}}\right)\end{equation}
gives
\begin{equation}\frac{PV}{k_BT}=N + \left(-\frac{N}{V}\int{f(r)d\vec{r}}\right){N} + \frac{N^2}{V}\frac{1}{2}\int{f(r)d\vec{r}}\end{equation}
%\begin{equation}\frac{PV}{k_BT}=N - \frac{N^2}{V}\frac{1}{2}\int{f(r)d\vec{r}}\end{equation}
\begin{equation}\frac{PV}{k_BT}=N\left(1-\frac{N}{V}\frac{1}{2}\int{f(r)d\vec{r}}\right)\end{equation}
\begin{equation}\frac{PV}{k_BT}=N\left(1+\frac{N}{V}B_2(T)\right) \end{equation}
where $B_2(T)$ is given by
\begin{equation}B_2=-\frac{1}{2}\int_{Volume}f(r)d\vec{r} \end{equation}

\color{red}This result was for low densities where higher terms in the expression for the excess free energy were neglected. The full virial equation is given by\color{black}
\begin{equation}\frac{PV}{k_BT}=N\left(1+\frac{N}{V}B_2(T)+\left(\frac{N}{V}\right)^2B_3(T)+ \cdot\cdot\cdot\right) \end{equation}

\[{}\]
%\[{}\]

\noindent \textbf{10. Derivation of the Excess Free Energy from Scaled Particle Theory}
The excess chemical potential of a homogeneous fluid corresdponds to the reversible work of inserting a sphere of radius R 
into a fluid consisting of hard spheres of radius R. 
\begin{displaymath}\mu_{ex}=W_{rev}(R){~~~~}\text{work to insert a hard sphere of radius R}\end{displaymath}
This result is obtained by considering the difference in the excess Helmholtz free energy when one atom is added to the system. 
It was shown by Widom to be related to the ensemble average of the Boltzman factor for the potential energy difference $V_{diff}$ 
of a system with N atoms and one with N+1 atoms. The resulting equation is known as Widom's Insertion Formula.
%\begin{displaymath}\mu_{ex}=\frac{\partial{F_{ex}}}{\partial{N}}\bigg|_{T,V}{~~~~~~}\end{displaymath}
\begin{equation}\mu_{ex}=\frac{F_{ex}(N+1)-F_{ex}(N)}{\Delta{N}}\bigg|_{T,V}\end{equation}
\begin{equation}\label{widom}{~}=-k_BT\ln\left(<e^{-\beta{V_{diff}}}>\right)\end{equation}
%Equation~\ref{widom} is called Widom's Insertion Formula. 
\color{red}For a system... $<e^{-\beta{V_{diff}}}> \rightarrow e^{-\beta{V_{diff}}}$, and Equation~\ref{widom} becomes 
\begin{equation}\mu_{ex}=-k_BT\ln\left(e^{-\beta{V_{diff}}}\right)=-k_BT(-\beta{V}_{diff})=V_{diff}=W_{rev}\end{equation}\color{black}
%In a randomly positioned (I think it has to be homogeneous!)
\indent In a collection of hard spheres making up a hard sphere fluid, no energy would be required to place a hard sphere at points 
surrounded by an empty space the size of a hard sphere. But a hard sphere cannot be placed at points where a hard sphere would 
overlap another hard sphere; the energy required to do that would be infinite. Thus, for a Hard Sphere fluid, Equation~\ref{widom} becomes
\begin{equation}\mu_{ex}=-k_BT\ln\left(P_{NoOverlap}e^{-\beta{V}_{NoOverlap}}+P_{Overlap}e^{-\beta{V}_{Overlap}}\right)\end{equation}
\begin{equation}=-k_BT\ln\left(P_{NoOverlap}e^{-\beta{0}}+P_{Overlap}e^{-\beta\infty}\right)\end{equation}
\begin{equation}=-k_BT\ln\left(P_{NoOverlap}(1)+P_{Overlap}(0)\right)\end{equation}
\begin{equation}=-k_BT\ln\left(P_{NoOverlap}\right)\end{equation}
where $P_{NoOverlap}$ is the probablity of positions existing in the hard fluid that are surrounded by an empty space the size of a hard sphere.

\begin{figure}[h!]
    \centering
    % FIXME \includegraphics[height=3cm]{Probability-of-overlap.pdf}
    \caption{No energy is needed to place a sphere at a position where there is no overlap such 
             as shown by the black dot at the center of the dotted region near the top. 
             But it is not possible to place a sphere at a position where an overlap occurs.}
    \label{fig:p_overlap}
  \end{figure}

For a homogeneous system, the probablity of positions existing in the hard fluid that are surrounded 
by an empty space the size of a hard sphere, $P_{NoOverlap}$, can be found by considering 
that $P_{NoOverlap}~=1-~P_{Overlap}$ where $P_{Overlap}$ is simply the number density times the volume of one sphere. 
%\color{red}Thus the chemical potential for a fluid with cavities having a radius equal to the radius R of a hard sphere is \color{black}
\color{red}Thus, considering cavities having a radius Ro equal to the radius R of a hard sphere, the chemical potential is found to be \color{black}
%\begin{equation}P_{Overlap}=nV_{sphere}\right)\end{equation}
\begin{equation}\mu_{ex}(Ro=R)=-k_BT\ln\left(1-nV_{sphere}\right)\end{equation}

%\begin{equation}\mu_{ex}(R)=-k_BT\ln\left(1-nV_{sphere}\right)\end{equation}

Using this result, scaled particle theory seeks a more general expression of the excess chemical potential of 
a hard sphere fluid by considering the work of inserting a hard sphere into a cavity of radius $R_o$ 
where the size of $R_o$ falls into three different cases: $R_o >> R$, $R_o > R$, and $R_o < R$. 

\begin{equation}\label{3_parts_to_mu_ex}{\mu_{ex}(R_o)=\left\{ \begin{array}{rc} P\frac{4\pi}{3}R_o^3 & R_o >> R 
\\ P\frac{4\pi}{3}R_o^3 + c_2R_o^2 + c_1R_o^1 + c_0R_o^0  & R_o > R \\ -k_BT(\ln(1-n\frac{4\pi}{3}(R_o+R)^3)) & R_o < R\end{array}\right.}\end{equation}

%The first function, valid for $R_o >> R$, yields the dominant term to which the second function, valid for $R_o > R$,  
%must tend toward in the limit that $R_o$ goes to infinity. By expanding the logrythmic term (ln(1-x)= ...) 
%in the last function, valid for $R_o < R$, the function can be expressed in terms of powers of $R_o$. 

%The first and last functions in Equation~\ref{3_parts_to_mu_ex} are derived from theory, and since they 
The first and last functions in Equation~\ref{3_parts_to_mu_ex} can be expressed in terms of powers of $R_o$, 
and so it is assumed that the middle function, which is valid for values of $R_o$ in between the two extremes, 
can also be expressed in terms of powers of $R_o$.  
The coeficients $c_2$, $c_1$ and $c_0$ in the middle function can be determnined be considering that this function (valid for $R_o > R$) 
must be continuous with the last function (valid for $R_o < R$) along with their first and second derivatives. 
The middle function must also tend toward the first function in the limit that $R_o$ goes to infinity. The chemical potential then becomes
\begin{equation}\mu_{ex}=-k_BT\left(\ln(1-\eta) + \frac{6\eta}{1-\eta}+\frac{9\eta^2}{2(1-\eta)^2}+\frac{PV\eta}{k_BTN}\right)\end{equation}
where $\eta$ is the packing fraction given by
\begin{displaymath}\eta = \frac{\mbox{Volume occupied by the spheres}}{\mbox{Total Volume}}=\frac{N\frac{4}{3}\pi{R}^3}{V}\end{displaymath} 

%The last two functions, valid for $R_o > R$, and $R_o < R$, and their first and second derivatives must be continuous at $R_o=R$. 
%From these considerations, factors $c_2$, $c_1$ and $c_0$ can be determnined, and using the relation 
%...ADD!...P, an expression is found for the reversible work done when $R_o=R$ which is equal to the excess chemical potential. 
%From the expression for pressure given by

The pressure can be obtained from \color{red}
\begin{equation}{\frac{\partial{P}}{\partial{N}}\bigg|_{T,V}V=N\frac{\partial\mu}{\partial{N}}\bigg|_{T,V}}\end{equation} \color{black}
%The excess Hemlholtz free energy is then found from the relation
and the Helmholz free energy from
\begin{displaymath}\mu_{ex}=\frac{\partial{F_{ex}}}{\partial{N}}\bigg|_{T,V}{~~~~~~}\text{or}{~~~~~~}F_{ex}=\int{\mu_{ex}dN}\end{displaymath}
which gives
%\begin{displaymath}F_{ex}=\end{displaymath}
\begin{equation}F_{excess}=k_BTN\left(-ln(1-\eta)+\frac{3\eta}{1-\eta}+\frac{3{\eta}^2}{2(1-\eta)^2}\right)\end{equation} 


\cite{fake}

\bibliography{thesis}
\bibliographystyle{plain}

\end{document}



