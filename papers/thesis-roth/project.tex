\documentclass[letterpaper,twocolumn,amsmath,amssymb,prb]{revtex4-1}
\usepackage{graphicx}% Include figure files
\usepackage{dcolumn}% Align table columns on decimal point
\usepackage{bm}% bold math
\usepackage{color}

%%%%%%%%%%%%%%%%%%%%%%%%%%%%%%%%%%%%%%%%%%%%%%%%%%%%%%%%%%%%
%% Definitions

% 1/k_B/T
\newcommand{\kT}{\ensuremath{k_BT}}

% vector
\newcommand{\rr}{\ensuremath{\mathbf{r}}}

% special values for n
\newcommand{\npart}{\ensuremath{n_{particular}}}
\newcommand{\nliq}{\ensuremath{n_{liquid}}}
\newcommand{\nvap}{\ensuremath{n_{vapor}}}

% Free energy terms
\newcommand{\fid}{\ensuremath{f_\text{ID}(T,n)}}
\newcommand{\fhs}{\ensuremath{f_\text{HS}(T,n)}}
\newcommand{\fdisp}{\ensuremath{f_\text{disp}(T,n)}}
\newcommand{\fattr}{\ensuremath{f_\text{attr}(T,n)}}

% pair potential
\newcommand{\1}{\ensuremath{\textbf{r}_1}}
\newcommand{\2}{\ensuremath{\textbf{r}_2}}
\newcommand{\3}{\ensuremath{\textbf{r}_3}}
\newcommand{\4}{\ensuremath{\textbf{r}_4}}

% fixme
\newcommand{\fixme}[1]{\textcolor{red}{\textbf{[#1]}}}
\newcommand{\davidsays}[1]{\textcolor{blue}{\textit{[#1]}}}

%%%%%%%%%%%%%%%%%%%%%%%%%%%%%%%%%%%%%%%%%%%%%%%%%%%%%%%%%%%%

\begin{document}
\title{Applying Renormalization Group Theory to the Square Well Liquid}

\author{Dan Roth}
\affiliation{Department of Physics, Oregon State University, Corvallis, OR
97331}

%%%%%%%%%%%%%%%%%%%%%%%%%%%%%%%%%%%%%%%%%%%%%%%%%%%%%%%%%%%%
\begin{abstract}

Near the critical point of a liquid, there are density \fixme{is
  saying $g(\rr)$ more accurate?} fluctuations at all length
scales. In this regime, mean field theories break down, for obvious
reasons. As one approaches the critical point, there becomes less and
less of a difference in free energy of the liquid and vapor states;
this causes difficulty in modeling liquid-vapor coexistence. This work
develops a method for calculating liquid-vapor coexistence, and
applies renormalization group theory to the square well liquid.

%\tableofcontents
\end{abstract}

\maketitle

%%%%%%%%%%%%%%%%%%%%%%%%%%%%%%%%%%%%%%%%%%%%%%%%%%%%%%%%%%%%
\section{Introduction}
\fixme{Write introduction; why do we do this? why is it useful? how is this relevant? what is the history?}

%%%%%%%%%%%%%%%%%%%%%%%%%%%%%%%%%%%%%%%%%%%%%%%%%%%%%%%%%%%%
\section{Theory}\label{sec:theory}

This work applies entirely to the homogeneous case---that is, there
are no walls containing our liquid, nor are there any sort of solutes
in the liquid. Since our fluid is of infinite extent, the free energy
is infinite. Therefore, I work with free energy density:
\begin{align}
  f(T,n) &\equiv \frac{F(T,n)}{V}
\end{align}

\subsection{Renormalization Group Theory}\label{subsec:RGT}
Renormalization group (RG) theory is an iterative procedure that was
developed to deal with large-scale fluctuations of the order parameter
near the critical point of a system. The formulation used here is
provided by Forte~\textit{et al}.\cite{Forte11}

\fixme{Discuss Wilson phase cell}

We begin with a baseline free enrgy $f_0(T,n)$. We assume this
baseline accounts for all fluctuations of wavelength $\lambda <
\lambda_0$. We then double the size of our phase cell and add a
correction to the free energy $\delta f_1(T,n)$. This correction
accounts for fluctuations with a wavelength $\lambda =
2\lambda_0$. The new free energy is $f_1(T,n) = f_0(T,n) + \delta
f_1(T,n)$. Then double the length scale again; $\delta f_2(T,n)$
accounts for fluctionations with $\lambda = 4\lambda_0$ and $f_2(T,n)
= f_1(T,n) + \delta f_2(T,n)$. As we continue this process, the
$i^{th}$ correction accounts for fluctuations with $\lambda =
2^i\lambda_0$, and $f_i(T,n) = f_{i-1}(T,n) + \delta f_i(T,n)$. In the
end, your free energy is
\begin{align}
  f(T,n) = f_0(T,n) + \sum_{i=1}^\infty\delta f_i(T,n) + \fattr
\end{align}

Where $\fattr$ is the contribution due to the longest-wavelength
density fluctuations. At this stage, I do not entirely understand why
Forte splits up $\fattr$.

Forte gives $\delta f_i(T,n)$ as
\begin{align}
  \delta f_i(T,n) &= -\frac{\kT}{V_D}\ln\left[ \frac{Z_D(T,n)}{Z_D^*(T,n} \right]
\end{align}
where
\begin{align}
  Z_D(T,n) &= \int dx\, \exp\left[ -\frac{V_D}{\kT}\left( \bar{f}_D(T,n) + \bar{u}_D(T,n \right) \right] \\
  Z_D^*(T,n) &= \int dx\, \exp\left[ -\frac{V_D}{\kT}\bar{f}_D(T,n) \right]
\end{align}
and
\begin{align}
  \bar{f}_D(T,n) &= \frac{f_{i-1}(T,n+x) + f_{i-1}(T,n-x)}{2} - f_{i-1}(T,n) \\
  \bar{u}_D(T,n) &= \frac{u(T,n+x,\lambda_D) + u(T,n-x,\lambda_D)}{2} - u(T,n,\lambda_D)
\end{align}
Finally, $u(T,n)$ is given by
\begin{align}
  \begin{split}
  u(T,n) = &{} g^\text{HS}(n_{eff};\sigma)\alpha - \left( \frac{4\pi^2}{2^{2i+1}L^2} \right)\alpha\omega^2 \\ &{} + \left( \frac{4\pi^2}{2^{2i+1}L^2} \right)^2\alpha\gamma
  \end{split}
\end{align}
with
\begin{align}
  \alpha &= \frac{2}{3}\pi\epsilon\sigma^3(\lambda^3 - 1) \\
  \omega^2 &= \frac{1}{5}\sigma^2\frac{\lambda^5 - 1}{\lambda^3 - 1} \\
  \gamma &= \frac{1}{70}\sigma^4\frac{\lambda^7 - 1}{\lambda^3 - 1}
\end{align}


\subsection{Thermodynamic perturbation theory}\label{subsec:TPT}

\fixme{And introduction to TPT---why do we use it?}\davidsays{We use
  TPT because it gives a good description of simple liquids, as long
  as you are far from the critical point.  Thus it covers most of our
  bases.  It is also an elegant theory, since it comes from a rigorous
  power series expansion in a small parameter... or one that one hopes
  is small. }

This discussion follows primarily from \textit{Theory of Simple
  Liquids}, by Hansen and McDonald\cite{Hansen06}.

Thermodynamic perturbation theory treats the intermolecular potential
as separable into two parts, one for short-range repulsion and another
for long-range attraction.

The pair potential $v(\1,\2)$ describes the interaction between two
hard spheres. For the square-well model, the pair potential is
\begin{align}
  v_\text{HS}(\1,\2) &=
    \begin{cases}
      \infty & r < \sigma \\
      0 & \sigma < r
    \end{cases}
\end{align}
\fixme{When $v_\lambda$ is derived below, the $w$
should be the attraction portion (i.e.~the $-\varepsilon$ between $\sigma$ and $\lambda\sigma$), and $v_{HS}$ should
just be the hard-sphere potential (without attraction).  That's how
the TPT works, it is the attraction that is being added in with the
$\lambda$ proportionality.}

Now, consider this to be the reference, and add a small perturbation:
\begin{align}
  v_\lambda(\1,\2) &= v_\text{HS}(\1,\2) + \lambda w(\1,\2) \label{eqn:small-perturbation}
\end{align}

The total potential energy of our system and its derivatives with resepect to $\lambda$---which become important later---are
\begin{align}
  V_N(\lambda) &= \sum_{i=1}^N\sum_{j>i}^N v_\lambda(\mathbf{r}_i,\mathbf{r}_j) \\ % \label{eq:VN} \\
  \hookrightarrow \frac{\partial V_N(\lambda)}{\partial\lambda} &= \sum_{i=1}^N\sum_{j>i}^N w(\mathbf{r}_i,\mathbf{r}_j) \nonumber \\
  &\equiv W_N \\
  \hookrightarrow \frac{\partial^2 V_N(\lambda)}{\partial\lambda^2} &= 0
\end{align}

The \emph{configuration integral} (the potential energy component of the partition function) and its derivative with respect to $\lambda$ are
\begin{align}
  Z_N(\lambda) &= \int d\rr^N\, e^{-\beta V_N(\lambda)} \\
  \hookrightarrow \frac{\partial Z_N(\lambda)}{\partial\lambda} &=  \int d\rr^N\, \frac{\partial}{\partial\lambda}e^{-\beta V_N(\lambda)} \\
  &= \int d\rr^N\, -\beta\frac{\partial V_N(\lambda)}{\partial\lambda}e^{-\beta V_N(\lambda)} \\
  &= \int d\rr^N\, -\beta W_N e^{-\beta V_N(\lambda)}
\end{align}
And the non-ideal part of the free energy (i.e.~the excess free energy) and its derivative are
\begin{align}
  F &= -\kT\ln\left( \frac{Z_N}{V^N} \right) \\
  \hookrightarrow \frac{\partial F}{\partial\lambda} &= \frac{V^N}{Z_N}\frac{\partial}{\partial\lambda}\left( \frac{Z_N}{V^N} \right) \\
  &= -\kT\left( \frac{1}{Z_N}\frac{\partial Z_N}{\partial\lambda} \right) \\
  &= \frac{1}{Z_N} \int d\rr^N\, W_N e^{-\beta V_N(\lambda)}\\
  &= \left\langle W_N \right\rangle_\lambda
\end{align}
(where $-\kT$ and $-\beta$ have divided out to $1$). Here, $V^N$ is
the total volume taken to the $N^{th}$ power, and
$\left\langle W_N \right\rangle_\lambda$ is the ensemble
average of $W_N$, with the subscript indicating some given
value of $\lambda$. We can simplify this even further by noting
\begin{align}
  \left\langle W_N \right\rangle &= \left\langle \sum_i\sum_jw(\mathbf{r}_i,\mathbf{r}_j) \right\rangle \\
  &= \frac{1}{2}N^2\left\langle w(\1,\2) \right\rangle
  \intertext{which gives us}
  \frac{\partial F}{\partial\lambda} &= \frac{1}{2}N^2\left\langle w(\1,\2) \right\rangle_\lambda
\end{align}

At this point, note that we have an exact equation with no approximation. Now take a series expansion:
\begin{align}\begin{split} % too long; use split environment, and align to a space (&{})
  \left\langle W_N(\lambda) \right\rangle_\lambda = &{} \left\langle W_N(\lambda)\right\rangle_{\lambda = 0} + (\lambda)\frac{\partial}{\partial\lambda}\left\langle W_N(\lambda) \right\rangle_{\lambda}\bigg|_{\lambda = 0} \\ &{} + \mathcal{O}(\lambda^2)
\end{split}\end{align}
The derivative term is
\begin{align}
  \frac{\partial}{\partial\lambda}\left\langle W_N(\lambda) \right\rangle_{\lambda} &= \left\langle V_N^{''} \right\rangle_\lambda - \beta\sigma_\lambda^2
\end{align}
where
\begin{align}
  \sigma_\lambda^2 &\equiv \left\langle \left[ W_N(\lambda) \right]^2 \right\rangle_\lambda - \left\langle W_N(\lambda) \right\rangle^2_\lambda
\end{align}
Now note that $W_N$ has no dependence on $\lambda$, therefore $W_N' = 0$, and
\begin{align}
  \frac{\partial}{\partial\lambda}\left\langle W_N(\lambda) \right\rangle_{\lambda} &= - \beta\sigma_\lambda^2 \\
  \therefore \left\langle W_N(\lambda) \right\rangle_\lambda &= \left\langle W_N(\lambda)\right\rangle_{\lambda = 0} - \lambda\beta\sigma_{\lambda=0}^2 + \mathcal{O}(\lambda^2)
\end{align}

\davidsays{The derivative is (pretty) easy if you write your ensemble
  average as a multidimensional integral, as in
  Eq.~\ref{eqn:dFdlambda}.  Write down $dF/d\lambda$ in terms of $Z_N$
  and an integral of Boltzmann factors times $W_N$.  Then when you
  take a derivative, it hits top and bottom, but each is an integral
  of an exponential, so those are easy to differentiate.  A little
  math will give you this variance formula.}

\fixme{Below this, I have not updated}

Putting this together, we have
\begin{align}
  \begin{split}
    \beta F = &{} \beta F_0 + \beta(1 - 0)\left\langle W_N(\lambda) \right\rangle_{\lambda = 0} \\ &{} - \beta\frac{1}{2}(1 - 0)^2\left(\left\langle W_N'(\lambda = 0) \right\rangle_{\lambda = 0} - \beta^2\sigma^2_{\lambda = 0} \right)
  \end{split} \\
  \begin{split}
    =&{}  \beta F_0 + \beta\left\langle W_N(\lambda) \right\rangle_{\lambda = 0} \\ &{} - \beta\frac{1}{2}\left(\left\langle W_N'(\lambda = 0) \right\rangle_{\lambda = 0} - \beta^2\sigma^2_{\lambda = 0} \right) \label{eqn:complicated-HTE}
  \end{split}
\end{align}
Now, we can define the total perturbation energy as
\begin{align}
  W_N &= \sum_{i=1}^N\sum_{j>i}^N w_\lambda(\mathbf{r}_i,\mathbf{r}_j)
\end{align}
Then, following from equation~\ref{eqn:small-perturbation},
\begin{align}
  W_N &= \frac{\partial}{\partial\lambda}\left(\sum_{i=1}^N\sum_{j>i}^N \lambda w(\mathbf{r}_i,\mathbf{r}_j)\right) \nonumber \\
  &= W_N \\
  \therefore W_N' &= 0
\end{align}
\davidsays{Ah ha! Here you do what I mentioned earlier.  I'd tend to
  put this earlier, so you can throw away the $V_N''$ term sooner.
  And also talking about $W_N$ with this definition I would find nicer
  and clearer than all the $V_N'$.}  With this in mind, we can
simplify equation~\ref{eqn:complicated-HTE} to the
\emph{high-temperature expansion}, first derived by
Zwanzig:\cite{Zwanzig54}
\begin{align}
  \begin{split}
    \beta F = &{} \beta F_0 + \beta\left\langle W_N \right\rangle_0  \\ &{} - \frac{1}{2}\beta^2\left( \left\langle W_N^2 \right\rangle_0 - \left\langle W_N \right\rangle_0^2 \right) + \mathcal{O}(\beta^3)
  \end{split}
\end{align}
Zwanzig further showed that the $n^{th}$ term could be written in terms of the mean fluctuations $\left\langle \left[ W_N - \left\langle W_N\right\rangle_0 \right]^\nu \right\rangle_0$ with $\nu \leq n$.

Writing the statistical averages in terms of the pair density
$n_\lambda^{(2)}(\1,\2)$ (i.e. the probability of having a sphere at
\1 and \2 \fixme{make this more clear?}\davidsays{I'd perhaps phrase
  it as ``probability per volume squared of having one sphere at \1
  and another at \2''}) gives us
\begin{align}
  \frac{\beta F}{N} &= \frac{\beta F_0}{N} + \frac{\beta}{2N}\int_0^1 d\lambda\, \iint d\1 d\2\, n_\lambda^{(2)}(\1,\2)w(\1,\2) \label{eqn:Henderson5-2-12}
\end{align}
\davidsays{Here's another place where all the $N$s on the bottom are
  muddying things up, I think.  (like the $\beta$s were before).  I
  also think this discussion is pretty redundant with parts of the
  previous derivation in terms of $V_N$.}  The pair density itself can
be expanded as
\begin{align}
  n_\lambda^{(2)}(\1,\2) &= n_0^{(2)}(\1,\2) + \lambda\frac{\partial n_\lambda^{(2)}(\1,\2)}{\partial\lambda}\bigg|_{\lambda = 0} + \mathcal{O}(\lambda^2)
\end{align}
Applying this to the \fixme{0th, 1st or 2nd, depending on how you
  think of it}\davidsays{it's ``second'' since ``terms'' are things
  that are added together, while factors are things that are
  multiplied together.  And in Eq~\ref{eqn:Henderson5-2-12} there are
  only two things added together.  (I figured this terminology out on
  my own some time after becoming a professor... it seems to be used,
  but not taught).} term in equation~\ref{eqn:Henderson5-2-12}, we
have
\begin{align}
  \frac{\beta F_1}{N} &= \frac{\beta}{2N}\iint d\1 d\2\, n_0^{(2)}(\1,\2)w(\1,\2) \\
  &= \frac{\beta n}{2}\int d\mathbf{r}_{12}\, g_0(\1,\2)w(\1,\2)
\end{align}
The second-order $\lambda$ term is quite a bit more complicated:
\begin{widetext}
 \begin{align}
   \begin{split}
   \frac{\beta F_2}{N} =&{} -\frac{\beta^2}{2}\left[ \frac{n}{2}\int d\2\, g_0(\1,\2)w^2(\1,\2) \right. \\ &{} + n^2 \iint d\2 d\3\, g_0^{(3)}(\1,\2,\3)w(\1,\2)w(\1,\3) \\ &{} \left. + \frac{n^3}{4}\iiint d\2 d\3 d\4\, \left( g_0^{(4)}(\1,\2,\3,\4) - g_0^{(2)}(\1,\2)g_0^{(2)}(\3,\4) \right)w(\1,\2)2(\3,\4) \right] \\ &{} - \frac{1}{4}S_0(0)\left[ \frac{\partial}{\partial n}\left( n^2\int d\2\, g_0(\1,\2)w(\1,\2) \right)^2 \right]
   \end{split}
 \end{align}
\end{widetext}
This term requires use to know the form of higher-order correlation functions and the structure function $S_0(k)$ of the reference system. These things are very difficult. Hence, we use approximations, as outlined in section~\ref{sub2sec:disp}.

\davidsays{ you can say something to the effect
  of ``in the next section we will see a simple approximation for this
  term.''  Hansen's explanation of TPT is not very clear, which makes
  it all the more valuable to make \emph{this} a clear and lucid
  explanation.  I'd love to see just a couple more steps at key
  places---largely where you were confused. This
  has been an excellent first shot, though! }


\subsection{Square well liquid free enrgy}\label{subsec:SW}
For the baseline, this work is based on the square well liquid free energy:\cite{Hughes13}
\begin{align}
  f_0(T,n) &= \fid + \fhs + \left( \fdisp - n a_1(n) \right) \ .
\end{align}
The long-range attractive term, $\fattr$ is given by
\begin{align}
  \fattr &= n a_1(n) \ .
\end{align}
$\fid$ is the ideal gas free energy, $\fhs$ is the hard sphere
repulsion, and $\fdisp$ is the attraction. The function $a_1(n)$ is
the first term from a high-temperature perturbation expansion,
described in further detail in sections~\ref{sub2sec:disp}
and~\ref{subsec:TPT}. Each function is outlined below.

\subsubsection{Ideal gas}\label{sub2sec:ID}
The ideal gas term is given as
\begin{align}
  \fid &= n\kT\left(\log(n) - 1\right)
\end{align}

\subsubsection{Hard sphere repulsion}\label{sub2sec:HS}
Hard sphere repulsive forces are dealt with by the White Bear version
of Fundamental-Mearuse Theory.\cite{Roth02} The hard sphere excess
free energy is given in the homogeneous case as
\begin{align}
  \fhs &= \kT\left( \Phi_1(n) + \Phi_2(n) + \Phi_3(n) \right)
\end{align}
The $\Phi_j(n)$ terms are given as
\begin{align}
  \Phi_1(n) &= -n_0(n)\ln(1 - n_3(n)) \\
  \Phi_2(n) &= \frac{n_1(n)n_2(n)}{1 - n_3(n)} \\
  \Phi_3(n) &= n_2^3(n)\left( \frac{n_3(n) + (1 - n_3(n))^2\ln(1 - n_3(n))}{36\pi n_3^2(n)(1 - n_3(n))^2} \right)
\end{align}
The \emph{fundamental measure densities} are
\begin{align}
  n_0(n) &= \frac{n_2}{4\pi R^2} \nonumber \\
  &= n \\
  n_1(n) &= \frac{n_2}{4\pi R} \nonumber \\
  &= nR \\
  n_2(n) &= n4\pi R^2 \\
  n_3(n) &= n\frac{4}{3}\pi R^3 \nonumber \\
  &= \eta
\end{align}
where $\eta$ is the packing fraction.

\subsubsection{Attraction}\label{sub2sec:disp}
The attractive free energy includes van der Waals attraction between
hard spheres and is founded in Thermodynamic Perturbation Theory
(TPT).
\begin{align}
  \fdisp &= n \left( a_1(n) + \frac{1}{\kT}a_2(n) \right)
\end{align}
The $a_j(n)$ terms come from TPT, outlined in section~\ref{subsec:TPT}.

The first term, $a_1$, is the mean-field dispersion interaction. The
second term, $a_2$, describes the effect of fluctuations resulting
from compression of the fluid due to the dispersion interaction
itself, and is approximated using the local compressibility
approximation (LCA), which assumes the energy fluctuation is simply
related to the compressibility of a hard-sphere reference
fluid.\cite{Barker76}

In Gil-Villegas' paper,\cite{gil-villegas97} HS and Dispersion are wrapped up into the
\textit{monomer} contribution, expressed in terms of energy densities
as:
\begin{align}
  f_\text{mono} = f_\text{HS} - \frac{\alpha^\text{VDW}n}{kT}
\end{align}
with $\alpha^\text{VDW}$ given by
\begin{align}
  \alpha^\text{VDW} &= 2\pi\epsilon\int_\sigma^\infty r^2\phi(r)\,dr \\
  \intertext{using reduced units of $x = r/\sigma$, we have}
  &= 2\pi\sigma^3\int_1^\infty x^2\phi(x)\,dx \\
  &= 3b^\text{VDW}\epsilon\int_1^\infty x^2\phi(x)\,dx
\end{align}
where $b^\text{VDW}$ is the van der Waals size parameter. It
corresponds to the volume excluded by two spherical particles of
volume $b$: $b^\text{VDW} = 4b = 4\left(\pi\sigma^3/6\right)$.

Then, the high-temp expansion is an expansion of the monomer term:
\begin{align}
  f_\text{mono} &= f_\text{HS} + \beta a_1 + \beta^2 a_2 + \cdots
\end{align}

The term $a_1$ is given by
\begin{align}
  a_1 &= -2\pi n \epsilon\int_\sigma^\infty r^2\phi(r)g^\text{HS}(r)\,dr \\
  &= -3 n  b^\text{VDW}\epsilon\int_1^\infty x^2\phi(x)g^\text{HS}(x)\,dx
\end{align}
If we assume random correlations between the particles' positions, for
all distances, we have $g^\text{HS}(r) = 1$. This yields
\begin{align}
  a_1^\text{VDW} = - n \alpha^\text{VDW},
\end{align}
the van der Waals mean-field energy.

The second-order term is quite a bit more complicated; one must have
knowledge of higher-order correlation functions to determine
$a_2$. This term describes the response of the attractive energy due
to the compression of the fluid from the attractive well. Gil-Villegas
\emph{et al.} base their expression on an approximation by Barker and
Henderson.\cite{Barker67} Barker and Henderson's approximation
considers fluctuations in the number of particles in the potential
well. In this way, the fluctuations in $a_2$ are related to the
pressure and compressibility of the liquid. Given pressure
$P^\text{HS}$ and isothermal compressibility $K^\text{HS} =
kT\left(\partial n /\partial P^\text{HS}\right)_T$, we have:
\begin{align}
  a_2 &= -\pi n \epsilon^2kT\int_0^\infty r^2\left[\phi(r)\right]^2\frac{\partial n  g^\text{HS}(r)}{\partial P^\text{HS}}\,dr \\
  &= -\pi n \epsilon^2K^\text{HS}\frac{\partial}{\partial n }\left[\int_\sigma^\infty r^2\left[\phi(r)\right]^2 n  g^\text{HS}\,dr\right] \\
  &= \frac{1}{2}\epsilon K^\text{HS}\frac{\partial}{\partial n }\left[-3 n  b^\text{VDW}\epsilon\int_1^\infty x^2\left[\phi(x)\right]^2 g^\text{HS}(x)\,dx \right]
\end{align}

We use the isothermal compressibility in terms of packing fraction
$\eta$, given by the Percus-Yevick exprssion:\cite{Barker76}
\begin{align}
  K^\text{HS} &= \frac{\left(1 - \eta\right)^4}{1 + 4\eta + 4\eta^2}
\end{align}
%%%%%%%%%%%%%%%%%%%%%%%%%%%%%%%%%%%%%%%%%%%%%%%%%%%%%%%%%%%%
\section{Methods}\label{sec:methods}

\subsection{Coexistence Curve Algorithm}\label{subsec:coexis}
Coexistence curves are found by modifying the chemical potential $\mu$
in the grand free energy per volume $\phi(T,n)/V$. Grand free energy
density is defined as
\begin{align}
  \frac{\Phi(T,n)}{V} &= \phi(T,n) \nonumber \\
                 &= f(T,n) - n\mu \nonumber \\
                 &= f(T,n) - n\frac{df(T,n)}{dn}\bigg|_{\npart}\ .
\end{align}
It is the value $\npart$ that we adjust to find coexistence curves.

\begin{figure}
  \centering
  \includegraphics[width=\columnwidth]{figs/SW-phi-lowT}
  \caption{Grand free energy per unit volume, SW, low temp}
  \label{fig:SW-phi-lowT}
\end{figure}

The grand free energy density generally has two distinct minima when
plotted versus density. (See fig.~\ref{fig:SW-phi-lowT} for an
example.) The liquid will be in liquid-vapor equilibrium when those
minima have the same value for $\phi(T,n)$. There is a local maximum
between those minima, and the value of the density at that maximum is
used for $\npart$. Find this value at a given temperature then plot
$\phi(T,n)$ vs $n$ at slightly higher temperature. Find the new value
for $\npart$ such that the two minima have the same value for
$\phi(T,n)$. (It is helpful to note that if $\npart$ is increased,
$\phi(T,\nliq)$ increases, and $\phi(T,\nliq)$ decreases if $\npart$
is decresed.) The program I wrote uses the previous temperature's
$\npart$ as a ``first guess,'' maximizing $\phi(T,\npart)$. From
there, the program adjusts $\npart$ until $\phi(T,\nvap) =
\phi(T,\nliq)$. When that condition is met, it uses this new value of
$\npart$ as the ``first guess'' at $\npart$ for the next higher
temperature.

The algorithm is as follows; start at some temperature $T=T_{start}$,
with an intial guess for $\npart$.
\begin{enumerate}
  \item Calculate $\nvap$ and $\nliq$ by calculating the minima in $\phi(T_{start},n)$ over appropriate regions
  \item Calculate $\phi(T_{start},\nvap)$ and $\phi(T_{start},\nliq)$ \label{while-start}
  \item Calculate \[\delta\mu = \frac{\phi(T_{start},\nvap) - \phi(T_{start},\nliq)}{\nliq - \nvap}\]
  \item Calculate $\phi^*(T_{start},n) = \phi(T_{start},n) + \delta\mu$
  \item Calculate a new $\npart$, $\npart^*$, based on $\phi^*(T_{start},n)$ by finding the local maximum between $\nvap$ and $\nliq$
  \item Calculate $\phi(T_{start},\npart^*)$
  \item Re-caluclate $\nvap$, $\nliq$, $\phi(T_{start},\nvap)$, and $\phi(T_{start},\nliq)$, using $\npart^*$
  \item Go back to step~\ref{while-start} and repeat until \[ \frac{\phi(T_{start},\nvap) - \phi(T_{start},\nliq)}{\phi(T_{start},\npart^*)} > tol  \] where $tol$ is some computational tolerance (I used $tol=10^{-5}$)
  \item Increase the temperature to $T = T + \delta T$ and repeat
\end{enumerate}

\subsection{Free Energies}\label{subsec:free-energies}
The $a_j(n)$ terms require even more simplification when
``converting'' from the mathematical formulation to Python code, to
rid ourselves of troublesome integrals. Gil-Villegas gives us
\newcommand\eff{\textit{eff}}
\begin{align}
  a_1^\text{SW}(n) &= a_1^\text{VDW}(n)g^\text{HS}(1;\eta_{\eff}) \\
  &= -4\eta\epsilon(\lambda^3 - 1)\frac{1 - \left( \eta_{\eff}/2 \right)}{(1 - \eta_{\eff})^3}
\end{align}
and
\begin{align}
  \eta_{\eff} &= c_1\eta + c_2\eta^2 + c_3\eta^3
\end{align}
with
\begin{align}
  \left( \begin{array}{c}
    c_1 \\
    c_2 \\
    c_3
    \end{array} \right)
  &= \left( \begin{array}{ccc}
    2.25855 & -1.50349 & 0.249434 \\
    -0.669270 & 1.40049 & -0.827739 \\
    10.1576 & -15.0427 & 5.30827
    \end{array} \right)
  \left( \begin{array}{c}
    1 \\
    \lambda \\
    \lambda^2
    \end{array} \right)
\end{align}

\subsection{RG partition functions}\label{subsec:fbar-ubar}
I use the midpoint method for evaluating the integrals of $\bar{f}_D$
and $\bar{u}_D$.

%%%%%%%%%%%%%%%%%%%%%%%%%%%%%%%%%%%%%%%%%%%%%%%%%%%%%%%%%%%%
\section{Results and discussion}

For this work, I use $\lambda_\text{SW} = 1.5$.

\begin{figure}
  \begin{center}
  \includegraphics[width=\columnwidth]{figs/coexistance_SW}
  \end{center}
  \caption{Liquid-vapor coexistance for a square well liquid.}
  \label{fig:coexistance_SW}
\end{figure}

\begin{figure}
  \begin{center}
  \includegraphics[width=\columnwidth]{figs/phi-RG-lowT}
  \end{center}
  \caption{Free energy density for RG at low temperature}
  \label{fig:phi-RG-lowT}
\end{figure}

\begin{figure}
  \begin{center}
  \includegraphics[width=\columnwidth]{figs/phi-RG-highT}
  \end{center}
  \caption{Free energy density for RG near critical temp}
  \label{fig:phi-RG-highT}
\end{figure}

Figures~\ref{fig:phi-RG-lowT} and~\ref{fig:phi-RG-highT} show the
grand free energy per unit volume at a low temperature, and near the
critical point. It is clear that the program is not de-bugged.


%%%%%%%%%%%%%%%%%%%%%%%%%%%%%%%%%%%%%%%%%%%%%%%%%%%%%%%%%%%%

\section{Conclusion}
I have implemented a the use of renormalization group theory in
calculations for the square well liquid. The program is not entirely
de-bugged. I have also developed an algorithm that calculates
liqiud-vapor coexistence curves, even close to the critical point.


\bibliographystyle{unsrt}
\bibliography{project} % Produces the bibliography via BibTeX.

\end{document}

