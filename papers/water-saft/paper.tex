\documentclass[twocolumn,amsmath,amssymb,prl]{revtex4-1}
\usepackage{graphicx}% Include figure files
\usepackage{dcolumn}% Align table columns on decimal point
\usepackage{bm}% bold math
\usepackage{color}

%%%%%%%%%%%%%%%%%%%%%%%%%%%%%%%%%%%%%%%%%%%%%%%%%%%%%%%%%%%%
%% Definitions
\def\re{\text{Re}}
\def\im{\text{Im}}
\def\ket#1{\vert #1 \rangle}
\def\cU{{\cal{U}}}
\def\cD{{\cal{D}}}
\def\re{\text{Re}}
\def\im{\text{Im}}
\newcommand{\red}[1]{{\bf \color{red} #1}}
\newcommand{\blue}[1]{{\bf \color{blue} #1}}
\newcommand{\green}[1]{{\bf \color{green} #1}}
\newcommand{\rr}{\textbf{r}}
\newcommand{\xx}{\textbf{r}}
\newcommand{\refnote}{\red{[ref]}}

\newcommand\etadisp{\ensuremath{\eta_\textit{d}}}
\newcommand\epsilondisp{\ensuremath{\epsilon_\textit{d}}}
\newcommand\epsilonassoc{\ensuremath{\epsilon_\textit{a}}}
\newcommand\kappaassoc{\ensuremath{\kappa_\textit{a}}}
\newcommand\lambdadisp{\ensuremath{\lambda_\textit{d}}}
\newcommand\lscale{\ensuremath{s_d}}

% fixme is intended to prioritize stuff that needs fixing.
\newcommand{\fixme}[1]{\textcolor{red}{[\emph{#1}]}}

%%%%%%%%%%%%%%%%%%%%%%%%%%%%%%%%%%%%%%%%%%%%%%%%%%%%%%%%%%%%

\begin{document}
\title{A Classical Density-Functional Theory for Describing Water Interfaces}

\author{Eric J. Krebs}
\affiliation{Department of Physics, Oregon State University,
  Corvallis, OR 97331, USA}
\author{Jeff B. Schulte}
\affiliation{Department of Physics, Oregon State University,
  Corvallis, OR 97331, USA}
\author{David Roundy}
\affiliation{Department of Physics, Oregon State University,
  Corvallis, OR 97331, USA}

\begin{abstract}
We develop an improved version of our old functional.
\end{abstract}
\maketitle

\section{Introduction}

Water, the universal solvent, is of critical practical importance, and
a continuum description of water is in high demand for a solvation
model.  A number of recent attempts to develop improved solvation
models for water have built on the approach of classical
density-functional theory (DFT)~\cite{jeanmairet2013molecular,
  zhao2011molecular, zhao2011new, ramirez2005direct,
  ramirez2005density, levesque2012solvation, levesque2012scalar}.
Classical DFT is based on a description of a fluid written as a free
energy functional of the density distribution.  There are two
approaches used to construct a classical DFT for water.  The first is
to choose a functional form that is fit to properties of the bulk
liquid at a given temperature and pressure
\cite{jeanmairet2013molecular} ~\fixme{cite Jeanmairet and similar}.
Using this approach, it is possible to construct a functional that
reproduces the exact second-order response function of the liquid
under the fitted conditions.  This class of functional, however, will
be less accurate at other temperatures or pressures---and in the
inhomogeneous scenarios in which solvation models are applied.  The
second approach is construct a functional by applying liquid-state
theory to a model system, and then fit the model to experimental data
such as the equation of state.  Classical DFT is a natural framework
for creating a more flexible theory of hydrophobicity that can readily
describe interaction of water with arbitrary external
potentials---such as potentials describing strong interactions with
solutes or surfaces.

A widely used family of models used in the development of classical
density functionals is based on Statistical Associating Fluid Theory
(SAFT). SAFT is a theory for fluids in which hydrogen bonding plays a
significant role\cite{chapman1989saft, muller2001molecular}. SAFT is
used to accurately model the equations of state of both pure fluids
and mixtures over a wide range of temperatures and pressures.
\fixme{Talk about and cite papers using SAFT to build DFTs.}

SAFT is based on Wertheim's first-order thermodynamic perturbation
theory (TPT1)\cite{wertheim1984fluidsI, wertheim1984fluidsII,
  wertheim1986fluidsIII, wertheim1986fluidsIV}, which models
\fixme{fluids} as hard-spheres with strong associative interactions at
sites on the surfaces of the spheres. This is ideal for water, since
its four hydrogen bond sites can be represented by these association
sites. The association sites have an effective range and a small
energy contribution when a site is within range of another
hard-sphere's site. The free energy due to this association depends on
the radial distribution function $g_\sigma^\text{HS}(\rr)$. Our
previous water functional used Yu and Wu's radial distribution
function \cite{yu2002fmt-dft-inhomogeneous-associating}, which was
found to agree less with hard-sphere Monte-Carlo simulation than a
distribution function developed by us in a recent paper
\cite{schulte2012using}. In our paper, we developed an approximation
for correlation at contact to describe the hard-sphere fluid.

In a recent paper \cite{hughes2013classical}, we constructed a
classical DFT for water which uses SAFT as a basis for a free energy
functional. We used Clark \emph{et al}'s \cite{clark2006developing}
five fitting parameters for an optimal cDFT for water and, in
addition, introduced a parameter for the length scale over which the
dispersion interaction is correlated.

\section{Method}

We modify our previous association free energy term to use our
$g_{\sigma}^\textit{A}$ rather than Yu and Wu's $g_\sigma$.  The
association free energy for our four-site model has the form:
\begin{align}
  F_\text{assoc}[n] &= k_BT \int n_\text{site}(\xx)
  \left(\ln X(\xx) - \frac{X(\xx)}{2} + \frac12\right) d\xx
\end{align}
where
\begin{align}
  n_\text{site}(\rr) &=
  \begin{cases}
    4 n(\rr) & \text{this work}\\
    4 n_0(\rr) \zeta(\rr) & \text{Ref.~\citenum{hughes2013classical}}
  \end{cases} 
\end{align}
is the density of bonding sites at position~$\rr$.  The four in
$n_\text{site}(\rr)$ comes from the four hydrogen bond sites,
$n_0(\rr)$ is the average density contacting point $\rr$, and
$\zeta(\xx)$ is a dimensionless measure of the density inhomogeneity
from Yu and Wu~\cite{yu2002fmt-dft-inhomogeneous-associating}.  The
functional $X(\rr)$ is the fraction of association sites \emph{not}
hydrogen-bonded, which is determined by the quadratic equation
\begin{align}
  X(\xx) &= \frac{\sqrt{1 + 2n_\text{site}'(\rr)
      \kappaassoc g^\textit{SW}_\sigma(\xx)
  \left(e^{-\beta\epsilonassoc} - 1\right)} - 1}
  {n_\text{site}'(\rr)
    \kappaassoc g^\textit{SW}_\sigma(\xx)
  \left(e^{-\beta\epsilonassoc} - 1\right)}, \label{eq:X}
\end{align}
where
\begin{align}
  n_\text{site}'(\rr) &=
  \begin{cases}
    \frac{4}{\pi\sigma^2} \int n(\rr')\delta(\sigma - |\rr-\rr'|) d\rr' & \text{this work}\\
    4 n_0(\rr) \zeta(\rr) & \text{Ref.~\citenum{hughes2013classical}}
  \end{cases} 
\end{align}
is the density of bonding sites that could bond to the sites~$n_\text{site}(\rr)$, and
\begin{align}
  g^\textit{SW}_\sigma(\xx) &= g^\textit{HS}_\sigma(\xx) +
  \frac{1}{4}\beta\left(\frac{\partial a_1}{\partial \etadisp(\xx)} -
  \frac{\lambdadisp}{3 \etadisp}\frac{\partial a_1}{\partial \lambdadisp}\right)\label{eq:gSW},
\end{align}
%This is equation 77 of gil-villegas-1997-SAFT-VR
where $g^\textit{SW}_\sigma$ is the correlation function evaluated at
contact for a hard-sphere fluid with a square-well dispersion
potential, and $a_1$ and $a_2$ are the two terms in the dispersion
free energy.  The radial distribution function $g^\textit{SW}_\sigma$ is
written as a perturbative correction to the hard-sphere fluid
correlation function $g^\textit{HS}_\sigma$. Four our old functional
we used Yu and Wu's radial distribution function, and for this paper we
use our $g^\textit{A}_\sigma$.

\begin{figure}
\begin{center}
\includegraphics[width=3.5in]{figs/surface-tension}
\end{center}
\caption{Comparison of Surface tension versus temperature for
  theoretical and experimental data. The experimental data is taken
  from NIST~\cite{nistwater}.  The length-scaling parameter $\lscale$
  is fit so that the theoretical surface tension will match the
  experimental surface tension near room temperature.}
\label{fig:surface-tension}
\end{figure}

As in Ref.~\citenum{hughes2013classical}, we use Clark's five
empirical parameters, and fit the calculated surface tension to
experimental surface tension at ambient conditions by tuning the
parameter $\lscale$, which adjusts the length-scale of the average
density used for the dispersion interaction.  With the improved
association term, we find these agree when $\lscale$ is 0.454, which
is an increase from the value of 0.353 found in
Ref.~\citenum{hughes2013classical}.  In order to explore further the
change made by the improved association term, we compared the new
functional with that of Ref.~\citenum{hughes2013classical} for the two
hydrophobic cases of hard rods and the hard spherical solutes.

\begin{figure}
\begin{center}
\includegraphics[width=3.5in]{figs/density-compare}
\end{center}
\caption{ Density profiles for a water around a single hydrophobic
  rod of radius 0.2 nm. The solid red profile is the density using our new functional
  and the dashed blue profile the result of our old functional.  For
  scale, under the profiles is a cartoon of a string of hard
  spheres touching in one dimension. The horizontal black dotted line
  is the bulk density for water.}
\label{fig:density-single-rod}
\end{figure}

\section{Results}

We will first discuss the case of a single hydrophobic rod immersed in
water. Figure~\ref{fig:density-single-rod} shows the density profile
of water near a rod with radius~0.1\AA\fixme{Eric, please figure this
  radius out}.  The density computed using the functional of this
paper is qualitatively similar to that from
Ref.~\citenum{hughes2013classical}, with a comparable density at
contact---consistent with having made only a moderate change in the
free energy.  The first density peak near the surface is consistently
higher than that from Ref.~\citenum{hughes2013classical}, and the peak
is not smooth.  This results from the improved accurately of the
$g_\sigma$ from Ref.~\citenum{schulte2012using}, since beyond the
first peak water molecules are unable to touch---or hydrogen bond
to---molecules at the surface of the hard rod.

\begin{figure}
\begin{center}
\includegraphics[width=3.5in]{figs/single-rod-broken-HB}
\end{center}
\caption{Broken hydrogen bonds per nanometer for various hydrophobic rods
  immeresed in water.  The solid red line uses our new functional
  while the dashed blue line uses the old one. For large enough rods,
  the graph increases linearly for both functionals.}
\label{fig:single-rod-broken-HB}
\end{figure}

A common test case for studying hydrophobic solutes in water is the
hard-sphere solute.  Figure~\ref{fig:spheres-broken-HB} shows results
for the number of broken hydrogen bonds caused by a hard-sphere soute,
as a function of the solute radius.  As in
Fig.~\ref{fig:single-rod-broken-HB}, the number of broken bonds scales
with surface area for large solutes, and the number of broken bonds is
about four times smaller than the number from the functional of
Ref.~\citenum{hughes2013classical}.  For solutes smaller than
3~\AA\ in radius, there is less than a tenth of a hydrogen bond
broken. This is consistent with the well-known fact that small solutes
(unlike large solutes) do not disrupt the hydrogen-bonding network of
water~\cite{chandler2005}.

\begin{figure}
\begin{center}
\includegraphics[width=3.5in]{figs/sphere-broken-HB}
\end{center}
\caption{Broken hydrogen bonds per for various hard spheres
  immeresed in water.  The solid red line uses our new functional
  while the dashed blue line uses the old one.}
\label{fig:spheres-broken-HB}
\end{figure}

\section{Conclusion}

We have modified the classical DFT for water developed by
Hughes~\emph{et al.}~\cite{hughes2013classical} with the more accurate
radial distribution function at contact developed by Schulte~\emph{et
  al.}~\cite{schulte2012using}, which affects the predicted hydrogen
bonding between water molecules.  We found that while this
modification has a relatively mild effect on the free energy and
density profiles, it predicts fewer broken hydrogen bonds around
hydrophobic solutes and at aqueous interfaces.

\bibliography{paper}% Produces the bibliography via BibTeX.

\clearpage

\section{Figures that are not included in the paper (but could be)}

\begin{figure}
\begin{center}
\includegraphics[width=3.5in]{figs/single-rod-X-plot}
\end{center}
\caption{ The average number of hydrogen bonds per molecule at
  distances from the center of hydrophobic rods of various
  radii. The solid lines are the results from using the assymetric
  $g_{\sigma}^A$, while the dotted lines are from Yu and Wu's $g_{\sigma}^{HS}$.}
\label{fig:single-rod-X}
\end{figure}

\begin{figure}
\begin{center}
\includegraphics[width=3.5in]{figs/spheres-X-plot}
\end{center}
\caption{ The average number of hydrogen bonds per molecule at
  distances from the center of spherical cavities of various
  radii. The solid lines are the results from using the assymetric
  $g_{\sigma}^A$, while the dotted lines are from Yu and Wu's $g_{\sigma}$.}
\label{fig:spheres-X}
\end{figure}

\end{document}
