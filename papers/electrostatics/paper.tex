% TODO:

% Write ``Liquid-vapor interface'' section A.

% Look up question-mark references (citations).

\documentclass[letterpaper,twocolumn,amsmath,amssymb,prb]{revtex4-1}
\usepackage{graphicx}% Include figure files
\usepackage{dcolumn}% Align table columns on decimal point
%\usepackage{bm}% bold math
\usepackage{color}

\usepackage{feynmp}

% The following code causes pdflatex (if given the -shell-escape flag)
% to automagically call mpost as needed for feynmp diagrams.
\DeclareGraphicsRule{*}{mps}{*}{}
\makeatletter
\def\endfmffile{%
  \fmfcmd{\p@rcent\space the end.^^J%
          end.^^J%
          endinput;}%
  \if@fmfio
    \immediate\closeout\@outfmf
  \fi
  \ifnum\pdfshellescape>\z@
    \immediate\write18{mpost \thefmffile}%
  \fi}
\makeatother

\newcommand{\red}[1]{{\bf \color{red} #1}}
\newcommand{\blue}[1]{{\bf \color{blue} #1}}
\newcommand{\green}[1]{{\bf \color{green} #1}}
\newcommand{\rr}{\textbf{r}}
\newcommand{\xx}{\textbf{x}}
\newcommand{\refnote}{\red{[ref]}}

\newcommand{\fixme}[1]{\red{[#1]}}

% needsworklater is used to annotate bits that need work, but that we
% can postpone for a while.
\newcommand{\needsworklater}[1]{\emph{[#1]}}
% needsworknow is intended to prioritize stuff that needs fixing.
\newcommand{\needsworknow}[1]{\textcolor{red}{[\emph{#1}]}}

\begin{document}
\title{A Classical Density Functional Electrostatic Interactions in Water }

\author{David Roundy}
\affiliation{Department of Physics, Oregon State University, Corvallis, OR 97331}

%%%%%%%%%%%%%%%%%%%%%%%%%%%%%%%%%%%%%%%%%%%%%%%%%%%%%%%%%%%%
\begin{abstract}
\needsworklater{ We develop a functional based on FMT~\cite{roth2002whitebear}
 to ... for SAFT}
\end{abstract}

\maketitle

%%%%%%%%%%%%%%%%%%%%%%%%%%%%%%%%%%%%%%%%%%%%%%%%%%%%%%%%%%%%
\section{Introduction}
We would like to create a classical density functional for water that
correctly describes its interaction with charged systems.  There are a
couple of effects that seem particularly important to get accurately
in order to provide a reasonable description for the solvation of
ions.



%% \begin{fmffile}{diagrams}
%% \begin{fmfchar*}(40,25)
%%   \fmfleft{em,ep} \fmflabel{$e^+$}{ep} \fmflabel{$e^-$}{em}
%%   \fmf{fermion}{em,Zee,ep}
%%   \fmf{photon,label=$\gamma,,Z$}{Zee,Zff}
%%   \fmf{fermion}{fb,Zff,f}
%%   \fmfright{fb,f} \fmflabel{$\bar f'$}{fb} \fmflabel{$f$}{f}
%%   \fmfdot{Zee,Zff}
%% \end{fmfchar*}
%% \end{fmffile}


\unitlength=1mm
\begin{fmffile}{powerseries}
\begin{equation}
  \parbox{15mm}{
    \begin{fmfgraph}(10,10)
      \fmfpen{thick}
      \fmfsurroundn{v}{2}
      \fmfvn{d.sh=circle,d.f=empty}{v}{2}
      \fmf{plain}{v1,v2}
    \end{fmfgraph}
  }
  +
  \parbox{30mm}{
    \hspace{1em}
    \begin{fmfgraph}(20,10)
      \fmfpen{thick}
      \fmfleft{v1}
      \fmfright{v3}
      \fmfv{d.sh=circle,d.f=empty}{v1,v3}
      \fmfv{d.sh=circle,d.f=full}{v2}
      \fmf{plain}{v1,v2,v3}
    \end{fmfgraph}
  }
  +
  \parbox{30mm}{
    \hspace{1em}
    \begin{fmfgraph}(20,10)
      \fmfpen{thick}
      \fmfleft{i}
      \fmfright{o}
      \fmfv{d.sh=circle,d.f=empty}{i,o}
      \fmf{phantom}{i,vmid,o}
      \fmffreeze
      \fmffixed{(0,1.0h)}{vmid,v1}
      \fmf{plain}{i,v1,o}
      \fmfv{d.sh=circle,d.f=full}{v1}
      \fmf{plain}{i,o}
    \end{fmfgraph}
  }
    + \cdots
\end{equation}
\end{fmffile}

First, we should include the nonlocality of the dielectric response:
water is unable to polarize at length scales small compared to its
molecular diameter.  This is important because many ions are in fact
of approximately the same size as water.

Secondly, the polarization of water saturates.  There are only so many
water molecules in a given volume, which means the polarization will
slowly asymptote at large electric fields.  This is relevant because
the  first solvation shell is likely to be close to fully polarized
around a small ion.

Either of these two  effects could be fudged using the other, but a
correct description should include both effects.

\bibliography{paper}% Produces the bibliography via BibTeX.

\end{document}

